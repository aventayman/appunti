\documentclass{report}

\input{preamble}
\input{macros}
\input{letterfonts}

\title{\Huge{Algebra Lineare e Geometria Analitica}\\Ingegneria dell'Automazione Industriale}
\author{\huge{Ayman Marpicati}}
\date{A.A. 2022/2023}

\begin{document}

\maketitle
\newpage% or \cleardoublepage
% \pdfbookmark[<level>]{<title>}{<dest>}
\pdfbookmark[section]{\contentsname}{toc}
\tableofcontents

\mprop{}{Siano \(\alpha \ r\) rispettivamente un piano e una retta di \(\EE_3(\RR)\) con \(\alpha \) non ortogonale a \(r\). Allora \(\exists ! \) piano \(\beta : \beta\) ortogonale \(\alpha\) e \(r \subseteq \beta.\)  }
\pf{Dimostrazione}{Dimostriamo l'esistenza: sia \(\beta = [P, V_1 + V_2 orto]\) dove \(r = [P, V_1]\) e \(\alpha = [Q, V_2]\). \begin{enumerate}
    \item \(\beta\) è un piano perché \(\dim(V_1 = 1), \ \dim(V_2 orto)= 1\) e \(V_1 \neq V_2 orto\) (poiché \(\alpha non orto r\) ) \( \implies \dim(V_1 + V_2 orto) = 2 \implies \beta\) è un piano.
    Per costruzione abbiamo che \(\beta \perp \alpha\), infatti lo spazio di traslazione di \(\beta\) è: \[
        V_1 \oplus V_2^{\perp} \supseteq V_2^{\perp} \text{  e \(V_2\) è lo spazio di traslazione di \(\alpha\)  }
    \] inoltre \(r \) 
\end{enumerate}}

\chapter{Geometria analitica in \(\EE_n(\RR)\) }
\dfn{}{In \(\EE_n(\RR)\) si dice \textbf{riferimento cartesiano ortogonale monometrico} la coppia \([0, \mcB]\): \begin{itemize}
    \item O è un punto di \(\EE_n(\RR)\) 
    \item \(\mcB=(e_1, e_2, ..., e_n)\) è una base ortonormale 
\end{itemize}}
\nt{\begin{enumerate}
    \item In \(\EE_n(\RR) \ (n= 2) \implies \mcB = (i,j)\) 
    \item In \(\EE_3(\RR) \ (n = 3) \implies \mcB = (i,j,k)\)
\end{enumerate}}

\dfn{Ortogonalità fra rette}{Siano \(r_1, r_2\) due rette di \(\EE_2(\RR)\) e sia \(r_1 = [P, f(v)] \quad v = l \cdot i + m \cdot j\), analogamente \(r_2 = [P, f'(v)] \quad v' = l' \cdot i + m' \cdot j\) \[
    v \perp v' \iff l \cdot l' + m \cdot m' = 0
    \] se \(r_1\) ha equazione \(ax + by + c = 0\) e \(r_2\) ha equazione \(a'x + b'y + c' = 0\) allora \(P.d.r_1 = [(-b,a)]\), e \(P.d.r_2 = [(-b', a')]\)  \[
    -b \cdot (-b') + a \cdot a' = bb' + aa' = 0
\]
Se abbiamo \(r_1, r_2\) rette in \(\EE_3(\RR)\) con \(P.d.r_1 = [(l,m,n)]\), \(P.d.r_2 = [(l',m',n')]\)  
\(r_1 \perp r_2 \iff v_1\) generatore della direzione di \(r_1 \perp v_2\) generatore della direzione di \(r_2\). \[
    v_1 = li + mj + nk \qquad v_2 = l'i + m'j + n'k
\] \[
    v_1 \perp v_2 \iff r_1 \perp r_2 \iff ll' + mm' + nn' = 0
\] Analogamente se \(r_1, r_2\) sono rette in \(\EE_n(\RR)\) con \(P.d.r_1 = [(x_1, x_2, ..., x_n)]\), \(P.d.r_2 = [(x_1', x_2', ..., x_n')]\)
}
\section{Direzione \(\perp\) ad un iperpiano}
\mprop{}{Sia \(r: ax + by + c = 0\) una retta di \(\EE_2(\RR)\). Allora \([(a, b)]\) è la classe dei parametri direttori della direzione ortogonale a \(r\).}
\pf{Dimostrazione}{\(P.d.r = [(-b,a)]\) e abbiamo che \((a, b) \cdot (-b, a)= 0\) oppure \((a \cdot i + b \cdot j) \cdot ( -b \cdot i + a \cdot j ) = 0 \implies [(a, b)] \perp r\).  }
\mprop{}{Sia \(\pi: ax + by + cz + d = 0\) un piano in \(\EE_3(\RR)\). Allora \([(a, b, c)]\) è la classe dei parametri direttori della direzione ortogonale a \(\pi\). }
\pf{Dimostrazione}{Sia \(v \in V_2 \ (\pi\) ha spazio di traslazione o \(V_2\)). Se \(v = (x, y, z) \implies ax + by + cz = 0 \iff (x, y, z) \cdot (a, b, c) = 0 \implies (a,b,c) \perp v \ \forall v \in V_2 \) }
Più in generale: sia \(S _{n-1}\) un iperpiano in \(\EE_n(\RR)\) di equazione cartesiana \(0 = a_1 x_1 + a_2 x_2 + ... + a_n x_n + a_0 \implies [(a_1, a_2, ..., a_n)]\) è la classe dei parametri direttori della direzione ortogonale a \(S _{n-1}\).

\mprop{Ortogonalità tra piani}{Siano \(\alpha: ax + by + cz + d = 0 \quad \beta: a'x + b'y + c'z + d' = 0\) due piani in \(\EE_3(\RR)\). Allora \(\alpha \perp \beta \iff a \cdot a' + b \cdot b' + c \cdot c' = 0\) }
\pf{Dimostrazione}{\(\alpha \perp \beta \iff V_2 \supseteq V_2'^{\perp}\) dove \(V_2\) è la giacitura di \(\alpha\) e \(V_2'\) è la giacitura di \(\beta\). \[V_2'^{\perp}= [\mcL((a', b',c')) ] \iff (a', b', c') \in V_2\]\((x, y, z) \in V_2 \iff ax + by + cz = 0\) e quindi \((a', b', c') \in V_2 \iff a \cdot a' + b \cdot b' + c \cdot c' = 0\) }
\mprop{Ortogonalità tra retta e piano}{Siano \(r:\) con \(P.d.r = [(l,m,n)]\) e sia \(\alpha\) di equazione \(ax + by + cz + d= 0\) una retta e un piano di \(\EE_3(\RR)\). Allora \(r \perp \alpha\) se e soltanto se \([(a,b,c)] = [(l,m,n)]\) }
\pf{Dimostrazione}{\(r \perp \alpha \iff V_1=V_2^{\perp}\) dove \(V_1\) è la direzione della retta e \(V_2\) è la giacitura di \(\alpha\). \[
        V_1 = \mcL((l,m,n)) = V_2 ^{\perp} = \mcL((a,b,c)) \iff [(a,b,c)] = [(l,m,n)]  
\]}

\end{document}
