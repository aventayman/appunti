\documentclass{report}

\input{preamble}
\input{macros}
\input{letterfonts}

\title{\Huge{Algebra Lineare e Geometria Analitica}\\Ingegneria dell'Automazione Industriale}
\author{\huge{Ayman Marpicati}}
\date{A.A. 2022/2023}

\begin{document}

\maketitle
\newpage% or \cleardoublepage
% \pdfbookmark[<level>]{<title>}{<dest>}
\pdfbookmark[section]{\contentsname}{toc}
\tableofcontents


\section{Classificazione delle quadriche}
\dfn{}{
Dati una quadrica generale \(Q\) e il piano improprio \(\alpha _{\infty}\) Se intersechiamo otteniamo una curva \[C_\infty = Q \cap \alpha _{\infty}\] detta \textbf{conica impropria di \(Q\)}.}
\dfn{}{Una conica generale \(Q\) si chiama
\begin{enumerate}
    \item \textbf{ellissoide}, se \(C_\infty\) è irriducibile e priva di punti reali
    \item \textbf{iperboloide}, se \(C_\infty\) è irriducibile con punti reali
    \item \textbf{paraboloide}, se \(C_\infty\) è irriducibile
\end{enumerate}}
\paragraph{Osservazione 1:} Per \(C_\infty\) non ha senso la distinzione in ellisse, parabola o iperbole perché tutti i suoi punti sono punti impropri.
\paragraph{Osservazione 2:} Il paraboloide, avendo \(C_\infty\) riducibile, è tangente con \(\alpha _{\infty}\).

\mprop{}{Sia \(Q: {^tX}AX = 0\) una quadrica irriducibile, allora \(C_\infty\) è riducibile se, e soltanto se, \(|A^{*}| = 0\), dove \[
A^{*} =
\left( \; \begin{matrix}
    a_{11} & a_{12} & a_{13} \\
    a_{12} & a_{22} & a_{23} \\
    a_{13} & a_{23} & a_{33} \\
\end{matrix} \; \right) 
\] }
\pf{Dimostrazione}{Sia \(C_\infty = Q \cap \alpha _\infty\), quindi \[
C_\infty :
\begin{cases}
    \ a_{11}x_1^2 + a_{22}x_2^2 + 2a_{12}x_1x_2 + a_{33}x_3^2 + 2a_{13}x_1x_3 + 2a_{23}x_2x_3 = 0 \quad  (1)\\
    \ x_4 = 0 \\
\end{cases}
\] \((1)\) è una quadrica \(Q'\) tale che la sua intersezione con \(\alpha _\infty\) è proprio la conica impropria \(C_\infty\) di \(Q\). Quindi la matrice della quadrica \(Q'\) è \[
A' = 
\left( \; \begin{matrix}
    a_{11} & a_{12} & a_{13} & 0 \\
    a_{12} & a_{22} & a_{23} & 0 \\
    a_{13} & a_{23} & a_{33} & 0 \\
    0 & 0 & 0 & 0 \\
\end{matrix} \; \right) 
\] \(|A'| = 0\), quindi \(Q'\) non è generale perché \(\rho(A') \le 3\). Per ipotesi \(C_\infty\) è riducibile, ora partiamo con la dimostrazione vera e propria. \\ "\(\implies \)" Supponiamo, per assurdo, che \(|A^{*}| \neq 0 \implies \rho(A') = 3 \implies Q'\) è un cono o un cilindro. Determiniamo il vertice di \(Q': A'X=0\). Scrivendo un sistema principale equivalente \[
s.p.e.:
\begin{cases}
    \ a_{11}x_1 + a_{12}x_2 + a_{13}x_3 = 0 \\
    \ a_{12}x_1 + \ldots  = 0 \\
    \ \ldots  = 0 \\
\end{cases}
\] troveremo facilmente il \(V=[(0,0,0,1)]\), che è il vertice ed è un punto proprio, quindi \(Q'\) è un cono. Quindi \(C_\infty = Q' \cap \alpha _\infty\) è la conica impropria di un cono, quindi \(C_\infty\) è irriducibile, che è un \textbf{assurdo!}. \\ "\(\impliedby \)" Abbiamo per ipotesi che \(|A^{*}| =0\), \(\rho(A') \le 2\), quindi \(Q'\) è riducibile, allora \(C_\infty= Q' \cap \alpha _\infty\) è sezione di una quadrica riducibile e quindi \(C_\infty\) è riducibile.}

\paragraph{Osservazione:}
\begin{enumerate}
    \item Per distinguere un cono o un cilindro abbiamo ora un criterio analitico, cioè 
        \begin{itemize}
            \item \(|A^{*}| = 0 \iff C_\infty \text{ è riducibile } \iff Q \text{ è cilindro }\) 
            \item \(|A^{*}| \neq 0 \iff Q \text{ è cono }\) 
        \end{itemize}
\item se \(Q\) invece è generale abbiamo che
    \begin{itemize}
        \item \(|A^{*}| = 0 \iff Q\) è paraboloide
        \item \(|A^{*}| \neq 0 \iff Q\) è ellissoide o iperboloide
    \end{itemize}
\end{enumerate}

\ex{}{\[
Q : x^2 - 3y^2 - z^2 - y = 0
\] \[ A = 
\left( \; \begin{matrix}
    1 & 0 & 0 & 0 \\
    0 & -3 & 0 & -\frac{1}{2} \\
    0 & 0 & -1 & 0 \\
    0 & -\frac{1}{2} & 0 & 0 \\
\end{matrix} \; \right) 
\] Possiamo dire che \[
|A| \neq 0 \implies Q \text{ generale } \quad |A^{*}| = 3 \neq 0 \implies Q \text{ o ellissoide o iperboloide }
\] \[
C_\infty :
\begin{cases}
    \ x_1^2 - 3x_2^2 - x_3^2 - x_2x_4 = 0 \\
    \ x_4 = 0 \\
\end{cases} \quad
\begin{cases}
    \ x_1^2-3x_2^2 - x_3^2 = 0 \\
    \ x_4 = 0 \\
\end{cases}\]
\[ P_\infty = [(1,0,1,0)] \in C_\infty \text{ il quale è reale}\implies Q \text{ è un paraboloide }
\] }

\subsubsection{Classificazione dei punti semplici di una quadrica}
Sia \(Q\) una quadrica irriducibile, sia un punto \(P \in Q\) semplice. Chiamiamo \(\alpha \) il piano tangente a \(Q\) in \(P\) e la conica \(C = Q \cap \alpha \), la quale è riducibile.
\dfn{}{Se \(C\) si riduce in due rette coincidenti, \(P\) si dice punto \textbf{parabolico}.}

\mprop{}{Se una quadrica irriducibile ha un punto semplice parabolico, allora tutti i punti semplici sono parabolici.}

\thm{}{Una quadrica irriducibile è un cono o un cilindro se, e soltanto se, i suoi punti semplici sono parabolici.}
\pf{Dimostrazione}{"\(\implies \)" Sappiamo per ipotesi che \(Q\) è un cono o un cilindro. Sia \(P \in Q\), un punto semplice, quindi \(P \neq V\), chiamiamo \(\alpha \) il piano tangente in \(P\). La conica \(C = Q \cap \alpha = r \cup s \subseteq Q\), di conseguenza \(r \subseteq Q \implies V \in r\) e \(s \subseteq Q \implies V \in s\). Inoltre \(P \in r\) e \(P \in s\). Ma quindi necessariamente \(r = \overline{PV} = s\). Quindi \(P\) è un punto parabolico. \\
"\(\impliedby \)" Per ipotesi abbiamo i punti semplici parabolici. Chiamiamo \(P\) un punto semplice di \(Q\) e \(\alpha \) il piano tangente a \(Q\) in \(P\). Allora \[
C = Q \cap \alpha =r \cup r
\] se \(P' \in r\) e semplice, allora \(\alpha \) è un piano passante per \(P'\) tale che \(Q \cap \alpha \) è riducibile in due rette passanti per \(P'\). Questo ci dice che allora \(\alpha \) è il piano tangente a \(Q\) anche in \(P'\). Sia \(\beta \) un piano con \(\beta  \neq \alpha \) e tale che \(r \subseteq \beta \). Chiamiamo inoltre \(C' = Q \cap \beta \), sicuramente \(r \subseteq C'\), questo significa che \(C'\) è riducibile, cioè \(C = r \cup s\). Ma \(r \neq s\) perché se fosse, per assurdo \(r = s\), allora in \(P\) avrei due piani tangenti distinti \(\alpha \) e \(\beta \), \textbf{assurdo!} (contro l'unicità del piano tangente). Sia \(\{V\} =r \cap s\). Sicuramente \(V\) è un punto doppio, perché se fosse semplice per \(V\) avremmo due piani tangenti distinti (nuovamente contro l'unicità del piano tangente). Su \(Q\) non possono esserci altri punti doppi distinti da \(V\) (perché per ipotesi \(Q\) è irriducibile). Quindi \(Q\) ammette esattamente un punto doppio, cioè \(Q\) è un cono o un cilindro.}

\paragraph{Osservazione:} Se \(Q\) è generale, sicuramente i suoi punti semplici non sono parabolici

\dfn{}{Sia \(Q\) una quadrica irriducibile, \(P \in Q\) un punto semplice reale, \(\alpha \) il piano tangente in \(P\) a \(Q\) e \(C = Q \cap \alpha \) riducibile. Abbiamo che un punto \(P\) è
\begin{enumerate}
    \item \textbf{parabolico}, se, e soltanto se, \(C\) si riduce in due rette coincidenti
    \item \textbf{iperbolico}, se, e soltanto se, \(C\) si riduce in due rette reali e distinte
    \item \textbf{ellittico}, se, e soltanto se, \(C\) si riduce in due rette immaginarie e coniugate
\end{enumerate}}

\mprop{}{Se una quadrica irriducibile \(Q\) ha un punto semplice reale parabolico, iperbolico o ellittico, allora tutti i suoi punti semplici reali sono dello stesso tipo.}

\dfn{}{La quadrica \(Q\) si dice 
\begin{enumerate}
    \item \textbf{parabolica} se i suoi punti semplici reali sono parabolici
    \item \textbf{iperbolica} se i suoi punti semplici sono iperbolici
    \item \textbf{ellittica} se i suoi punti semplici reali sono ellittici
\end{enumerate}}

\mprop{}{I punti semplici reali di un ellissoide sono necessariamente ellittici.}
\pf{Dimostrazione}{Sia \(Q\) un ellissoide, \(P\) un punto semplice reale e supponiamo, per assurdo, che \(P\) sia iperbolico. Chiamiamo \(\alpha \) il piano tangente in \(P\) e \(C = Q \cap \alpha = r \cup s\) con \(r,s\) reali e distinte. Sappiamo che \(r \subseteq Q\) e \[\{P_\infty\} r \cap \alpha \subseteq Q \cap \alpha _\infty = C_\infty\]sarebbe un punto reale sulla \(C_\infty\) di un ellissoide, \textbf{assurdo!} Quindi \(P\) è ellittico.}

\paragraph{Osservazione:} Ricapitolando abbiamo che, se \(Q\) è generale, allora può essere
\begin{enumerate}
    \item ellissoide (ellittico)
    \item iperboloide
        \begin{enumerate}
            \item ellittico
            \item iperbolico
        \end{enumerate}
    \item paraboloide
        \begin{enumerate}
            \item ellittico
            \item iperbolico
        \end{enumerate}
\end{enumerate}
\end{document}
