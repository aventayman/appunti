\documentclass{report}

\input{preamble}
\input{macros}
\input{letterfonts}

\title{\Huge{Algebra Lineare e Geometria Analitica}\\Ingegneria dell'Automazione Industriale}
\author{\huge{Ayman Marpicati}}
\date{A.A. 2022/2023}

\begin{document}

\maketitle
\newpage% or \cleardoublepage
% \pdfbookmark[<level>]{<title>}{<dest>}
\pdfbookmark[section]{\contentsname}{toc}
\tableofcontents





Per determinare le coordinate del centro dobbiamo scegliere due punti \(X_{\infty} = [(1, 0, 0)]\), punto improprio dell'asse \(x\), e \(Y_{\infty} = [(0,1,0)]\), punto improprio dell'asse \(y\). La polare di \(X_{\infty}\) è \[
    (1,0,0)
\left( \; \begin{matrix}
    a_{11} & a_{12} & a_{13} \\
    a_{12} & a_{22} & a_{23} \\
    a_{13} & a_{23} & a_{33} \\
\end{matrix} \; \right)
\left( \; \begin{matrix} x_1 \\ x_2\\ x_3 \end{matrix} \; \right) = 0 \qquad (a_{11}, a_{12}, a_{13}) \left( \; \begin{matrix} x_1 \\ x_2\\ x_3 \end{matrix} \; \right) = a_{11}x_1+a_{12}x_2 + a_{13} x_3 = 0
\] Analogamente la polare di \(y_{\infty}\) è  \[
a_{12}x_1 + a_{22}x_2 + a_{23}x_3 = 0
\] \[
\begin{cases}
    \ a_{11}x_1+a_{12}x_2+a_{13}x_3 = 0 \qquad P_1 \\
    \ a_{12}x_1+a_{22}x_2+a_{23}x_3=0 \qquad P_2 \\
\end{cases}
\] 
Il centro \(C\) è proprio se \(P_1\) e \(P_2\) non sono paralleli. Se \[
\left| \; \begin{matrix}
    a_{11} & a_{12} \\
    a_{12} & a_{22} \\
\end{matrix} \; \right| = |A^{*}| \neq 0
\] Il centro è un punto proprio. Quindi il centro è un punto proprio se \(C\) è un ellisse o un'iperbole. Quindi in questo caso i diametri sono un fascio proprio di rette di centro \(C\). \[
F: \ \lambda (a_{11}x_1+a_{12}x_2+a_{13}x_3) + \mu (a_{12}x_1 + a_{22}x_2 + a_{23}x_3) = 0
\] \textbf{Equazione del fascio dei diametri.} Se \(C\) è una parabola \(\implies |A^{*}| = 0 \implies P_1 \) parallelo a \(P_2 \implies  \) il centro è un punto improprio. \(\implies \) i diametri formano un fascio improprio di equazione \[
a_{11}x_1+a_{12}x_2 + kx_3 = 0 \quad \text{con} \quad k \in \CC
\] fascio improprio dei diametri della parabola.

\section{Asintoti di una conica}
\dfn{Asintoti}{Si dicono \textbf{asintoti} di una conica le rette proprie tangenti alla conica nei suoi punti impropri.}
\paragraph{Osservazione:} Gli asintoti di una conica sono quindi le rette polari nei suoi punti impropri. Gli asintoti sono quindi dei diametri e passano per il centro. Se il centro è proprio (cioè se \(C\) è un'ellisse o un'iperbole) gli asintoti sono le rette che congiungono il centro con i punti impropri di \(C\).

\mprop{}{La parabola è una conica con centro improprio e priva di asintoti.}
\pf{Dimostrazione}{Sia \(C\) una parabola \(\implies C\) è tangente alla retta impropria in un punto che chiamiamo \(P_{\infty}\). Quindi la retta polare di \(P_{\infty}\) è \(r_{\infty} \implies \) il polo della \(r_{\infty}\) è \(P_{\infty} \implies \) il punto \(P_\infty\) è il centro della parabola. Osserviamo che \(C\) ha solo un punto improprio \(P_\infty \implies \) ammette solo una tangente nel suo punto improprio. Ma \(t\) è la \(r_\infty \implies \) la \(r_{\infty}\) non è un asintoto.}

\dfn{Coniche a centro}{Diremo che l'iperbole e l'ellisse sono coniche \textbf{a centro}, mentre la parabola è detta conica \textbf{non a centro}.}

\section{Proprietà metriche}
\dfn{Iperbole equilatera}{Un'iperbole si dice \textbf{equilatera} se i suoi asintoti sono ortogonali.}

\mprop{}{Una conica generale è un'iperbole equilatera se, e soltanto se, \(a_{11} + a_{22} = 0\).}

\ex{}{Si stabiliscano i valori di \(k \in \RR:\) \[
C: \ 2kx^2 + 2 (k-2) xy - 4 y ^2 + 2x + 1 = 0
\] sia un'iperbole equilatera. \\ 
\begin{enumerate}
    \item \(2k = -(-4) \to k = 2\)
    \item Sostituiamo dentro all'equazione e scriviamola in forma omogenea \[ 4x_1 ^2 + 0 x_1 x_2 - 4 x_2 ^2 + 2 x_1 x_3 + x_3 ^2 = 0 \quad 
        A = 
\left| \; \begin{matrix}
    4 & 0 & 1 \\
    0 & -4 & 0 \\
    1 & 0 & 1 \\
\end{matrix} \; \right| \neq 0
\] \(k = 2\) dà luogo ad un'iperbole equilatera.
\end{enumerate}}

\dfn{Ortogonale al punto improprio}{Diremo che la retta \(p\) di parametri direttori \([(l', m')]\) è ortogonale al punto improprio \(P: [(l,m,0)]\) se \(ll' + mm' = 0\).}

\dfn{Asse di una conica}{Si dice \textbf{asse} di una conica ogni diametro ortogonale al proprio polo.}

\dfn{Vertici}{Si dicono \textbf{vertici} le intersezioni proprie della conica con i propri assi.}

\section{Condizioni analitiche}
\mprop{}{Gli assi di una conica a centro (ellisse o iperbole) sono due e sono ortogonali tra loro, a meno che non si tratti di una circonferenza generalizzata, in tal caso tutti i diametri sono assi.}
\pf{Dimostrazione}{Per definizione i diametri sono le polari dei punti impropri. Dato \(P_\infty : [(l,m,0)]\) \[
\left( \; \begin{matrix}
    l & m & 0 \\
\end{matrix} \; \right)
\left( \; \begin{matrix}
    a_{11} & a_{12} & a_{13} \\
    a_{12} & a_{22} & a_{23} \\
    a_{13} & a_{23} & a_{33} \\
\end{matrix} \; \right)
\left( \; \begin{matrix} x_1 \\ x_2\\ x_3 \end{matrix} \; \right) = 0
\] Il generico diametro è: \[
\left( \; \begin{matrix}
    la_{11} + m a_{12} & l a_{12} + m a_{22} & l a_{13} + m a_{23} \\
\end{matrix} \; \right)
\left( \; \begin{matrix} x_1 \\ x_2\\ x_3 \end{matrix} \; \right) = 0
\] \[
(la_{11} + m a_{12}) x_1 + (la_{12} + ma_{22}) x_2 + (la_{13} + ma_{23}) x_3 = 0
\] \[
p.d.d : \ [(-la_{12} - m a_{22}, l a_{11} + ma_{12})]
\] Il polo di \(d\) è \(P_\infty: [(l,m,0)]\). \(d\) è un asse se è ortogonale a \(P_\infty\) ovvero se \[
l ( -la_{12}-ma_{22} ) + m(la_{11} + ma_{12}) = 0
\] \[
-l^2 a_{12} + ml (-a_{22}+a_{11}) + m^2 a_{12} = 0 \qquad l^2a_{12} + ml(a_{22}- a_{11}) - m^2 a_{12} = 0 
\] \[
a_{12} \left( \frac{l}{m} \right) ^2 + \frac{l}{m} (a_{22} - a_{11}) - a_{12} = 0
\] Se \(a_{12}=0\) e \(a_{22} = a_{11}\) l'equazione è risolta da tutte le coppie \((l,m)\). Quindi se la conica è una circonferenza generalizzata tutti i diametri sono assi. I due assi hanno polo \(P_\infty : [(l',m',0)]\) e \(Q_\infty : [(l '', m '', 0)]\). Sia \(p'\) l'asse associato al polo \(P_\infty\) e sia \(A_\infty\) il suo punto improprio. Sia \(a\) la retta che congiunge il centro al punto improprio \(rt(C, P_\infty)\), per ipotesi \(a \perp p'\). \(a\) contiene \(P_\infty\) che è il polo di \(p'\), quindi per il principio di reciprocità \(p'\) contiene il polo di \(a\). Il polo di \(a\) è improprio (perché \(a\) è diametro) \(\implies \) il punto improprio di \(a\) è \(A_\infty\), ma \(A_\infty \) è ortogonale alla direzione di \(a \implies a\) è un asse. Quindi i due assi sono ortogonali.}

\mprop{}{La parabola ha un unico asse e un solo vertice \(v\). Inoltre la tangente alla parabola in \(v\) è ortogonale all'asse.}

\pf{Dimostrazione}{Il punto \(P_\infty\) di una parabola è \([(-a_{12}, a_{11}, 0)]\). I \(p.d.d = [(-a_{12}, a_{11})]\). La direzione ortogonale è data da \([(a_{11}, a_{12})]\), quindi il punto \(P_\infty\) è \([(a_{11}, a_{12}, 0)]\). Da cui segue che l'asse è unico ed è la polare di \((a_{11}, a_{12}, 0)\). Sostituendo nell'equazione del fascio improprio dei diametri abbiamo che l'asse ha equazione: \[
a_{11}(a_{11}x_1+ a_{12}x_2 + a_{13}x_3) + a_{12}(a_{12}x_1 + a_{22}x_2 + a_{23}x_3) = 0
\] Per il teorema dell'ordine \(a\) interseca la parabola \(C\) in due punti, ma uno è \(P_\infty\) quindi l'altro punto sarà l'unico vertice della parabola. \\ Ora dimostriamo la seconda parte del teorema. \(v \in a\) che è il polo di \(t\). Per il principio di reciprocità \(t\) contiene il polo di \(a\), ovvero \(P_{\infty} \in t\). Ma \(P_\infty\) è ortogonale ad \(a \implies t \perp a\).}

\chapter{Ampliamento di \(A_3(\RR )\)}
Chiamiamo con \(\tilde{A}_{3}(\RR ) \) lo spazio reale affine ampliato. I punti possono essere
\begin{itemize}
    \item \textbf{propri} \(A\) dei punti di \(A_3(\RR )\) 
    \item \textbf{impropri} \(A_\infty\) direzioni delle rette, (spazi di traslazione di dimensione 1)
\end{itemize}
Le rette possono essere
\begin{itemize}
    \item \textbf{proprie} rette di \(A_3(\RR )\) ciascuna estesa con il suo punto improprio (ovvero la sua direzione)
    \item \textbf{improprie} sono le giaciture dei piani (spazi di traslazione di dimensione 2)
\end{itemize}
I piani possono essere
\begin{itemize}
    \item \textbf{propri} i piani di \(A_3(\RR )\) ciascuno esteso con la sua retta impropria (ovvero la sua giacitura)
    \item \textbf{piano improprio} \(A_\infty\) il luogo dei punti impropri
\end{itemize}

\mprop{}{
Diamo una serie di conseguenze senza dimostrazione
\begin{enumerate}
    \item due rette parallele hanno la stessa direzione e quindi hanno lo stesso punto improprio
    \item due piani paralleli hanno la stessa giacitura e quindi hanno la stessa retta impropria
    \item il piano improprio contiene tutte e sole le rette improprie
    \item ogni retta impropria contiene un solo punto improprio (la sua direzione)
    \item ogni piano proprio contiene \(\infty^{1}\) punti impropri, ovvero una retta (la sua giacitura).
\end{enumerate}}

\section{Geometria analitica in \(\tilde{A}_{3}(\RR) \)}
Indichiamo con \[ \frac{\RR ^{4} \backslash  \{(0,0,0,0)\}}{\rho } \] cioè l'insieme delle quaterne definite a meno di un fattore di proporzionalità reale e non nullo. In cui \(\rho \) indica la relazione di equivalenza data dalla proporzionalità. Quindi consideriamo due terne equivalenti se sono proporzionali.

\mprop{}{Sia \(RA = [O, B]\) un riferimento affine di \(A_{3}(\RR) \) e sia \[
\phi : A \cup A_\infty \quad  \to \quad \frac{\RR ^{4} \backslash  \{(0,0,0,0)\}}{\rho }
\] sia \(P \in A\) di coordinate \((x,y,z)\) \[
\phi(P) = [(x,y,z,1)]
\] sia \(P \in A_\infty\) corrispondente alla direzione \([(l,m,n)]\) \[
\phi (P) = [(l,m,n,0)]
\] la mappa \(\phi\) è una biiezione e le coordinate indotte da \(\phi\) sono chiamate \textbf{coordinate omogenee}.}

\ex{}{\[
        Q = [(2,0,3,-2)] \qquad -2 \neq 0 \implies Q \ \text{è proprio}
\] \[
Q = \left[ \left( \frac{2}{-2}, \frac{0}{-2}, - \frac{3}{2}, 1 \right)  \right] \implies Q = \left( -1, 0, - \frac{3}{2} \right) 
\] \[
P = [(2,1,0,0)] \implies [(2,1,0)]
\] }

\dfn{Rappresentazione dei piani}{
\[
ax_1 + bx_2 + cx_3 + dx_4 = 0 \quad \text{con} \quad (a,b,c,d) \neq (0,0,0,0)
\] questa è l'equazione omogenea dei piani in \(\tilde{A}_{3}(\RR) \) (e si ottiene in modo analogo all'equazione omogenea delle rette in \(\tilde{A}_{2}(\RR) \)).
}
\paragraph{Osservazione:}
\begin{enumerate}
    \item se \((a,b,c) \neq (0,0,0)\) allora il piano è proprio ed ha equazione affine \[
ax + by + cz + d = 0
\] 
    \item se \((a,b,c) = (0,0,0)\) allora \(d \neq 0\) e otteniamo \(x_4 = 0\) (che definisce il piano improprio).
\end{enumerate}

\dfn{Rappresentazione di rette}{Una retta è intersezione di 2 piani distinti \[
r:
\begin{cases}
    \ ax_1 + bx_2+ cx_3 + dx_4 = 0 \\
    \ a'x_1 + b'x_2+ c'x_3 + d'x_4 = 0 \\
\end{cases} \quad \text{con} \quad \rho
\left( \; \begin{matrix}
    a & b & c & d \\
    a' & b' & c' & d' \\
\end{matrix} \; \right) = 2
\] questa è la rappresentazione della generica retta di \(\tilde{A}_{3}(\RR)\)}
\begin{itemize}
    \item se \[
    \rho
\left( \; \begin{matrix}
    a & b & c \\
    a' & b' & c' \\
\end{matrix} \; \right) = 2     
    \] \(r\) è propria \[
\begin{cases}
    \ ax + by + cz + d = 0 \\
    \ a'x + b'y + c'z + d' = 0 \\
\end{cases}
    \]
\item se \[
    \rho
\left( \; \begin{matrix}
    a & b & c \\
    a' & b' & c' \\
\end{matrix} \; \right) = 1     
    \] ho due casi possibili
    \begin{itemize}
        \item i due piani sono paralleli e distinti
        \item uno dei due è il piano improprio e quindi \(x_4 = 0\)
    \end{itemize}
    in entrambi i casi \(r\) è impropria
\end{itemize} 

\section{Complessificazione di \(\tilde{A}_{3}(\CC)\)}
\(\tilde{A}_{3}(\CC) \) è lo spazio ampliato e complessificato. I suoi punti sono le quaterne di \[
\frac{\CC ^{4} \backslash  \{(0,0,0,0)\}}{\rho }
\] cioè le classi di proporzionalità delle quaterne complesse. La relazione di proporzionalità è chiaramente da intendersi in \(\CC\). All'interno dello spazio definiamo
 \begin{itemize}
    \item le \textbf{rette} sono i punti tali che \[
\begin{cases}
    \ ax_1+ bx_2 + cx_3+dx_4 =0 \\
    \ a'x_1 + b'x_2 + c'x_3 + d'x_4 = 0 \\
\end{cases} \quad \text{con}\quad a,a',b,b',c,c',d,d' \in C
    \] e tali che \[
    \rho
\left( \; \begin{matrix}
    a & b & c & d \\
    a' & b' & c' & d' \\
\end{matrix} \; \right) = 2
    \]
    \item un \textbf{piano} è costituito dai punti \[
    ax_1 + bx_2+cx_3+dx_4 = 0 \quad \text{con}\quad (a,b,c,d) \in \CC^{4} \backslash \{(0,0,0,0)\} 
    \] 
\end{itemize}

\dfn{Punti, rette e piani reali}{In \(\tilde{A}_{3}(\CC) \) i punti, le rette e i piani si dicono \textbf{reali} se ammettono almeno una rappresentazione con coefficienti reali. Si dicono immaginari altrimenti.}

\dfn{Rette immaginarie di prima e seconda specie}{In \(\tilde{A}_{3}(\CC) \) una retta \(r\) immaginaria è detta \textbf{immaginaria di prima specie} se è complanare con la propria coniugata \(\overline{r}\). \(r\) è detta  \textbf{immaginaria di seconda specie} se è sghemba con \(\overline{r}\).}

\mprop{}{
\begin{enumerate}
    \item La retta congiungente due punti immaginari e coniugati è reale
    \item se una retta (o un piano) reale contiene un punto \(P\) immaginario allora contiene anche \(\overline{P}\)
    \item se \(P\) è immaginario l'unica retta reale per \(P\) è \(rt(P, \overline{P})\)
    \item l'intersezione tra un piano \(\pi \) immaginario e \(\overline{\pi }\) è una retta reale
    \item un piano \(\pi \) immaginario contiene un'unica retta reale \(: \pi \cap \overline{\pi }\)
    \item se \(r\) è una retta immaginaria allora
        \begin{enumerate}
            \item \(r\) è contenuta in al più un piano reale
            \item \(r\) contiene al più un punto immaginario
        \end{enumerate}
        in particolare se \(r\) è immaginaria di prima specie il piano contenente \(r\) e \(\overline{r}\) è reale e \(r \cap \overline{r}\) è un punto reale. Se invece \(r\) è immaginaria di seconda specie allora \(r\) non è contenuta in alcuno piano reale e non contiene alcun punto reale.
\end{enumerate}
}
\dfn{Superfici algebriche reali in \(\tilde{A}_{3}(\CC) \)}{\textbf{Una superficie algebrica reale di \(\tilde{A}_{3}(\CC) \)} è l'insieme delle classi di autosoluzioni complesse di un'equazione del tipo \[
F(x_1, x_2, x_3, x_4) = 0 \quad \text{ove} \quad F \ \text{è un polinomio omogeneo a coefficienti reali in } x_1, x_2, x_3, x_4
\]  il grado di \(F\) è chiamato ordine della superficie. Se \(F\) è fattorizzabile in polinomi di grado positivo la superficie si dice riducibile in componenti \[
\text{fattori di \(F\)} \leftrightarrow \text{componenti della superficie}
\] }

\thm{Primo teorema dell'ordine}{L'ordine di una superficie algebrica \(\Sigma\) reale è uguale al numero di punti in comune a \(\Sigma\) e a una qualsiasi retta \(r\) non contenuta in \(\Sigma\) a patto di contarli con la dovuta molteplicità.}

\cor{}{\[\text{Se }|r \cap \Sigma| > \text{ord}(\Sigma) \implies r \subseteq \Sigma\].}

\thm{Secondo teorema dell'ordine}{L'intersezione tra una superficie algebrica reale \(\Sigma\) e un piano \(\alpha \) non componente di \(\Sigma\) è una curva dello stesso ordine di \(\Sigma\).}

\cor{}{Se \(\Sigma \cap \pi \) contiene una curva \(C\) con \(\text{ord}(C) > \text{ord}(\Sigma) \implies \pi \) è componente di \(\Sigma\).}

\dfn{}{In \(\tilde{A}_{3}(\CC) \), data una superficie algebrica reale \(\Sigma\), un punto \(P \in \Sigma\) è detto \textbf{r-uplo} se la generica retta per \(P\) ha molteplicità di intersezione con \(\Sigma\) in \(P\) uguale a \(r\).
\begin{itemize}
    \item se \(r = 1\) \(P\) è detto \textbf{semplice} 
    \item se \(r > 1\) \(P\) è detto \textbf{multiplo} 
\end{itemize}}

\thm{}{I punti multipli di una curva algebrica reale di equazione \(F(x_1, x_2, x_3, x_4)\) sono le classi di autosoluzioni del sistema \[
\begin{cases}
    \ \frac{\partial F}{\partial x_1} = 0 \\
    \ \frac{\partial F}{\partial x_2} = 0 \\
    \ \frac{\partial F}{\partial x_3} = 0 \\
    \ \frac{\partial F}{\partial x_4} = 0 \\
\end{cases}
\] }

\chapter{Quadriche}
\dfn{Quadrica}{Si dice \textbf{quadrica} una superficie algebrica reale del secondo ordine. Analiticamente si indica come \[
a_{11}x_1^2 + a_{12}x_1x_2 + 2a_{13}x_1x_3 + 2a_{14}x_1x_4 + a_{22}x_2^2 + a_{23}x_2x_3 + 2a_{24}x_2x_4 + 2a_{34}x_3x_4 + a_{33}x_3^2 + a_{44}x_4^2 = 0
\] con almeno un \(a_{ij}\neq 0\). Ponendo \[
X =
\left( \; \begin{matrix}
    x_1 \\
    x_2 \\
    x_3 \\
    x_4 \\
\end{matrix} \; \right) \quad \text{si ha che} \quad A =
\left( \; \begin{matrix}
    a_{11} & a_{12} & a_{13} & a_{14} \\
    a_{12} & a_{22} & a_{23} & a_{24} \\
    a_{13} & a_{23} & a_{33} & a_{34} \\
    a_{14} & a_{24} & a_{34} & a_{44} \\
\end{matrix} \; \right) 
\] è tale che \[
Q : \ {^tX}AX = \ul{0} 
\] Quindi dipende da 10 coefficienti e abbiamo \(\infty^{9}\) quadriche.}

\mprop{}{Se una quadrica è riducibile, si riduce in due piani che possono essere reali e coincidenti, reali e distinti o immaginari e coniugati. Inoltre tutte le sue sezioni sono riducibili.}
\pf{Dimostrazione}{\(F\) è di  secondo grado (\(Q\) è del second'ordine), quindi se si fattorizza in due polinomi di primo grado, essendo \(F\) reale, le possibilità sono quelle elencate.
Sia \(Q = \alpha \cup \beta \) e sia \(\gamma\) un terzo piano abbiamo che \[
Q \cap \gamma = (\alpha \cup \beta ) \cap \gamma = (\alpha \cap \gamma) \cup (\beta \cap \gamma)
\] è unione di due rette, quindi è riducibile.}

\section{Coni e cilindri}
\dfn{Cono e cilindro}{Si dice \textbf{cono} quadrico il luogo delle rette che proiettano dal punto \(V\), chiamato \textbf{vertice}, i punti di una conica generale \(C\), chiamata \textbf{direttrice}, dove \(C\) appartiene ad un piano non contenente il \(V\). Se \(V\) è proprio otteniamo un \textbf{cono}, se \(V\) è improprio otteniamo un \textbf{cilindro}.}

\subsubsection{Punti multipli di una quadrica}
\thm{}{Una quadrica non ha punti tripli e i punti multipli di una quadrica sono i punti doppi.}
\pf{Dimostrazione}{Poiché la quadrica \(Q\) ha ordine 2, per il primo teorema dell'ordine \(r\) non può intersecare \(Q\) in un punto \(P\) con molteplicità 3.}

\thm{}{Una quadrica \(Q\) ha almeno 2 punti doppi se, e soltanto se, è riducibile.}
\pf{Dimostrazione}{"\(\implies \)" Siano \(R\) e \(S\) due punti doppi distinti e sia \(H \in Q\), ma non appartenente a \(rt(R,S)\). Prima di tutto osserviamo che \(rt(R,S)\) ha molteplicità di intersezione con \(Q\) almeno di \(2 + 2 = 4\) (\(|R| +  |S| \)). Quindi per il primo teorema dell'ordine la \(rt(R,S) \subseteq Q\). Allo stesso modo \(rt(R,H)\) (ma analogamente anche \(rt(S,H)\)) ha molteplicità di intersezione con \(Q\), almeno di \(1 + 2 = 3 > 2 \implies \) per il primo teorema dell'ordine  \(rt(R,H) \subseteq Q\), ugualmente per \(rt(S,H) \subseteq Q\). Chiamiamo \(\pi \) il piano contenente  \(R, S\) e \(H\). \[Q \cap \pi \supseteq \underbrace{rt(R,S) \cup rt(R,H) \cup rt(S,H)}_{\text{curva \(C\) di ordine \(3\)}} \]quindi poiché \(\ord(C) > \ord(Q) = 2\) per il secondo teorema dell'ordine il piano \(\pi \) è componente di \(Q\), per questo motivo \(Q\) è riducibile. \\
"\(\impliedby \)" Sia \(Q = \alpha \cup \beta \) e sia \(P \in \alpha \cap \beta \). Osserviamo che data \(r\) retta passante per \(P\) non in \(\alpha \cup \beta \) abbiamo che \(r \cap (Q) = r \cap (\alpha \cup \beta ) = (r \cap \alpha ) \cup (r \cap \beta )\), cioè l'unione dello stesso punto, quindi \(P\) è punto doppio. Di conseguenza abbiamo che ogni punto di \(\alpha \cap \beta \) è doppio e abbiamo due possibili casi
\begin{itemize}
    \item \(\infty^{1}\) punti (se \(\alpha \neq \beta \))
    \item \(\infty^2\) punti (se \(\alpha = \beta \))
\end{itemize}}

\thm{}{Una quadrica ha un unico punto doppio se, e soltanto se, è un cono o un cilindro quadrico.}
\pf{Dimostrazione}{"\(\implies \)" Sia \(V\) l'unico punto doppio della quadrica \(Q\). Ora dimostriamo prima di tutto che tutte le rette \(r\) contenute in \(Q\) passano per \(V\). Sia, per assurdo, \(r\) contenuta in \(Q\) con \(v \notin r\). Siano \(A, B \in r\) due punti distinti. Osserviamo che la retta \(rt(V,A)\) ha molteplicità di intersezione con \(Q\) pari ad almeno 1 in \(A\) e esattamente 2 in \(V\), quindi ha molteplicità di intersezione almeno 3. Quindi per il primo teorema dell'ordine \(rt(V,A) \subseteq Q\). Analogamente \(rt(V,B)\) è contenuta in \(Q\). Chiamiamo \(\pi \) il piano contenente \(r\) e \(V\). \[
Q \cap \pi \supseteq \underbrace{r \cup rt(V,A) \cup rt(V,B)}_{\text{curva \(C\) di ordine \(3\)}} 
\] poiché \(\ord(C) >\ord(Q) \implies \pi \subseteq Q\). Quindi \(\pi \) è componente di \(Q\), di conseguenza \(Q\) è riducibile e ha almeno \(\infty^{1}\) punti doppi. \textbf{Assurdo!} Perciò tutte le rette di \(Q\) passano per \(V\). Sia \(\alpha \) piano non contenente \(V\). \(\alpha \) non è componente di \(Q\), poiché \(Q\) è irriducibile, perciò \(\alpha  \cap Q\) è una conica (per il secondo teorema dell'ordine). Poiché \(C\) non si riduce in due rette \(C\) è generale. Sia ora \(P \in C\) la retta \(rt(P,V)\) ha molteplicità di intersezione con \(Q\) di almeno \(1 + 2 = 3 > \ord(Q) = 2\), quindi per il primo teorema dell'ordine \(rt(P,V) \subseteq Q\) per ogni punto di \(C\). Di conseguenza \(Q\) è un cono o un cilindro quadrico.\\
"\(\impliedby \)" Sia \(Q\) un cono o un cilindro quadrico con vertice \(V\). \(Q\) ha al più un punto doppio, altrimenti sarebbe riducibile. Sia \(r\) una retta non contenuta in \(Q\) e passante per \(V\), l'unico punto di intersezione è \(r \cap Q = V\). Poiché per il primo teorema dell'ordine la somma delle intersezioni (contate con la dovuta molteplicità) è 2, segue che \(v\) è doppio.}

\section{Condizioni analitiche}
\dfn{}{Una quadrica \(Q \in \tilde{A}_{3}(\CC) \) si dice 
\begin{itemize}
    \item \textbf{generale} se è priva di punti doppi
    \item \textbf{semplicemente degenere} se ha 1 unico punto doppio (cono o cilindro)
    \item \textbf{doppiamente degenere} se ha \(\infty^{1}\) punti doppi
    \item \textbf{tre volte degenere} se ha \(\infty^2\) punti doppi
\end{itemize} 
Inoltre le quadriche doppiamente e tre volte degeneri sono \textbf{riducibili}.}

\mprop{}{I punti doppi di una quadrica \(Q: {^tX}AX = \ul{0} \) sono le classi di autosoluzioni del sistema omogeneo \(AX = \ul{0} \).}
\thm{}{Sia la quadrica \(Q : {^tX}AX = \ul{0} \). Abbiamo le seguenti possibilità
\begin{itemize}
    \item Se \(\rho(A) = 4\), allora \(Q\) è generale
    \item Se \(\rho(A) = 3\), allora \(Q\) è semplicemente degenere
    \item Se \(\rho(A) = 2\), allora \(Q\) è doppiamente degenere
    \item se \(\rho(A) = 1\), allora \(Q\) è tre volte degenere
\end{itemize}}

\subsubsection{Sezioni piane riducibili}
Dati data una quadrica \(Q\) e un piano \(\pi \) abbiamo \(C = Q \cap \pi \), se \(\pi \not\subseteq Q\), allora \(C\) è una conica per il secondo teorema dell'ordine. 
\nt{Se \(Q\) è una quadrica riducibile, allora \(C\) è riducibile.}

\thm{}{Sia \(Q\) una quadrica irriducibile (cioè cono, cilindro o quadrica generale) e sia \(P \in Q\) e sia \(\alpha \) un piano contenente \(P\). Allora
\begin{itemize}
    \item se \(P\) è doppio, allora \(P\) è doppio anche per \(C = Q \cap \pi \), quindi \(C\) è riducibile
    \item se \(P\) è un punto semplice, allora \(P\) è doppio per \(C = Q \cap \alpha \) se, e soltanto se, \(\alpha \) è il piano tangente in \(P\) a \(Q\), quindi \(C\) è riducibile 
\end{itemize}}
\nt{Se \(Q\) è generale, allora le sezioni piane di \(Q \cap \alpha \) sono riducibili se, e soltanto se, \(\alpha \) è un piano tangente a \(Q\).}
\section{Conica impropria di una quadrica irriducibile}
\subsubsection{Cono o cilindro}
\mprop{}{Sia \(Q\) un cono e sia \(C_{\infty} = Q \cap \pi _\infty\) la sua conica impropria, allora
\begin{enumerate}
    \item \(C_\infty\) è una conica generale
    \item se \(C_\infty\) è reale, il cono ha generatrici reali ed è detto \textbf{a falda reale}
    \item se \(C_\infty\) non ha punti reali, allora l'unico punto reale di \(Q\) è il vertice \(V\) del cono, quindi il cono ha generatrici a coppie immaginarie e coniugate ed è detto \textbf{privo di falda reale}
\end{enumerate}}

\mprop{}{La conica impropria \(C_\infty = Q \cap \pi _\infty\) di un cilindro \(Q\) è riducibile.}
\pf{Dimostrazione}{\(V\), vertice del cilindro, appartiene a \(\pi _{\infty}\), quindi \(V\) è doppio anche in \(Q \cap \pi _\infty = C\), di conseguenza \(C\) ha un punto doppio ed è riducibile.}

\subsubsection{Classificazione affine dei cilindri}
Un cilindro \(Q\) è detto 
\begin{enumerate}
    \item \textbf{iperbolico}, se \(C_\infty\) è unione di due rette reali e distinte
    \item \textbf{ellittico}, se \(C_\infty\) è unione di due rette immaginarie e coniugate
    \item \textbf{parabolico}, se \(C_\infty\) è unione di una retta contata 2 volte
\end{enumerate}
\end{document}
