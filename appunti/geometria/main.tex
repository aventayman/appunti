\documentclass{report}

\input{preamble}
\input{macros}
\input{letterfonts}

\title{\Huge{Algebra Lineare e Geometria Analitica}\\Ingegneria dell'Automazione Industriale}
\author{\huge{Ayman Marpicati}}
\date{A.A. 2022/2023}

\begin{document}

\maketitle
\newpage% or \cleardoublepage
% \pdfbookmark[<level>]{<title>}{<dest>}
\pdfbookmark[section]{\contentsname}{toc}
\tableofcontents


\paragraph{Osservazione:} L'equazione della generica conica di \(\tilde{A}_{2}(\CC)\) dipende da 6 coefficienti definiti a meno di un fattore di proporzionalità. Quindi le coniche di \(\tilde{A}_{2}(\CC)\) sono \(\infty^{5}\).

\mprop{}{Sia \(C\) una conica di \(\tilde{A}_{2}(\CC)\) riducibile. Allora \(C\) è unione di 2 rette
\begin{enumerate}
    \item reali e distinte
    \item reali e coincidenti
    \item immaginarie e coniugate
\end{enumerate}}

\pf{Dimostrazione}{Sia \(C\) conica associata al polinomio \(F=(x_1,x_2, x_3) = 0\). Se \(C\) è riducibile \(F=(x_1,x_2, x_3) = F_1=(x_1,x_2, x_3) \cdot F_2=(x_1,x_2, x_3)\) dove \(F_1\) e \(F_2\) hanno grado unitario, quindi rappresentano delle rette e di conseguenza \(C\) è unione di due rette \(r_1\) e \(r_2\). Se \(r_1\) e \(r_2\) sono entrambe reali siamo nei casi \(1\) o \(2\). Se invece \(r_1\) è immaginaria, \(\overline{r_1}\) è ancora componente di \(C\) (per ogni \(P \in r_1, \ \overline{P}\in C\)), ma \(r_1 \neq \overline{r_1} \implies \overline{r_1}=r_2\implies C\) si riduce in due rette immaginarie e coniugate.}

\paragraph{Osservazione:} Se \(r\) è immaginaria anche \(\overline{r}\) lo è. Infatti \(r \neq \overline{r}\) e quindi \(\overline{r} \neq \overline{\overline{r}} = r\).

\mprop{}{In \(\tilde{A}_{2}(\CC) \) una conica
\begin{enumerate}
    \item non ha punti tripli
    \item ha un punto doppio se, e soltanto se, è riducibile. E abbiamo due possibilità
        \begin{enumerate}
            \item ha solo un punto doppio \(P\) e si riduce in due rette distinte per \(P\)
            \item ha almeno due punti doppi allora ne ha \(\infty^{1}\) e si fattorizza in una retta
                reale contata due volte
        \end{enumerate}
\end{enumerate}}

\pf{Dimostrazione}{"\(\implies \)" Per ipotesi \(C\) ha punto doppio \(P\). Sia \(R \in C\) e consideriamo la retta \(r = rt(P,R)\), se non fosse componente avrebbe \[
|r \cap C| \ge 2 + 1 = 3 \quad \text{intersezioni con \(C\)}
\] \textbf{Assurdo!} Questo è in contraddizione con il teorema dell'ordine.

"\(\impliedby \)" Sia \(C\) riducibile. Allora  \(C = r_1 \cup r_2\). Sia \(P \in r_1 \cap  r_2\) e sia \(r\) una retta per \(P\) diversa \(r_1\) e da \(r_2\). Quindi \(r \cap C = P\). Per il teorema dell'ordine \(P\) ha molteplicità doppia e abbiamo due casi
\begin{enumerate}
    \item se \(r_1=r_2\) abbiamo \(\infty^{1}\) punti doppi e \(C = r_1 \cup  r_1\)
    \item altrimenti abbiamo un \textbf{solo} \(P\) punto doppio che è \(r_1\cap r_2\)
\end{enumerate}
Dobbiamo dimostrare che esiste un solo punto. Siano per assurdo \(P_1\) e \(P_2\) punti doppi e sia \(C=r_1 \cup r_2\) con \(r_1\neq r_2\). Sia \(Q \in r_2\) con \(P_2 \in r_1\), allora \[
|rt(P_2, Q) \cap C| \ge \underbrace{2}_{P_2} + \underbrace{1}_{Q} 
\] Per il teorema dell'ordine \(rt(P_2, Q)\) è componente. \textbf{Assurdo!} Perché avremmo 3 componenti \((r_1, r_2, rt(P_2, Q))\).} 

\dfn{Coniche generali o degeneri}{Una conica si dice
 \begin{itemize}
    \item \textbf{generale} se è priva di punti doppi \(\implies \) se non è riducibile
    \item \textbf{semplicemente degenere} se ha un solo punto doppio \(\implies C = r_1\cup r_2\) con \(r_1\neq r_2\) 
    \item \textbf{doppiamente degenere} se ha \(\infty^{1}\) punti doppi \(\implies C = r \cup r\)
\end{itemize}}

\thm{}{In \(\tilde{A}_{2}(\CC) \) i punti doppi di una conica \(C\) si trovano considerando le classi di autosoluzioni del sistema omogeneo \[
AX = \ul{0}
\] dove \(A\) è la matrice associata a \(C\).}

\pf{Dimostrazione}{\[
C: F(x_1, x_2, x_3) = 0 \quad \text{dove \(F\) è:} 
\]
\[
a_{11}x_1^2+ 2a_{12}x_1x_2+2a_{13}x_1x_3+a_{22}x_2^2+2a_{23}x_2x_3+a_{33}x_3^2=0
\] i punti doppi si trovano risolvendo 
\[
\begin{cases}
    \ \frac{\partial F}{\partial x_1}= 2a_{11}x_1+2a_{12}x_2+2a_{13}x_3=0\\
    \ \frac{\partial F}{\partial x_2}= 2a_{12}x_1+2a_{22}x_2+2a_{23}x_3=0\\
    \ \frac{\partial F}{\partial x_3} = 2a_{13}x_1+2a_{23}x_2+2a_{33}x_3=0\\
\end{cases}
\] Possiamo dividere tutti i fattori per 2 \[
\left( \; \begin{matrix}
    a_{11} & a_{12} & a_{13} \\
    a_{12} & a_{22} & a_{23} \\
    a_{13} & a_{23} & a_{33} \\
\end{matrix} \; \right) \cdot \left( \; \begin{matrix} x_1\\ x_2\\ x_3 \end{matrix} \; \right) = \left( \; \begin{matrix} 0\\ 0\\ 0\\ \end{matrix} \; \right)
\] \[
\implies AX = \ul{0} 
\]  
}

\thm{}{In \(\tilde{A}_{2}(\CC) \) una conica \(C : {^tX}A X = 0\) è 
\begin{enumerate}
    \item generale se \(\rho(A) = 3\) 
    \item semplicemente degenere se \(\rho(A) = 2\) 
    \item doppiamente degenere se \(\rho(A) = 1\)
\end{enumerate}}

\pf{Dimostrazione}{  
Dimostriamo tutti i casi singolarmente:
\begin{enumerate}
    \item \(C\) è generale se non ha punti doppi. Se  \(AX = \ul{0} \) ha solo la soluzione nulla \( \iff \rho(A) =3\).
    \item \(C\) è semplicemente degenere se ha un solo punto doppio. \( \iff AX = \ul{0} \) ha \(\infty^{1} \iff \rho(A) = 2\)
    \item \(C\) è doppiamente degenere se ha \(\infty^{1}\) punti doppi \(\iff AX = \ul{0} \) ha \(\infty^{2}\) soluzioni (se \([(x_1, x_2, x_3)]\) è soluzione \([(2x_1, 2x_2, 2x_3)]\) è lo stesso punto doppio) \(\iff \rho(A) =1\)
\end{enumerate}
}

\section{Classificazione affine di una conica generale}

Sia \(C\) una conica di \(\tilde{A}_{2}(\CC) \) e \(r\) una retta osserviamo che \(r \cap C =\)
\begin{enumerate}
    \item due punti reali e distinti
    \item un punto reale con molteplicità doppia
    \item due punti immaginari e coniugati
\end{enumerate}
Se consideriamo la \(r_{\infty}\) questa casistica ci dà la classificazione affine delle coniche generali.

\dfn{Ellisse, iperbole e parabola}{Sia \(C\) una conica di \(\tilde{A}_{2}(\CC) \) e sia \(C\) generale. Allora \(C \cap r_{\infty}\) è data dai due punti \(P, Q\) (non necessariamente distinti) e \(C\) si dice:
\begin{enumerate}
    \item \textbf{ellisse} se \(P\) e \(Q\) sono immaginari e coniugati;
    \item \textbf{iperbole} se \(P\) e \(Q\) sono reali e distinti;
    \item \textbf{parabola} se \(P\) e \(Q\) sono reali e coincidenti.
\end{enumerate}}

\section{Condizioni analitiche}
Sia \(C\) una conica generale di equazione \[
a_{11}x_1^2+ 2a_{12}x_1x_2+2a_{13}x_1x_3+a_{22}x_2^2+2a_{23}x_2x_3+a_{33}x_3^2=0
\] 
La \(r_{\infty}\) ha equazione \(x_{3}=0\) \[
\begin{cases}
    \ a_{11}x_1^2+2a_{12}x_1x_2+a_{22}x_2^2= 0 = C \cap r_{\infty} \\
    \ x_3 = 0 \\
\end{cases}
\]
Almeno uno fra \(x_1, x_2 \neq 0\). Supponiamo \(x_2 \neq 0\) e dividiamo per \(x_2^2\) \[
a_{11} \left( \frac{x_1}{x_2} \right) ^2 + 2a_{12} \frac{x_1}{x_2} + a_{22} = 0
\] 
La risolviamo in \(\frac{x_1}{x_2}\). Se 
\begin{enumerate}
    \item \(\frac{\Delta}{4} > 0\) abbiamo due soluzioni reali e distinte \(\implies \) \textbf{iperbole};
    \item \(\frac{\Delta}{4} = 0\) abbiamo due soluzioni coincidenti \(\implies \) \textbf{parabola};
    \item \(\frac{\Delta}{4} < 0\) abbiamo due soluzioni immaginarie e coniugate \(\implies \) \textbf{ellisse}.
\end{enumerate}

\[
\frac{\Delta}{4} = \left( \frac{b}{2} \right) ^2 - ac = \left( \frac{2a_{12}}{2} \right) ^2 - a_{11} a_{22} = a_{12}^2 - a_{11} a_{22}
\] \[
A = 
\left( \; \begin{matrix}
    a_{11} & a_{12} & a_{13} \\
    a_{12} & a_{22} & a_{23} \\
    a_{13} & a_{23} & a_{33} \\
\end{matrix} \; \right)
\quad \text{poniamo} \quad A^* =
\left( \; \begin{matrix}
    a_{11} & a_{12} \\
    a_{12} & a_{22} \\
\end{matrix} \; \right) \]
\[
|A^{*}| = a_{11}a_{22}-a_{12}^2= - \frac{\Delta}{4}
\] Se \(C\) è una conica generale \((|A| = 0)\) allora si applicano le casistiche sopra elencate.

\dfn{Polarità associata ad una conica}{Data una conica \(C: {^tX}AX = 0\) e dati due punti del piano \((\tilde{A}_{2}(\CC) )\) \[
        P' = [(x_1', x_2', x_3')] \quad e \quad P '' = [(x_1 '', x_2 '', x_3 '')]
\] si dice che \(P'\) è coniugato a \( P ''\) rispetto a \(C\) se \[
{^tX'AX '' = 0 \quad con \quad X' = \left( \; \begin{matrix} x'_1\\ x_2'\\ x'_3 \end{matrix} \; \right)} \quad  e \quad X '' = \left( \; \begin{matrix} x''_1\\ x_2 ''\\ x''_3 \end{matrix} \; \right)
\] }

\paragraph{Osservazione:} Sia \(P'\) coniugato a \(P ''\), ovvero \[
    {^tX'} A X '' = 0 \implies {^t({^tX'}AX '')} = 0 = {^tX ''} {^tA}{^t({^tX'})} = {^tX ''} A X' = 0 \implies P '' \text{ è coniugato a } P' 
\] Quindi la relazione di coniugio è simmetrica \(\implies \) potremo dire semplicemente che \(P'\) e \(P ''\) sono coniugati.

\dfn{Polare}{Sia \(C\) una conica di \(\tilde{A}_{2}(\CC) \) e sia \(P \in \tilde{A}_{2}(\CC) \) si dice \textbf{polare} di \(P\) rispetto a \(C\) il luogo dei punti \(Q\) coniugati di \(P\) rispetto alla conica \(C\).}

\mprop{}{In \(\tilde{A}_{2}(\CC) \) la polare di un punto \(P\) rispetto ad una conica generale \textbf{è una retta}.}
\pf{Dimostrazione}{Siano \([(x_1', x_2', x_3')] = P\) allora \(Q = [(x_1, x_2, x_3)]\) appartiene alla polare di \(P\) se e soltanto se \[
        (x_1', x_2', x_3') A \left( \; \begin{matrix} x_1\\ x_2\\ x_3 \end{matrix} \; \right) = 0 \quad (x_1', x_2', x_3') A = (a,b,c)
\] \[
(a, b, c) \left( \; \begin{matrix} x_1\\ x_2\\ x_3 \end{matrix} \; \right) = ax_1+bx_2+cx_3=0
\] Dimostriamo che \((a,b,c) \neq \ul{0} \). Sia per assurdo \((a,b,c) = \ul{0} \implies (x_1', x_2', x_3') A = \ul{0} \) \[
\iff {^tA} \left( \; \begin{matrix} x'_1\\ x_2'\\ x'_3 \end{matrix} \; \right) \iff A \left( \; \begin{matrix} x'_1\\ x_2'\\ x'_3 \end{matrix} \; \right) = \ul{0} 
\] \(\left( \; \begin{matrix} x_1'\\ x_2'\\ x_3' \end{matrix} \; \right)\) è le coordinate di un punto doppio \(\implies P\) è un punto doppio di \(C \implies \) ma \(C\) è generale \(\implies \) \textbf{assurdo!} \(\implies (a,b,c) \neq \ul{0} \implies ax_1 + bx_2+ cx_3 = 0\) è una retta. Essa è chiamata retta polare di \(P\) rispetto a \(C\).}
\dfn{}{\(P\) è detto polo della sua retta polare. La relazione \textbf{polo} \(\leftrightarrow\) \textbf{polare} è detta polarità ed è una biiezione. }

\subsection{Principio di reciprocità}
Sia \(C\) una conica generale di \(\tilde{A}_{2}(\CC) \), sia \(P \in \tilde{A}_{2}(\CC) \) e sia \(p\) la polare di \(P\). Allora
\begin{enumerate}
    \item le polari dei punti di \(p\) passano per \(P\).
        \pf{Dimostrazione}{Sia \(Q \in p \implies Q, P\) sono coniugati \(\implies P \in q\) di \(Q\).}
    \item i poli delle rette per \(P\) appartengono a \(p\).
        \pf{Dimostrazione}{Sia \(q\) una retta per \(P\). Il polo \(Q\) di \(q\) è coniugato a tutti i punti di \(q \implies Q\) è coniugato a \(P \implies Q \in p\).}
\end{enumerate}

\mprop{}{Sia \(C\) una conica generale di \(\tilde{A}_{2}(\CC) \). Allora
\begin{enumerate}
    \item sia \(P \in C \implies \) la polare \(p\) di \(P\) è la retta tangente a \(C\) in \(P\).
        \pf{Dimostrazione}{Sia \(P\) di coordinate \(x_P = \left( \; \begin{matrix} x'_1\\ x_2'\\ x'_3 \end{matrix} \; \right)\) appartenente alla conica allora la polare di \(P\) ha equazione \({^tX_P}A \left( \; \begin{matrix} x_1\\ x_2\\ x_3 \end{matrix} \; \right) = 0\) che è la formula della retta tangente a \(C\) in \(P\).}
    \item Sia \(P \notin C\). La polare di \(P\) è la congiungente dei due punti \(T_1\) e \(T_2\) ottenuti intersecando le tangenti \(t_1\) e \(t_2\) alla conica per \(P\).
        \pf{Dimostrazione}{\(T_1 \in C \implies \) la polare di \(T_1\) rispetto a \(C\) è \(t_1\). \(P \in t_1 \implies P\) appartiene alla polare di \(T_1\). Quindi per il principio di reciprocità \(T_1\) appartiene alla polare di \(P\) \(\implies T_1 \in p\). Analogamente \(T_2 \in C \implies \) la polare di \(T_2\) è \(t_2\) e \(P \in t_2 \implies T_2 \in p\). Quindi \(T_1, T_2 \in p \implies p\) è la congiungente di \(T_1\) e \(T_2\).}
\end{enumerate}}

\paragraph{Osservazione:} Equivalentemente il punto 2 si può riscrivere
 \mprop{}{Se \(P \notin C\) la sua polare \(p\) si ottiene congiungendo i punti \(T_1\) e \(T_2\) di tangenza delle tangenti per \(P\).}

 \dfn{Centro e diametri di una conica}{Si dice \textbf{centro} di una conica generale di \(\tilde{A}_{2}(\CC) \) il polo della retta impropria. Si dicono diametri di una conica generale le rette polari dei punti impropri.}
\paragraph{Osservazione:} Per il principio di reciprocità i diametri passano per il centro della conica. Quindi sono il fascio proprio (se c'è proprio) di rette per \(C\).

Per determinare le coordinate del centro dobbiamo scegliere due punti \(X_{\infty} = [(1, 0, 0)]\), punto improprio dell'asse \(x\), e \(Y_{\infty} = [(0,1,0)]\), punto improprio dell'asse \(y\). La polare di \(X_{\infty}\) è \[
    (1,0,0)
\left( \; \begin{matrix}
    a_{11} & a_{12} & a_{13} \\
    a_{12} & a_{22} & a_{23} \\
    a_{13} & a_{23} & a_{33} \\
\end{matrix} \; \right)
\left( \; \begin{matrix} x_1 \\ x_2\\ x_3 \end{matrix} \; \right) = 0 \qquad (a_{11}, a_{12}, a_{13}) \left( \; \begin{matrix} x_1 \\ x_2\\ x_3 \end{matrix} \; \right) = a_{11}x_1+a_{12}x_2 + a_{13} x_3 = 0
\] Analogamente la polare di \(y_{\infty}\) è  \[
a_{12}x_1 + a_{22}x_2 + a_{23}x_3 = 0
\] \[
\begin{cases}
    \ a_{11}x_1+a_{12}x_2+a_{13}x_3 = 0 \qquad P_1 \\
    \ a_{12}x_1+a_{22}x_2+a_{23}x_3=0 \qquad P_2 \\
\end{cases}
\] 
Il centro \(C\) è proprio se \(P_1\) e \(P_2\) non sono paralleli. Se \[
\left| \; \begin{matrix}
    a_{11} & a_{12} \\
    a_{12} & a_{22} \\
\end{matrix} \; \right| = |A^{*}| \neq 0
\] Il centro è un punto proprio. Quindi il centro è un punto proprio se \(C\) è un ellisse o un'iperbole. Quindi in questo caso i diametri sono un fascio proprio di rette di centro \(C\). \[
F: \ \lambda (a_{11}x_1+a_{12}x_2+a_{13}x_3) + \mu (a_{12}x_1 + a_{22}x_2 + a_{23}x_3) = 0
\] \textbf{Equazione del fascio dei diametri.} Se \(C\) è una parabola \(\implies |A^{*}| = 0 \implies P_1 \) parallelo a \(P_2 \implies  \) il centro è un punto improprio. \(\implies \) i diametri formano un fascio improprio di equazione \[
a_{11}x_1+a_{12}x_2 + kx_3 = 0 \quad \text{con} \quad k \in \CC
\] fascio improprio dei diametri della parabola.

\section{Asintoti di una conica}
\dfn{Asintoti}{Si dicono \textbf{asintoti} di una conica le rette proprie tangenti alla conica nei suoi punti impropri.}
\paragraph{Osservazione:} Gli asintoti di una conica sono quindi le rette polari nei suoi punti impropri. Gli asintoti sono quindi dei diametri e passano per il centro. Se il centro è proprio (cioè se \(C\) è un'ellisse o un'iperbole) gli asintoti sono le rette che congiungono il centro con i punti impropri di \(C\).

\mprop{}{La parabola è una conica con centro improprio e priva di asintoti.}
\pf{Dimostrazione}{Sia \(C\) una parabola \(\implies C\) è tangente alla retta impropria in un punto che chiamiamo \(P_{\infty}\). Quindi la retta polare di \(P_{\infty}\) è \(r_{\infty} \implies \) il polo della \(r_{\infty}\) è \(P_{\infty} \implies \) il punto \(P_\infty\) è il centro della parabola. Osserviamo che \(C\) ha solo un punto improprio \(P_\infty \implies \) ammette solo una tangente nel suo punto improprio. Ma \(t\) è la \(r_\infty \implies \) la \(r_{\infty}\) non è un asintoto.}

\dfn{Coniche a centro}{Diremo che l'iperbole e l'ellisse sono coniche \textbf{a centro}, mentre la parabola è detta conica \textbf{non a centro}.}

\section{Proprietà metriche}
\dfn{Iperbole equilatera}{Un'iperbole si dice \textbf{equilatera} se i suoi asintoti sono ortogonali.}

\mprop{}{Una conica generale è un'iperbole equilatera se, e soltanto se, \(a_{11} + a_{22} = 0\).}

\ex{}{Si stabiliscano i valori di \(k \in \RR:\) \[
C: \ 2kx^2 + 2 (k-2) xy - 4 y ^2 + 2x + 1 = 0
\] sia un'iperbole equilatera. \\ 
\begin{enumerate}
    \item \(2k = -(-4) \to k = 2\)
    \item Sostituiamo dentro all'equazione e scriviamola in forma omogenea \[ 4x_1 ^2 + 0 x_1 x_2 - 4 x_2 ^2 + 2 x_1 x_3 + x_3 ^2 = 0 \quad 
        A = 
\left| \; \begin{matrix}
    4 & 0 & 1 \\
    0 & -4 & 0 \\
    1 & 0 & 1 \\
\end{matrix} \; \right| \neq 0
\] \(k = 2\) dà luogo ad un'iperbole equilatera.
\end{enumerate}}

\dfn{Ortogonale al punto improprio}{Diremo che la retta \(p\) di parametri direttori \([(l', m')]\) è ortogonale al punto improprio \(P: [(l,m,0)]\) se \(ll' + mm' = 0\).}

\dfn{Asse di una conica}{Si dice \textbf{asse} di una conica ogni diametro ortogonale al proprio polo.}

\dfn{Vertici}{Si dicono \textbf{vertici} le intersezioni proprie della conica con i propri assi.}

\section{Condizioni analitiche}
\mprop{}{Gli assi di una conica a centro (ellisse o iperbole) sono due e sono ortogonali tra loro, a meno che non si tratti di una circonferenza generalizzata, in tal caso tutti i diametri sono assi.}
\pf{Dimostrazione}{Per definizione i diametri sono le polari dei punti impropri. Dato \(P_\infty : [(l,m,0)]\) \[
\left( \; \begin{matrix}
    l & m & 0 \\
\end{matrix} \; \right)
\left( \; \begin{matrix}
    a_{11} & a_{12} & a_{13} \\
    a_{12} & a_{22} & a_{23} \\
    a_{13} & a_{23} & a_{33} \\
\end{matrix} \; \right)
\left( \; \begin{matrix} x_1 \\ x_2\\ x_3 \end{matrix} \; \right) = 0
\] Il generico diametro è: \[
\left( \; \begin{matrix}
    la_{11} + m a_{12} & l a_{12} + m a_{22} & l a_{13} + m a_{23} \\
\end{matrix} \; \right)
\left( \; \begin{matrix} x_1 \\ x_2\\ x_3 \end{matrix} \; \right) = 0
\] \[
(la_{11} + m a_{12}) x_1 + (la_{12} + ma_{22}) x_2 + (la_{13} + ma_{23}) x_3 = 0
\] \[
p.d.d : \ [(-la_{12} - m a_{22}, l a_{11} + ma_{12})]
\] Il polo di \(d\) è \(P_\infty: [(l,m,0)]\). \(d\) è un asse se è ortogonale a \(P_\infty\) ovvero se \[
l ( -la_{12}-ma_{22} ) + m(la_{11} + ma_{12}) = 0
\] \[
-l^2 a_{12} + ml (-a_{22}+a_{11}) + m^2 a_{12} = 0 \qquad l^2a_{12} + ml(a_{22}- a_{11}) - m^2 a_{12} = 0 
\] \[
a_{12} \left( \frac{l}{m} \right) ^2 + \frac{l}{m} (a_{22} - a_{11}) - a_{12} = 0
\] Se \(a_{12}=0\) e \(a_{22} = a_{11}\) l'equazione è risolta da tutte le coppie \((l,m)\). Quindi se la conica è una circonferenza generalizzata tutti i diametri sono assi. I due assi hanno polo \(P_\infty : [(l',m',0)]\) e \(Q_\infty : [(l '', m '', 0)]\). Sia \(p'\) l'asse associato al polo \(P_\infty\) e sia \(A_\infty\) il suo punto improprio. Sia \(a\) la retta che congiunge il centro al punto improprio \(rt(C, P_\infty)\), per ipotesi \(a \perp p'\). \(a\) contiene \(P_\infty\) che è il polo di \(p'\), quindi per il principio di reciprocità \(p'\) contiene il polo di \(a\). Il polo di \(a\) è improprio (perché \(a\) è diametro) \(\implies \) il punto improprio di \(a\) è \(A_\infty\), ma \(A_\infty \) è ortogonale alla direzione di \(a \implies a\) è un asse. Quindi i due assi sono ortogonali.}

\mprop{}{La parabola ha un unico asse e un solo vertice \(v\). Inoltre la tangente alla parabola in \(v\) è ortogonale all'asse.}

\pf{Dimostrazione}{Il punto \(P_\infty\) di una parabola è \([(-a_{12}, a_{11}, 0)]\). I \(p.d.d = [(-a_{12}, a_{11})]\). La direzione ortogonale è data da \([(a_{11}, a_{12})]\), quindi il punto \(P_\infty\) è \([(a_{11}, a_{12}, 0)]\). Da cui segue che l'asse è unico ed è la polare di \((a_{11}, a_{12}, 0)\). Sostituendo nell'equazione del fascio improprio dei diametri abbiamo che l'asse ha equazione: \[
a_{11}(a_{11}x_1+ a_{12}x_2 + a_{13}x_3) + a_{12}(a_{12}x_1 + a_{22}x_2 + a_{23}x_3) = 0
\] Per il teorema dell'ordine \(a\) interseca la parabola \(C\) in due punti, ma uno è \(P_\infty\) quindi l'altro punto sarà l'unico vertice della parabola. \\ Ora dimostriamo la seconda parte del teorema. \(v \in a\) che è il polo di \(t\). Per il principio di reciprocità \(t\) contiene il polo di \(a\), ovvero \(P_{\infty} \in t\). Ma \(P_\infty\) è ortogonale ad \(a \implies t \perp a\).}

\chapter{Ampliamento di \(A_3(\RR )\)}
Chiamiamo con \(\tilde{A}_{3}(\RR ) \) lo spazio reale affine ampliato. I punti possono essere
\begin{itemize}
    \item \textbf{propri} \(A\) dei punti di \(A_3(\RR )\) 
    \item \textbf{impropri} \(A_\infty\) direzioni delle rette, (spazi di traslazione di dimensione 1)
\end{itemize}
Le rette possono essere
\begin{itemize}
    \item \textbf{proprie} rette di \(A_3(\RR )\) ciascuna estesa con il suo punto improprio (ovvero la sua direzione)
    \item \textbf{improprie} sono le giaciture dei piani (spazi di traslazione di dimensione 2)
\end{itemize}
I piani possono essere
\begin{itemize}
    \item \textbf{propri} i piani di \(A_3(\RR )\) ciascuno esteso con la sua retta impropria (ovvero la sua giacitura)
    \item \textbf{piano improprio} \(A_\infty\) il luogo dei punti impropri
\end{itemize}

\mprop{}{
Diamo una serie di conseguenze senza dimostrazione
\begin{enumerate}
    \item due rette parallele hanno la stessa direzione e quindi hanno lo stesso punto improprio
    \item due piani paralleli hanno la stessa giacitura e quindi hanno la stessa retta impropria
    \item il piano improprio contiene tutte e sole le rette improprie
    \item ogni retta impropria contiene un solo punto improprio (la sua direzione)
    \item ogni piano proprio contiene \(\infty^{1}\) punti impropri, ovvero una retta (la sua giacitura).
\end{enumerate}}

\section{Geometria analitica in \(\tilde{A}_{3}(\RR) \)}
Indichiamo con \[ \frac{\RR ^{4} \backslash  \{(0,0,0,0)\}}{\rho } \] cioè l'insieme delle quaterne definite a meno di un fattore di proporzionalità reale e non nullo. In cui \(\rho \) indica la relazione di equivalenza data dalla proporzionalità. Quindi consideriamo due terne equivalenti se sono proporzionali.

\mprop{}{Sia \(RA = [O, B]\) un riferimento affine di \(A_{3}(\RR) \) e sia \[
\phi : A \cup A_\infty \quad  \to \quad \frac{\RR ^{4} \backslash  \{(0,0,0,0)\}}{\rho }
\] sia \(P \in A\) di coordinate \((x,y,z)\) \[
\phi(P) = [(x,y,z,1)]
\] sia \(P \in A_\infty\) corrispondente alla direzione \([(l,m,n)]\) \[
\phi (P) = [(l,m,n,0)]
\] la mappa \(\phi\) è una biiezione e le coordinate indotte da \(\phi\) sono chiamate \textbf{coordinate omogenee}.}

\ex{}{\[
        Q = [(2,0,3,-2)] \qquad -2 \neq 0 \implies Q \ \text{è proprio}
\] \[
Q = \left[ \left( \frac{2}{-2}, \frac{0}{-2}, - \frac{3}{2}, 1 \right)  \right] \implies Q = \left( -1, 0, - \frac{3}{2} \right) 
\] \[
P = [(2,1,0,0)] \implies [(2,1,0)]
\] }

\dfn{Rappresentazione dei piani}{
\[
ax_1 + bx_2 + cx_3 + dx_4 = 0 \quad \text{con} \quad (a,b,c,d) \neq (0,0,0,0)
\] questa è l'equazione omogenea dei piani in \(\tilde{A}_{3}(\RR) \) (e si ottiene in modo analogo all'equazione omogenea delle rette in \(\tilde{A}_{2}(\RR) \)).
}
\paragraph{Osservazione:}
\begin{enumerate}
    \item se \((a,b,c) \neq (0,0,0)\) allora il piano è proprio ed ha equazione affine \[
ax + by + cz + d = 0
\] 
    \item se \((a,b,c) = (0,0,0)\) allora \(d \neq 0\) e otteniamo \(x_4 = 0\) (che definisce il piano improprio).
\end{enumerate}

\dfn{Rappresentazione di rette}{Una retta è intersezione di 2 piani distinti \[
r:
\begin{cases}
    \ ax_1 + bx_2+ cx_3 + dx_4 = 0 \\
    \ a'x_1 + b'x_2+ c'x_3 + d'x_4 = 0 \\
\end{cases} \quad \text{con} \quad \rho
\left( \; \begin{matrix}
    a & b & c & d \\
    a' & b' & c' & d' \\
\end{matrix} \; \right) = 2
\] questa è la rappresentazione della generica retta di \(\tilde{A}_{3}(\RR)\)}
\begin{itemize}
    \item se \[
    \rho
\left( \; \begin{matrix}
    a & b & c \\
    a' & b' & c' \\
\end{matrix} \; \right) = 2     
    \] \(r\) è propria \[
\begin{cases}
    \ ax + by + cz + d = 0 \\
    \ a'x + b'y + c'z + d' = 0 \\
\end{cases}
    \]
\item se \[
    \rho
\left( \; \begin{matrix}
    a & b & c \\
    a' & b' & c' \\
\end{matrix} \; \right) = 1     
    \] ho due casi possibili
    \begin{itemize}
        \item i due piani sono paralleli e distinti
        \item uno dei due è il piano improprio e quindi \(x_4 = 0\)
    \end{itemize}
    in entrambi i casi \(r\) è impropria
\end{itemize} 

\section{Complessificazione di \(\tilde{A}_{3}(\CC)\)}
\(\tilde{A}_{3}(\CC) \) è lo spazio ampliato e complessificato. I suoi punti sono le quaterne di \[
\frac{\CC ^{4} \backslash  \{(0,0,0,0)\}}{\rho }
\] cioè le classi di proporzionalità delle quaterne complesse. La relazione di proporzionalità è chiaramente da intendersi in \(\CC\). All'interno dello spazio definiamo
 \begin{itemize}
    \item le \textbf{rette} sono i punti tali che \[
\begin{cases}
    \ ax_1+ bx_2 + cx_3+dx_4 =0 \\
    \ a'x_1 + b'x_2 + c'x_3 + d'x_4 = 0 \\
\end{cases} \quad \text{con}\quad a,a',b,b',c,c',d,d' \in C
    \] e tali che \[
    \rho
\left( \; \begin{matrix}
    a & b & c & d \\
    a' & b' & c' & d' \\
\end{matrix} \; \right) = 2
    \]
    \item un \textbf{piano} è costituito dai punti \[
    ax_1 + bx_2+cx_3+dx_4 = 0 \quad \text{con}\quad (a,b,c,d) \in \CC^{4} \backslash \{(0,0,0,0)\} 
    \] 
\end{itemize}

\dfn{Punti, rette e piani reali}{In \(\tilde{A}_{3}(\CC) \) i punti, le rette e i piani si dicono \textbf{reali} se ammettono almeno una rappresentazione con coefficienti reali. Si dicono immaginari altrimenti.}

\dfn{Rette immaginarie di prima e seconda specie}{In \(\tilde{A}_{3}(\CC) \) una retta \(r\) immaginaria è detta \textbf{immaginaria di prima specie} se è complanare con la propria coniugata \(\overline{r}\). \(r\) è detta  \textbf{immaginaria di seconda specie} se è sghemba con \(\overline{r}\).}

\mprop{}{
\begin{enumerate}
    \item La retta congiungente due punti immaginari e coniugati è reale
    \item se una retta (o un piano) reale contiene un punto \(P\) immaginario allora contiene anche \(\overline{P}\)
    \item se \(P\) è immaginario l'unica retta reale per \(P\) è \(rt(P, \overline{P})\)
    \item l'intersezione tra un piano \(\pi \) immaginario e \(\overline{\pi }\) è una retta reale
    \item un piano \(\pi \) immaginario contiene un'unica retta reale \(: \pi \cap \overline{\pi }\)
    \item se \(r\) è una retta immaginaria allora
        \begin{enumerate}
            \item \(r\) è contenuta in al più un piano reale
            \item \(r\) contiene al più un punto immaginario
        \end{enumerate}
        in particolare se \(r\) è immaginaria di prima specie il piano contenente \(r\) e \(\overline{r}\) è reale e \(r \cap \overline{r}\) è un punto reale. Se invece \(r\) è immaginaria di seconda specie allora \(r\) non è contenuta in alcuno piano reale e non contiene alcun punto reale.
\end{enumerate}
}
\dfn{Superfici algebriche reali in \(\tilde{A}_{3}(\CC) \)}{\textbf{Una superficie algebrica reale di \(\tilde{A}_{3}(\CC) \)} è l'insieme delle classi di autosoluzioni complesse di un'equazione del tipo \[
F(x_1, x_2, x_3, x_4) = 0 \quad \text{ove} \quad F \ \text{è un polinomio omogeneo a coefficienti reali in } x_1, x_2, x_3, x_4
\]  il grado di \(F\) è chiamato ordine della superficie. Se \(F\) è fattorizzabile in polinomi di grado positivo la superficie si dice riducibile in componenti \[
\text{fattori di \(F\)} \leftrightarrow \text{componenti della superficie}
\] }

\thm{Primo teorema dell'ordine}{L'ordine di una superficie algebrica \(\Sigma\) reale è uguale al numero di punti in comune a \(\Sigma\) e a una qualsiasi retta \(r\) non contenuta in \(\Sigma\) a patto di contarli con la dovuta molteplicità.}

\cor{}{\[\text{Se }|r \cap \Sigma| > \text{ord}(\Sigma) \implies r \subseteq \Sigma\].}

\thm{Secondo teorema dell'ordine}{L'intersezione tra una superficie algebrica reale \(\Sigma\) e un piano \(\alpha \) non componente di \(\Sigma\) è una curva dello stesso ordine di \(\Sigma\).}

\cor{}{Se \(\Sigma \cap \pi \) contiene una curva \(C\) con \(\text{ord}(C) > \text{ord}(\Sigma) \implies \pi \) è componente di \(\Sigma\).}

\dfn{}{In \(\tilde{A}_{3}(\CC) \), data una superficie algebrica reale \(\Sigma\), un punto \(P \in \Sigma\) è detto \textbf{r-uplo} se la generica retta per \(P\) ha molteplicità di intersezione con \(\Sigma\) in \(P\) uguale a \(r\).
\begin{itemize}
    \item se \(r = 1\) \(P\) è detto \textbf{semplice} 
    \item se \(r > 1\) \(P\) è detto \textbf{multiplo} 
\end{itemize}}

\thm{}{I punti multipli di una curva algebrica reale di equazione \(F(x_1, x_2, x_3, x_4)\) sono le classi di autosoluzioni del sistema \[
\begin{cases}
    \ \frac{\partial F}{\partial x_1} = 0 \\
    \ \frac{\partial F}{\partial x_2} = 0 \\
    \ \frac{\partial F}{\partial x_3} = 0 \\
    \ \frac{\partial F}{\partial x_4} = 0 \\
\end{cases}
\] }
\end{document}
