\documentclass{report}

\input{preamble}
\input{macros}
\input{letterfonts}

\title{\Huge{Algebra Lineare e Geometria Analitica}\\Ingegneria dell'Automazione Industriale}
\author{\huge{Ayman Marpicati}}
\date{A.A. 2022/2023}

\begin{document}

\maketitle
\newpage% or \cleardoublepage
% \pdfbookmark[<level>]{<title>}{<dest>}
\pdfbookmark[section]{\contentsname}{toc}
\tableofcontents

\mprop{}{Siano \(\alpha \ r\) rispettivamente un piano e una retta di \(\EE_3(\RR)\) con \(\alpha \) non ortogonale a \(r\). Allora \(\exists ! \) piano \(\beta : \beta\) ortogonale \(\alpha\) e \(r \subseteq \beta.\)  }
\pf{Dimostrazione}{Dimostriamo l'esistenza: sia \(\beta = [P, V_1 + V_2 orto]\) dove \(r = [P, V_1]\) e \(\alpha = [Q, V_2]\). \begin{enumerate}
    \item \(\beta\) è un piano perché \(\dim(V_1 = 1), \ \dim(V_2 orto)= 1\) e \(V_1 \neq V_2 orto\) (poiché \(\alpha non orto r\) ) \( \implies \dim(V_1 + V_2 orto) = 2 \implies \beta\) è un piano.
    Per costruzione abbiamo che \(\beta \perp \alpha\), infatti lo spazio di traslazione di \(\beta\) è: \[
        V_1 \oplus V_2^{\perp} \supseteq V_2^{\perp} \text{  e \(V_2\) è lo spazio di traslazione di \(\alpha\)  }
    \] inoltre \(r \) 
\end{enumerate}}

\chapter{Geometria analitica in \(\EE_n(\RR)\) }
\dfn{}{In \(\EE_n(\RR)\) si dice \textbf{riferimento cartesiano ortogonale monometrico} la coppia \([0, \mcB]\): \begin{itemize}
    \item O è un punto di \(\EE_n(\RR)\) 
    \item \(\mcB=(e_1, e_2, ..., e_n)\) è una base ortonormale 
\end{itemize}}
\nt{\begin{enumerate}
    \item In \(\EE_n(\RR) \ (n= 2) \implies \mcB = (i,j)\) 
    \item In \(\EE_3(\RR) \ (n = 3) \implies \mcB = (i,j,k)\)
\end{enumerate}}

\dfn{Ortogonalità fra rette}{Siano \(r_1, r_2\) due rette di \(\EE_2(\RR)\) e sia \(r_1 = [P, f(v)] \quad v = l \cdot i + m \cdot j\), analogamente \(r_2 = [P, f'(v)] \quad v' = l' \cdot i + m' \cdot j\) \[
    v \perp v' \iff l \cdot l' + m \cdot m' = 0
    \] se \(r_1\) ha equazione \(ax + by + c = 0\) e \(r_2\) ha equazione \(a'x + b'y + c' = 0\) allora \(P.d.r_1 = [(-b,a)]\), e \(P.d.r_2 = [(-b', a')]\)  \[
    -b \cdot (-b') + a \cdot a' = bb' + aa' = 0
\]
Se abbiamo \(r_1, r_2\) rette in \(\EE_3(\RR)\) con \(P.d.r_1 = [(l,m,n)]\), \(P.d.r_2 = [(l',m',n')]\)  
\(r_1 \perp r_2 \iff v_1\) generatore della direzione di \(r_1 \perp v_2\) generatore della direzione di \(r_2\). \[
    v_1 = li + mj + nk \qquad v_2 = l'i + m'j + n'k
\] \[
    v_1 \perp v_2 \iff r_1 \perp r_2 \iff ll' + mm' + nn' = 0
\] Analogamente se \(r_1, r_2\) sono rette in \(\EE_n(\RR)\) con \(P.d.r_1 = [(x_1, x_2, ..., x_n)]\), \(P.d.r_2 = [(x_1', x_2', ..., x_n')]\)
}
\section{Direzione \(\perp\) ad un iperpiano}
\mprop{}{Sia \(r: ax + by + c = 0\) una retta di \(\EE_2(\RR)\). Allora \([(a, b)]\) è la classe dei parametri direttori della direzione ortogonale a \(r\).}
\pf{Dimostrazione}{\(P.d.r = [(-b,a)]\) e abbiamo che \((a, b) \cdot (-b, a)= 0\) oppure \((a \cdot i + b \cdot j) \cdot ( -b \cdot i + a \cdot j ) = 0 \implies [(a, b)] \perp r\).  }
\mprop{}{Sia \(\pi: ax + by + cz + d = 0\) un piano in \(\EE_3(\RR)\). Allora \([(a, b, c)]\) è la classe dei parametri direttori della direzione ortogonale a \(\pi\). }
\pf{Dimostrazione}{Sia \(v \in V_2 \ (\pi\) ha spazio di traslazione o \(V_2\)). Se \(v = (x, y, z) \implies ax + by + cz = 0 \iff (x, y, z) \cdot (a, b, c) = 0 \implies (a,b,c) \perp v \ \forall v \in V_2 \) }
Più in generale: sia \(S _{n-1}\) un iperpiano in \(\EE_n(\RR)\) di equazione cartesiana \(0 = a_1 x_1 + a_2 x_2 + ... + a_n x_n + a_0 \implies [(a_1, a_2, ..., a_n)]\) è la classe dei parametri direttori della direzione ortogonale a \(S _{n-1}\).

\mprop{Ortogonalità tra piani}{Siano \(\alpha: ax + by + cz + d = 0 \quad \beta: a'x + b'y + c'z + d' = 0\) due piani in \(\EE_3(\RR)\). Allora \(\alpha \perp \beta \iff a \cdot a' + b \cdot b' + c \cdot c' = 0\) }
\pf{Dimostrazione}{\(\alpha \perp \beta \iff V_2 \supseteq V_2'^{\perp}\) dove \(V_2\) è la giacitura di \(\alpha\) e \(V_2'\) è la giacitura di \(\beta\). \[V_2'^{\perp}= [\mcL((a', b',c')) ] \iff (a', b', c') \in V_2\]\((x, y, z) \in V_2 \iff ax + by + cz = 0\) e quindi \((a', b', c') \in V_2 \iff a \cdot a' + b \cdot b' + c \cdot c' = 0\) }
\mprop{Ortogonalità tra retta e piano}{Siano \(r:\) con \(P.d.r = [(l,m,n)]\) e sia \(\alpha\) di equazione \(ax + by + cz + d= 0\) una retta e un piano di \(\EE_3(\RR)\). Allora \(r \perp \alpha\) se e soltanto se \([(a,b,c)] = [(l,m,n)]\) }
\pf{Dimostrazione}{\(r \perp \alpha \iff V_1=V_2^{\perp}\) dove \(V_1\) è la direzione della retta e \(V_2\) è la giacitura di \(\alpha\). \[
        V_1 = \mcL((l,m,n)) = V_2 ^{\perp} = \mcL((a,b,c)) \iff [(a,b,c)] = [(l,m,n)]  
\]}

\dfn{Distanza tra 2 punti in \(\EE_{3}(\RR)\) }{Siano \(P = (x_{1} , x_{2}, ..., x_{n} )\) e \(Q = (x_{1} ' x_2 ', ..., x_n ')\). La distanza tra \(P\) e \(Q\) è la norma del vettore \(\vec{PQ}\) \[
        d(P,Q) = ||\vec{PQ}|| = \sqrt{\vec{PQ} \cdot \vec{PQ}}
\]     \[
    \vec{{PQ}} = (x_1'-x_1) e_1 + ... + (x_n' - x_n) e_n
\] \[
    ||\vec{{PQ}}|| = \sqrt{(x_1' - x_1)^{2} + ... + (x_n'-x_n)^{2}  }
\]}

\dfn{Caso \(\EE_{2}(\RR)\) }{\[
    P = (x, y) \quad Q = (x', y')
\] \[
    \vec{{PQ}} = (x'- x) i + (y' - y)j
\] \[
    d (P, Q) = \sqrt{(x' - x)^{2} + (y'-y)^{2} }
\] }
Caso \(\EE_{3}(\RR)\) da aggiungere

\dfn{Distanza tra punto e retta}{Siano \(P = (x_0, y_0)\) e \(r = [Q, V_1]\) rispettivamente un punto e una retta in \(\EE_{2}(\RR)\). Definiamo la \textbf{distanza tra il punto \(P\) e la retta \(r\)} come la distanza tra \(P\) e il punto \(H\), piede della perpendicolare per \(P\) a \(r\) (cioè l'intersezione tra \(r\) e la retta perpendicolare a \(r\) passante per \(P\)).}

Determiniamo \(||\vec{{PH}}||\). Se \(r\) ha equazione \(ax + by +c = 0\) allora \(V_1^{\perp} = \mcL(a \cdot i + b \cdot j)  \).\[
    \text{Posta} \quad n = [P, V_1^{\perp}] \implies n = [P, \mcL(a i + bj) ]
\]  \[
    H = n \cap r \text{ è la proiezione di \(P\) su \(r\). (è l'intersezione tra \(r\) e la retta per \(P^{\perp} \)).}
\] Sia \(P' = (x', y')\) un generico punti su \(r\). \[ax' + by' + c = 0\] \(PH\) è la componente di \(PP'\) lungo \(v\). \(PP' = (x'-x_0) i + (y' - y_0)j\). \[
    \vec{{PH}} = \frac{PP' \cdot v}{ v \cdot v } v
\] \[
d (P, H) = d (P, r) = || \vec{{PH}} || = ||(\frac{\vec{{PP'}} \cdot v}{v \cdot v} v)|| = [...] = \frac{|ax_0 + by_0 + c|}{\sqrt{a^{2} + b^{2} }} 
\] Da completare 

\dfn{Distanza punto piano}{Siano \(P = (x_0, y_0, x_0)\) e \(\alpha : ax + by + cz + d = 0\) un punto e un piano di \(\EE_{3}(\RR)\). Definiamo \textbf{la distanza} \(d(P, \alpha)\) come la distanza tra \(P\) e il punto \(H\) intersezione tra \(\alpha\) e la retta per \(p \perp \alpha \).}
\pf{Dimostrazione}{\(d(P, \alpha )= d (P, H) = ||\vec{{PH}}||\). Analogamente al caso piano abbiamo che \[
    d(P, \alpha ) = \frac{|ax_0 + by_0 + cz_0 + d|}{\sqrt{a^{2} + b^{2} + c^{2} }}
\]  }

\dfn{Distanza tra un punto e una retta in \(\EE_{3}(\RR)\) }{Siano \(P\) e \(r = [Q, V_1]\) un punto e una retta in \(\EE_{3}(\RR)\). Sia \(\alpha \) il piano per \(P\) ortogonale a \(r\) e sia \(H\) l'intersezione tra \(r\) e \(\alpha \). Definiamo \(d(P, r) = d(P, H) = ||\vec{{PH}}||\).}
\ex{}{In \(\EE_{3}(\RR)\) determiniamo la distanza di \(P = (3, 0, 1)\) da \(r : 
\begin{cases}
    x + y = 1 \\
    z = 2 \\
\end{cases}
\)  \[
\begin{cases}
    x = 1-t \\
    y = t \\
    z = 2 \\
\end{cases} \quad P.d.r = [(-1, 1, 0)]=[(a,b,c)] \qquad \alpha: -x + y + 0 \cdot z + d = 0
\] \[
    \text{ Imponiamo il passaggio per \(P\)}: \quad -3 + 0 + d = 0 \quad d = 3 \quad \alpha : -x + y + 3 = 0
\] \[
    \alpha \cap r :
\begin{cases}
    x + y = 1 \\
    -x + y + 3 = 0 \\
    z = 2 \\
\end{cases} \quad
\begin{cases}
    x+y=1 \\
    0x + 2y= -2 \\
    z=2 \\
\end{cases} \implies x = 2; \ y = -1
\]\[
    H: (2, -1, 2) \quad d(P,r) = ||\vec{{PH}}|| = \vec{{PH}} = (-1) i + (-1) j + k = -1 -j + k
\]}

\dfn{Retta di minima distanza}{Si dice \textbf{retta di minima distanza} tra due rette \(r, s\) sghembe in \(\EE_{3}(\RR)\) una retta ortogonale e incidente sia a \(r\) che a \(s\).}

\mprop{}{La retta di minima distanza tra \(r\) e \(s\) esiste ed è unica.}

\dfn{Distanza tra due rette sghembe in \(\EE_{3}(\RR)\)}{Definiamo \textbf{la distanza tra due rette \(r\) e \(s\) sghembe in \(\EE_{3}(\RR)\)} come la distanza tra i punti \(R\) e \(S\) ottenuti intersecando la retta \(t\) di minima distanza tra \(r\) e \(s\) con \(r\) e \(s\).}

\dfn{Assi}{In \(\EE_{2}(\RR)\) dati due punti \(P,Q\), si dice \textbf{asse} del segmento \(\ol{PQ}\) la retta passante per il punto medio di \(P\) e \(Q\) e ortogonale al segmento \(\ol{PQ}\).  }
\mprop{}{L'asse di un segmento \(\ol{PQ}\) è il luogo dei punti equidistanti da \(P\) e da \(Q\).}
\pf{Dimostrazione}{Dobbiamo dimostrare che \(||\vec{PH} || = ||\vec{QH} || \quad \forall H \in a\) (asse di \(\ol{PQ}\)). \[
    \vec{PH} = \vec{PM} + \vec{MH} \quad e \quad \vec{QH} = \vec{QM} + \vec{MH} 
\] \[
    ||\vec{PH} || = \sqrt{||PM || ^{2} + ||MH||^{2}  } \quad ||\vec{QH} || = \sqrt{||QM|| ^{2} + ||MH|| ^{2} } \quad \text{ ma } \quad ||PM||= ||QM|| 
\] \[
    ||\vec{PH} || = \sqrt{||PM|| ^{2} + ||MH|| ^{2} } = \sqrt{||QM|| ^{2} + ||MH|| ^{2} } = ||\vec{QH} || 
\]}

\ex{}{Determiniamo l'asse di \(P=(1,1)\) e \(Q=(2, -4)\). Il punto \(M = (\frac{3}{2}, -\frac{3}{2})\) \[
    \vec{PQ} = (2-1) i + (-4-1) j = 1 - 5 j = (1, -5)
\] \(r \perp \vec{PQ} \) per \(M\)  è del tipo \[
    x -5y + c = 0 \quad \text{e passa per \(M\) }
\] \[
    \frac{3}{2} + \frac{15}{2} + c = 0 \quad c = -9 \implies r : \ x - 5y -9 = 0
\]Alternativamente \[
    r: \ H \in r \iff d(H,P) = d(H, Q)
\]se \(H = (x, y)\) \[
    \sqrt{(x-1)^{2} + (y-1) ^{2} } = \sqrt{(x-2)^{2} + (y + 4)^{2} }
\] \[
    x^{2} - 2x + 1 + y^{2}  -2y + 1 = x^{2} - 4x + 4 + y^{2} + 8y + 16 \implies r: \ 2x -10y -18 = 0
\]}

\dfn{Piano assiale}{In \(\EE_{3}(\RR)\) si dice \textbf{piano assiale} del segmento \(\ol{PQ}\) il piano \(\alpha \) passante per il punto medio di \(P\) e \(Q\) e ortogonale al segmento \(\ol{PQ}\).}

\mprop{}{Il piano assiale del segmento \(\ol{PQ}\) è il luogo dei punti equidistanti tra \(P\) e \(Q\).}

\section{Circonferenze in \(\EE_{2}(\RR)\) }
\dfn{Circonferenza}{Dato un punto \(C = (x_0, y_0)\) in \(\EE_{2}(\RR)\) e dato \(r\) numero reale positivoSi dice circonferenza di centro \(C\) e raggio \(r\) il luogo dei punti aventi distanza \(r\) da \(C\). }

Sia \(P=(x, y)\) appartenente alla circonferenza di centro \(C\) e raggio \(r\). \[
        d(P,C) = \sqrt{(x-x_0)^{2}x^{2} + y^{2} + 2ax + 2by + c = 0 + (y-y_0)^{2}} = r \iff (x-x_0)^{2} + (y-y_0)^{2} = r^{2}  
\] \[
    x^{2} + y^{2} + 2ax + 2by + c = 0 \iff x^{2} + y^{2} - 2x_0x - 2y_0y + (x_0^{2} + y_0^{2} - r^{2} ) = 0
\] 

\mprop{Equazione cartesiana di una circonferenza}{Tutte e sole le circonferenze si rappresentano come \(x^{2} + y^{2} + 2ax + 2by + c = 0\) con \(a^{2} + b^{2} - c > 0\) e avremo che \(C = (-a, -b)\) e \(r = \sqrt{a^{2} + b^{2} -c}\)  }

\nt{Se \(r\) fosse 0, \(a^{2} + b^{2} - c = 0 \implies  x^{2} + y^{2} + 2ax + 2by + c = 0\) è rappresentata solo da \(C = (-a, -b)\).}
\mprop{}{Per tre punti non allineati in \(\EE_{2}(\RR)\) passa un unica circonferenza.}

\section{Sfere in \(\EE_{3}(\RR)\) }
\dfn{Sfera}{Sia \(C: (x_{0}, y_{0}, z_{0} )\) e sia \(r\) un numero reale positivo. Si dice \textbf{sfera} di raggio \(C\) e di centro \(r\) il luogo dei punti aventi distanza \(r\) da \(C\). }

Sia \(P:(x,y,z)\) appartenente alla sfera, allora \[
    d(P,C) = \sqrt{(x-x_0)^{2} + (y-y_0)^{2} + (z-z_0)^{2}} = r \iff (x-x_0)^{2} + (y-y_0)^{2} + (z-z_0)^{2} = r^{2}
\]
\nt{Una sfera è una superficie algebrica reale (Analogamente una circonferenza è una curva algebrica reale).}

\mprop{Equazione cartesiana di una sfera}{Tutte le sfere si rappresentano come \(x^{2} + y^{2} + z^{2} + 2ax + 2by + 2cz + d = 0\) con \(a^{2} + b^{2} + c^{2} > 0\) e avremo che \(C = (-a, -b, -c)\) e \( r = \sqrt{a^{2} + b^{2} + c^{2} - d}\)  }
\nt{Se \(a^{2} + b^{2} + c^{2} - d = 0 \implies x^{2} + y^{2} + z^{2} + 2ax + 2by + 2cz + d = 0\) è realizzata dal solo centro \(C = (-a, -b, -c)\). }

\mprop{}{Siano \(A, B, C, D\) quattro punti non complanari di \(\EE_{3}(\RR)\). Per \(A, B, C, D\) passa un'unica sfera}

Il centro della sfera si trova intersecando i piani assiali dei quattro punti. Il raggio è la distanza del centro da uno qualsiasi dei quattro punti.

\section{Circonferenze in \(\EE_{3}(\RR)\) }
\dfn{Circonferenza in \(\EE_{3}(\RR)\) }{Dati un piano \(\alpha \), un suo punto \(C\) e un numero reale positivo \(r\). Si dice \textbf{circonferenza} di raggio \(C\) e raggio \(r\) il luogo dei punti di \(\alpha \) aventi distanza \(r\) da \(C\).}

\nt{Una circonferenza appartiene a infinite sfere. Quindi per tre punti non allineati passano infinite sfere.}

\mprop{}{Tutte e sole le circonferenze di \(\EE_{3}(\RR)\) ammettono una rappresentazione del tipo \[
\begin{cases}
    ax + by + cz + d = 0 \quad \rightarrow \quad \text{piano \(\alpha \) } \\
    (x- x_0)^{2} + (y - y_0)^{2} + (z-z_0)^{2} = r'^{2} \\ 
\end{cases}
\] \[
    d(C', \alpha ) < r' \quad \text{dove} \quad C' = (x_0, y_0, z_0) \qquad 
    \frac{|ax_0 + by_0 + cz_0 + d| }{\sqrt{a^{2} + b^{2} + c^{2} }} < r' 
\] ci sono infinite rappresentazioni ma solo una con il centro \(C\) della circonferenza coincidente con il centro \(C'\) della sfera.}
Il centro della circonferenza \(C\) si trova intersecando il piano \(\alpha \) con la retta per il centro della sfera \(C'\) perpendicolarmente ad \(\alpha \).
Utilizziamo il teorema di Pitagora. Conosciamo \(\ol{CC'}\) e conosciamo anche il raggio \(r'\) della sfera. Quindi \[
    r = \sqrt{r'^{2} - \ol{CC'}^{2}}
\]
\nt{Una circonferenza si può ottenere anche intersecando altre superfici superfici con un piano.}

\ex{}{Si consideri \[
\begin{cases}
    x^{2} + y^{2} = 7 \\
    z = 3 \quad \rightarrow \quad \alpha \\
\end{cases} \quad \text{è una circonferenza?} 
\] \[
    x^{2} + y^{2} + z^{2} - z^{2} = 7 \quad  \text{ e siccome \(z = 3\)  } 
\begin{cases}
    x^{2} + y^{2} + z^{2} = 16 \\
    z = 3 \\
\end{cases} \quad \text{che descrive una circonferenza.}
\] Ed è una curva algebrica reale di \(\EE_{3}(\RR)\).}

\chapter{Ampliamento di \(\AA_{2}(\RR)\)}
In \(\AA_{2}(\RR)\) date due rette \(r\) e \(s\) o sono parallele o si intersecano in un punto. Se sono parallele \[
        r = [P,V_1] \ e \ s = [Q, W_1] \implies V_1 = W_1
    \]quindi la direzione \(V_1\) è l'elemento comune a tutte le rette parallele a \(r\). Poiché il parallelismo è una relazione di equivalenza tra le rette del piano. 

\section{Ampliamento (proiettivo) di \(\AA_{2}(\RR)\)}
\dfn{\(\~{\AA}_2(\RR)\) }{
    \begin{itemize}
    \item Punti propri che sono tutti e soli i punti di \(\AA_{2}(\RR)\)
    \item I punti impropri che sono le direzioni delle rette del piano ovvero i sottospazi di \(\RR^{2} \) di dimensione 1.
\end{itemize} Da cui il nome proprio/improprio per i fasci di rette del piano.}
\dfn{Rette}{\begin{itemize}
    \item le rette proprie: le rette di \(\AA_{2}(\RR)\) unite al loro punto improprio;
    \item una retta impropria: tutti i punti impropri (\(r_{\infty}\))
\end{itemize}}

Sia \(P_{\infty} \in r_{\infty} \implies \) non definiamo il vettore \(\vec{QP_{\infty} } \) con \(Q\) proprio/improprio.

La funzione \(f : A \times A \rightarrow V_{2}(\\RR) \) 
\textbf{da completare}

\mprop{}{Due rette distinte di \(\~{\AA}_{2}(\RR)\) sono sempre incidenti.}
\pf{Dimostrazione}{Siano \(r\) e \(s\) due rette distinte di \(\~{\AA}_{2}(\RR)\) \begin{enumerate}
    \item \(r\) e \(s\) sono proprie e non parallele tra loro \( \implies \) \(r\) è incidente a \(s\) in \(\AA_{2}(\RR)\) \(\subseteq \) \(\~{\AA}_{2}(\RR)\) e il punto improprio di \(r\) è diverso da quello di \(s\).
    \item \(r\) e \(s\) sono proprie ma \(r\) è parallelo a \(s\). \\
        \(r \cap s = \emptyset\) in \(\AA_{2}(\RR)\) ma \(r\) e \(s\) hanno la stessa direzione \( \implies \) lo stesso punto improprio.
    \item \(r\) è propria e \(s = r_{\infty} \). \\ \(r \cap s = r \cap r_{\infty} \) è il punto improprio di \(r\). 
\end{enumerate}}

\textbf{Completare con il quinto assioma e la spiegazione di a due tilda di r}
\nt{Tutte le rette proprie contengono un solo punto improprio e \(r_{\infty} \) contiene solo punti impropri.}

\mprop{}{Per due punti distinti di \(\~{\AA}_{2}(\RR)\) passa un'unica retta. }
\pf{Dimostrazione}{Siano \(A\) e \(B\) i due punti distinti considerati: \begin{enumerate}
    \item \(A, B\) sono entrambi propri \[
        \implies \exists ! r \in \AA_{2}(\RR) \text{  per  } A \ e \ B
    \] \[
        \exists ! r \quad \text{propria per} A \ e \ B
    \] la \(r_{\infty}\) non contiene \(A\) e \(B\) \( \implies \exists ! \ r\) per \(A\) e \(B\).
    \item \(A\) è proprio e \(B\) è improprio (o viceversa). \( \implies B\) è la direazione \(V_1\) \( \implies \exists !\) retta per \(\) \textbf{completare} 
    \item \(A\) e \(B\) sono entrambi impropri. Nessuna retta propria li contiene entrambi (ogni retta propria ha solo 1 punto improprio) \( \implies A, B \in r_{\infty} \) che è l'unica che li contiene entrambi.
\end{enumerate}}
\end{document}
