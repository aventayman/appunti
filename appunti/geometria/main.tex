\documentclass{report}

%%%%%%%%%%%%%%%%%%%%%%%%%%%%%%%%%
% PACKAGE IMPORTS
%%%%%%%%%%%%%%%%%%%%%%%%%%%%%%%%%


\usepackage[tmargin=2cm,rmargin=1in,lmargin=1in,margin=0.85in,bmargin=2cm,footskip=.2in]{geometry}
\usepackage{amsmath,amsfonts,amsthm,amssymb,mathtools}
\usepackage[varbb]{newpxmath}
\usepackage{xfrac}
\usepackage[makeroom]{cancel}
\usepackage{mathtools}
\usepackage{bookmark}
\usepackage{enumitem}
\usepackage{hyperref,theoremref}
\hypersetup{
	pdftitle={Assignment},
	colorlinks=true, linkcolor=doc!90,
	bookmarksnumbered=true,
	bookmarksopen=true
}
\usepackage{pgfplots}
\usepackage[most,many,breakable]{tcolorbox}
\usepackage{xcolor}
\usepackage{varwidth}
\usepackage{varwidth}
\usepackage{etoolbox}
%\usepackage{authblk}
\usepackage{nameref}
\usepackage{multicol,array}
\usepackage{tikz-cd}
\usepackage[ruled,vlined,linesnumbered]{algorithm2e}
\usepackage{comment} % enables the use of multi-line comments (\ifx \fi) 
\usepackage{import}
\usepackage{xifthen}
\usepackage{pdfpages}
\usepackage{transparent}
\usepackage{fancyhdr}

\usepackage[italian]{babel}

\newcommand\mycommfont[1]{\footnotesize\ttfamily\textcolor{blue}{#1}}
\SetCommentSty{mycommfont}
\newcommand{\incfig}[1]{%
    \def\svgwidth{\columnwidth}
    \import{./figures/}{#1.pdf_tex}
}

\usepackage{tikzsymbols}
\renewcommand\qedsymbol{$\Laughey$}

\usepackage{import}
\usepackage{pdfpages}
\usepackage{transparent}
\usepackage{xcolor}

\newcommand{\incfig}[2][1]{%
    \def\svgwidth{#1\columnwidth}
    \import{./figures/}{#2.pdf_tex}
}

\pdfsuppresswarningpagegroup=1
%\usepackage{import}
%\usepackage{xifthen}
%\usepackage{pdfpages}
%\usepackage{transparent}


%%%%%%%%%%%%%%%%%%%%%%%%%%%%%%
% SELF MADE COLORS
%%%%%%%%%%%%%%%%%%%%%%%%%%%%%%



\definecolor{myg}{RGB}{56, 140, 70}
\definecolor{myb}{RGB}{45, 111, 177}
\definecolor{myr}{RGB}{199, 68, 64}
\definecolor{mytheorembg}{HTML}{F2F2F9}
\definecolor{mytheoremfr}{HTML}{00007B}
\definecolor{mylenmabg}{HTML}{FFFAF8}
\definecolor{mylenmafr}{HTML}{983b0f}
\definecolor{mypropbg}{HTML}{f2fbfc}
\definecolor{mypropfr}{HTML}{191971}
\definecolor{myexamplebg}{HTML}{F2FBF8}
\definecolor{myexamplefr}{HTML}{88D6D1}
\definecolor{myexampleti}{HTML}{2A7F7F}
\definecolor{mydefinitbg}{HTML}{E5E5FF}
\definecolor{mydefinitfr}{HTML}{3F3FA3}
\definecolor{notesgreen}{RGB}{0,162,0}
\definecolor{myp}{RGB}{197, 92, 212}
\definecolor{mygr}{HTML}{2C3338}
\definecolor{myred}{RGB}{127,0,0}
\definecolor{myyellow}{RGB}{169,121,69}
\definecolor{myexercisebg}{HTML}{F2FBF8}
\definecolor{myexercisefg}{HTML}{88D6D1}


%%%%%%%%%%%%%%%%%%%%%%%%%%%%
% TCOLORBOX SETUPS
%%%%%%%%%%%%%%%%%%%%%%%%%%%%

\setlength{\parindent}{1cm}
%================================
% THEOREM BOX
%================================

\tcbuselibrary{theorems,skins,hooks}
\newtcbtheorem[number within=section]{Theorem}{Teorema}
{%
	enhanced,
	breakable,
	colback = mytheorembg,
	frame hidden,
	boxrule = 0sp,
	borderline west = {2pt}{0pt}{mytheoremfr},
	sharp corners,
	detach title,
	before upper = \tcbtitle\par\smallskip,
	coltitle = mytheoremfr,
	fonttitle = \bfseries\sffamily,
	description font = \mdseries,
	separator sign none,
	segmentation style={solid, mytheoremfr},
}
{th}

\tcbuselibrary{theorems,skins,hooks}
\newtcbtheorem[number within=chapter]{theorem}{Teorema}
{%
	enhanced,
	breakable,
	colback = mytheorembg,
	frame hidden,
	boxrule = 0sp,
	borderline west = {2pt}{0pt}{mytheoremfr},
	sharp corners,
	detach title,
	before upper = \tcbtitle\par\smallskip,
	coltitle = mytheoremfr,
	fonttitle = \bfseries\sffamily,
	description font = \mdseries,
	separator sign none,
	segmentation style={solid, mytheoremfr},
}
{th}


\tcbuselibrary{theorems,skins,hooks}
\newtcolorbox{Theoremcon}
{%
	enhanced
	,breakable
	,colback = mytheorembg
	,frame hidden
	,boxrule = 0sp
	,borderline west = {2pt}{0pt}{mytheoremfr}
	,sharp corners
	,description font = \mdseries
	,separator sign none
}

%================================
% Corollery
%================================
\tcbuselibrary{theorems,skins,hooks}
\newtcbtheorem[number within=section]{Corollary}{Corollario}
{%
	enhanced
	,breakable
	,colback = myp!10
	,frame hidden
	,boxrule = 0sp
	,borderline west = {2pt}{0pt}{myp!85!black}
	,sharp corners
	,detach title
	,before upper = \tcbtitle\par\smallskip
	,coltitle = myp!85!black
	,fonttitle = \bfseries\sffamily
	,description font = \mdseries
	,separator sign none
	,segmentation style={solid, myp!85!black}
}
{th}
\tcbuselibrary{theorems,skins,hooks}
\newtcbtheorem[number within=chapter]{corollary}{Corollario}
{%
	enhanced
	,breakable
	,colback = myp!10
	,frame hidden
	,boxrule = 0sp
	,borderline west = {2pt}{0pt}{myp!85!black}
	,sharp corners
	,detach title
	,before upper = \tcbtitle\par\smallskip
	,coltitle = myp!85!black
	,fonttitle = \bfseries\sffamily
	,description font = \mdseries
	,separator sign none
	,segmentation style={solid, myp!85!black}
}
{th}


%================================
% LENMA
%================================

\tcbuselibrary{theorems,skins,hooks}
\newtcbtheorem[number within=section]{Lenma}{Lemma}
{%
	enhanced,
	breakable,
	colback = mylenmabg,
	frame hidden,
	boxrule = 0sp,
	borderline west = {2pt}{0pt}{mylenmafr},
	sharp corners,
	detach title,
	before upper = \tcbtitle\par\smallskip,
	coltitle = mylenmafr,
	fonttitle = \bfseries\sffamily,
	description font = \mdseries,
	separator sign none,
	segmentation style={solid, mylenmafr},
}
{th}

\tcbuselibrary{theorems,skins,hooks}
\newtcbtheorem[number within=chapter]{lenma}{Lemma}
{%
	enhanced,
	breakable,
	colback = mylenmabg,
	frame hidden,
	boxrule = 0sp,
	borderline west = {2pt}{0pt}{mylenmafr},
	sharp corners,
	detach title,
	before upper = \tcbtitle\par\smallskip,
	coltitle = mylenmafr,
	fonttitle = \bfseries\sffamily,
	description font = \mdseries,
	separator sign none,
	segmentation style={solid, mylenmafr},
}
{th}


%================================
% PROPOSITION
%================================

\tcbuselibrary{theorems,skins,hooks}
\newtcbtheorem[number within=section]{Prop}{Proposizione}
{%
	enhanced,
	breakable,
	colback = mypropbg,
	frame hidden,
	boxrule = 0sp,
	borderline west = {2pt}{0pt}{mypropfr},
	sharp corners,
	detach title,
	before upper = \tcbtitle\par\smallskip,
	coltitle = mypropfr,
	fonttitle = \bfseries\sffamily,
	description font = \mdseries,
	separator sign none,
	segmentation style={solid, mypropfr},
}
{th}

\tcbuselibrary{theorems,skins,hooks}
\newtcbtheorem[number within=chapter]{prop}{Proposizione}
{%
	enhanced,
	breakable,
	colback = mypropbg,
	frame hidden,
	boxrule = 0sp,
	borderline west = {2pt}{0pt}{mypropfr},
	sharp corners,
	detach title,
	before upper = \tcbtitle\par\smallskip,
	coltitle = mypropfr,
	fonttitle = \bfseries\sffamily,
	description font = \mdseries,
	separator sign none,
	segmentation style={solid, mypropfr},
}
{th}


%================================
% CLAIM
%================================

\tcbuselibrary{theorems,skins,hooks}
\newtcbtheorem[number within=section]{claim}{Claim}
{%
	enhanced
	,breakable
	,colback = myg!10
	,frame hidden
	,boxrule = 0sp
	,borderline west = {2pt}{0pt}{myg}
	,sharp corners
	,detach title
	,before upper = \tcbtitle\par\smallskip
	,coltitle = myg!85!black
	,fonttitle = \bfseries\sffamily
	,description font = \mdseries
	,separator sign none
	,segmentation style={solid, myg!85!black}
}
{th}



%================================
% Exercise
%================================

\tcbuselibrary{theorems,skins,hooks}
\newtcbtheorem[number within=section]{Exercise}{Esercizio}
{%
	enhanced,
	breakable,
	colback = myexercisebg,
	frame hidden,
	boxrule = 0sp,
	borderline west = {2pt}{0pt}{myexercisefg},
	sharp corners,
	detach title,
	before upper = \tcbtitle\par\smallskip,
	coltitle = myexercisefg,
	fonttitle = \bfseries\sffamily,
	description font = \mdseries,
	separator sign none,
	segmentation style={solid, myexercisefg},
}
{th}

\tcbuselibrary{theorems,skins,hooks}
\newtcbtheorem[number within=chapter]{exercise}{Esercizio}
{%
	enhanced,
	breakable,
	colback = myexercisebg,
	frame hidden,
	boxrule = 0sp,
	borderline west = {2pt}{0pt}{myexercisefg},
	sharp corners,
	detach title,
	before upper = \tcbtitle\par\smallskip,
	coltitle = myexercisefg,
	fonttitle = \bfseries\sffamily,
	description font = \mdseries,
	separator sign none,
	segmentation style={solid, myexercisefg},
}
{th}

%================================
% EXAMPLE BOX
%================================

\newtcbtheorem[number within=section]{Example}{Esempio}
{%
	colback = myexamplebg
	,breakable
	,colframe = myexamplefr
	,coltitle = myexampleti
	,boxrule = 1pt
	,sharp corners
	,detach title
	,before upper=\tcbtitle\par\smallskip
	,fonttitle = \bfseries
	,description font = \mdseries
	,separator sign none
	,description delimiters parenthesis
}
{ex}

\newtcbtheorem[number within=chapter]{example}{Esempio}
{%
	colback = myexamplebg
	,breakable
	,colframe = myexamplefr
	,coltitle = myexampleti
	,boxrule = 1pt
	,sharp corners
	,detach title
	,before upper=\tcbtitle\par\smallskip
	,fonttitle = \bfseries
	,description font = \mdseries
	,separator sign none
	,description delimiters parenthesis
}
{ex}

%================================
% DEFINITION BOX
%================================

\newtcbtheorem[number within=section]{Definition}{Definizione}{enhanced,
	before skip=2mm,after skip=2mm, colback=red!5,colframe=red!80!black,boxrule=0.5mm,
	attach boxed title to top left={xshift=1cm,yshift*=1mm-\tcboxedtitleheight}, varwidth boxed title*=-3cm,
	boxed title style={frame code={
					\path[fill=tcbcolback]
					([yshift=-1mm,xshift=-1mm]frame.north west)
					arc[start angle=0,end angle=180,radius=1mm]
					([yshift=-1mm,xshift=1mm]frame.north east)
					arc[start angle=180,end angle=0,radius=1mm];
					\path[left color=tcbcolback!60!black,right color=tcbcolback!60!black,
						middle color=tcbcolback!80!black]
					([xshift=-2mm]frame.north west) -- ([xshift=2mm]frame.north east)
					[rounded corners=1mm]-- ([xshift=1mm,yshift=-1mm]frame.north east)
					-- (frame.south east) -- (frame.south west)
					-- ([xshift=-1mm,yshift=-1mm]frame.north west)
					[sharp corners]-- cycle;
				},interior engine=empty,
		},
	fonttitle=\bfseries,
	title={#2},#1}{def}
\newtcbtheorem[number within=chapter]{definition}{Definizione}{enhanced,
	before skip=2mm,after skip=2mm, colback=red!5,colframe=red!80!black,boxrule=0.5mm,
	attach boxed title to top left={xshift=1cm,yshift*=1mm-\tcboxedtitleheight}, varwidth boxed title*=-3cm,
	boxed title style={frame code={
					\path[fill=tcbcolback]
					([yshift=-1mm,xshift=-1mm]frame.north west)
					arc[start angle=0,end angle=180,radius=1mm]
					([yshift=-1mm,xshift=1mm]frame.north east)
					arc[start angle=180,end angle=0,radius=1mm];
					\path[left color=tcbcolback!60!black,right color=tcbcolback!60!black,
						middle color=tcbcolback!80!black]
					([xshift=-2mm]frame.north west) -- ([xshift=2mm]frame.north east)
					[rounded corners=1mm]-- ([xshift=1mm,yshift=-1mm]frame.north east)
					-- (frame.south east) -- (frame.south west)
					-- ([xshift=-1mm,yshift=-1mm]frame.north west)
					[sharp corners]-- cycle;
				},interior engine=empty,
		},
	fonttitle=\bfseries,
	title={#2},#1}{def}



%================================
% Solution BOX
%================================

\makeatletter
\newtcbtheorem{question}{Problema}{enhanced,
	breakable,
	colback=white,
	colframe=myb!80!black,
	attach boxed title to top left={yshift*=-\tcboxedtitleheight},
	fonttitle=\bfseries,
	title={#2},
	boxed title size=title,
	boxed title style={%
			sharp corners,
			rounded corners=northwest,
			colback=tcbcolframe,
			boxrule=0pt,
		},
	underlay boxed title={%
			\path[fill=tcbcolframe] (title.south west)--(title.south east)
			to[out=0, in=180] ([xshift=5mm]title.east)--
			(title.center-|frame.east)
			[rounded corners=\kvtcb@arc] |-
			(frame.north) -| cycle;
		},
	#1
}{def}
\makeatother

%================================
% SOLUTION BOX
%================================

\makeatletter
\newtcolorbox{solution}{enhanced,
	breakable,
	colback=white,
	colframe=myg!80!black,
	attach boxed title to top left={yshift*=-\tcboxedtitleheight},
	title=Solution,
	boxed title size=title,
	boxed title style={%
			sharp corners,
			rounded corners=northwest,
			colback=tcbcolframe,
			boxrule=0pt,
		},
	underlay boxed title={%
			\path[fill=tcbcolframe] (title.south west)--(title.south east)
			to[out=0, in=180] ([xshift=5mm]title.east)--
			(title.center-|frame.east)
			[rounded corners=\kvtcb@arc] |-
			(frame.north) -| cycle;
		},
}
\makeatother

%================================
% Question BOX
%================================

\makeatletter
\newtcbtheorem{qstion}{Question}{enhanced,
	breakable,
	colback=white,
	colframe=mygr,
	attach boxed title to top left={yshift*=-\tcboxedtitleheight},
	fonttitle=\bfseries,
	title={#2},
	boxed title size=title,
	boxed title style={%
			sharp corners,
			rounded corners=northwest,
			colback=tcbcolframe,
			boxrule=0pt,
		},
	underlay boxed title={%
			\path[fill=tcbcolframe] (title.south west)--(title.south east)
			to[out=0, in=180] ([xshift=5mm]title.east)--
			(title.center-|frame.east)
			[rounded corners=\kvtcb@arc] |-
			(frame.north) -| cycle;
		},
	#1
}{def}
\makeatother

\newtcbtheorem[number within=chapter]{wconc}{Wrong Concept}{
	breakable,
	enhanced,
	colback=white,
	colframe=myr,
	arc=0pt,
	outer arc=0pt,
	fonttitle=\bfseries\sffamily\large,
	colbacktitle=myr,
	attach boxed title to top left={},
	boxed title style={
			enhanced,
			skin=enhancedfirst jigsaw,
			arc=3pt,
			bottom=0pt,
			interior style={fill=myr}
		},
	#1
}{def}



%================================
% NOTE BOX
%================================


\usetikzlibrary{arrows,calc,shadows.blur}
\tcbuselibrary{skins}
\newtcolorbox{note}[1][]{%
	enhanced jigsaw,
	colback=gray!20!white,%
	colframe=gray!80!black,
	size=small,
	boxrule=1pt,
	title=\textbf{N.B.},
	halign title=flush center,
	coltitle=black,
	breakable,
	drop shadow=black!50!white,
	attach boxed title to top left={xshift=1cm,yshift=-\tcboxedtitleheight/2,yshifttext=-\tcboxedtitleheight/2},
	minipage boxed title=1.5cm,
	boxed title style={%
			colback=white,
			size=fbox,
			boxrule=1pt,
			boxsep=2pt,
			underlay={%
					\coordinate (dotA) at ($(interior.west) + (-0.5pt,0)$);
					\coordinate (dotB) at ($(interior.east) + (0.5pt,0)$);
					\begin{scope}
						\clip (interior.north west) rectangle ([xshift=3ex]interior.east);
						\filldraw [white, blur shadow={shadow opacity=60, shadow yshift=-.75ex}, rounded corners=2pt] (interior.north west) rectangle (interior.south east);
					\end{scope}
					\begin{scope}[gray!80!black]
						\fill (dotA) circle (2pt);
						\fill (dotB) circle (2pt);
					\end{scope}
				},
		},
	#1,
}

%%%%%%%%%%%%%%%%%%%%%%%%%%%%%%
% SELF MADE COMMANDS
%%%%%%%%%%%%%%%%%%%%%%%%%%%%%%


\newcommand{\thm}[2]{\begin{Theorem}{#1}{}#2\end{Theorem}}
\newcommand{\cor}[2]{\begin{Corollary}{#1}{}#2\end{Corollary}}
\newcommand{\mlenma}[2]{\begin{Lenma}{#1}{}#2\end{Lenma}}
\newcommand{\mprop}[2]{\begin{Prop}{#1}{}#2\end{Prop}}
\newcommand{\clm}[3]{\begin{claim}{#1}{#2}#3\end{claim}}
\newcommand{\wc}[2]{\begin{wconc}{#1}{}\setlength{\parindent}{1cm}#2\end{wconc}}
\newcommand{\thmcon}[1]{\begin{Theoremcon}{#1}\end{Theoremcon}}
\newcommand{\ex}[2]{\begin{Example}{#1}{}#2\end{Example}}
\newcommand{\dfn}[2]{\begin{Definition}[colbacktitle=red!75!black]{#1}{}#2\end{Definition}}
\newcommand{\dfnc}[2]{\begin{definition}[colbacktitle=red!75!black]{#1}{}#2\end{definition}}
\newcommand{\qs}[2]{\begin{question}{#1}{}#2\end{question}}
\newcommand{\pf}[2]{\begin{myproof}[#1]#2\end{myproof}}
\newcommand{\nt}[1]{\begin{note}#1\end{note}}

\newcommand*\circled[1]{\tikz[baseline=(char.base)]{
		\node[shape=circle,draw,inner sep=1pt] (char) {#1};}}
\newcommand\getcurrentref[1]{%
	\ifnumequal{\value{#1}}{0}
	{??}
	{\the\value{#1}}%
}
\newcommand{\getCurrentSectionNumber}{\getcurrentref{section}}
\newenvironment{myproof}[1][\proofname]{%
	\proof[\bfseries #1: ]%
}{\endproof}

\newcommand{\mclm}[2]{\begin{myclaim}[#1]#2\end{myclaim}}
\newenvironment{myclaim}[1][\claimname]{\proof[\bfseries #1: ]}{}

\newcounter{mylabelcounter}

\makeatletter
\newcommand{\setword}[2]{%
	\phantomsection
	#1\def\@currentlabel{\unexpanded{#1}}\label{#2}%
}
\makeatother




\tikzset{
	symbol/.style={
			draw=none,
			every to/.append style={
					edge node={node [sloped, allow upside down, auto=false]{$#1$}}}
		}
}


% deliminators
\DeclarePairedDelimiter{\abs}{\lvert}{\rvert}
\DeclarePairedDelimiter{\norm}{\lVert}{\rVert}

\DeclarePairedDelimiter{\ceil}{\lceil}{\rceil}
\DeclarePairedDelimiter{\floor}{\lfloor}{\rfloor}
\DeclarePairedDelimiter{\round}{\lfloor}{\rceil}

\newsavebox\diffdbox
\newcommand{\slantedromand}{{\mathpalette\makesl{d}}}
\newcommand{\makesl}[2]{%
\begingroup
\sbox{\diffdbox}{$\mathsurround=0pt#1\mathrm{#2}$}%
\pdfsave
\pdfsetmatrix{1 0 0.2 1}%
\rlap{\usebox{\diffdbox}}%
\pdfrestore
\hskip\wd\diffdbox
\endgroup
}
\newcommand{\dd}[1][]{\ensuremath{\mathop{}\!\ifstrempty{#1}{%
\slantedromand\@ifnextchar^{\hspace{0.2ex}}{\hspace{0.1ex}}}%
{\slantedromand\hspace{0.2ex}^{#1}}}}
\ProvideDocumentCommand\dv{o m g}{%
  \ensuremath{%
    \IfValueTF{#3}{%
      \IfNoValueTF{#1}{%
        \frac{\dd #2}{\dd #3}%
      }{%
        \frac{\dd^{#1} #2}{\dd #3^{#1}}%
      }%
    }{%
      \IfNoValueTF{#1}{%
        \frac{\dd}{\dd #2}%
      }{%
        \frac{\dd^{#1}}{\dd #2^{#1}}%
      }%
    }%
  }%
}
\providecommand*{\pdv}[3][]{\frac{\partial^{#1}#2}{\partial#3^{#1}}}
%  - others
\DeclareMathOperator{\Lap}{\mathcal{L}}
\DeclareMathOperator{\Var}{Var} % varience
\DeclareMathOperator{\Cov}{Cov} % covarience
\DeclareMathOperator{\E}{E} % expected

% Since the amsthm package isn't loaded

% I prefer the slanted \leq
\let\oldleq\leq % save them in case they're every wanted
\let\oldgeq\geq
\renewcommand{\leq}{\leqslant}
\renewcommand{\geq}{\geqslant}

% % redefine matrix env to allow for alignment, use r as default
% \renewcommand*\env@matrix[1][r]{\hskip -\arraycolsep
%     \let\@ifnextchar\new@ifnextchar
%     \array{*\c@MaxMatrixCols #1}}


%\usepackage{framed}
%\usepackage{titletoc}
%\usepackage{etoolbox}
%\usepackage{lmodern}


%\patchcmd{\tableofcontents}{\contentsname}{\sffamily\contentsname}{}{}

%\renewenvironment{leftbar}
%{\def\FrameCommand{\hspace{6em}%
%		{\color{myyellow}\vrule width 2pt depth 6pt}\hspace{1em}}%
%	\MakeFramed{\parshape 1 0cm \dimexpr\textwidth-6em\relax\FrameRestore}\vskip2pt%
%}
%{\endMakeFramed}

%\titlecontents{chapter}
%[0em]{\vspace*{2\baselineskip}}
%{\parbox{4.5em}{%
%		\hfill\Huge\sffamily\bfseries\color{myred}\thecontentspage}%
%	\vspace*{-2.3\baselineskip}\leftbar\textsc{\small\chaptername~\thecontentslabel}\\\sffamily}
%{}{\endleftbar}
%\titlecontents{section}
%[8.4em]
%{\sffamily\contentslabel{3em}}{}{}
%{\hspace{0.5em}\nobreak\itshape\color{myred}\contentspage}
%\titlecontents{subsection}
%[8.4em]
%{\sffamily\contentslabel{3em}}{}{}  
%{\hspace{0.5em}\nobreak\itshape\color{myred}\contentspage}



%%%%%%%%%%%%%%%%%%%%%%%%%%%%%%%%%%%%%%%%%%%
% TABLE OF CONTENTS
%%%%%%%%%%%%%%%%%%%%%%%%%%%%%%%%%%%%%%%%%%%

\usepackage{tikz}
\definecolor{doc}{RGB}{0,60,110}
\usepackage{titletoc}
\contentsmargin{0cm}
\titlecontents{chapter}[3.7pc]
{\addvspace{30pt}%
	\begin{tikzpicture}[remember picture, overlay]%
		\draw[fill=doc!60,draw=doc!60] (-7,-.1) rectangle (-0.9,.5);%
		\pgftext[left,x=-3.5cm,y=0.2cm]{\color{white}\Large\sc\bfseries Capitolo\ \thecontentslabel};%
	\end{tikzpicture}\color{doc!60}\large\sc\bfseries}%
{}
{}
{\;\titlerule\;\large\sc\bfseries Pagina \thecontentspage
	\begin{tikzpicture}[remember picture, overlay]
		\draw[fill=doc!60,draw=doc!60] (2pt,0) rectangle (4,0.1pt);
	\end{tikzpicture}}%
\titlecontents{section}[3.7pc]
{\addvspace{2pt}}
{\contentslabel[\thecontentslabel]{2pc}}
{}
{\hfill\small \thecontentspage}
[]
\titlecontents*{subsection}[3.7pc]
{\addvspace{-1pt}\small}
{}
{}
{\ --- \small\thecontentspage}
[ \textbullet\ ][]

\makeatletter
\renewcommand{\tableofcontents}{%
	\chapter*{%
	  \vspace*{-20\p@}%
	  \begin{tikzpicture}[remember picture, overlay]%
		  \pgftext[right,x=15cm,y=0.2cm]{\color{doc!60}\Huge\sc\bfseries \contentsname};%
		  \draw[fill=doc!60,draw=doc!60] (13,-.75) rectangle (20,1);%
		  \clip (13,-.75) rectangle (20,1);
		  \pgftext[right,x=15cm,y=0.2cm]{\color{white}\Huge\sc\bfseries \contentsname};%
	  \end{tikzpicture}}%
	\@starttoc{toc}}
\makeatother


%From M275 "Topology" at SJSU
\newcommand{\id}{\mathrm{id}}
\newcommand{\taking}[1]{\xrightarrow{#1}}
\newcommand{\inv}{^{-1}}

%From M170 "Introduction to Graph Theory" at SJSU
\DeclareMathOperator{\diam}{diam}
\DeclareMathOperator{\ord}{ord}
\newcommand{\defeq}{\overset{\mathrm{def}}{=}}

%From the USAMO .tex files
\newcommand{\ts}{\textsuperscript}
\newcommand{\dg}{^\circ}
\newcommand{\ii}{\item}

% % From Math 55 and Math 145 at Harvard
% \newenvironment{subproof}[1][Proof]{%
% \begin{proof}[#1] \renewcommand{\qedsymbol}{$\blacksquare$}}%
% {\end{proof}}

\newcommand{\liff}{\leftrightarrow}
\newcommand{\lthen}{\rightarrow}
\newcommand{\opname}{\operatorname}
\newcommand{\surjto}{\twoheadrightarrow}
\newcommand{\injto}{\hookrightarrow}
\newcommand{\On}{\mathrm{On}} % ordinals
\DeclareMathOperator{\img}{im} % Image
\DeclareMathOperator{\Img}{Im} % Image
\DeclareMathOperator{\coker}{coker} % Cokernel
\DeclareMathOperator{\Coker}{Coker} % Cokernel
\DeclareMathOperator{\Ker}{Ker} % Kernel
\DeclareMathOperator{\rank}{rank}
\DeclareMathOperator{\Spec}{Spec} % spectrum
\DeclareMathOperator{\Tr}{Tr} % trace
\DeclareMathOperator{\pr}{pr} % projection
\DeclareMathOperator{\ext}{ext} % extension
\DeclareMathOperator{\pred}{pred} % predecessor
\DeclareMathOperator{\dom}{dom} % domain
\DeclareMathOperator{\ran}{ran} % range
\DeclareMathOperator{\Hom}{Hom} % homomorphism
\DeclareMathOperator{\Mor}{Mor} % morphisms
\DeclareMathOperator{\End}{End} % endomorphism

\newcommand{\eps}{\epsilon}
\newcommand{\veps}{\varepsilon}
\newcommand{\ol}{\overline}
\newcommand{\ul}{\underline}
\newcommand{\wt}{\widetilde}
\newcommand{\wh}{\widehat}
\newcommand{\vocab}[1]{\textbf{\color{blue} #1}}
\providecommand{\half}{\frac{1}{2}}
\newcommand{\dang}{\measuredangle} %% Directed angle
\newcommand{\ray}[1]{\overrightarrow{#1}}
\newcommand{\seg}[1]{\overline{#1}}
\newcommand{\arc}[1]{\wideparen{#1}}
\DeclareMathOperator{\cis}{cis}
\DeclareMathOperator*{\lcm}{lcm}
\DeclareMathOperator*{\argmin}{arg min}
\DeclareMathOperator*{\argmax}{arg max}
\newcommand{\cycsum}{\sum_{\mathrm{cyc}}}
\newcommand{\symsum}{\sum_{\mathrm{sym}}}
\newcommand{\cycprod}{\prod_{\mathrm{cyc}}}
\newcommand{\symprod}{\prod_{\mathrm{sym}}}
\newcommand{\Qed}{\begin{flushright}\qed\end{flushright}}
\newcommand{\parinn}{\setlength{\parindent}{1cm}}
\newcommand{\parinf}{\setlength{\parindent}{0cm}}
% \newcommand{\norm}{\|\cdot\|}
\newcommand{\inorm}{\norm_{\infty}}
\newcommand{\opensets}{\{V_{\alpha}\}_{\alpha\in I}}
\newcommand{\oset}{V_{\alpha}}
\newcommand{\opset}[1]{V_{\alpha_{#1}}}
\newcommand{\lub}{\text{lub}}
\newcommand{\del}[2]{\frac{\partial #1}{\partial #2}}
\newcommand{\Del}[3]{\frac{\partial^{#1} #2}{\partial^{#1} #3}}
\newcommand{\deld}[2]{\dfrac{\partial #1}{\partial #2}}
\newcommand{\Deld}[3]{\dfrac{\partial^{#1} #2}{\partial^{#1} #3}}
\newcommand{\lm}{\lambda}
\newcommand{\uin}{\mathbin{\rotatebox[origin=c]{90}{$\in$}}}
\newcommand{\usubset}{\mathbin{\rotatebox[origin=c]{90}{$\subset$}}}
\newcommand{\lt}{\left}
\newcommand{\rt}{\right}
\newcommand{\bs}[1]{\boldsymbol{#1}}
\newcommand{\exs}{\exists}
\newcommand{\st}{\strut}
\newcommand{\dps}[1]{\displaystyle{#1}}

\newcommand{\sol}{\setlength{\parindent}{0cm}\textbf{\textit{Solution:}}\setlength{\parindent}{1cm} }
\newcommand{\solve}[1]{\setlength{\parindent}{0cm}\textbf{\textit{Solution: }}\setlength{\parindent}{1cm}#1 \Qed}

% Things Lie
\newcommand{\kb}{\mathfrak b}
\newcommand{\kg}{\mathfrak g}
\newcommand{\kh}{\mathfrak h}
\newcommand{\kn}{\mathfrak n}
\newcommand{\ku}{\mathfrak u}
\newcommand{\kz}{\mathfrak z}
\DeclareMathOperator{\Ext}{Ext} % Ext functor
\DeclareMathOperator{\Tor}{Tor} % Tor functor
\newcommand{\gl}{\opname{\mathfrak{gl}}} % frak gl group
\renewcommand{\sl}{\opname{\mathfrak{sl}}} % frak sl group chktex 6

% More script letters etc.
\newcommand{\SA}{\mathcal A}
\newcommand{\SB}{\mathcal B}
\newcommand{\SC}{\mathcal C}
\newcommand{\SF}{\mathcal F}
\newcommand{\SG}{\mathcal G}
\newcommand{\SH}{\mathcal H}
\newcommand{\OO}{\mathcal O}

\newcommand{\SCA}{\mathscr A}
\newcommand{\SCB}{\mathscr B}
\newcommand{\SCC}{\mathscr C}
\newcommand{\SCD}{\mathscr D}
\newcommand{\SCE}{\mathscr E}
\newcommand{\SCF}{\mathscr F}
\newcommand{\SCG}{\mathscr G}
\newcommand{\SCH}{\mathscr H}

% Mathfrak primes
\newcommand{\km}{\mathfrak m}
\newcommand{\kp}{\mathfrak p}
\newcommand{\kq}{\mathfrak q}

% number sets
\newcommand{\RR}[1][]{\ensuremath{\ifstrempty{#1}{\mathbb{R}}{\mathbb{R}^{#1}}}}
\newcommand{\NN}[1][]{\ensuremath{\ifstrempty{#1}{\mathbb{N}}{\mathbb{N}^{#1}}}}
\newcommand{\ZZ}[1][]{\ensuremath{\ifstrempty{#1}{\mathbb{Z}}{\mathbb{Z}^{#1}}}}
\newcommand{\QQ}[1][]{\ensuremath{\ifstrempty{#1}{\mathbb{Q}}{\mathbb{Q}^{#1}}}}
\newcommand{\CC}[1][]{\ensuremath{\ifstrempty{#1}{\mathbb{C}}{\mathbb{C}^{#1}}}}
\newcommand{\PP}[1][]{\ensuremath{\ifstrempty{#1}{\mathbb{P}}{\mathbb{P}^{#1}}}}
\newcommand{\HH}[1][]{\ensuremath{\ifstrempty{#1}{\mathbb{H}}{\mathbb{H}^{#1}}}}
\newcommand{\FF}[1][]{\ensuremath{\ifstrempty{#1}{\mathbb{F}}{\mathbb{F}^{#1}}}}
% expected value
\newcommand{\EE}{\ensuremath{\mathbb{E}}}
\newcommand{\charin}{\text{ char }}
\DeclareMathOperator{\sign}{sign}
\DeclareMathOperator{\Aut}{Aut}
\DeclareMathOperator{\Inn}{Inn}
\DeclareMathOperator{\Syl}{Syl}
\DeclareMathOperator{\Gal}{Gal}
\DeclareMathOperator{\GL}{GL} % General linear group
\DeclareMathOperator{\SL}{SL} % Special linear group

%---------------------------------------
% BlackBoard Math Fonts :-
%---------------------------------------

%Captital Letters
\newcommand{\bbA}{\mathbb{A}}	\newcommand{\bbB}{\mathbb{B}}
\newcommand{\bbC}{\mathbb{C}}	\newcommand{\bbD}{\mathbb{D}}
\newcommand{\bbE}{\mathbb{E}}	\newcommand{\bbF}{\mathbb{F}}
\newcommand{\bbG}{\mathbb{G}}	\newcommand{\bbH}{\mathbb{H}}
\newcommand{\bbI}{\mathbb{I}}	\newcommand{\bbJ}{\mathbb{J}}
\newcommand{\bbK}{\mathbb{K}}	\newcommand{\bbL}{\mathbb{L}}
\newcommand{\bbM}{\mathbb{M}}	\newcommand{\bbN}{\mathbb{N}}
\newcommand{\bbO}{\mathbb{O}}	\newcommand{\bbP}{\mathbb{P}}
\newcommand{\bbQ}{\mathbb{Q}}	\newcommand{\bbR}{\mathbb{R}}
\newcommand{\bbS}{\mathbb{S}}	\newcommand{\bbT}{\mathbb{T}}
\newcommand{\bbU}{\mathbb{U}}	\newcommand{\bbV}{\mathbb{V}}
\newcommand{\bbW}{\mathbb{W}}	\newcommand{\bbX}{\mathbb{X}}
\newcommand{\bbY}{\mathbb{Y}}	\newcommand{\bbZ}{\mathbb{Z}}

%---------------------------------------
% MathCal Fonts :-
%---------------------------------------

%Captital Letters
\newcommand{\mcA}{\mathcal{A}}	\newcommand{\mcB}{\mathcal{B}}
\newcommand{\mcC}{\mathcal{C}}	\newcommand{\mcD}{\mathcal{D}}
\newcommand{\mcE}{\mathcal{E}}	\newcommand{\mcF}{\mathcal{F}}
\newcommand{\mcG}{\mathcal{G}}	\newcommand{\mcH}{\mathcal{H}}
\newcommand{\mcI}{\mathcal{I}}	\newcommand{\mcJ}{\mathcal{J}}
\newcommand{\mcK}{\mathcal{K}}	\newcommand{\mcL}{\mathcal{L}}
\newcommand{\mcM}{\mathcal{M}}	\newcommand{\mcN}{\mathcal{N}}
\newcommand{\mcO}{\mathcal{O}}	\newcommand{\mcP}{\mathcal{P}}
\newcommand{\mcQ}{\mathcal{Q}}	\newcommand{\mcR}{\mathcal{R}}
\newcommand{\mcS}{\mathcal{S}}	\newcommand{\mcT}{\mathcal{T}}
\newcommand{\mcU}{\mathcal{U}}	\newcommand{\mcV}{\mathcal{V}}
\newcommand{\mcW}{\mathcal{W}}	\newcommand{\mcX}{\mathcal{X}}
\newcommand{\mcY}{\mathcal{Y}}	\newcommand{\mcZ}{\mathcal{Z}}


%---------------------------------------
% Bold Math Fonts :-
%---------------------------------------

%Captital Letters
\newcommand{\bmA}{\boldsymbol{A}}	\newcommand{\bmB}{\boldsymbol{B}}
\newcommand{\bmC}{\boldsymbol{C}}	\newcommand{\bmD}{\boldsymbol{D}}
\newcommand{\bmE}{\boldsymbol{E}}	\newcommand{\bmF}{\boldsymbol{F}}
\newcommand{\bmG}{\boldsymbol{G}}	\newcommand{\bmH}{\boldsymbol{H}}
\newcommand{\bmI}{\boldsymbol{I}}	\newcommand{\bmJ}{\boldsymbol{J}}
\newcommand{\bmK}{\boldsymbol{K}}	\newcommand{\bmL}{\boldsymbol{L}}
\newcommand{\bmM}{\boldsymbol{M}}	\newcommand{\bmN}{\boldsymbol{N}}
\newcommand{\bmO}{\boldsymbol{O}}	\newcommand{\bmP}{\boldsymbol{P}}
\newcommand{\bmQ}{\boldsymbol{Q}}	\newcommand{\bmR}{\boldsymbol{R}}
\newcommand{\bmS}{\boldsymbol{S}}	\newcommand{\bmT}{\boldsymbol{T}}
\newcommand{\bmU}{\boldsymbol{U}}	\newcommand{\bmV}{\boldsymbol{V}}
\newcommand{\bmW}{\boldsymbol{W}}	\newcommand{\bmX}{\boldsymbol{X}}
\newcommand{\bmY}{\boldsymbol{Y}}	\newcommand{\bmZ}{\boldsymbol{Z}}
%Small Letters
\newcommand{\bma}{\boldsymbol{a}}	\newcommand{\bmb}{\boldsymbol{b}}
\newcommand{\bmc}{\boldsymbol{c}}	\newcommand{\bmd}{\boldsymbol{d}}
\newcommand{\bme}{\boldsymbol{e}}	\newcommand{\bmf}{\boldsymbol{f}}
\newcommand{\bmg}{\boldsymbol{g}}	\newcommand{\bmh}{\boldsymbol{h}}
\newcommand{\bmi}{\boldsymbol{i}}	\newcommand{\bmj}{\boldsymbol{j}}
\newcommand{\bmk}{\boldsymbol{k}}	\newcommand{\bml}{\boldsymbol{l}}
\newcommand{\bmm}{\boldsymbol{m}}	\newcommand{\bmn}{\boldsymbol{n}}
\newcommand{\bmo}{\boldsymbol{o}}	\newcommand{\bmp}{\boldsymbol{p}}
\newcommand{\bmq}{\boldsymbol{q}}	\newcommand{\bmr}{\boldsymbol{r}}
\newcommand{\bms}{\boldsymbol{s}}	\newcommand{\bmt}{\boldsymbol{t}}
\newcommand{\bmu}{\boldsymbol{u}}	\newcommand{\bmv}{\boldsymbol{v}}
\newcommand{\bmw}{\boldsymbol{w}}	\newcommand{\bmx}{\boldsymbol{x}}
\newcommand{\bmy}{\boldsymbol{y}}	\newcommand{\bmz}{\boldsymbol{z}}

%---------------------------------------
% Scr Math Fonts :-
%---------------------------------------

\newcommand{\sA}{{\mathscr{A}}}   \newcommand{\sB}{{\mathscr{B}}}
\newcommand{\sC}{{\mathscr{C}}}   \newcommand{\sD}{{\mathscr{D}}}
\newcommand{\sE}{{\mathscr{E}}}   \newcommand{\sF}{{\mathscr{F}}}
\newcommand{\sG}{{\mathscr{G}}}   \newcommand{\sH}{{\mathscr{H}}}
\newcommand{\sI}{{\mathscr{I}}}   \newcommand{\sJ}{{\mathscr{J}}}
\newcommand{\sK}{{\mathscr{K}}}   \newcommand{\sL}{{\mathscr{L}}}
\newcommand{\sM}{{\mathscr{M}}}   \newcommand{\sN}{{\mathscr{N}}}
\newcommand{\sO}{{\mathscr{O}}}   \newcommand{\sP}{{\mathscr{P}}}
\newcommand{\sQ}{{\mathscr{Q}}}   \newcommand{\sR}{{\mathscr{R}}}
\newcommand{\sS}{{\mathscr{S}}}   \newcommand{\sT}{{\mathscr{T}}}
\newcommand{\sU}{{\mathscr{U}}}   \newcommand{\sV}{{\mathscr{V}}}
\newcommand{\sW}{{\mathscr{W}}}   \newcommand{\sX}{{\mathscr{X}}}
\newcommand{\sY}{{\mathscr{Y}}}   \newcommand{\sZ}{{\mathscr{Z}}}


%---------------------------------------
% Math Fraktur Font
%---------------------------------------

%Captital Letters
\newcommand{\mfA}{\mathfrak{A}}	\newcommand{\mfB}{\mathfrak{B}}
\newcommand{\mfC}{\mathfrak{C}}	\newcommand{\mfD}{\mathfrak{D}}
\newcommand{\mfE}{\mathfrak{E}}	\newcommand{\mfF}{\mathfrak{F}}
\newcommand{\mfG}{\mathfrak{G}}	\newcommand{\mfH}{\mathfrak{H}}
\newcommand{\mfI}{\mathfrak{I}}	\newcommand{\mfJ}{\mathfrak{J}}
\newcommand{\mfK}{\mathfrak{K}}	\newcommand{\mfL}{\mathfrak{L}}
\newcommand{\mfM}{\mathfrak{M}}	\newcommand{\mfN}{\mathfrak{N}}
\newcommand{\mfO}{\mathfrak{O}}	\newcommand{\mfP}{\mathfrak{P}}
\newcommand{\mfQ}{\mathfrak{Q}}	\newcommand{\mfR}{\mathfrak{R}}
\newcommand{\mfS}{\mathfrak{S}}	\newcommand{\mfT}{\mathfrak{T}}
\newcommand{\mfU}{\mathfrak{U}}	\newcommand{\mfV}{\mathfrak{V}}
\newcommand{\mfW}{\mathfrak{W}}	\newcommand{\mfX}{\mathfrak{X}}
\newcommand{\mfY}{\mathfrak{Y}}	\newcommand{\mfZ}{\mathfrak{Z}}
%Small Letters
\newcommand{\mfa}{\mathfrak{a}}	\newcommand{\mfb}{\mathfrak{b}}
\newcommand{\mfc}{\mathfrak{c}}	\newcommand{\mfd}{\mathfrak{d}}
\newcommand{\mfe}{\mathfrak{e}}	\newcommand{\mff}{\mathfrak{f}}
\newcommand{\mfg}{\mathfrak{g}}	\newcommand{\mfh}{\mathfrak{h}}
\newcommand{\mfi}{\mathfrak{i}}	\newcommand{\mfj}{\mathfrak{j}}
\newcommand{\mfk}{\mathfrak{k}}	\newcommand{\mfl}{\mathfrak{l}}
\newcommand{\mfm}{\mathfrak{m}}	\newcommand{\mfn}{\mathfrak{n}}
\newcommand{\mfo}{\mathfrak{o}}	\newcommand{\mfp}{\mathfrak{p}}
\newcommand{\mfq}{\mathfrak{q}}	\newcommand{\mfr}{\mathfrak{r}}
\newcommand{\mfs}{\mathfrak{s}}	\newcommand{\mft}{\mathfrak{t}}
\newcommand{\mfu}{\mathfrak{u}}	\newcommand{\mfv}{\mathfrak{v}}
\newcommand{\mfw}{\mathfrak{w}}	\newcommand{\mfx}{\mathfrak{x}}
\newcommand{\mfy}{\mathfrak{y}}	\newcommand{\mfz}{\mathfrak{z}}


\title{\Huge{Algebra Lineare e Geometria Analitica}\\Ingegneria dell'Automazione Industriale}
\author{\huge{Ayman Marpicati}}
\date{A.A. 2022/2023}

\begin{document}

\maketitle
\newpage% or \cleardoublepage
% \pdfbookmark[<level>]{<title>}{<dest>}
\pdfbookmark[section]{\contentsname}{toc}
\tableofcontents





Per determinare le coordinate del centro dobbiamo scegliere due punti \(X_{\infty} = [(1, 0, 0)]\), punto improprio dell'asse \(x\), e \(Y_{\infty} = [(0,1,0)]\), punto improprio dell'asse \(y\). La polare di \(X_{\infty}\) è \[
    (1,0,0)
\left( \; \begin{matrix}
    a_{11} & a_{12} & a_{13} \\
    a_{12} & a_{22} & a_{23} \\
    a_{13} & a_{23} & a_{33} \\
\end{matrix} \; \right)
\left( \; \begin{matrix} x_1 \\ x_2\\ x_3 \end{matrix} \; \right) = 0 \qquad (a_{11}, a_{12}, a_{13}) \left( \; \begin{matrix} x_1 \\ x_2\\ x_3 \end{matrix} \; \right) = a_{11}x_1+a_{12}x_2 + a_{13} x_3 = 0
\] Analogamente la polare di \(y_{\infty}\) è  \[
a_{12}x_1 + a_{22}x_2 + a_{23}x_3 = 0
\] \[
\begin{cases}
    \ a_{11}x_1+a_{12}x_2+a_{13}x_3 = 0 \qquad P_1 \\
    \ a_{12}x_1+a_{22}x_2+a_{23}x_3=0 \qquad P_2 \\
\end{cases}
\] 
Il centro \(C\) è proprio se \(P_1\) e \(P_2\) non sono paralleli. Se \[
\left| \; \begin{matrix}
    a_{11} & a_{12} \\
    a_{12} & a_{22} \\
\end{matrix} \; \right| = |A^{*}| \neq 0
\] Il centro è un punto proprio. Quindi il centro è un punto proprio se \(C\) è un ellisse o un'iperbole. Quindi in questo caso i diametri sono un fascio proprio di rette di centro \(C\). \[
F: \ \lambda (a_{11}x_1+a_{12}x_2+a_{13}x_3) + \mu (a_{12}x_1 + a_{22}x_2 + a_{23}x_3) = 0
\] \textbf{Equazione del fascio dei diametri.} Se \(C\) è una parabola \(\implies |A^{*}| = 0 \implies P_1 \) parallelo a \(P_2 \implies  \) il centro è un punto improprio. \(\implies \) i diametri formano un fascio improprio di equazione \[
a_{11}x_1+a_{12}x_2 + kx_3 = 0 \quad \text{con} \quad k \in \CC
\] fascio improprio dei diametri della parabola.

\section{Asintoti di una conica}
\dfn{Asintoti}{Si dicono \textbf{asintoti} di una conica le rette proprie tangenti alla conica nei suoi punti impropri.}
\paragraph{Osservazione:} Gli asintoti di una conica sono quindi le rette polari nei suoi punti impropri. Gli asintoti sono quindi dei diametri e passano per il centro. Se il centro è proprio (cioè se \(C\) è un'ellisse o un'iperbole) gli asintoti sono le rette che congiungono il centro con i punti impropri di \(C\).

\mprop{}{La parabola è una conica con centro improprio e priva di asintoti.}
\pf{Dimostrazione}{Sia \(C\) una parabola \(\implies C\) è tangente alla retta impropria in un punto che chiamiamo \(P_{\infty}\). Quindi la retta polare di \(P_{\infty}\) è \(r_{\infty} \implies \) il polo della \(r_{\infty}\) è \(P_{\infty} \implies \) il punto \(P_\infty\) è il centro della parabola. Osserviamo che \(C\) ha solo un punto improprio \(P_\infty \implies \) ammette solo una tangente nel suo punto improprio. Ma \(t\) è la \(r_\infty \implies \) la \(r_{\infty}\) non è un asintoto.}

\dfn{Coniche a centro}{Diremo che l'iperbole e l'ellisse sono coniche \textbf{a centro}, mentre la parabola è detta conica \textbf{non a centro}.}

\section{Proprietà metriche}
\dfn{Iperbole equilatera}{Un'iperbole si dice \textbf{equilatera} se i suoi asintoti sono ortogonali.}

\mprop{}{Una conica generale è un'iperbole equilatera se, e soltanto se, \(a_{11} + a_{22} = 0\).}

\ex{}{Si stabiliscano i valori di \(k \in \RR:\) \[
C: \ 2kx^2 + 2 (k-2) xy - 4 y ^2 + 2x + 1 = 0
\] sia un'iperbole equilatera. \\ 
\begin{enumerate}
    \item \(2k = -(-4) \to k = 2\)
    \item Sostituiamo dentro all'equazione e scriviamola in forma omogenea \[ 4x_1 ^2 + 0 x_1 x_2 - 4 x_2 ^2 + 2 x_1 x_3 + x_3 ^2 = 0 \quad 
        A = 
\left| \; \begin{matrix}
    4 & 0 & 1 \\
    0 & -4 & 0 \\
    1 & 0 & 1 \\
\end{matrix} \; \right| \neq 0
\] \(k = 2\) dà luogo ad un'iperbole equilatera.
\end{enumerate}}

\dfn{Ortogonale al punto improprio}{Diremo che la retta \(p\) di parametri direttori \([(l', m')]\) è ortogonale al punto improprio \(P: [(l,m,0)]\) se \(ll' + mm' = 0\).}

\dfn{Asse di una conica}{Si dice \textbf{asse} di una conica ogni diametro ortogonale al proprio polo.}

\dfn{Vertici}{Si dicono \textbf{vertici} le intersezioni proprie della conica con i propri assi.}

\section{Condizioni analitiche}
\mprop{}{Gli assi di una conica a centro (ellisse o iperbole) sono due e sono ortogonali tra loro, a meno che non si tratti di una circonferenza generalizzata, in tal caso tutti i diametri sono assi.}
\pf{Dimostrazione}{Per definizione i diametri sono le polari dei punti impropri. Dato \(P_\infty : [(l,m,0)]\) \[
\left( \; \begin{matrix}
    l & m & 0 \\
\end{matrix} \; \right)
\left( \; \begin{matrix}
    a_{11} & a_{12} & a_{13} \\
    a_{12} & a_{22} & a_{23} \\
    a_{13} & a_{23} & a_{33} \\
\end{matrix} \; \right)
\left( \; \begin{matrix} x_1 \\ x_2\\ x_3 \end{matrix} \; \right) = 0
\] Il generico diametro è: \[
\left( \; \begin{matrix}
    la_{11} + m a_{12} & l a_{12} + m a_{22} & l a_{13} + m a_{23} \\
\end{matrix} \; \right)
\left( \; \begin{matrix} x_1 \\ x_2\\ x_3 \end{matrix} \; \right) = 0
\] \[
(la_{11} + m a_{12}) x_1 + (la_{12} + ma_{22}) x_2 + (la_{13} + ma_{23}) x_3 = 0
\] \[
p.d.d : \ [(-la_{12} - m a_{22}, l a_{11} + ma_{12})]
\] Il polo di \(d\) è \(P_\infty: [(l,m,0)]\). \(d\) è un asse se è ortogonale a \(P_\infty\) ovvero se \[
l ( -la_{12}-ma_{22} ) + m(la_{11} + ma_{12}) = 0
\] \[
-l^2 a_{12} + ml (-a_{22}+a_{11}) + m^2 a_{12} = 0 \qquad l^2a_{12} + ml(a_{22}- a_{11}) - m^2 a_{12} = 0 
\] \[
a_{12} \left( \frac{l}{m} \right) ^2 + \frac{l}{m} (a_{22} - a_{11}) - a_{12} = 0
\] Se \(a_{12}=0\) e \(a_{22} = a_{11}\) l'equazione è risolta da tutte le coppie \((l,m)\). Quindi se la conica è una circonferenza generalizzata tutti i diametri sono assi. I due assi hanno polo \(P_\infty : [(l',m',0)]\) e \(Q_\infty : [(l '', m '', 0)]\). Sia \(p'\) l'asse associato al polo \(P_\infty\) e sia \(A_\infty\) il suo punto improprio. Sia \(a\) la retta che congiunge il centro al punto improprio \(rt(C, P_\infty)\), per ipotesi \(a \perp p'\). \(a\) contiene \(P_\infty\) che è il polo di \(p'\), quindi per il principio di reciprocità \(p'\) contiene il polo di \(a\). Il polo di \(a\) è improprio (perché \(a\) è diametro) \(\implies \) il punto improprio di \(a\) è \(A_\infty\), ma \(A_\infty \) è ortogonale alla direzione di \(a \implies a\) è un asse. Quindi i due assi sono ortogonali.}

\mprop{}{La parabola ha un unico asse e un solo vertice \(v\). Inoltre la tangente alla parabola in \(v\) è ortogonale all'asse.}

\pf{Dimostrazione}{Il punto \(P_\infty\) di una parabola è \([(-a_{12}, a_{11}, 0)]\). I \(p.d.d = [(-a_{12}, a_{11})]\). La direzione ortogonale è data da \([(a_{11}, a_{12})]\), quindi il punto \(P_\infty\) è \([(a_{11}, a_{12}, 0)]\). Da cui segue che l'asse è unico ed è la polare di \((a_{11}, a_{12}, 0)\). Sostituendo nell'equazione del fascio improprio dei diametri abbiamo che l'asse ha equazione: \[
a_{11}(a_{11}x_1+ a_{12}x_2 + a_{13}x_3) + a_{12}(a_{12}x_1 + a_{22}x_2 + a_{23}x_3) = 0
\] Per il teorema dell'ordine \(a\) interseca la parabola \(C\) in due punti, ma uno è \(P_\infty\) quindi l'altro punto sarà l'unico vertice della parabola. \\ Ora dimostriamo la seconda parte del teorema. \(v \in a\) che è il polo di \(t\). Per il principio di reciprocità \(t\) contiene il polo di \(a\), ovvero \(P_{\infty} \in t\). Ma \(P_\infty\) è ortogonale ad \(a \implies t \perp a\).}

\chapter{Ampliamento di \(A_3(\RR )\)}
Chiamiamo con \(\tilde{A}_{3}(\RR ) \) lo spazio reale affine ampliato. I punti possono essere
\begin{itemize}
    \item \textbf{propri} \(A\) dei punti di \(A_3(\RR )\) 
    \item \textbf{impropri} \(A_\infty\) direzioni delle rette, (spazi di traslazione di dimensione 1)
\end{itemize}
Le rette possono essere
\begin{itemize}
    \item \textbf{proprie} rette di \(A_3(\RR )\) ciascuna estesa con il suo punto improprio (ovvero la sua direzione)
    \item \textbf{improprie} sono le giaciture dei piani (spazi di traslazione di dimensione 2)
\end{itemize}
I piani possono essere
\begin{itemize}
    \item \textbf{propri} i piani di \(A_3(\RR )\) ciascuno esteso con la sua retta impropria (ovvero la sua giacitura)
    \item \textbf{piano improprio} \(A_\infty\) il luogo dei punti impropri
\end{itemize}

\mprop{}{
Diamo una serie di conseguenze senza dimostrazione
\begin{enumerate}
    \item due rette parallele hanno la stessa direzione e quindi hanno lo stesso punto improprio
    \item due piani paralleli hanno la stessa giacitura e quindi hanno la stessa retta impropria
    \item il piano improprio contiene tutte e sole le rette improprie
    \item ogni retta impropria contiene un solo punto improprio (la sua direzione)
    \item ogni piano proprio contiene \(\infty^{1}\) punti impropri, ovvero una retta (la sua giacitura).
\end{enumerate}}

\section{Geometria analitica in \(\tilde{A}_{3}(\RR) \)}
Indichiamo con \[ \frac{\RR ^{4} \backslash  \{(0,0,0,0)\}}{\rho } \] cioè l'insieme delle quaterne definite a meno di un fattore di proporzionalità reale e non nullo. In cui \(\rho \) indica la relazione di equivalenza data dalla proporzionalità. Quindi consideriamo due terne equivalenti se sono proporzionali.

\mprop{}{Sia \(RA = [O, B]\) un riferimento affine di \(A_{3}(\RR) \) e sia \[
\phi : A \cup A_\infty \quad  \to \quad \frac{\RR ^{4} \backslash  \{(0,0,0,0)\}}{\rho }
\] sia \(P \in A\) di coordinate \((x,y,z)\) \[
\phi(P) = [(x,y,z,1)]
\] sia \(P \in A_\infty\) corrispondente alla direzione \([(l,m,n)]\) \[
\phi (P) = [(l,m,n,0)]
\] la mappa \(\phi\) è una biiezione e le coordinate indotte da \(\phi\) sono chiamate \textbf{coordinate omogenee}.}

\ex{}{\[
        Q = [(2,0,3,-2)] \qquad -2 \neq 0 \implies Q \ \text{è proprio}
\] \[
Q = \left[ \left( \frac{2}{-2}, \frac{0}{-2}, - \frac{3}{2}, 1 \right)  \right] \implies Q = \left( -1, 0, - \frac{3}{2} \right) 
\] \[
P = [(2,1,0,0)] \implies [(2,1,0)]
\] }

\dfn{Rappresentazione dei piani}{
\[
ax_1 + bx_2 + cx_3 + dx_4 = 0 \quad \text{con} \quad (a,b,c,d) \neq (0,0,0,0)
\] questa è l'equazione omogenea dei piani in \(\tilde{A}_{3}(\RR) \) (e si ottiene in modo analogo all'equazione omogenea delle rette in \(\tilde{A}_{2}(\RR) \)).
}
\paragraph{Osservazione:}
\begin{enumerate}
    \item se \((a,b,c) \neq (0,0,0)\) allora il piano è proprio ed ha equazione affine \[
ax + by + cz + d = 0
\] 
    \item se \((a,b,c) = (0,0,0)\) allora \(d \neq 0\) e otteniamo \(x_4 = 0\) (che definisce il piano improprio).
\end{enumerate}

\dfn{Rappresentazione di rette}{Una retta è intersezione di 2 piani distinti \[
r:
\begin{cases}
    \ ax_1 + bx_2+ cx_3 + dx_4 = 0 \\
    \ a'x_1 + b'x_2+ c'x_3 + d'x_4 = 0 \\
\end{cases} \quad \text{con} \quad \rho
\left( \; \begin{matrix}
    a & b & c & d \\
    a' & b' & c' & d' \\
\end{matrix} \; \right) = 2
\] questa è la rappresentazione della generica retta di \(\tilde{A}_{3}(\RR)\)}
\begin{itemize}
    \item se \[
    \rho
\left( \; \begin{matrix}
    a & b & c \\
    a' & b' & c' \\
\end{matrix} \; \right) = 2     
    \] \(r\) è propria \[
\begin{cases}
    \ ax + by + cz + d = 0 \\
    \ a'x + b'y + c'z + d' = 0 \\
\end{cases}
    \]
\item se \[
    \rho
\left( \; \begin{matrix}
    a & b & c \\
    a' & b' & c' \\
\end{matrix} \; \right) = 1     
    \] ho due casi possibili
    \begin{itemize}
        \item i due piani sono paralleli e distinti
        \item uno dei due è il piano improprio e quindi \(x_4 = 0\)
    \end{itemize}
    in entrambi i casi \(r\) è impropria
\end{itemize} 

\section{Complessificazione di \(\tilde{A}_{3}(\CC)\)}
\(\tilde{A}_{3}(\CC) \) è lo spazio ampliato e complessificato. I suoi punti sono le quaterne di \[
\frac{\CC ^{4} \backslash  \{(0,0,0,0)\}}{\rho }
\] cioè le classi di proporzionalità delle quaterne complesse. La relazione di proporzionalità è chiaramente da intendersi in \(\CC\). All'interno dello spazio definiamo
 \begin{itemize}
    \item le \textbf{rette} sono i punti tali che \[
\begin{cases}
    \ ax_1+ bx_2 + cx_3+dx_4 =0 \\
    \ a'x_1 + b'x_2 + c'x_3 + d'x_4 = 0 \\
\end{cases} \quad \text{con}\quad a,a',b,b',c,c',d,d' \in C
    \] e tali che \[
    \rho
\left( \; \begin{matrix}
    a & b & c & d \\
    a' & b' & c' & d' \\
\end{matrix} \; \right) = 2
    \]
    \item un \textbf{piano} è costituito dai punti \[
    ax_1 + bx_2+cx_3+dx_4 = 0 \quad \text{con}\quad (a,b,c,d) \in \CC^{4} \backslash \{(0,0,0,0)\} 
    \] 
\end{itemize}

\dfn{Punti, rette e piani reali}{In \(\tilde{A}_{3}(\CC) \) i punti, le rette e i piani si dicono \textbf{reali} se ammettono almeno una rappresentazione con coefficienti reali. Si dicono immaginari altrimenti.}

\dfn{Rette immaginarie di prima e seconda specie}{In \(\tilde{A}_{3}(\CC) \) una retta \(r\) immaginaria è detta \textbf{immaginaria di prima specie} se è complanare con la propria coniugata \(\overline{r}\). \(r\) è detta  \textbf{immaginaria di seconda specie} se è sghemba con \(\overline{r}\).}

\mprop{}{
\begin{enumerate}
    \item La retta congiungente due punti immaginari e coniugati è reale
    \item se una retta (o un piano) reale contiene un punto \(P\) immaginario allora contiene anche \(\overline{P}\)
    \item se \(P\) è immaginario l'unica retta reale per \(P\) è \(rt(P, \overline{P})\)
    \item l'intersezione tra un piano \(\pi \) immaginario e \(\overline{\pi }\) è una retta reale
    \item un piano \(\pi \) immaginario contiene un'unica retta reale \(: \pi \cap \overline{\pi }\)
    \item se \(r\) è una retta immaginaria allora
        \begin{enumerate}
            \item \(r\) è contenuta in al più un piano reale
            \item \(r\) contiene al più un punto immaginario
        \end{enumerate}
        in particolare se \(r\) è immaginaria di prima specie il piano contenente \(r\) e \(\overline{r}\) è reale e \(r \cap \overline{r}\) è un punto reale. Se invece \(r\) è immaginaria di seconda specie allora \(r\) non è contenuta in alcuno piano reale e non contiene alcun punto reale.
\end{enumerate}
}
\dfn{Superfici algebriche reali in \(\tilde{A}_{3}(\CC) \)}{\textbf{Una superficie algebrica reale di \(\tilde{A}_{3}(\CC) \)} è l'insieme delle classi di autosoluzioni complesse di un'equazione del tipo \[
F(x_1, x_2, x_3, x_4) = 0 \quad \text{ove} \quad F \ \text{è un polinomio omogeneo a coefficienti reali in } x_1, x_2, x_3, x_4
\]  il grado di \(F\) è chiamato ordine della superficie. Se \(F\) è fattorizzabile in polinomi di grado positivo la superficie si dice riducibile in componenti \[
\text{fattori di \(F\)} \leftrightarrow \text{componenti della superficie}
\] }

\thm{Primo teorema dell'ordine}{L'ordine di una superficie algebrica \(\Sigma\) reale è uguale al numero di punti in comune a \(\Sigma\) e a una qualsiasi retta \(r\) non contenuta in \(\Sigma\) a patto di contarli con la dovuta molteplicità.}

\cor{}{\[\text{Se }|r \cap \Sigma| > \text{ord}(\Sigma) \implies r \subseteq \Sigma\].}

\thm{Secondo teorema dell'ordine}{L'intersezione tra una superficie algebrica reale \(\Sigma\) e un piano \(\alpha \) non componente di \(\Sigma\) è una curva dello stesso ordine di \(\Sigma\).}

\cor{}{Se \(\Sigma \cap \pi \) contiene una curva \(C\) con \(\text{ord}(C) > \text{ord}(\Sigma) \implies \pi \) è componente di \(\Sigma\).}

\dfn{}{In \(\tilde{A}_{3}(\CC) \), data una superficie algebrica reale \(\Sigma\), un punto \(P \in \Sigma\) è detto \textbf{r-uplo} se la generica retta per \(P\) ha molteplicità di intersezione con \(\Sigma\) in \(P\) uguale a \(r\).
\begin{itemize}
    \item se \(r = 1\) \(P\) è detto \textbf{semplice} 
    \item se \(r > 1\) \(P\) è detto \textbf{multiplo} 
\end{itemize}}

\thm{}{I punti multipli di una curva algebrica reale di equazione \(F(x_1, x_2, x_3, x_4)\) sono le classi di autosoluzioni del sistema \[
\begin{cases}
    \ \frac{\partial F}{\partial x_1} = 0 \\
    \ \frac{\partial F}{\partial x_2} = 0 \\
    \ \frac{\partial F}{\partial x_3} = 0 \\
    \ \frac{\partial F}{\partial x_4} = 0 \\
\end{cases}
\] }
\end{document}
