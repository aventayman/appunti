\documentclass{report}

\input{preamble}
\input{macros}
\input{letterfonts}

\title{\Huge{Algebra Lineare e Geometria Analitica}\\Ingegneria dell'Automazione Industriale}
\author{\huge{Ayman Marpicati}}
\date{A.A. 2022/2023}

\begin{document}

\maketitle
\newpage% or \cleardoublepage
% \pdfbookmark[<level>]{<title>}{<dest>}
\pdfbookmark[section]{\contentsname}{toc}
\tableofcontents


\section{Sezioni piane irriducibili di una quadrica}
Sia \(Q\) una quadrica irriducibile, \(\pi \) un piano e \(C = Q \cap \pi\), come possiamo riconoscere la conica che si ottiene? Per riconoscere \(C\) dobbiamo determinare i suoi punti impropri: \[
\{P_\infty, Q_\infty\} = C \cap \alpha _\infty = (Q \cap \pi ) \cap \alpha _\infty = (\underbrace{Q\cap \alpha _\infty}_{C_\infty} ) \cap \underbrace{(\pi \cap \alpha _\infty)}_{\text{retta impropria \\ di } \pi } = C_\infty \cap r_\infty
\]
\subsubsection{Coni e cilindri}
\paragraph{Osservazione:} Tutte le sezioni piane irriducibili di un cilindro sono:
\begin{itemize}
    \item se il cilindro è ellittico \(\to \) ellissi
    \item se il cilindro è parabolico \(\to \) parabole
    \item se il cilindro è iperbolico \(\to \) iperboli
\end{itemize}
Tutte le coniche generali sono ottenibili come sezione di un cono.

\subsubsection{Quadriche generali}

\subsubsection{Studio analitico}


\end{document}
