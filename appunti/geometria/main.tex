\documentclass{report}

%%%%%%%%%%%%%%%%%%%%%%%%%%%%%%%%%
% PACKAGE IMPORTS
%%%%%%%%%%%%%%%%%%%%%%%%%%%%%%%%%


\usepackage[tmargin=2cm,rmargin=1in,lmargin=1in,margin=0.85in,bmargin=2cm,footskip=.2in]{geometry}
\usepackage{amsmath,amsfonts,amsthm,amssymb,mathtools}
\usepackage[varbb]{newpxmath}
\usepackage{xfrac}
\usepackage[makeroom]{cancel}
\usepackage{mathtools}
\usepackage{bookmark}
\usepackage{enumitem}
\usepackage{hyperref,theoremref}
\hypersetup{
	pdftitle={Assignment},
	colorlinks=true, linkcolor=doc!90,
	bookmarksnumbered=true,
	bookmarksopen=true
}
\usepackage{pgfplots}
\usepackage[most,many,breakable]{tcolorbox}
\usepackage{xcolor}
\usepackage{varwidth}
\usepackage{varwidth}
\usepackage{etoolbox}
%\usepackage{authblk}
\usepackage{nameref}
\usepackage{multicol,array}
\usepackage{tikz-cd}
\usepackage[ruled,vlined,linesnumbered]{algorithm2e}
\usepackage{comment} % enables the use of multi-line comments (\ifx \fi) 
\usepackage{import}
\usepackage{xifthen}
\usepackage{pdfpages}
\usepackage{transparent}
\usepackage{fancyhdr}

\usepackage[italian]{babel}

\newcommand\mycommfont[1]{\footnotesize\ttfamily\textcolor{blue}{#1}}
\SetCommentSty{mycommfont}
\newcommand{\incfig}[1]{%
    \def\svgwidth{\columnwidth}
    \import{./figures/}{#1.pdf_tex}
}

\usepackage{tikzsymbols}
\renewcommand\qedsymbol{$\Laughey$}

\usepackage{import}
\usepackage{pdfpages}
\usepackage{transparent}
\usepackage{xcolor}

\newcommand{\incfig}[2][1]{%
    \def\svgwidth{#1\columnwidth}
    \import{./figures/}{#2.pdf_tex}
}

\pdfsuppresswarningpagegroup=1
%\usepackage{import}
%\usepackage{xifthen}
%\usepackage{pdfpages}
%\usepackage{transparent}


%%%%%%%%%%%%%%%%%%%%%%%%%%%%%%
% SELF MADE COLORS
%%%%%%%%%%%%%%%%%%%%%%%%%%%%%%



\definecolor{myg}{RGB}{56, 140, 70}
\definecolor{myb}{RGB}{45, 111, 177}
\definecolor{myr}{RGB}{199, 68, 64}
\definecolor{mytheorembg}{HTML}{F2F2F9}
\definecolor{mytheoremfr}{HTML}{00007B}
\definecolor{mylenmabg}{HTML}{FFFAF8}
\definecolor{mylenmafr}{HTML}{983b0f}
\definecolor{mypropbg}{HTML}{f2fbfc}
\definecolor{mypropfr}{HTML}{191971}
\definecolor{myexamplebg}{HTML}{F2FBF8}
\definecolor{myexamplefr}{HTML}{88D6D1}
\definecolor{myexampleti}{HTML}{2A7F7F}
\definecolor{mydefinitbg}{HTML}{E5E5FF}
\definecolor{mydefinitfr}{HTML}{3F3FA3}
\definecolor{notesgreen}{RGB}{0,162,0}
\definecolor{myp}{RGB}{197, 92, 212}
\definecolor{mygr}{HTML}{2C3338}
\definecolor{myred}{RGB}{127,0,0}
\definecolor{myyellow}{RGB}{169,121,69}
\definecolor{myexercisebg}{HTML}{F2FBF8}
\definecolor{myexercisefg}{HTML}{88D6D1}


%%%%%%%%%%%%%%%%%%%%%%%%%%%%
% TCOLORBOX SETUPS
%%%%%%%%%%%%%%%%%%%%%%%%%%%%

\setlength{\parindent}{1cm}
%================================
% THEOREM BOX
%================================

\tcbuselibrary{theorems,skins,hooks}
\newtcbtheorem[number within=section]{Theorem}{Teorema}
{%
	enhanced,
	breakable,
	colback = mytheorembg,
	frame hidden,
	boxrule = 0sp,
	borderline west = {2pt}{0pt}{mytheoremfr},
	sharp corners,
	detach title,
	before upper = \tcbtitle\par\smallskip,
	coltitle = mytheoremfr,
	fonttitle = \bfseries\sffamily,
	description font = \mdseries,
	separator sign none,
	segmentation style={solid, mytheoremfr},
}
{th}

\tcbuselibrary{theorems,skins,hooks}
\newtcbtheorem[number within=chapter]{theorem}{Teorema}
{%
	enhanced,
	breakable,
	colback = mytheorembg,
	frame hidden,
	boxrule = 0sp,
	borderline west = {2pt}{0pt}{mytheoremfr},
	sharp corners,
	detach title,
	before upper = \tcbtitle\par\smallskip,
	coltitle = mytheoremfr,
	fonttitle = \bfseries\sffamily,
	description font = \mdseries,
	separator sign none,
	segmentation style={solid, mytheoremfr},
}
{th}


\tcbuselibrary{theorems,skins,hooks}
\newtcolorbox{Theoremcon}
{%
	enhanced
	,breakable
	,colback = mytheorembg
	,frame hidden
	,boxrule = 0sp
	,borderline west = {2pt}{0pt}{mytheoremfr}
	,sharp corners
	,description font = \mdseries
	,separator sign none
}

%================================
% Corollery
%================================
\tcbuselibrary{theorems,skins,hooks}
\newtcbtheorem[number within=section]{Corollary}{Corollario}
{%
	enhanced
	,breakable
	,colback = myp!10
	,frame hidden
	,boxrule = 0sp
	,borderline west = {2pt}{0pt}{myp!85!black}
	,sharp corners
	,detach title
	,before upper = \tcbtitle\par\smallskip
	,coltitle = myp!85!black
	,fonttitle = \bfseries\sffamily
	,description font = \mdseries
	,separator sign none
	,segmentation style={solid, myp!85!black}
}
{th}
\tcbuselibrary{theorems,skins,hooks}
\newtcbtheorem[number within=chapter]{corollary}{Corollario}
{%
	enhanced
	,breakable
	,colback = myp!10
	,frame hidden
	,boxrule = 0sp
	,borderline west = {2pt}{0pt}{myp!85!black}
	,sharp corners
	,detach title
	,before upper = \tcbtitle\par\smallskip
	,coltitle = myp!85!black
	,fonttitle = \bfseries\sffamily
	,description font = \mdseries
	,separator sign none
	,segmentation style={solid, myp!85!black}
}
{th}


%================================
% LENMA
%================================

\tcbuselibrary{theorems,skins,hooks}
\newtcbtheorem[number within=section]{Lenma}{Lemma}
{%
	enhanced,
	breakable,
	colback = mylenmabg,
	frame hidden,
	boxrule = 0sp,
	borderline west = {2pt}{0pt}{mylenmafr},
	sharp corners,
	detach title,
	before upper = \tcbtitle\par\smallskip,
	coltitle = mylenmafr,
	fonttitle = \bfseries\sffamily,
	description font = \mdseries,
	separator sign none,
	segmentation style={solid, mylenmafr},
}
{th}

\tcbuselibrary{theorems,skins,hooks}
\newtcbtheorem[number within=chapter]{lenma}{Lemma}
{%
	enhanced,
	breakable,
	colback = mylenmabg,
	frame hidden,
	boxrule = 0sp,
	borderline west = {2pt}{0pt}{mylenmafr},
	sharp corners,
	detach title,
	before upper = \tcbtitle\par\smallskip,
	coltitle = mylenmafr,
	fonttitle = \bfseries\sffamily,
	description font = \mdseries,
	separator sign none,
	segmentation style={solid, mylenmafr},
}
{th}


%================================
% PROPOSITION
%================================

\tcbuselibrary{theorems,skins,hooks}
\newtcbtheorem[number within=section]{Prop}{Proposizione}
{%
	enhanced,
	breakable,
	colback = mypropbg,
	frame hidden,
	boxrule = 0sp,
	borderline west = {2pt}{0pt}{mypropfr},
	sharp corners,
	detach title,
	before upper = \tcbtitle\par\smallskip,
	coltitle = mypropfr,
	fonttitle = \bfseries\sffamily,
	description font = \mdseries,
	separator sign none,
	segmentation style={solid, mypropfr},
}
{th}

\tcbuselibrary{theorems,skins,hooks}
\newtcbtheorem[number within=chapter]{prop}{Proposizione}
{%
	enhanced,
	breakable,
	colback = mypropbg,
	frame hidden,
	boxrule = 0sp,
	borderline west = {2pt}{0pt}{mypropfr},
	sharp corners,
	detach title,
	before upper = \tcbtitle\par\smallskip,
	coltitle = mypropfr,
	fonttitle = \bfseries\sffamily,
	description font = \mdseries,
	separator sign none,
	segmentation style={solid, mypropfr},
}
{th}


%================================
% CLAIM
%================================

\tcbuselibrary{theorems,skins,hooks}
\newtcbtheorem[number within=section]{claim}{Claim}
{%
	enhanced
	,breakable
	,colback = myg!10
	,frame hidden
	,boxrule = 0sp
	,borderline west = {2pt}{0pt}{myg}
	,sharp corners
	,detach title
	,before upper = \tcbtitle\par\smallskip
	,coltitle = myg!85!black
	,fonttitle = \bfseries\sffamily
	,description font = \mdseries
	,separator sign none
	,segmentation style={solid, myg!85!black}
}
{th}



%================================
% Exercise
%================================

\tcbuselibrary{theorems,skins,hooks}
\newtcbtheorem[number within=section]{Exercise}{Esercizio}
{%
	enhanced,
	breakable,
	colback = myexercisebg,
	frame hidden,
	boxrule = 0sp,
	borderline west = {2pt}{0pt}{myexercisefg},
	sharp corners,
	detach title,
	before upper = \tcbtitle\par\smallskip,
	coltitle = myexercisefg,
	fonttitle = \bfseries\sffamily,
	description font = \mdseries,
	separator sign none,
	segmentation style={solid, myexercisefg},
}
{th}

\tcbuselibrary{theorems,skins,hooks}
\newtcbtheorem[number within=chapter]{exercise}{Esercizio}
{%
	enhanced,
	breakable,
	colback = myexercisebg,
	frame hidden,
	boxrule = 0sp,
	borderline west = {2pt}{0pt}{myexercisefg},
	sharp corners,
	detach title,
	before upper = \tcbtitle\par\smallskip,
	coltitle = myexercisefg,
	fonttitle = \bfseries\sffamily,
	description font = \mdseries,
	separator sign none,
	segmentation style={solid, myexercisefg},
}
{th}

%================================
% EXAMPLE BOX
%================================

\newtcbtheorem[number within=section]{Example}{Esempio}
{%
	colback = myexamplebg
	,breakable
	,colframe = myexamplefr
	,coltitle = myexampleti
	,boxrule = 1pt
	,sharp corners
	,detach title
	,before upper=\tcbtitle\par\smallskip
	,fonttitle = \bfseries
	,description font = \mdseries
	,separator sign none
	,description delimiters parenthesis
}
{ex}

\newtcbtheorem[number within=chapter]{example}{Esempio}
{%
	colback = myexamplebg
	,breakable
	,colframe = myexamplefr
	,coltitle = myexampleti
	,boxrule = 1pt
	,sharp corners
	,detach title
	,before upper=\tcbtitle\par\smallskip
	,fonttitle = \bfseries
	,description font = \mdseries
	,separator sign none
	,description delimiters parenthesis
}
{ex}

%================================
% DEFINITION BOX
%================================

\newtcbtheorem[number within=section]{Definition}{Definizione}{enhanced,
	before skip=2mm,after skip=2mm, colback=red!5,colframe=red!80!black,boxrule=0.5mm,
	attach boxed title to top left={xshift=1cm,yshift*=1mm-\tcboxedtitleheight}, varwidth boxed title*=-3cm,
	boxed title style={frame code={
					\path[fill=tcbcolback]
					([yshift=-1mm,xshift=-1mm]frame.north west)
					arc[start angle=0,end angle=180,radius=1mm]
					([yshift=-1mm,xshift=1mm]frame.north east)
					arc[start angle=180,end angle=0,radius=1mm];
					\path[left color=tcbcolback!60!black,right color=tcbcolback!60!black,
						middle color=tcbcolback!80!black]
					([xshift=-2mm]frame.north west) -- ([xshift=2mm]frame.north east)
					[rounded corners=1mm]-- ([xshift=1mm,yshift=-1mm]frame.north east)
					-- (frame.south east) -- (frame.south west)
					-- ([xshift=-1mm,yshift=-1mm]frame.north west)
					[sharp corners]-- cycle;
				},interior engine=empty,
		},
	fonttitle=\bfseries,
	title={#2},#1}{def}
\newtcbtheorem[number within=chapter]{definition}{Definizione}{enhanced,
	before skip=2mm,after skip=2mm, colback=red!5,colframe=red!80!black,boxrule=0.5mm,
	attach boxed title to top left={xshift=1cm,yshift*=1mm-\tcboxedtitleheight}, varwidth boxed title*=-3cm,
	boxed title style={frame code={
					\path[fill=tcbcolback]
					([yshift=-1mm,xshift=-1mm]frame.north west)
					arc[start angle=0,end angle=180,radius=1mm]
					([yshift=-1mm,xshift=1mm]frame.north east)
					arc[start angle=180,end angle=0,radius=1mm];
					\path[left color=tcbcolback!60!black,right color=tcbcolback!60!black,
						middle color=tcbcolback!80!black]
					([xshift=-2mm]frame.north west) -- ([xshift=2mm]frame.north east)
					[rounded corners=1mm]-- ([xshift=1mm,yshift=-1mm]frame.north east)
					-- (frame.south east) -- (frame.south west)
					-- ([xshift=-1mm,yshift=-1mm]frame.north west)
					[sharp corners]-- cycle;
				},interior engine=empty,
		},
	fonttitle=\bfseries,
	title={#2},#1}{def}



%================================
% Solution BOX
%================================

\makeatletter
\newtcbtheorem{question}{Problema}{enhanced,
	breakable,
	colback=white,
	colframe=myb!80!black,
	attach boxed title to top left={yshift*=-\tcboxedtitleheight},
	fonttitle=\bfseries,
	title={#2},
	boxed title size=title,
	boxed title style={%
			sharp corners,
			rounded corners=northwest,
			colback=tcbcolframe,
			boxrule=0pt,
		},
	underlay boxed title={%
			\path[fill=tcbcolframe] (title.south west)--(title.south east)
			to[out=0, in=180] ([xshift=5mm]title.east)--
			(title.center-|frame.east)
			[rounded corners=\kvtcb@arc] |-
			(frame.north) -| cycle;
		},
	#1
}{def}
\makeatother

%================================
% SOLUTION BOX
%================================

\makeatletter
\newtcolorbox{solution}{enhanced,
	breakable,
	colback=white,
	colframe=myg!80!black,
	attach boxed title to top left={yshift*=-\tcboxedtitleheight},
	title=Solution,
	boxed title size=title,
	boxed title style={%
			sharp corners,
			rounded corners=northwest,
			colback=tcbcolframe,
			boxrule=0pt,
		},
	underlay boxed title={%
			\path[fill=tcbcolframe] (title.south west)--(title.south east)
			to[out=0, in=180] ([xshift=5mm]title.east)--
			(title.center-|frame.east)
			[rounded corners=\kvtcb@arc] |-
			(frame.north) -| cycle;
		},
}
\makeatother

%================================
% Question BOX
%================================

\makeatletter
\newtcbtheorem{qstion}{Question}{enhanced,
	breakable,
	colback=white,
	colframe=mygr,
	attach boxed title to top left={yshift*=-\tcboxedtitleheight},
	fonttitle=\bfseries,
	title={#2},
	boxed title size=title,
	boxed title style={%
			sharp corners,
			rounded corners=northwest,
			colback=tcbcolframe,
			boxrule=0pt,
		},
	underlay boxed title={%
			\path[fill=tcbcolframe] (title.south west)--(title.south east)
			to[out=0, in=180] ([xshift=5mm]title.east)--
			(title.center-|frame.east)
			[rounded corners=\kvtcb@arc] |-
			(frame.north) -| cycle;
		},
	#1
}{def}
\makeatother

\newtcbtheorem[number within=chapter]{wconc}{Wrong Concept}{
	breakable,
	enhanced,
	colback=white,
	colframe=myr,
	arc=0pt,
	outer arc=0pt,
	fonttitle=\bfseries\sffamily\large,
	colbacktitle=myr,
	attach boxed title to top left={},
	boxed title style={
			enhanced,
			skin=enhancedfirst jigsaw,
			arc=3pt,
			bottom=0pt,
			interior style={fill=myr}
		},
	#1
}{def}



%================================
% NOTE BOX
%================================


\usetikzlibrary{arrows,calc,shadows.blur}
\tcbuselibrary{skins}
\newtcolorbox{note}[1][]{%
	enhanced jigsaw,
	colback=gray!20!white,%
	colframe=gray!80!black,
	size=small,
	boxrule=1pt,
	title=\textbf{N.B.},
	halign title=flush center,
	coltitle=black,
	breakable,
	drop shadow=black!50!white,
	attach boxed title to top left={xshift=1cm,yshift=-\tcboxedtitleheight/2,yshifttext=-\tcboxedtitleheight/2},
	minipage boxed title=1.5cm,
	boxed title style={%
			colback=white,
			size=fbox,
			boxrule=1pt,
			boxsep=2pt,
			underlay={%
					\coordinate (dotA) at ($(interior.west) + (-0.5pt,0)$);
					\coordinate (dotB) at ($(interior.east) + (0.5pt,0)$);
					\begin{scope}
						\clip (interior.north west) rectangle ([xshift=3ex]interior.east);
						\filldraw [white, blur shadow={shadow opacity=60, shadow yshift=-.75ex}, rounded corners=2pt] (interior.north west) rectangle (interior.south east);
					\end{scope}
					\begin{scope}[gray!80!black]
						\fill (dotA) circle (2pt);
						\fill (dotB) circle (2pt);
					\end{scope}
				},
		},
	#1,
}

%%%%%%%%%%%%%%%%%%%%%%%%%%%%%%
% SELF MADE COMMANDS
%%%%%%%%%%%%%%%%%%%%%%%%%%%%%%


\newcommand{\thm}[2]{\begin{Theorem}{#1}{}#2\end{Theorem}}
\newcommand{\cor}[2]{\begin{Corollary}{#1}{}#2\end{Corollary}}
\newcommand{\mlenma}[2]{\begin{Lenma}{#1}{}#2\end{Lenma}}
\newcommand{\mprop}[2]{\begin{Prop}{#1}{}#2\end{Prop}}
\newcommand{\clm}[3]{\begin{claim}{#1}{#2}#3\end{claim}}
\newcommand{\wc}[2]{\begin{wconc}{#1}{}\setlength{\parindent}{1cm}#2\end{wconc}}
\newcommand{\thmcon}[1]{\begin{Theoremcon}{#1}\end{Theoremcon}}
\newcommand{\ex}[2]{\begin{Example}{#1}{}#2\end{Example}}
\newcommand{\dfn}[2]{\begin{Definition}[colbacktitle=red!75!black]{#1}{}#2\end{Definition}}
\newcommand{\dfnc}[2]{\begin{definition}[colbacktitle=red!75!black]{#1}{}#2\end{definition}}
\newcommand{\qs}[2]{\begin{question}{#1}{}#2\end{question}}
\newcommand{\pf}[2]{\begin{myproof}[#1]#2\end{myproof}}
\newcommand{\nt}[1]{\begin{note}#1\end{note}}

\newcommand*\circled[1]{\tikz[baseline=(char.base)]{
		\node[shape=circle,draw,inner sep=1pt] (char) {#1};}}
\newcommand\getcurrentref[1]{%
	\ifnumequal{\value{#1}}{0}
	{??}
	{\the\value{#1}}%
}
\newcommand{\getCurrentSectionNumber}{\getcurrentref{section}}
\newenvironment{myproof}[1][\proofname]{%
	\proof[\bfseries #1: ]%
}{\endproof}

\newcommand{\mclm}[2]{\begin{myclaim}[#1]#2\end{myclaim}}
\newenvironment{myclaim}[1][\claimname]{\proof[\bfseries #1: ]}{}

\newcounter{mylabelcounter}

\makeatletter
\newcommand{\setword}[2]{%
	\phantomsection
	#1\def\@currentlabel{\unexpanded{#1}}\label{#2}%
}
\makeatother




\tikzset{
	symbol/.style={
			draw=none,
			every to/.append style={
					edge node={node [sloped, allow upside down, auto=false]{$#1$}}}
		}
}


% deliminators
\DeclarePairedDelimiter{\abs}{\lvert}{\rvert}
\DeclarePairedDelimiter{\norm}{\lVert}{\rVert}

\DeclarePairedDelimiter{\ceil}{\lceil}{\rceil}
\DeclarePairedDelimiter{\floor}{\lfloor}{\rfloor}
\DeclarePairedDelimiter{\round}{\lfloor}{\rceil}

\newsavebox\diffdbox
\newcommand{\slantedromand}{{\mathpalette\makesl{d}}}
\newcommand{\makesl}[2]{%
\begingroup
\sbox{\diffdbox}{$\mathsurround=0pt#1\mathrm{#2}$}%
\pdfsave
\pdfsetmatrix{1 0 0.2 1}%
\rlap{\usebox{\diffdbox}}%
\pdfrestore
\hskip\wd\diffdbox
\endgroup
}
\newcommand{\dd}[1][]{\ensuremath{\mathop{}\!\ifstrempty{#1}{%
\slantedromand\@ifnextchar^{\hspace{0.2ex}}{\hspace{0.1ex}}}%
{\slantedromand\hspace{0.2ex}^{#1}}}}
\ProvideDocumentCommand\dv{o m g}{%
  \ensuremath{%
    \IfValueTF{#3}{%
      \IfNoValueTF{#1}{%
        \frac{\dd #2}{\dd #3}%
      }{%
        \frac{\dd^{#1} #2}{\dd #3^{#1}}%
      }%
    }{%
      \IfNoValueTF{#1}{%
        \frac{\dd}{\dd #2}%
      }{%
        \frac{\dd^{#1}}{\dd #2^{#1}}%
      }%
    }%
  }%
}
\providecommand*{\pdv}[3][]{\frac{\partial^{#1}#2}{\partial#3^{#1}}}
%  - others
\DeclareMathOperator{\Lap}{\mathcal{L}}
\DeclareMathOperator{\Var}{Var} % varience
\DeclareMathOperator{\Cov}{Cov} % covarience
\DeclareMathOperator{\E}{E} % expected

% Since the amsthm package isn't loaded

% I prefer the slanted \leq
\let\oldleq\leq % save them in case they're every wanted
\let\oldgeq\geq
\renewcommand{\leq}{\leqslant}
\renewcommand{\geq}{\geqslant}

% % redefine matrix env to allow for alignment, use r as default
% \renewcommand*\env@matrix[1][r]{\hskip -\arraycolsep
%     \let\@ifnextchar\new@ifnextchar
%     \array{*\c@MaxMatrixCols #1}}


%\usepackage{framed}
%\usepackage{titletoc}
%\usepackage{etoolbox}
%\usepackage{lmodern}


%\patchcmd{\tableofcontents}{\contentsname}{\sffamily\contentsname}{}{}

%\renewenvironment{leftbar}
%{\def\FrameCommand{\hspace{6em}%
%		{\color{myyellow}\vrule width 2pt depth 6pt}\hspace{1em}}%
%	\MakeFramed{\parshape 1 0cm \dimexpr\textwidth-6em\relax\FrameRestore}\vskip2pt%
%}
%{\endMakeFramed}

%\titlecontents{chapter}
%[0em]{\vspace*{2\baselineskip}}
%{\parbox{4.5em}{%
%		\hfill\Huge\sffamily\bfseries\color{myred}\thecontentspage}%
%	\vspace*{-2.3\baselineskip}\leftbar\textsc{\small\chaptername~\thecontentslabel}\\\sffamily}
%{}{\endleftbar}
%\titlecontents{section}
%[8.4em]
%{\sffamily\contentslabel{3em}}{}{}
%{\hspace{0.5em}\nobreak\itshape\color{myred}\contentspage}
%\titlecontents{subsection}
%[8.4em]
%{\sffamily\contentslabel{3em}}{}{}  
%{\hspace{0.5em}\nobreak\itshape\color{myred}\contentspage}



%%%%%%%%%%%%%%%%%%%%%%%%%%%%%%%%%%%%%%%%%%%
% TABLE OF CONTENTS
%%%%%%%%%%%%%%%%%%%%%%%%%%%%%%%%%%%%%%%%%%%

\usepackage{tikz}
\definecolor{doc}{RGB}{0,60,110}
\usepackage{titletoc}
\contentsmargin{0cm}
\titlecontents{chapter}[3.7pc]
{\addvspace{30pt}%
	\begin{tikzpicture}[remember picture, overlay]%
		\draw[fill=doc!60,draw=doc!60] (-7,-.1) rectangle (-0.9,.5);%
		\pgftext[left,x=-3.5cm,y=0.2cm]{\color{white}\Large\sc\bfseries Capitolo\ \thecontentslabel};%
	\end{tikzpicture}\color{doc!60}\large\sc\bfseries}%
{}
{}
{\;\titlerule\;\large\sc\bfseries Pagina \thecontentspage
	\begin{tikzpicture}[remember picture, overlay]
		\draw[fill=doc!60,draw=doc!60] (2pt,0) rectangle (4,0.1pt);
	\end{tikzpicture}}%
\titlecontents{section}[3.7pc]
{\addvspace{2pt}}
{\contentslabel[\thecontentslabel]{2pc}}
{}
{\hfill\small \thecontentspage}
[]
\titlecontents*{subsection}[3.7pc]
{\addvspace{-1pt}\small}
{}
{}
{\ --- \small\thecontentspage}
[ \textbullet\ ][]

\makeatletter
\renewcommand{\tableofcontents}{%
	\chapter*{%
	  \vspace*{-20\p@}%
	  \begin{tikzpicture}[remember picture, overlay]%
		  \pgftext[right,x=15cm,y=0.2cm]{\color{doc!60}\Huge\sc\bfseries \contentsname};%
		  \draw[fill=doc!60,draw=doc!60] (13,-.75) rectangle (20,1);%
		  \clip (13,-.75) rectangle (20,1);
		  \pgftext[right,x=15cm,y=0.2cm]{\color{white}\Huge\sc\bfseries \contentsname};%
	  \end{tikzpicture}}%
	\@starttoc{toc}}
\makeatother


%From M275 "Topology" at SJSU
\newcommand{\id}{\mathrm{id}}
\newcommand{\taking}[1]{\xrightarrow{#1}}
\newcommand{\inv}{^{-1}}

%From M170 "Introduction to Graph Theory" at SJSU
\DeclareMathOperator{\diam}{diam}
\DeclareMathOperator{\ord}{ord}
\newcommand{\defeq}{\overset{\mathrm{def}}{=}}

%From the USAMO .tex files
\newcommand{\ts}{\textsuperscript}
\newcommand{\dg}{^\circ}
\newcommand{\ii}{\item}

% % From Math 55 and Math 145 at Harvard
% \newenvironment{subproof}[1][Proof]{%
% \begin{proof}[#1] \renewcommand{\qedsymbol}{$\blacksquare$}}%
% {\end{proof}}

\newcommand{\liff}{\leftrightarrow}
\newcommand{\lthen}{\rightarrow}
\newcommand{\opname}{\operatorname}
\newcommand{\surjto}{\twoheadrightarrow}
\newcommand{\injto}{\hookrightarrow}
\newcommand{\On}{\mathrm{On}} % ordinals
\DeclareMathOperator{\img}{im} % Image
\DeclareMathOperator{\Img}{Im} % Image
\DeclareMathOperator{\coker}{coker} % Cokernel
\DeclareMathOperator{\Coker}{Coker} % Cokernel
\DeclareMathOperator{\Ker}{Ker} % Kernel
\DeclareMathOperator{\rank}{rank}
\DeclareMathOperator{\Spec}{Spec} % spectrum
\DeclareMathOperator{\Tr}{Tr} % trace
\DeclareMathOperator{\pr}{pr} % projection
\DeclareMathOperator{\ext}{ext} % extension
\DeclareMathOperator{\pred}{pred} % predecessor
\DeclareMathOperator{\dom}{dom} % domain
\DeclareMathOperator{\ran}{ran} % range
\DeclareMathOperator{\Hom}{Hom} % homomorphism
\DeclareMathOperator{\Mor}{Mor} % morphisms
\DeclareMathOperator{\End}{End} % endomorphism

\newcommand{\eps}{\epsilon}
\newcommand{\veps}{\varepsilon}
\newcommand{\ol}{\overline}
\newcommand{\ul}{\underline}
\newcommand{\wt}{\widetilde}
\newcommand{\wh}{\widehat}
\newcommand{\vocab}[1]{\textbf{\color{blue} #1}}
\providecommand{\half}{\frac{1}{2}}
\newcommand{\dang}{\measuredangle} %% Directed angle
\newcommand{\ray}[1]{\overrightarrow{#1}}
\newcommand{\seg}[1]{\overline{#1}}
\newcommand{\arc}[1]{\wideparen{#1}}
\DeclareMathOperator{\cis}{cis}
\DeclareMathOperator*{\lcm}{lcm}
\DeclareMathOperator*{\argmin}{arg min}
\DeclareMathOperator*{\argmax}{arg max}
\newcommand{\cycsum}{\sum_{\mathrm{cyc}}}
\newcommand{\symsum}{\sum_{\mathrm{sym}}}
\newcommand{\cycprod}{\prod_{\mathrm{cyc}}}
\newcommand{\symprod}{\prod_{\mathrm{sym}}}
\newcommand{\Qed}{\begin{flushright}\qed\end{flushright}}
\newcommand{\parinn}{\setlength{\parindent}{1cm}}
\newcommand{\parinf}{\setlength{\parindent}{0cm}}
% \newcommand{\norm}{\|\cdot\|}
\newcommand{\inorm}{\norm_{\infty}}
\newcommand{\opensets}{\{V_{\alpha}\}_{\alpha\in I}}
\newcommand{\oset}{V_{\alpha}}
\newcommand{\opset}[1]{V_{\alpha_{#1}}}
\newcommand{\lub}{\text{lub}}
\newcommand{\del}[2]{\frac{\partial #1}{\partial #2}}
\newcommand{\Del}[3]{\frac{\partial^{#1} #2}{\partial^{#1} #3}}
\newcommand{\deld}[2]{\dfrac{\partial #1}{\partial #2}}
\newcommand{\Deld}[3]{\dfrac{\partial^{#1} #2}{\partial^{#1} #3}}
\newcommand{\lm}{\lambda}
\newcommand{\uin}{\mathbin{\rotatebox[origin=c]{90}{$\in$}}}
\newcommand{\usubset}{\mathbin{\rotatebox[origin=c]{90}{$\subset$}}}
\newcommand{\lt}{\left}
\newcommand{\rt}{\right}
\newcommand{\bs}[1]{\boldsymbol{#1}}
\newcommand{\exs}{\exists}
\newcommand{\st}{\strut}
\newcommand{\dps}[1]{\displaystyle{#1}}

\newcommand{\sol}{\setlength{\parindent}{0cm}\textbf{\textit{Solution:}}\setlength{\parindent}{1cm} }
\newcommand{\solve}[1]{\setlength{\parindent}{0cm}\textbf{\textit{Solution: }}\setlength{\parindent}{1cm}#1 \Qed}

% Things Lie
\newcommand{\kb}{\mathfrak b}
\newcommand{\kg}{\mathfrak g}
\newcommand{\kh}{\mathfrak h}
\newcommand{\kn}{\mathfrak n}
\newcommand{\ku}{\mathfrak u}
\newcommand{\kz}{\mathfrak z}
\DeclareMathOperator{\Ext}{Ext} % Ext functor
\DeclareMathOperator{\Tor}{Tor} % Tor functor
\newcommand{\gl}{\opname{\mathfrak{gl}}} % frak gl group
\renewcommand{\sl}{\opname{\mathfrak{sl}}} % frak sl group chktex 6

% More script letters etc.
\newcommand{\SA}{\mathcal A}
\newcommand{\SB}{\mathcal B}
\newcommand{\SC}{\mathcal C}
\newcommand{\SF}{\mathcal F}
\newcommand{\SG}{\mathcal G}
\newcommand{\SH}{\mathcal H}
\newcommand{\OO}{\mathcal O}

\newcommand{\SCA}{\mathscr A}
\newcommand{\SCB}{\mathscr B}
\newcommand{\SCC}{\mathscr C}
\newcommand{\SCD}{\mathscr D}
\newcommand{\SCE}{\mathscr E}
\newcommand{\SCF}{\mathscr F}
\newcommand{\SCG}{\mathscr G}
\newcommand{\SCH}{\mathscr H}

% Mathfrak primes
\newcommand{\km}{\mathfrak m}
\newcommand{\kp}{\mathfrak p}
\newcommand{\kq}{\mathfrak q}

% number sets
\newcommand{\RR}[1][]{\ensuremath{\ifstrempty{#1}{\mathbb{R}}{\mathbb{R}^{#1}}}}
\newcommand{\NN}[1][]{\ensuremath{\ifstrempty{#1}{\mathbb{N}}{\mathbb{N}^{#1}}}}
\newcommand{\ZZ}[1][]{\ensuremath{\ifstrempty{#1}{\mathbb{Z}}{\mathbb{Z}^{#1}}}}
\newcommand{\QQ}[1][]{\ensuremath{\ifstrempty{#1}{\mathbb{Q}}{\mathbb{Q}^{#1}}}}
\newcommand{\CC}[1][]{\ensuremath{\ifstrempty{#1}{\mathbb{C}}{\mathbb{C}^{#1}}}}
\newcommand{\PP}[1][]{\ensuremath{\ifstrempty{#1}{\mathbb{P}}{\mathbb{P}^{#1}}}}
\newcommand{\HH}[1][]{\ensuremath{\ifstrempty{#1}{\mathbb{H}}{\mathbb{H}^{#1}}}}
\newcommand{\FF}[1][]{\ensuremath{\ifstrempty{#1}{\mathbb{F}}{\mathbb{F}^{#1}}}}
% expected value
\newcommand{\EE}{\ensuremath{\mathbb{E}}}
\newcommand{\charin}{\text{ char }}
\DeclareMathOperator{\sign}{sign}
\DeclareMathOperator{\Aut}{Aut}
\DeclareMathOperator{\Inn}{Inn}
\DeclareMathOperator{\Syl}{Syl}
\DeclareMathOperator{\Gal}{Gal}
\DeclareMathOperator{\GL}{GL} % General linear group
\DeclareMathOperator{\SL}{SL} % Special linear group

%---------------------------------------
% BlackBoard Math Fonts :-
%---------------------------------------

%Captital Letters
\newcommand{\bbA}{\mathbb{A}}	\newcommand{\bbB}{\mathbb{B}}
\newcommand{\bbC}{\mathbb{C}}	\newcommand{\bbD}{\mathbb{D}}
\newcommand{\bbE}{\mathbb{E}}	\newcommand{\bbF}{\mathbb{F}}
\newcommand{\bbG}{\mathbb{G}}	\newcommand{\bbH}{\mathbb{H}}
\newcommand{\bbI}{\mathbb{I}}	\newcommand{\bbJ}{\mathbb{J}}
\newcommand{\bbK}{\mathbb{K}}	\newcommand{\bbL}{\mathbb{L}}
\newcommand{\bbM}{\mathbb{M}}	\newcommand{\bbN}{\mathbb{N}}
\newcommand{\bbO}{\mathbb{O}}	\newcommand{\bbP}{\mathbb{P}}
\newcommand{\bbQ}{\mathbb{Q}}	\newcommand{\bbR}{\mathbb{R}}
\newcommand{\bbS}{\mathbb{S}}	\newcommand{\bbT}{\mathbb{T}}
\newcommand{\bbU}{\mathbb{U}}	\newcommand{\bbV}{\mathbb{V}}
\newcommand{\bbW}{\mathbb{W}}	\newcommand{\bbX}{\mathbb{X}}
\newcommand{\bbY}{\mathbb{Y}}	\newcommand{\bbZ}{\mathbb{Z}}

%---------------------------------------
% MathCal Fonts :-
%---------------------------------------

%Captital Letters
\newcommand{\mcA}{\mathcal{A}}	\newcommand{\mcB}{\mathcal{B}}
\newcommand{\mcC}{\mathcal{C}}	\newcommand{\mcD}{\mathcal{D}}
\newcommand{\mcE}{\mathcal{E}}	\newcommand{\mcF}{\mathcal{F}}
\newcommand{\mcG}{\mathcal{G}}	\newcommand{\mcH}{\mathcal{H}}
\newcommand{\mcI}{\mathcal{I}}	\newcommand{\mcJ}{\mathcal{J}}
\newcommand{\mcK}{\mathcal{K}}	\newcommand{\mcL}{\mathcal{L}}
\newcommand{\mcM}{\mathcal{M}}	\newcommand{\mcN}{\mathcal{N}}
\newcommand{\mcO}{\mathcal{O}}	\newcommand{\mcP}{\mathcal{P}}
\newcommand{\mcQ}{\mathcal{Q}}	\newcommand{\mcR}{\mathcal{R}}
\newcommand{\mcS}{\mathcal{S}}	\newcommand{\mcT}{\mathcal{T}}
\newcommand{\mcU}{\mathcal{U}}	\newcommand{\mcV}{\mathcal{V}}
\newcommand{\mcW}{\mathcal{W}}	\newcommand{\mcX}{\mathcal{X}}
\newcommand{\mcY}{\mathcal{Y}}	\newcommand{\mcZ}{\mathcal{Z}}


%---------------------------------------
% Bold Math Fonts :-
%---------------------------------------

%Captital Letters
\newcommand{\bmA}{\boldsymbol{A}}	\newcommand{\bmB}{\boldsymbol{B}}
\newcommand{\bmC}{\boldsymbol{C}}	\newcommand{\bmD}{\boldsymbol{D}}
\newcommand{\bmE}{\boldsymbol{E}}	\newcommand{\bmF}{\boldsymbol{F}}
\newcommand{\bmG}{\boldsymbol{G}}	\newcommand{\bmH}{\boldsymbol{H}}
\newcommand{\bmI}{\boldsymbol{I}}	\newcommand{\bmJ}{\boldsymbol{J}}
\newcommand{\bmK}{\boldsymbol{K}}	\newcommand{\bmL}{\boldsymbol{L}}
\newcommand{\bmM}{\boldsymbol{M}}	\newcommand{\bmN}{\boldsymbol{N}}
\newcommand{\bmO}{\boldsymbol{O}}	\newcommand{\bmP}{\boldsymbol{P}}
\newcommand{\bmQ}{\boldsymbol{Q}}	\newcommand{\bmR}{\boldsymbol{R}}
\newcommand{\bmS}{\boldsymbol{S}}	\newcommand{\bmT}{\boldsymbol{T}}
\newcommand{\bmU}{\boldsymbol{U}}	\newcommand{\bmV}{\boldsymbol{V}}
\newcommand{\bmW}{\boldsymbol{W}}	\newcommand{\bmX}{\boldsymbol{X}}
\newcommand{\bmY}{\boldsymbol{Y}}	\newcommand{\bmZ}{\boldsymbol{Z}}
%Small Letters
\newcommand{\bma}{\boldsymbol{a}}	\newcommand{\bmb}{\boldsymbol{b}}
\newcommand{\bmc}{\boldsymbol{c}}	\newcommand{\bmd}{\boldsymbol{d}}
\newcommand{\bme}{\boldsymbol{e}}	\newcommand{\bmf}{\boldsymbol{f}}
\newcommand{\bmg}{\boldsymbol{g}}	\newcommand{\bmh}{\boldsymbol{h}}
\newcommand{\bmi}{\boldsymbol{i}}	\newcommand{\bmj}{\boldsymbol{j}}
\newcommand{\bmk}{\boldsymbol{k}}	\newcommand{\bml}{\boldsymbol{l}}
\newcommand{\bmm}{\boldsymbol{m}}	\newcommand{\bmn}{\boldsymbol{n}}
\newcommand{\bmo}{\boldsymbol{o}}	\newcommand{\bmp}{\boldsymbol{p}}
\newcommand{\bmq}{\boldsymbol{q}}	\newcommand{\bmr}{\boldsymbol{r}}
\newcommand{\bms}{\boldsymbol{s}}	\newcommand{\bmt}{\boldsymbol{t}}
\newcommand{\bmu}{\boldsymbol{u}}	\newcommand{\bmv}{\boldsymbol{v}}
\newcommand{\bmw}{\boldsymbol{w}}	\newcommand{\bmx}{\boldsymbol{x}}
\newcommand{\bmy}{\boldsymbol{y}}	\newcommand{\bmz}{\boldsymbol{z}}

%---------------------------------------
% Scr Math Fonts :-
%---------------------------------------

\newcommand{\sA}{{\mathscr{A}}}   \newcommand{\sB}{{\mathscr{B}}}
\newcommand{\sC}{{\mathscr{C}}}   \newcommand{\sD}{{\mathscr{D}}}
\newcommand{\sE}{{\mathscr{E}}}   \newcommand{\sF}{{\mathscr{F}}}
\newcommand{\sG}{{\mathscr{G}}}   \newcommand{\sH}{{\mathscr{H}}}
\newcommand{\sI}{{\mathscr{I}}}   \newcommand{\sJ}{{\mathscr{J}}}
\newcommand{\sK}{{\mathscr{K}}}   \newcommand{\sL}{{\mathscr{L}}}
\newcommand{\sM}{{\mathscr{M}}}   \newcommand{\sN}{{\mathscr{N}}}
\newcommand{\sO}{{\mathscr{O}}}   \newcommand{\sP}{{\mathscr{P}}}
\newcommand{\sQ}{{\mathscr{Q}}}   \newcommand{\sR}{{\mathscr{R}}}
\newcommand{\sS}{{\mathscr{S}}}   \newcommand{\sT}{{\mathscr{T}}}
\newcommand{\sU}{{\mathscr{U}}}   \newcommand{\sV}{{\mathscr{V}}}
\newcommand{\sW}{{\mathscr{W}}}   \newcommand{\sX}{{\mathscr{X}}}
\newcommand{\sY}{{\mathscr{Y}}}   \newcommand{\sZ}{{\mathscr{Z}}}


%---------------------------------------
% Math Fraktur Font
%---------------------------------------

%Captital Letters
\newcommand{\mfA}{\mathfrak{A}}	\newcommand{\mfB}{\mathfrak{B}}
\newcommand{\mfC}{\mathfrak{C}}	\newcommand{\mfD}{\mathfrak{D}}
\newcommand{\mfE}{\mathfrak{E}}	\newcommand{\mfF}{\mathfrak{F}}
\newcommand{\mfG}{\mathfrak{G}}	\newcommand{\mfH}{\mathfrak{H}}
\newcommand{\mfI}{\mathfrak{I}}	\newcommand{\mfJ}{\mathfrak{J}}
\newcommand{\mfK}{\mathfrak{K}}	\newcommand{\mfL}{\mathfrak{L}}
\newcommand{\mfM}{\mathfrak{M}}	\newcommand{\mfN}{\mathfrak{N}}
\newcommand{\mfO}{\mathfrak{O}}	\newcommand{\mfP}{\mathfrak{P}}
\newcommand{\mfQ}{\mathfrak{Q}}	\newcommand{\mfR}{\mathfrak{R}}
\newcommand{\mfS}{\mathfrak{S}}	\newcommand{\mfT}{\mathfrak{T}}
\newcommand{\mfU}{\mathfrak{U}}	\newcommand{\mfV}{\mathfrak{V}}
\newcommand{\mfW}{\mathfrak{W}}	\newcommand{\mfX}{\mathfrak{X}}
\newcommand{\mfY}{\mathfrak{Y}}	\newcommand{\mfZ}{\mathfrak{Z}}
%Small Letters
\newcommand{\mfa}{\mathfrak{a}}	\newcommand{\mfb}{\mathfrak{b}}
\newcommand{\mfc}{\mathfrak{c}}	\newcommand{\mfd}{\mathfrak{d}}
\newcommand{\mfe}{\mathfrak{e}}	\newcommand{\mff}{\mathfrak{f}}
\newcommand{\mfg}{\mathfrak{g}}	\newcommand{\mfh}{\mathfrak{h}}
\newcommand{\mfi}{\mathfrak{i}}	\newcommand{\mfj}{\mathfrak{j}}
\newcommand{\mfk}{\mathfrak{k}}	\newcommand{\mfl}{\mathfrak{l}}
\newcommand{\mfm}{\mathfrak{m}}	\newcommand{\mfn}{\mathfrak{n}}
\newcommand{\mfo}{\mathfrak{o}}	\newcommand{\mfp}{\mathfrak{p}}
\newcommand{\mfq}{\mathfrak{q}}	\newcommand{\mfr}{\mathfrak{r}}
\newcommand{\mfs}{\mathfrak{s}}	\newcommand{\mft}{\mathfrak{t}}
\newcommand{\mfu}{\mathfrak{u}}	\newcommand{\mfv}{\mathfrak{v}}
\newcommand{\mfw}{\mathfrak{w}}	\newcommand{\mfx}{\mathfrak{x}}
\newcommand{\mfy}{\mathfrak{y}}	\newcommand{\mfz}{\mathfrak{z}}


\title{\Huge{Algebra Lineare e Geometria Analitica}\\Ingegneria dell'Automazione Industriale}
\author{\huge{Ayman Marpicati}}
\date{A.A. 2022/2023}

\begin{document}

\maketitle
\newpage% or \cleardoublepage
% \pdfbookmark[<level>]{<title>}{<dest>}
\pdfbookmark[section]{\contentsname}{toc}
\tableofcontents

\mprop{}{Siano \(\alpha \ r\) rispettivamente un piano e una retta di \(\EE_3(\RR)\) con \(\alpha \) non ortogonale a \(r\). Allora \(\exists ! \) piano \(\beta : \beta\) ortogonale \(\alpha\) e \(r \subseteq \beta.\)  }
\pf{Dimostrazione}{Dimostriamo l'esistenza: sia \(\beta = [P, V_1 + V_2 orto]\) dove \(r = [P, V_1]\) e \(\alpha = [Q, V_2]\). \begin{enumerate}
    \item \(\beta\) è un piano perché \(\dim(V_1 = 1), \ \dim(V_2 orto)= 1\) e \(V_1 \neq V_2 orto\) (poiché \(\alpha non orto r\) ) \( \implies \dim(V_1 + V_2 orto) = 2 \implies \beta\) è un piano.
    Per costruzione abbiamo che \(\beta \perp \alpha\), infatti lo spazio di traslazione di \(\beta\) è: \[
        V_1 \oplus V_2^{\perp} \supseteq V_2^{\perp} \text{  e \(V_2\) è lo spazio di traslazione di \(\alpha\)  }
    \] inoltre \(r \) 
\end{enumerate}}

\chapter{Geometria analitica in \(\EE_n(\RR)\) }
\dfn{}{In \(\EE_n(\RR)\) si dice \textbf{riferimento cartesiano ortogonale monometrico} la coppia \([0, \mcB]\): \begin{itemize}
    \item O è un punto di \(\EE_n(\RR)\) 
    \item \(\mcB=(e_1, e_2, ..., e_n)\) è una base ortonormale 
\end{itemize}}
\nt{\begin{enumerate}
    \item In \(\EE_n(\RR) \ (n= 2) \implies \mcB = (i,j)\) 
    \item In \(\EE_3(\RR) \ (n = 3) \implies \mcB = (i,j,k)\)
\end{enumerate}}

\dfn{Ortogonalità fra rette}{Siano \(r_1, r_2\) due rette di \(\EE_2(\RR)\) e sia \(r_1 = [P, f(v)] \quad v = l \cdot i + m \cdot j\), analogamente \(r_2 = [P, f'(v)] \quad v' = l' \cdot i + m' \cdot j\) \[
    v \perp v' \iff l \cdot l' + m \cdot m' = 0
    \] se \(r_1\) ha equazione \(ax + by + c = 0\) e \(r_2\) ha equazione \(a'x + b'y + c' = 0\) allora \(P.d.r_1 = [(-b,a)]\), e \(P.d.r_2 = [(-b', a')]\)  \[
    -b \cdot (-b') + a \cdot a' = bb' + aa' = 0
\]
Se abbiamo \(r_1, r_2\) rette in \(\EE_3(\RR)\) con \(P.d.r_1 = [(l,m,n)]\), \(P.d.r_2 = [(l',m',n')]\)  
\(r_1 \perp r_2 \iff v_1\) generatore della direzione di \(r_1 \perp v_2\) generatore della direzione di \(r_2\). \[
    v_1 = li + mj + nk \qquad v_2 = l'i + m'j + n'k
\] \[
    v_1 \perp v_2 \iff r_1 \perp r_2 \iff ll' + mm' + nn' = 0
\] Analogamente se \(r_1, r_2\) sono rette in \(\EE_n(\RR)\) con \(P.d.r_1 = [(x_1, x_2, ..., x_n)]\), \(P.d.r_2 = [(x_1', x_2', ..., x_n')]\)
}
\section{Direzione \(\perp\) ad un iperpiano}
\mprop{}{Sia \(r: ax + by + c = 0\) una retta di \(\EE_2(\RR)\). Allora \([(a, b)]\) è la classe dei parametri direttori della direzione ortogonale a \(r\).}
\pf{Dimostrazione}{\(P.d.r = [(-b,a)]\) e abbiamo che \((a, b) \cdot (-b, a)= 0\) oppure \((a \cdot i + b \cdot j) \cdot ( -b \cdot i + a \cdot j ) = 0 \implies [(a, b)] \perp r\).  }
\mprop{}{Sia \(\pi: ax + by + cz + d = 0\) un piano in \(\EE_3(\RR)\). Allora \([(a, b, c)]\) è la classe dei parametri direttori della direzione ortogonale a \(\pi\). }
\pf{Dimostrazione}{Sia \(v \in V_2 \ (\pi\) ha spazio di traslazione o \(V_2\)). Se \(v = (x, y, z) \implies ax + by + cz = 0 \iff (x, y, z) \cdot (a, b, c) = 0 \implies (a,b,c) \perp v \ \forall v \in V_2 \) }
Più in generale: sia \(S _{n-1}\) un iperpiano in \(\EE_n(\RR)\) di equazione cartesiana \(0 = a_1 x_1 + a_2 x_2 + ... + a_n x_n + a_0 \implies [(a_1, a_2, ..., a_n)]\) è la classe dei parametri direttori della direzione ortogonale a \(S _{n-1}\).

\mprop{Ortogonalità tra piani}{Siano \(\alpha: ax + by + cz + d = 0 \quad \beta: a'x + b'y + c'z + d' = 0\) due piani in \(\EE_3(\RR)\). Allora \(\alpha \perp \beta \iff a \cdot a' + b \cdot b' + c \cdot c' = 0\) }
\pf{Dimostrazione}{\(\alpha \perp \beta \iff V_2 \supseteq V_2'^{\perp}\) dove \(V_2\) è la giacitura di \(\alpha\) e \(V_2'\) è la giacitura di \(\beta\). \[V_2'^{\perp}= [\mcL((a', b',c')) ] \iff (a', b', c') \in V_2\]\((x, y, z) \in V_2 \iff ax + by + cz = 0\) e quindi \((a', b', c') \in V_2 \iff a \cdot a' + b \cdot b' + c \cdot c' = 0\) }
\mprop{Ortogonalità tra retta e piano}{Siano \(r:\) con \(P.d.r = [(l,m,n)]\) e sia \(\alpha\) di equazione \(ax + by + cz + d= 0\) una retta e un piano di \(\EE_3(\RR)\). Allora \(r \perp \alpha\) se e soltanto se \([(a,b,c)] = [(l,m,n)]\) }
\pf{Dimostrazione}{\(r \perp \alpha \iff V_1=V_2^{\perp}\) dove \(V_1\) è la direzione della retta e \(V_2\) è la giacitura di \(\alpha\). \[
        V_1 = \mcL((l,m,n)) = V_2 ^{\perp} = \mcL((a,b,c)) \iff [(a,b,c)] = [(l,m,n)]  
\]}

\dfn{Distanza tra 2 punti in \(\EE_{3}(\RR)\) }{Siano \(P = (x_{1} , x_{2}, ..., x_{n} )\) e \(Q = (x_{1} ' x_2 ', ..., x_n ')\). La distanza tra \(P\) e \(Q\) è la norma del vettore \(\vec{PQ}\) \[
        d(P,Q) = ||\vec{PQ}|| = \sqrt{\vec{PQ} \cdot \vec{PQ}}
\]     \[
    \vec{{PQ}} = (x_1'-x_1) e_1 + ... + (x_n' - x_n) e_n
\] \[
    ||\vec{{PQ}}|| = \sqrt{(x_1' - x_1)^{2} + ... + (x_n'-x_n)^{2}  }
\]}

\dfn{Caso \(\EE_{2}(\RR)\) }{\[
    P = (x, y) \quad Q = (x', y')
\] \[
    \vec{{PQ}} = (x'- x) i + (y' - y)j
\] \[
    d (P, Q) = \sqrt{(x' - x)^{2} + (y'-y)^{2} }
\] }
Caso \(\EE_{3}(\RR)\) da aggiungere

\dfn{Distanza tra punto e retta}{Siano \(P = (x_0, y_0)\) e \(r = [Q, V_1]\) rispettivamente un punto e una retta in \(\EE_{2}(\RR)\). Definiamo la \textbf{distanza tra il punto \(P\) e la retta \(r\)} come la distanza tra \(P\) e il punto \(H\), piede della perpendicolare per \(P\) a \(r\) (cioè l'intersezione tra \(r\) e la retta perpendicolare a \(r\) passante per \(P\)).}

Determiniamo \(||\vec{{PH}}||\). Se \(r\) ha equazione \(ax + by +c = 0\) allora \(V_1^{\perp} = \mcL(a \cdot i + b \cdot j)  \).\[
    \text{Posta} \quad n = [P, V_1^{\perp}] \implies n = [P, \mcL(a i + bj) ]
\]  \[
    H = n \cap r \text{ è la proiezione di \(P\) su \(r\). (è l'intersezione tra \(r\) e la retta per \(P^{\perp} \)).}
\] Sia \(P' = (x', y')\) un generico punti su \(r\). \[ax' + by' + c = 0\] \(PH\) è la componente di \(PP'\) lungo \(v\). \(PP' = (x'-x_0) i + (y' - y_0)j\). \[
    \vec{{PH}} = \frac{PP' \cdot v}{ v \cdot v } v
\] \[
d (P, H) = d (P, r) = || \vec{{PH}} || = ||(\frac{\vec{{PP'}} \cdot v}{v \cdot v} v)|| = [...] = \frac{|ax_0 + by_0 + c|}{\sqrt{a^{2} + b^{2} }} 
\] Da completare 

\dfn{Distanza punto piano}{Siano \(P = (x_0, y_0, x_0)\) e \(\alpha : ax + by + cz + d = 0\) un punto e un piano di \(\EE_{3}(\RR)\). Definiamo \textbf{la distanza} \(d(P, \alpha)\) come la distanza tra \(P\) e il punto \(H\) intersezione tra \(\alpha\) e la retta per \(p \perp \alpha \).}
\pf{Dimostrazione}{\(d(P, \alpha )= d (P, H) = ||\vec{{PH}}||\). Analogamente al caso piano abbiamo che \[
    d(P, \alpha ) = \frac{|ax_0 + by_0 + cz_0 + d|}{\sqrt{a^{2} + b^{2} + c^{2} }}
\]  }

\dfn{Distanza tra un punto e una retta in \(\EE_{3}(\RR)\) }{Siano \(P\) e \(r = [Q, V_1]\) un punto e una retta in \(\EE_{3}(\RR)\). Sia \(\alpha \) il piano per \(P\) ortogonale a \(r\) e sia \(H\) l'intersezione tra \(r\) e \(\alpha \). Definiamo \(d(P, r) = d(P, H) = ||\vec{{PH}}||\).}
\ex{}{In \(\EE_{3}(\RR)\) determiniamo la distanza di \(P = (3, 0, 1)\) da \(r : 
\begin{cases}
    x + y = 1 \\
    z = 2 \\
\end{cases}
\)  \[
\begin{cases}
    x = 1-t \\
    y = t \\
    z = 2 \\
\end{cases} \quad P.d.r = [(-1, 1, 0)]=[(a,b,c)] \qquad \alpha: -x + y + 0 \cdot z + d = 0
\] \[
    \text{ Imponiamo il passaggio per \(P\)}: \quad -3 + 0 + d = 0 \quad d = 3 \quad \alpha : -x + y + 3 = 0
\] \[
    \alpha \cap r :
\begin{cases}
    x + y = 1 \\
    -x + y + 3 = 0 \\
    z = 2 \\
\end{cases} \quad
\begin{cases}
    x+y=1 \\
    0x + 2y= -2 \\
    z=2 \\
\end{cases} \implies x = 2; \ y = -1
\]\[
    H: (2, -1, 2) \quad d(P,r) = ||\vec{{PH}}|| = \vec{{PH}} = (-1) i + (-1) j + k = -1 -j + k
\]}

\dfn{Retta di minima distanza}{Si dice \textbf{retta di minima distanza} tra due rette \(r, s\) sghembe in \(\EE_{3}(\RR)\) una retta ortogonale e incidente sia a \(r\) che a \(s\).}

\mprop{}{La retta di minima distanza tra \(r\) e \(s\) esiste ed è unica.}

\dfn{Distanza tra due rette sghembe in \(\EE_{3}(\RR)\)}{Definiamo \textbf{la distanza tra due rette \(r\) e \(s\) sghembe in \(\EE_{3}(\RR)\)} come la distanza tra i punti \(R\) e \(S\) ottenuti intersecando la retta \(t\) di minima distanza tra \(r\) e \(s\) con \(r\) e \(s\).}

\dfn{Assi}{In \(\EE_{2}(\RR)\) dati due punti \(P,Q\), si dice \textbf{asse} del segmento \(\ol{PQ}\) la retta passante per il punto medio di \(P\) e \(Q\) e ortogonale al segmento \(\ol{PQ}\).  }
\mprop{}{L'asse di un segmento \(\ol{PQ}\) è il luogo dei punti equidistanti da \(P\) e da \(Q\).}
\pf{Dimostrazione}{Dobbiamo dimostrare che \(||\vec{PH} || = ||\vec{QH} || \quad \forall H \in a\) (asse di \(\ol{PQ}\)). \[
    \vec{PH} = \vec{PM} + \vec{MH} \quad e \quad \vec{QH} = \vec{QM} + \vec{MH} 
\] \[
    ||\vec{PH} || = \sqrt{||PM || ^{2} + ||MH||^{2}  } \quad ||\vec{QH} || = \sqrt{||QM|| ^{2} + ||MH|| ^{2} } \quad \text{ ma } \quad ||PM||= ||QM|| 
\] \[
    ||\vec{PH} || = \sqrt{||PM|| ^{2} + ||MH|| ^{2} } = \sqrt{||QM|| ^{2} + ||MH|| ^{2} } = ||\vec{QH} || 
\]}

\ex{}{Determiniamo l'asse di \(P=(1,1)\) e \(Q=(2, -4)\). Il punto \(M = (\frac{3}{2}, -\frac{3}{2})\) \[
    \vec{PQ} = (2-1) i + (-4-1) j = 1 - 5 j = (1, -5)
\] \(r \perp \vec{PQ} \) per \(M\)  è del tipo \[
    x -5y + c = 0 \quad \text{e passa per \(M\) }
\] \[
    \frac{3}{2} + \frac{15}{2} + c = 0 \quad c = -9 \implies r : \ x - 5y -9 = 0
\]Alternativamente \[
    r: \ H \in r \iff d(H,P) = d(H, Q)
\]se \(H = (x, y)\) \[
    \sqrt{(x-1)^{2} + (y-1) ^{2} } = \sqrt{(x-2)^{2} + (y + 4)^{2} }
\] \[
    x^{2} - 2x + 1 + y^{2}  -2y + 1 = x^{2} - 4x + 4 + y^{2} + 8y + 16 \implies r: \ 2x -10y -18 = 0
\]}

\dfn{Piano assiale}{In \(\EE_{3}(\RR)\) si dice \textbf{piano assiale} del segmento \(\ol{PQ}\) il piano \(\alpha \) passante per il punto medio di \(P\) e \(Q\) e ortogonale al segmento \(\ol{PQ}\).}

\mprop{}{Il piano assiale del segmento \(\ol{PQ}\) è il luogo dei punti equidistanti tra \(P\) e \(Q\).}

\section{Circonferenze in \(\EE_{2}(\RR)\) }
\dfn{Circonferenza}{Dato un punto \(C = (x_0, y_0)\) in \(\EE_{2}(\RR)\) e dato \(r\) numero reale positivoSi dice circonferenza di centro \(C\) e raggio \(r\) il luogo dei punti aventi distanza \(r\) da \(C\). }

Sia \(P=(x, y)\) appartenente alla circonferenza di centro \(C\) e raggio \(r\). \[
        d(P,C) = \sqrt{(x-x_0)^{2}x^{2} + y^{2} + 2ax + 2by + c = 0 + (y-y_0)^{2}} = r \iff (x-x_0)^{2} + (y-y_0)^{2} = r^{2}  
\] \[
    x^{2} + y^{2} + 2ax + 2by + c = 0 \iff x^{2} + y^{2} - 2x_0x - 2y_0y + (x_0^{2} + y_0^{2} - r^{2} ) = 0
\] 

\mprop{Equazione cartesiana di una circonferenza}{Tutte e sole le circonferenze si rappresentano come \(x^{2} + y^{2} + 2ax + 2by + c = 0\) con \(a^{2} + b^{2} - c > 0\) e avremo che \(C = (-a, -b)\) e \(r = \sqrt{a^{2} + b^{2} -c}\)  }

\nt{Se \(r\) fosse 0, \(a^{2} + b^{2} - c = 0 \implies  x^{2} + y^{2} + 2ax + 2by + c = 0\) è rappresentata solo da \(C = (-a, -b)\).}
\mprop{}{Per tre punti non allineati in \(\EE_{2}(\RR)\) passa un unica circonferenza.}

\section{Sfere in \(\EE_{3}(\RR)\) }
\dfn{Sfera}{Sia \(C: (x_{0}, y_{0}, z_{0} )\) e sia \(r\) un numero reale positivo. Si dice \textbf{sfera} di raggio \(C\) e di centro \(r\) il luogo dei punti aventi distanza \(r\) da \(C\). }

Sia \(P:(x,y,z)\) appartenente alla sfera, allora \[
    d(P,C) = \sqrt{(x-x_0)^{2} + (y-y_0)^{2} + (z-z_0)^{2}} = r \iff (x-x_0)^{2} + (y-y_0)^{2} + (z-z_0)^{2} = r^{2}
\]
\nt{Una sfera è una superficie algebrica reale (Analogamente una circonferenza è una curva algebrica reale).}

\mprop{Equazione cartesiana di una sfera}{Tutte le sfere si rappresentano come \(x^{2} + y^{2} + z^{2} + 2ax + 2by + 2cz + d = 0\) con \(a^{2} + b^{2} + c^{2} > 0\) e avremo che \(C = (-a, -b, -c)\) e \( r = \sqrt{a^{2} + b^{2} + c^{2} - d}\)  }
\nt{Se \(a^{2} + b^{2} + c^{2} - d = 0 \implies x^{2} + y^{2} + z^{2} + 2ax + 2by + 2cz + d = 0\) è realizzata dal solo centro \(C = (-a, -b, -c)\). }

\mprop{}{Siano \(A, B, C, D\) quattro punti non complanari di \(\EE_{3}(\RR)\). Per \(A, B, C, D\) passa un'unica sfera}

Il centro della sfera si trova intersecando i piani assiali dei quattro punti. Il raggio è la distanza del centro da uno qualsiasi dei quattro punti.

\section{Circonferenze in \(\EE_{3}(\RR)\) }
\dfn{Circonferenza in \(\EE_{3}(\RR)\) }{Dati un piano \(\alpha \), un suo punto \(C\) e un numero reale positivo \(r\). Si dice \textbf{circonferenza} di raggio \(C\) e raggio \(r\) il luogo dei punti di \(\alpha \) aventi distanza \(r\) da \(C\).}

\nt{Una circonferenza appartiene a infinite sfere. Quindi per tre punti non allineati passano infinite sfere.}

\mprop{}{Tutte e sole le circonferenze di \(\EE_{3}(\RR)\) ammettono una rappresentazione del tipo \[
\begin{cases}
    ax + by + cz + d = 0 \quad \rightarrow \quad \text{piano \(\alpha \) } \\
    (x- x_0)^{2} + (y - y_0)^{2} + (z-z_0)^{2} = r'^{2} \\ 
\end{cases}
\] \[
    d(C', \alpha ) < r' \quad \text{dove} \quad C' = (x_0, y_0, z_0) \qquad 
    \frac{|ax_0 + by_0 + cz_0 + d| }{\sqrt{a^{2} + b^{2} + c^{2} }} < r' 
\] ci sono infinite rappresentazioni ma solo una con il centro \(C\) della circonferenza coincidente con il centro \(C'\) della sfera.}
Il centro della circonferenza \(C\) si trova intersecando il piano \(\alpha \) con la retta per il centro della sfera \(C'\) perpendicolarmente ad \(\alpha \).
Utilizziamo il teorema di Pitagora. Conosciamo \(\ol{CC'}\) e conosciamo anche il raggio \(r'\) della sfera. Quindi \[
    r = \sqrt{r'^{2} - \ol{CC'}^{2}}
\]
\nt{Una circonferenza si può ottenere anche intersecando altre superfici superfici con un piano.}

\ex{}{Si consideri \[
\begin{cases}
    x^{2} + y^{2} = 7 \\
    z = 3 \quad \rightarrow \quad \alpha \\
\end{cases} \quad \text{è una circonferenza?} 
\] \[
    x^{2} + y^{2} + z^{2} - z^{2} = 7 \quad  \text{ e siccome \(z = 3\)  } 
\begin{cases}
    x^{2} + y^{2} + z^{2} = 16 \\
    z = 3 \\
\end{cases} \quad \text{che descrive una circonferenza.}
\] Ed è una curva algebrica reale di \(\EE_{3}(\RR)\).}

\chapter{Ampliamento di \(\AA_{2}(\RR)\)}
In \(\AA_{2}(\RR)\) date due rette \(r\) e \(s\) o sono parallele o si intersecano in un punto. Se sono parallele \[
        r = [P,V_1] \ e \ s = [Q, W_1] \implies V_1 = W_1
    \]quindi la direzione \(V_1\) è l'elemento comune a tutte le rette parallele a \(r\). Poiché il parallelismo è una relazione di equivalenza tra le rette del piano. 

\section{Ampliamento (proiettivo) di \(\AA_{2}(\RR)\)}
\dfn{\(\tilde{\AA}_2(\RR)\) }{
    \begin{itemize}
    \item Punti propri che sono tutti e soli i punti di \(\AA_{2}(\RR)\)
    \item I punti impropri che sono le direzioni delle rette del piano ovvero i sottospazi di \(\RR^{2} \) di dimensione 1.
\end{itemize} Da cui il nome proprio/improprio per i fasci di rette del piano.}
\dfn{Rette}{\begin{itemize}
    \item le rette proprie: le rette di \(\AA_{2}(\RR)\) unite al loro punto improprio;
    \item una retta impropria: tutti i punti impropri (\(r_{\infty}\))
\end{itemize}}

Sia \(P_{\infty} \in r_{\infty} \implies \) non definiamo il vettore \(\vec{QP_{\infty} } \) con \(Q\) proprio/improprio.

La funzione \(f : A \times A \rightarrow V_{2}(\\RR) \) 
\textbf{da completare}

\mprop{}{Due rette distinte di \(\tilde{\AA}_{2}(\RR)\) sono sempre incidenti.}
\pf{Dimostrazione}{Siano \(r\) e \(s\) due rette distinte di \(\tilde{\AA}_{2}(\RR)\) \begin{enumerate}
    \item \(r\) e \(s\) sono proprie e non parallele tra loro \( \implies \) \(r\) è incidente a \(s\) in \(\AA_{2}(\RR)\) \(\subseteq \) \(\tilde{\AA}_{2}(\RR)\) e il punto improprio di \(r\) è diverso da quello di \(s\).
    \item \(r\) e \(s\) sono proprie ma \(r\) è parallelo a \(s\). \\
        \(r \cap s = \emptyset\) in \(\AA_{2}(\RR)\) ma \(r\) e \(s\) hanno la stessa direzione \( \implies \) lo stesso punto improprio.
    \item \(r\) è propria e \(s = r_{\infty} \). \\ \(r \cap s = r \cap r_{\infty} \) è il punto improprio di \(r\). 
\end{enumerate}}

\textbf{Completare con il quinto assioma e la spiegazione di a due tilda di r}
\nt{Tutte le rette proprie contengono un solo punto improprio e \(r_{\infty} \) contiene solo punti impropri.}

\mprop{}{Per due punti distinti di \(\tilde{\AA}_{2}(\RR)\) passa un'unica retta. }
\pf{Dimostrazione}{Siano \(A\) e \(B\) i due punti distinti considerati: \begin{enumerate}
    \item \(A, B\) sono entrambi propri \[
        \implies \exists ! r \in \AA_{2}(\RR) \text{  per  } A \ e \ B
    \] \[
        \exists ! r \quad \text{propria per} A \ e \ B
    \] la \(r_{\infty}\) non contiene \(A\) e \(B\) \( \implies \exists ! \ r\) per \(A\) e \(B\).
    \item \(A\) è proprio e \(B\) è improprio (o viceversa). \( \implies B\) è la direazione \(V_1\) \( \implies \exists !\) retta per \(\) \textbf{completare} 
    \item \(A\) e \(B\) sono entrambi impropri. Nessuna retta propria li contiene entrambi (ogni retta propria ha solo 1 punto improprio) \( \implies A, B \in r_{\infty} \) che è l'unica che li contiene entrambi.
\end{enumerate}}

\section{Geometria analitica in \(\tilde{\AA}_{2}(\RR) \)}
Enti definiti "a meno di un fattore" di proporzionalità:
\begin{enumerate}
    \item equazioni di rette in \(\AA_{2}(\RR) \) \[
            ax + by + c = 0 \qquad \text{se} \ [(a,b,c)] = [(a',b', c')]
    \] \[
    \implies r' : \ a'x + b'y + c' = 0 \qquad \text{è coincidente a } r
    \] 
\item \([(l,m)] = P.d.r\) è come classe di equivalenza 
\end{enumerate}
Indichiamo con \(\rho \) la relazione di equivalenza data dalla proporzionalità su \(\RR^{3}\backslash \{(0,0,0)\} \) 
\[
    \frac{\RR^{3}\backslash \{(0,0,0)\}}{\rho} = \{[(x,y,z)] : \ x,y,z \in \RR^{3} \ e \ (x,y,z) \neq \ul{0} \} 
\] 
Quest'insieme definirà le coordinate dei punti.
\[
\tilde\AA_{2}(\RR) = \AA_{2}(\RR) \cup \AA_{\infty}
\] Fissiamo il riferimento affine in \(\AA_{2}(\RR) \) \[
\phi: \ \AA_2 \cup \AA_{\infty} \to \frac{\RR^{3}\backslash \{(0,0,0)\}}{\rho} 
\] 
\begin{itemize}
    \item se \(P\) è proprio \((x,y) : \ \phi(x, y) = [(x,y,1)]\)
    \item se \(P\) è improprio \(P: [(l,m)] \quad \phi (P) = [(l,m,0)]\)
\end{itemize}

\mprop{}{\(\phi\) è una biiezione tra \(\tilde{\AA}_{2}(\RR) \to \frac{\RR^{3}\backslash \{(0,0,0)\}}{\rho} \)}

\paragraph{Osservazione:} Sia \(P\) di coordinate omogenee \([(x_1,x_2,x_3)]\) con \(x_3\neq 0\) \[
    \left[ \left( \frac{x_1}{x_3}, \frac{x_2}{x_3}, 1 \right)  \right] 
\] \(P\) è proprio \(P = (x,y) = (\frac{x_1}{x_3}, \frac{x_2}{x_3})\) \\
Sia invece \(x_3= 0\) 
\[
    P = [(x_1,x_2, 0 )] \qquad [(l,m)] = [(x_1,x_2)]
\] \(P\) non ha coordinate affini (non omogenee) \(\implies \) è improprio.

\section{Rappresentazione delle rette in \(\tilde{\AA}_{2}(\RR) \)}
Sia \(RA [O, B = (e_1, e_2)]\) un riferimento affine di \(\AA_{2}(\RR) \). In \(\AA_{2}(\RR) \) l'equazione cartesiana di una retta è \(ax+by+c = 0\) con \((a,b) \neq (0,0)\). Sui punti propri \(P = \left[ \left( \frac{x_1}{x_3}, \frac{x_2}{x_3}, 1 \right)  \right] \) dovrà valere \(ax+by+c = 0\)\[
a \left( \frac{x_1}{x_3} \right) + b \left( \frac{x_2}{x_3} \right) + c = 0 \qquad ax_1+ bx_2+ cx_3=0
\] Il punto improprio di \(ax+by+c=0\) è \([(-b, a, 0)]\). Sostituiamo in \(ax_1+ bx_2+ cx_3=0\) \([(-b, a, 0)]\) \[
a(-b) + b a + 0 = 0
\] \(\implies \) \(ax_1+ bx_2+ cx_3=0\) è l' equazione omogenea di una retta \(r\) di \(\tilde{\AA}_{2}(\RR) \). \\
Siano ora \((a,b) = (0,0)\), allora \(ax_1+ bx_2+ cx_3=0\) si riduce a \(0 x_1+0x_2+cx_3=0\) con \(c \neq 0, \ cx_3 = 0, \ x_3 = 0\) è la \(r_{\infty}\) perché rispettata da tutti e soli i punti impropri. 
L'equazione \(ax_1+bx_2+cx_3= 0\) con \((a,b,c) \neq (0,0,0)\) rappresenta, in ogni caso, una retta di \(\tilde{\AA}_{2}(\RR) \). Di conseguenza è l'equazione cartesiana di una retta di \(\tilde{\AA}_{2}(\RR)\).

\section{Complessificazione di \(\tilde{\AA}_{2}(\RR) \)}

\(\tilde{\AA}_{2}(\CC)\) = piano affine ampliato e \textbf{complessificato}.
\paragraph{Osservazione:} 
\begin{itemize}
    \item \textbf{punti}: terne (\(\neq \ul{0} \)) di numeri complessi determinati a meno di un fattore di proporzionalità complesso e non nullo.
\[
\frac{\CC^{3}\backslash \{(0,0,0)\} }{\rho }
\] 
    \item \textbf{rette}: luogo delle autosoluzioni (soluzioni non nulle) di un'equazione del tipo \[
    ax_1+bx_2+cx_3= 0 \quad \text{con} \quad (a,b,c) \neq \ul{0} \ e \ a,b,c \in \CC
    \] 
\end{itemize}

\dfn{Punti e rette in \(\tilde{\AA}_{2}(\CC)\)}{In \(\tilde{\AA}_{2}(\CC)\) si dicono:
\begin{itemize}
    \item \textbf{punti} e \textbf{rette reali} i punti e le rette che ammettono una rappresentazione reale
    \item \textbf{punti} e \textbf{rette immaginari} i punti e le rette che ammettono solo rappresentazioni immaginarie
\end{itemize}}

\ex{}{\(P:[(4,3+i, 1)]\) è immaginario.
\pf{Dimostrazione}{Sia \(a+ib \in \CC: \ P = (4(a+ib), (3+i) (a+ib), a+ib)\) con \(x_1,x_2, x_3\) reali, quindi \(x_3 = a+ib \implies x_3= a\). \(P=[(4a, (3+i)a, a)]\), ma \(a \neq 0 \ (3+i)a\) non è reale \(\implies P\) non è reale.}}

\dfn{Coniugati}{Si dicono \textbf{coniugati} due enti (punti, rette ecc\ldots) che ammettono rappresentazioni coniugate.}

\mprop{}{Un ente geometrico (punto, retta, curva ecc\ldots ) è reale se, e soltanto se, coincide con il proprio coniugato.}

\nt{Una retta reale ha infiniti punti immaginari
\ex{}{\(x_1= 0  \implies [(0,x_2,x_3)] \implies [(0, a + ib, 1)]\) sono tutti immaginari.}}

\paragraph{Osservazione:} Se un'equazione reale è realizzata da un punto \(P \implies \overline{P}\) è soluzione se \(r\) è reale e \(P \in r \implies \overline{P}\in \overline{r} = r\).

\mprop{}{La retta che congiunge due punti \(P\) e \(\overline{P}\) immaginari e coniugati è reale.}
\pf{Dimostrazione}{\(P \in r\) e \(\overline{P}\in r\) per costruzione. Poiché \(P \neq  \overline{P}\) \(r = rt(P, \overline{P})\). Poiché \(P \in r \implies \overline{P} \in \overline{r}\) e \(\overline{P} \in r \implies \overline{\overline{P}} \in \overline{r}\), ma \(\overline{\overline{P}}=P \in \overline{r}\). \(\implies \overline{P}\) e \(P \in \overline{r} \implies \overline{r}=rt(P, \overline{P})\). Per l'unicità della retta per \(P\) e \(\overline{P}\) \(\implies r = \overline{r} \implies r\) è reale.}

\mprop{}{Per un punto \(P\) immaginario \((P \neq \overline{P})\) passa un'unica retta reale.}
\pf{Dimostrazione}{La retta \(rt(P, \overline{P})\) è reale per la proposizione precedente. Supponiamo per assurdo che \(\exists \ s \neq rt(P, \overline{P})\) reale per \(P\). \(\implies \overline{P} \in  s\) poiché \(s\) è reale.  \(s = rt(P, \overline{P})\) che è \textbf{assurdo!}. Quindi esiste ed è unica la retta \(r\) reale per \(P\).}

\mprop{}{Due rette immaginarie e coniugate si intersecano in un punto reale di \(\tilde{\AA}_{2}(\CC) \).}
\pf{Dimostrazione}{Sia \(r\) immaginaria \(\implies r \neq \overline{r}\). Sia \(P = r \implies \overline{P}\in \overline{r}\). \(P \in \overline{r}\implies \overline{P}\in \overline{r}=r\). \(\overline{P}\in r \cap \overline{r} = P \implies \overline{P}=P \implies P\) è reale.}

\mprop{}{Ogni retta \(r\) immaginaria ha un unico punto reale in \(\tilde{\AA}_{2}(\CC)\).}

\pf{Dimostrazione}{Per ipotesi \(r \neq \overline{r} \implies \exists \ P\) intersezione di \(r\) e \(\overline{r}\). Quindi per la proposizione precedente \(P\) è reale. Sia \(S \in r\) un punti reale. Essendo reale \(S = \overline{S} \implies S \in \overline{r} \implies S \implies S \in r \cap \overline{r}\). Quindi per l'unicità del punto di intersezione, \(S = P\).}

\dfn{Curve algebriche reali in \(\tilde{\AA}_{2}(\CC) \)}{Curva algebrica reale di \(\tilde{\AA}_{2}(\CC) \) è il luogo delle autosoluzioni di un'equazione del tipo \[
F(x_1, x_2, x_3) = 0
\] dove \(F(x_1, x_2, x_3) = 0\) è un polinomio omogeneo a coefficienti reali nelle variabili \(x_1, x_2, x_3\).}

\paragraph{Osservazione:} Ogni curva algebrica reale di \(\tilde{\AA}_{2}(\CC)\) che contiene un punto  \(P\) contiene anche \(\overline{P}\).

\ex{}{Per esempio prendiamo una circonferenza\[
x^2+2ax + y^2+2by+c=0 \qquad r^2=a^2+ b ^2 - c = 0
\] \[
(x+a)^2+(y+b)^2=0 \qquad 
C: \ (-a, -b)
\] Ora consideriamo l'equazione \[
(x+a)^2+(y+b)^2=0 \qquad \text{in} \quad \AA_{2}(\CC) \quad x=\frac{x_1}{x_3} \quad y = \frac{x_2}{x_3} 
\] \[
\left( \frac{x_1}{x_3} + a \right) ^2+ \left( \frac{x_2}{x_3}+b \right) ^2=0
\] Moltiplichiamo dentro entrambi i membri per \(x_3\) e sviluppiamo\[
x_1^2+2ax_1x_3+a^2x_3^2 + x_2^2+2bx_2x_3+b ^2 x_3^2=0
\] \[
1+2a+a^2+1+2b+b ^2=0
\] }

\dfn{Curva riducibile}{In \(\tilde{\AA}_{2}(\CC) \) una curva \(F(x_1, x_2, x_3)\) si dice riducibile se \(F\) è il prodotto di polinomi di grado più basso.}
\ex{}{\[F(x_1, x_2, x_3) = F_1(x_1, x_2, x_3)^{n_1} \cdot  F_2(x_1, x_2, x_3)^{n_2} \cdot F_3(x_1, x_2, x_3)^{n_3}\] \[
\deg(F) = n_1 \deg(F_1) + \ldots + n_t \deg (F_t)
\] }

\paragraph{Osservazione:} Geometricamente una curva riducibile si riduce in componenti ottenute uguagliando a zero i vari fattori.

\dfn{Ordine}{Si dice \textbf{ordine} di una curva algebrica in \(\tilde{\AA}_{2}(\CC) \) il grado del polinomio \(F\) che la definisce.}

\thm{Teorema dell'ordine}{L'ordine di una curva algebrica reale è uguale al numero di intersezioni in comune con una qualsiasi retta \(r\) di \(\tilde{\AA}_{2}(\CC) \) a patto che 
\begin{enumerate}
    \item \(r\) non sia componente della curva
    \item le intersezioni siano contate con la loro molteplicità
\end{enumerate}}

\dfn{Punti semplici ed r-upli}{Sia \(C\) una curva algebrica di \(\tilde{\AA}_{2}(\CC) \) e sia \(P \in C\)
\begin{itemize}
    \item \(P\) si dice \textbf{semplice} se la generica retta per \(P\) interseca \(C\) in \(P\) con molteplicità unitaria ed esiste un'unica retta, chiamata retta tangente, con molteplicità di intersezione in \(P\) maggiore di 1.
    \item \(P\) si dice \textbf{r-uplo} (doppio, triplo, ecc\ldots ) se la generica retta per \(P\) interseca \(C\) in \(P\) con molteplicità \(r\), ed esistono \(r\) (contate con la loro molteplicità) rette con molteplicità di intersezione in \(P\) maggiore di \(r\) (rette tangenti).
\end{itemize}}

\mprop{}{Sia \(C\) una curva algebrica reale di \(\tilde{\AA}_{2}(\CC) \). Se una retta \(r\) ha più di \(n\) intersezioni con \(n\) l'ordine di \(C\), allora \(r\) è componente di \(C\).}
\pf{Dimostrazione}{Per il teorema dell'ordine se \(r\) non fosse componente della curva \(C\) avrebbe esattamente \(n\) intersezioni con \(C\) (a patto di contarle con la dovuta molteplicità).}

\mprop{}{Sia \(C\) una curva algebrica reale di \(\tilde{\AA}_{2}(\CC) \) di ordine \(n\). Allora \(C\) non possiede punti (\(n + 1\))-upli.}
\pf{Dimostrazione}{Dato che \(C\) è di ordine \(n \implies  \exists r \in \tilde{\AA}_{2}(\CC) \) non componente di \(C\) passante per un punto dato di \(C\). Sia, per assurdo, \(P\) un punto \((n+1)\)-uplo. \[
|r \cap C| \ge n+1 \quad  \text{perché passa per \(P\)}
\] ma dato che \(r\) non è componente, il teorema dell'ordine \[
|r \cap C| = n < n+1
\] \textbf{Assurdo!} }

\mprop{}{Sia \(C\) una curva algebrica reale di \(\tilde{\AA}_{2}(\CC)\) di ordine \(n\). \(C\) ha un punto \(n\)-uplo  \(P\) se, e soltanto se, \(C\) è unione di \(n\) rette (contate con la dovuta molteplicità) per \(P\).}
\pf{Dimostrazione}{"\(\implies \)"Sia \(P \neq Q \in C\) e sia \(r\) la retta \(rt(P,Q)\). Supponiamo per assurdo \(r\) non sia componente allora per il teorema dell'ordine \[
n = |r \cap C| \ge \underbrace{n}_{\in P}  + \underbrace{1}_{\in Q}
\] \textbf{Assurdo!} Quindi per ogni punto \(Q \in C\) la retta \(PQ\) è componente \(\implies \) \(C\) è unione di rette per \(P\). Quindi queste rette sono \(n = \deg(F) = \) ordine di \(C\). \\
"\(\impliedby \)" Sia \(C\) unione di \(n\) rette per \(P\). Allora la generica retta per \(P\) non componente di \(C\) interseca \(C\) sono in \(P\) \(\implies \) \(P\) è punto \(n\)-uplo.
}
\dfn{Punto multiplo}{Sia \(C\) una curva algebrica reale di \(\tilde{\AA}_{2}(\CC)\) e sia \(P \in C\). Se \(P\) non è un punto semplice allora si dice \textbf{punto multiplo}.}
\newpage

\thm{}{Sia \(C\) una curva algebrica reale di \(\tilde{\AA}_{2}(\CC)\) di ordine \(n\) e sia \(F(x_1, x_2, x_3) = 0\) il polinomio omogeneo che la definisce. I punti multipli di \(C\) sono le classi di autosoluzioni del sistema associato alle derivate: \[
\begin{cases}
    \ \frac{dF}{dx_1}= 0 \\
    \ \frac{dF}{dx_2}=0 \\
    \ \frac{dF}{dx_3}=0 \\
\end{cases}
\] }

\ex{}{\[
x_1^2+2x_2^2+3x_1x_3-3x_2x_3=0
\] \[
\begin{cases}
    \ \frac{dF}{dx_1}= 2x_1+3x_3=0 \\
    \ \frac{dF}{dx_2}= 4x_2-3x_3=0 \\
    \ \frac{dF}{dx_3} = 3x_1-3x_2 = 0 \\
\end{cases}
\] \[
A = 
\begin{pmatrix}
    2 & 0 & 3 \\
    0 & 4 & -3 \\
    3 & -3 & 0 \\
\end{pmatrix} \qquad |A| \neq 0
\] }

\chapter{Coniche in \(\tilde{\AA}_{2}(\CC)\)}
\dfn{Conica}{Si dice \textbf{conica} una curva algebrica reale di \(\tilde{\AA}_{2}(\CC)\) (curva piana) del secondo ordine. Una conica si rappresenta eguagliando a \(0\) un polinomio omogeneo \(F\) di secondo grado nelle variabili  \(x_1, x_2, x_3\), a coefficienti reali. La generica equazione della conica è \[
C: a_{11}x_1^2+ 2a_{12}x_1x_2+2a_{13}x_1x_3+a_{22}x_2^2+2a_{23}x_2x_3+a_{33}x_3^2=0
\] Se chiamiamo \[
X = \begin{pmatrix} x_1\\ x_2 \\ x_3 \end{pmatrix} \quad A =
\begin{pmatrix}
    a_{11} & a_{12} & a_{13} \\
    a_{12} & a_{22} & a_{23} \\
    a_{13} & a_{23} & a_{33} \\
\end{pmatrix}
\] Possiamo riscrivere l'equazione come prodotto righe per colonne \[
{^tX}AX = 0
\] \(A\) è una matrice reale e simmetrica ed è detta \textbf{matrice della conica}.} 

\ex{}{Consideriamo la conica \[
-x_1^2+ax_1x_2+5x_2^2-3x_2x_3+6x_3^2=0
\] \[
A = 
\begin{pmatrix}
    -1 & 2 & 0 \\
    2 & 5 & -\frac{3}{2} \\
    0 & -\frac{3}{2} & 6 \\
\end{pmatrix}
\] Ora facciamo il prodotto  \[
\begin{pmatrix}
    x_1 & x_2 & x_3 \\
\end{pmatrix} \cdot 
\begin{pmatrix}
    -1 & 2 & 0 \\
    2 & 5 & -\frac{3}{2} \\
    0 & -\frac{3}{2} & 6 \\
\end{pmatrix} \cdot 
\begin{pmatrix} x_1\\ x_2\\ x_3 \end{pmatrix} = 0
\] \[
\begin{pmatrix}
    -x_1+2x_2 & 2x_1+5x_2-\frac{3}{2}x_3 & -\frac{3}{2}x_2+6x_3 \\
\end{pmatrix} \cdot \begin{pmatrix} x_1\\ x_2\\ x_3 \end{pmatrix} = 0
\] \[
x_1(-x_1+2x_2) + x_2\left(2x_1+5x_2- \frac{3}{2}x_3\right) + x_3\left(-\frac{3}{2}x_2 + 6x_3\right) = 0
\] \[
-x_1^2+4x_1x_2+5x_2^2-3x_2x_3+6x_3^2=0
\] }

\paragraph{Osservazione:} L'equazione della generica conica di \(\tilde{\AA}_{2}(\CC)\) dipende da 6 coefficienti definiti a meno di un fattore di proporzionalità. Quindi le coniche di \(\tilde{\AA}_{2}(\CC)\) sono \(\infty^{5}\).

\mprop{}{Sia \(C\) una conica di \(\tilde{\AA}_{2}(\CC)\) riducibile. Allora \(C\) è unione di 2 rette
\begin{enumerate}
    \item reali e distinte
    \item reali e coincidenti
    \item immaginarie e coniugate
\end{enumerate}}

\pf{Dimostrazione}{Sia \(C\) conica associata al polinomio \(F=(x_1,x_2, x_3) = 0\). Se \(C\) è riducibile \(F=(x_1,x_2, x_3) = F_1=(x_1,x_2, x_3) \cdot F_2=(x_1,x_2, x_3)\) dove \(F_1\) e \(F_2\) hanno grado unitario, quindi rappresentano delle rette e di conseguenza \(C\) è unione di due rette \(r_1\) e \(r_2\). Se \(r_1\) e \(r_2\) sono entrambe reali siamo nei casi \(1\) o \(2\). Se invece \(r_1\) è immaginaria, \(\overline{r_1}\) è ancora componente di \(C\) (per ogni \(P \in r_1, \ \overline{P}\in C\)), ma \(r_1 \neq \overline{r_1} \implies \overline{r_1}=r_2\implies C\) si riduce in due rette immaginarie e coniugate.}

\paragraph{Osservazione:} Se \(r\) è immaginaria anche \(\overline{r}\) lo è. Infatti \(r \neq \overline{r}\) e quindi \(\overline{r} \neq \overline{\overline{r}} = r\).

\mprop{}{In \(\tilde{\AA}_{2}(\CC) \) una conica
\begin{enumerate}
    \item non ha punti tripli
    \item ha un punto doppio se, e soltanto se, è riducibile. E abbiamo due possibilità
        \begin{enumerate}
            \item ha solo un punto doppio \(P\) e si riduce in due rette distinte per \(P\)
            \item ha almeno due punti doppi allora ne ha \(\infty^{1}\) e si fattorizza in una retta
                reale contata due volte
        \end{enumerate}
\end{enumerate}}

\pf{Dimostrazione}{"\(\implies \)" Per ipotesi \(C\) ha punto doppio \(P\). Sia \(R \in C\) e consideriamo la retta \(r = rt(P,R)\), se non fosse componente avrebbe \[
|r \cap C| \ge 2 + 1 = 3 \quad \text{intersezioni con \(C\)}
\] \textbf{Assurdo!} Questo è in contraddizione con il teorema dell'ordine.

"\(\impliedby \)" Sia \(C\) riducibile. Allora  \(C = r_1 \cup r_2\). Sia \(P \in r_1 \cap  r_2\) e sia \(r\) una retta per \(P\) diversa \(r_1\) e da \(r_2\). Quindi \(r \cap C = P\). Per il teorema dell'ordine \(P\) ha molteplicità doppia e abbiamo due casi
\begin{enumerate}
    \item se \(r_1=r_2\) abbiamo \(\infty^{1}\) punti doppi e \(C = r_1 \cup  r_1\)
    \item altrimenti abbiamo un \textbf{solo} \(P\) punto doppio che è \(r_1\cap r_2\)
\end{enumerate}
Dobbiamo dimostrare che esiste un solo punto. Siano per assurdo \(P_1\) e \(P_2\) punti doppi e sia \(C=r_1 \cup r_2\) con \(r_1\neq r_2\). Sia \(Q \in r_2\) con \(P_2 \in r_1\), allora \[
|rt(P_2, Q) \cap C| \ge \underbrace{2}_{P_2} + \underbrace{1}_{Q} 
\] Per il teorema dell'ordine \(rt(P_2, Q)\) è componente. \textbf{Assurdo!} Perché avremmo 3 componenti \((r_1, r_2, rt(P_2, Q))\).} 

\dfn{Coniche generali o degeneri}{Una conica si dice
 \begin{itemize}
    \item \textbf{generale} se è priva di punti doppi \(\implies \) se non è riducibile
    \item \textbf{semplicemente degenere} se ha un solo punto doppio \(\implies C = r_1\cup r_2\) con \(r_1\neq r_2\) 
    \item \textbf{doppiamente degenere} se ha \(\infty^{1}\) punti doppi \(\implies C = r \cup r\)
\end{itemize}}

\thm{}{In \(\tilde{\AA}_{2}(\CC) \) i punti doppi di una conica \(C\) si trovano considerando le classi di autosoluzioni del sistema omogeneo \[
AX = \ul{0}
\] dove \(A\) è la matrice associata a \(C\).}

\pf{Dimostrazione}{\[
C: F(x_1, x_2, x_3) = 0 \quad \text{dove \(F\) è:} 
\]
\[
a_{11}x_1^2+ 2a_{12}x_1x_2+2a_{13}x_1x_3+a_{22}x_2^2+2a_{23}x_2x_3+a_{33}x_3^2=0
\] i punti doppi si trovano risolvendo 
\[
\begin{cases}
    \ \frac{dF}{dx_1}= 2a_{11}x_1+2a_{12}x_2+2a_{13}x_3=0\\
    \ \frac{dF}{dx_2}= 2a_{12}x_1+2a_{22}x_2+2a_{23}x_3=0\\
    \ \frac{dF}{dx_3} = 2a_{13}x_1+2a_{23}x_2+2a_{33}x_3=0\\
\end{cases}
\] Possiamo dividere tutti i fattori per 2 \[
\begin{pmatrix}
    a_{11} & a_{12} & a_{13} \\
    a_{12} & a_{22} & a_{23} \\
    a_{13} & a_{23} & a_{33} \\
\end{pmatrix} \cdot \begin{pmatrix} x_1\\ x_2\\ x_3 \end{pmatrix} = \begin{pmatrix} 0\\ 0\\ 0\\ \end{pmatrix}
\] \[
\implies AX = \ul{0} 
\]  
}

\thm{}{In \(\tilde{\AA}_{2}(\CC) \) una conica \(C : {^tX}A X = 0\) è 
\begin{enumerate}
    \item generale se \(\rho(A) = 3\) 
    \item semplicemente degenere se \(\rho(A) = 2\) 
    \item doppiamente degenere se \(\rho(A) = 1\)
\end{enumerate}}

\pf{Dimostrazione}{  
Dimostriamo tutti i casi singolarmente:
\begin{enumerate}
    \item \(C\) è generale se non ha punti doppi. Se  \(AX = \ul{0} \) ha solo la soluzione nulla \( \iff \rho(A) =3\).
    \item \(C\) è semplicemente degenere se ha un solo punto doppio. \( \iff AX = \ul{0} \) ha \(\infty^{1} \iff \rho(A) = 2\)
    \item \(C\) è doppiamente degenere se ha \(\infty^{1}\) punti doppi \(\iff AX = \ul{0} \) ha \(\infty^{2}\) soluzioni (se \([(x_1, x_2, x_3)]\) è soluzione \([(2x_1, 2x_2, 2x_3)]\) è lo stesso punto doppio) \(\iff \rho(A) =1\)
\end{enumerate}
}

\section{Classificazione affine di una conica generale}

Sia \(C\) una conica di \(\tilde{\AA}_{2}(\CC) \) e \(r\) una retta osserviamo che \(r \cap C =\)
\begin{enumerate}
    \item due punti reali e distinti
    \item un punto reale con molteplicità doppia
    \item due punti immaginari e coniugati
\end{enumerate}
Se consideriamo la \(r_{\infty}\) questa casistica ci dà la classificazione affine delle coniche generali.

\dfn{Ellisse, iperbole e parabola}{Sia \(C\) una conica di \(\tilde{\AA}_{2}(\CC) \) e sia \(C\) generale. Allora \(C \cap r_{\infty}\) è data dai due punti \(P, Q\) (non necessariamente distinti) e \(C\) si dice:
\begin{enumerate}
    \item \textbf{ellisse} se \(P\) e \(Q\) sono immaginari e coniugati;
    \item \textbf{iperbole} se \(P\) e \(Q\) sono reali e distinti;
    \item \textbf{parabola} se \(P\) e \(Q\) sono reali e coincidenti.
\end{enumerate}}

\section{Condizioni analitiche}
Sia \(C\) una conica generale di equazione \[
a_{11}x_1^2+ 2a_{12}x_1x_2+2a_{13}x_1x_3+a_{22}x_2^2+2a_{23}x_2x_3+a_{33}x_3^2=0
\] 
La \(r_{\infty}\) ha equazione \(x_{3}=0\) \[
\begin{cases}
    \ a_{11}x_1^2+2a_{12}x_1x_2+a_{22}x_2^2= 0 = C \cap r_{\infty} \\
    \ x_3 = 0 \\
\end{cases}
\]
Almeno uno fra \(x_1, x_2 \neq 0\). Supponiamo \(x_2 \neq 0\) e dividiamo per \(x_2^2\) \[
a_{11} \left( \frac{x_1}{x_2} \right) ^2 + 2a_{12} \frac{x_1}{x_2} + a_{22} = 0
\] 
La risolviamo in \(\frac{x_1}{x_2}\). Se 
\begin{enumerate}
    \item \(\frac{\Delta}{4} > 0\) abbiamo due soluzioni reali e distinte \(\implies \) \textbf{iperbole};
    \item \(\frac{\Delta}{4} = 0\) abbiamo due soluzioni coincidenti \(\implies \) \textbf{parabola};
    \item \(\frac{\Delta}{4} < 0\) abbiamo due soluzioni immaginarie e coniugate \(\implies \) \textbf{ellisse}.
\end{enumerate}

\[
\frac{\Delta}{4} = \left( \frac{b}{2} \right) ^2 - ac = \left( \frac{2a_{12}}{2} \right) ^2 - a_{11} a_{22} = a_{12}^2 - a_{11} a_{22}
\] \[
A = 
\begin{pmatrix}
    a_{11} & a_{12} & a_{13} \\
    a_{12} & a_{22} & a_{23} \\
    a_{13} & a_{23} & a_{33} \\
\end{pmatrix}
\quad \text{poniamo} \quad A^* =
\begin{pmatrix}
    a_{11} & a_{12} \\
    a_{12} & a_{22} \\
\end{pmatrix} \]
\[
|A^{*}| = a_{11}a_{22}-a_{12}^2= - \frac{\Delta}{4}
\] Se \(C\) è una conica generale \((|A| = 0)\) allora si applicano le casistiche sopra elencate.

\dfn{Polarità associata ad una conica}{Data una conica \(C: {^tX}AX = 0\) e dati due punti del piano \((\tilde{\AA}_{2}(\CC) )\) \[
        P' = [(x_1', x_2', x_3')] \quad e \quad P '' = [(x_1 '', x_2 '', x_3 '')]
\] si dice che \(P'\) è coniugato a \( P ''\) rispetto a \(C\) se \[
{^tX'AX '' = 0 \quad con \quad X' = \begin{pmatrix} x'_1\\ x_2'\\ x'_3 \end{pmatrix}} \quad  e \quad X '' = \begin{pmatrix} x''_1\\ x_2 ''\\ x''_3 \end{pmatrix}
\] }

\paragraph{Osservazione:} Sia \(P'\) coniugato a \(P ''\), ovvero \[
    {^tX'} A X '' = 0 \implies {^t({^tX'}AX '')} = 0 = {^tX ''} {^tA}{^t({^tX'})} = {^tX ''} A X' = 0 \implies P '' \text{ è coniugato a } P' 
\] Quindi la relazione di coniugio è simmetrica \(\implies \) potremo dire semplicemente che \(P'\) e \(P ''\) sono coniugati.

\dfn{Polare}{Sia \(C\) una conica di \(\tilde{\AA}_{2}(\CC) \) e sia \(P \in \tilde{\AA}_{2}(\CC) \) si dice \textbf{polare} di \(P\) rispetto a \(C\) il luogo dei punti \(Q\) coniugati di \(P\) rispetto alla conica \(C\).}

\mprop{}{In \(\tilde{\AA}_{2}(\CC) \) la polare di un punto \(P\) rispetto ad una conica generale \textbf{è una retta}.}
\pf{Dimostrazione}{Siano \([(x_1', x_2', x_3')] = P\) allora \(Q = [(x_1, x_2, x_3)]\) appartiene alla polare di \(P\) se e soltanto se \[
        (x_1', x_2', x_3') A \begin{pmatrix} x_1\\ x_2\\ x_3 \end{pmatrix} = 0 \quad (x_1', x_2', x_3') A = (a,b,c)
\] \[
(a, b, c) \begin{pmatrix} x_1\\ x_2\\ x_3 \end{pmatrix} = ax_1+bx_2+cx_3=0
\] Dimostriamo che \((a,b,c) \neq \ul{0} \). Sia per assurdo \((a,b,c) = \ul{0} \implies (x_1', x_2', x_3') A = \ul{0} \) \[
\iff {^tA} \begin{pmatrix} x'_1\\ x_2'\\ x'_3 \end{pmatrix} \iff A \begin{pmatrix} x'_1\\ x_2'\\ x'_3 \end{pmatrix} = \ul{0} 
\] \(\begin{pmatrix} x_1'\\ x_2'\\ x_3' \end{pmatrix}\) è le coordinate di un punto doppio \(\implies P\) è un punto doppio di \(C \implies \) ma \(C\) è generale \(\implies \) \textbf{assurdo!} \(\implies (a,b,c) \neq \ul{0} \implies ax_1 + bx_2+ cx_3 = 0\) è una retta. Essa è chiamata retta polare di \(P\) rispetto a \(C\).}
\dfn{}{\(P\) è detto polo della sua retta polare. La relazione \textbf{polo} \(\leftrightarrow\) \textbf{polare} è detta polarità ed è una biiezione. }

\subsection{Principio di reciprocità}
Sia \(C\) una conica generale di \(\tilde{\AA}_{2}(\CC) \), sia \(P \in \tilde{\AA}_{2}(\CC) \) e sia \(p\) la polare di \(P\). Allora
\begin{enumerate}
    \item le polari dei punti di \(p\) passano per \(P\).
        \pf{Dimostrazione}{Sia \(Q \in p \implies Q, P\) sono coniugati \(\implies P \in q\) di \(Q\).}
    \item i poli delle rette per \(P\) appartengono a \(p\).
        \pf{Dimostrazione}{Sia \(q\) una retta per \(P\). Il polo \(Q\) di \(q\) è coniugato a tutti i punti di \(q \implies Q\) è coniugato a \(P \implies Q \in p\).}
\end{enumerate}

\mprop{}{Sia \(C\) una conica generale di \(\tilde{\AA}_{2}(\CC) \). Allora
\begin{enumerate}
    \item sia \(P \in C \implies \) la polare \(p\) di \(P\) è la retta tangente a \(C\) in \(P\).
        \pf{Dimostrazione}{Sia \(P\) di coordinate \(x_P = \begin{pmatrix} x'_1\\ x_2'\\ x'_3 \end{pmatrix}\) appartenente alla conica allora la polare di \(P\) ha equazione \({^tX_P}A \begin{pmatrix} x_1\\ x_2\\ x_3 \end{pmatrix} = 0\) che è la formula della retta tangente a \(C\) in \(P\).}
    \item Sia \(P \notin C\). La polare di \(P\) è la congiungente dei due punti \(T_1\) e \(T_2\) ottenuti intersecando le tangenti \(t_1\) e \(t_2\) alla conica per \(P\).
        \pf{Dimostrazione}{\(T_1 \in C \implies \) la polare di \(T_1\) rispetto a \(C\) è \(t_1\). \(P \in t_1 \implies P\) appartiene alla polare di \(T_1\). Quindi per il principio di reciprocità \(T_1\) appartiene alla polare di \(P\) \(\implies T_1 \in p\). Analogamente \(T_2 \in C \implies \) la polare di \(T_2\) è \(t_2\) e \(P \in t_2 \implies T_2 \in p\). Quindi \(T_1, T_2 \in p \implies p\) è la congiungente di \(T_1\) e \(T_2\).}
\end{enumerate}}

\paragraph{Osservazione:} Equivalentemente il punto 2 si può riscrivere
 \mprop{}{Se \(P \notin C\) la sua polare \(p\) si ottiene congiungendo i punti \(T_1\) e \(T_2\) di tangenza delle tangenti per \(P\).}

 \dfn{Centro e diametri di una conica}{Si dice \textbf{centro} di una conica generale di \(\tilde{\AA}_{2}(\CC) \) il polo della retta impropria. Si dicono diametri di una conica generale le rette polari dei punti impropri.}
\paragraph{Osservazione:} Per il principio di reciprocità i diametri passano per il centro della conica. Quindi sono il fascio proprio (se c'è proprio) di rette per \(C\).

Per determinare le coordinate del centro dobbiamo scegliere due punti \(X_{\infty} = [(1, 0, 0)]\), punto improprio dell'asse \(x\), e \(Y_{\infty} = [(0,1,0)]\), punto improprio dell'asse \(y\). La polare di \(X_{\infty}\) è \[
    (1,0,0)
\begin{pmatrix}
    a_{11} & a_{12} & a_{13} \\
    a_{12} & a_{22} & a_{23} \\
    a_{13} & a_{23} & a_{33} \\
\end{pmatrix}
\begin{pmatrix} x_1 \\ x_2\\ x_3 \end{pmatrix} = 0 \qquad (a_{11}, a_{12}, a_{13}) \begin{pmatrix} x_1 \\ x_2\\ x_3 \end{pmatrix} = a_{11}x_1+a_{12}x_2 + a_{13} x_3 = 0
\] Analogamente la polare di \(y_{\infty}\) è  \[
a_{12}x_1 + a_{22}x_2 + a_{23}x_3 = 0
\] \[
\begin{cases}
    \ a_{11}x_1+a_{12}x_2+a_{13}x_3 = 0 \qquad P_1 \\
    \ a_{12}x_1+a_{22}x_2+a_{23}x_3=0 \qquad P_2 \\
\end{cases}
\] 
Il centro \(C\) è proprio se \(P_1\) e \(P_2\) non sono paralleli. Se \[
\begin{dmatrix}
    a_{11} & a_{12} \\
    a_{12} & a_{22} \\
\end{dmatrix} = |A^{*}| \neq 0
\] Il centro è un punto proprio. Quindi il centro è un punto proprio se \(C\) è un ellisse o un'iperbole. Quindi in questo caso i diametri sono un fascio proprio di rette di centro \(C\). \[
F: \ \lambda (a_{11}x_1+a_{12}x_2+a_{13}x_3) + \mu (a_{12}x_1 + a_{22}x_2 + a_{23}x_3) = 0
\] \textbf{Equazione del fascio dei diametri.} Se \(C\) è una parabola \(\implies |A^{*}| = 0 \implies P_1 \) parallelo a \(P_2 \implies  \) il centro è un punto improprio. \(\implies \) i diametri formano un fascio improprio di equazione \[
a_{11}x_1+a_{12}x_2 + kx_3 = 0 \quad \text{con} \quad k \in \CC
\] fascio improprio dei diametri della parabola.
\end{document}
