\documentclass[twoside]{report}

\input{preamble}
\input{macros}
\input{letterfonts}

\title{\textbf{\Huge{Appunti di}\\Fisica Sperimentale}}
\author{\Large{Ayman Marpicati}}
\date{\normalsize{Università degli Studi di Brescia}\\A.A. 2022/2023}
\setlength\parindent{0pt}

\begin{document}


\maketitle
\cleardoublepage
% \pdfbookmark[<level>]{<title>}{<dest>}
\pdfbookmark[section]{\contentsname}{toc}
\tableofcontents
\null\newpage

\setlength{\headheight}{15pt}

\pagestyle{fancy}
%... Then configure it.
\fancyhead{} % clear all header fields
\fancyhead[LO]{\rightmark}
\fancyhead[RO]{\thepage}
\fancyhead[RE]{\leftmark}
\fancyhead[LE]{\thepage}
\fancyfoot{} % clear all footer fields

\chapter{Introduzione}
\section{Le grandezze fisiche}
In fisica esistono delle grandezze fondamentali che ci permettono di esprimere le altre grandezze in funzione delle prime. Infatti esiste una branca della fisica, cioè la \textbf{metrologia}, che si occupa della definizione delle grandezze fondamentali.
\subsubsection{Tempo}
Per misurare il tempo utilizziamo i \textbf{secondi} (\(s\)). Per moltissimi anni il secondo era stato definito come \(1s = 1/86400 d\), tuttavia non era sufficiente ai fini della precisione, quindi dagli anni '50 è stato definito il secondo in base all'oscillazione di un atomo di Cesio.

\subsubsection{Lunghezza}
Per misurare la lunghezza utilizziamo il \textbf{metro} (\(m\)). Inizialmente il metro campione era detenuto dalla Francia ed era definito attraverso la lunghezza del meridiano che passava per Parigi, successivamente venne definito il metro attraverso un interferometro, dal 1980 invece il metro viene definito attraverso la velocità della luce, cioè \(1m = 1/c\), dove \(c = 2.99792458 \cdot 10^{8}m/s\).

\subsubsection{Massa}
Per misurare la massa utilizziamo il \textbf{chilogrammo} (\(kg\)). Inizialmente era definito attraverso il peso di un decimetro cubo di acqua distillata a 4 gradi, ma nel 2019 è stato ridefinito attraverso la costante di Plank per questioni di precisione.



\end{document}
