\documentclass[twoside]{report}

%%%%%%%%%%%%%%%%%%%%%%%%%%%%%%%%%
% PACKAGE IMPORTS
%%%%%%%%%%%%%%%%%%%%%%%%%%%%%%%%%


\usepackage[tmargin=2cm,rmargin=1in,lmargin=1in,margin=0.85in,bmargin=2cm,footskip=.2in]{geometry}
\usepackage{amsmath,amsfonts,amsthm,amssymb,mathtools}
\usepackage[varbb]{newpxmath}
\usepackage{xfrac}
\usepackage[makeroom]{cancel}
\usepackage{mathtools}
\usepackage{bookmark}
\usepackage{enumitem}
\usepackage{hyperref,theoremref}
\hypersetup{
	pdftitle={Assignment},
	colorlinks=true, linkcolor=doc!90,
	bookmarksnumbered=true,
	bookmarksopen=true
}
\usepackage{pgfplots}
\usepackage[most,many,breakable]{tcolorbox}
\usepackage{xcolor}
\usepackage{varwidth}
\usepackage{varwidth}
\usepackage{etoolbox}
%\usepackage{authblk}
\usepackage{nameref}
\usepackage{multicol,array}
\usepackage{tikz-cd}
\usepackage[ruled,vlined,linesnumbered]{algorithm2e}
\usepackage{comment} % enables the use of multi-line comments (\ifx \fi) 
\usepackage{import}
\usepackage{xifthen}
\usepackage{pdfpages}
\usepackage{transparent}
\usepackage{fancyhdr}

\usepackage[italian]{babel}

\newcommand\mycommfont[1]{\footnotesize\ttfamily\textcolor{blue}{#1}}
\SetCommentSty{mycommfont}
\newcommand{\incfig}[1]{%
    \def\svgwidth{\columnwidth}
    \import{./figures/}{#1.pdf_tex}
}

\usepackage{tikzsymbols}
\renewcommand\qedsymbol{$\Laughey$}

\usepackage{import}
\usepackage{pdfpages}
\usepackage{transparent}
\usepackage{xcolor}

\newcommand{\incfig}[2][1]{%
    \def\svgwidth{#1\columnwidth}
    \import{./figures/}{#2.pdf_tex}
}

\pdfsuppresswarningpagegroup=1
%\usepackage{import}
%\usepackage{xifthen}
%\usepackage{pdfpages}
%\usepackage{transparent}


%%%%%%%%%%%%%%%%%%%%%%%%%%%%%%
% SELF MADE COLORS
%%%%%%%%%%%%%%%%%%%%%%%%%%%%%%



\definecolor{myg}{RGB}{56, 140, 70}
\definecolor{myb}{RGB}{45, 111, 177}
\definecolor{myr}{RGB}{199, 68, 64}
\definecolor{mytheorembg}{HTML}{F2F2F9}
\definecolor{mytheoremfr}{HTML}{00007B}
\definecolor{mylenmabg}{HTML}{FFFAF8}
\definecolor{mylenmafr}{HTML}{983b0f}
\definecolor{mypropbg}{HTML}{f2fbfc}
\definecolor{mypropfr}{HTML}{191971}
\definecolor{myexamplebg}{HTML}{F2FBF8}
\definecolor{myexamplefr}{HTML}{88D6D1}
\definecolor{myexampleti}{HTML}{2A7F7F}
\definecolor{mydefinitbg}{HTML}{E5E5FF}
\definecolor{mydefinitfr}{HTML}{3F3FA3}
\definecolor{notesgreen}{RGB}{0,162,0}
\definecolor{myp}{RGB}{197, 92, 212}
\definecolor{mygr}{HTML}{2C3338}
\definecolor{myred}{RGB}{127,0,0}
\definecolor{myyellow}{RGB}{169,121,69}
\definecolor{myexercisebg}{HTML}{F2FBF8}
\definecolor{myexercisefg}{HTML}{88D6D1}


%%%%%%%%%%%%%%%%%%%%%%%%%%%%
% TCOLORBOX SETUPS
%%%%%%%%%%%%%%%%%%%%%%%%%%%%

\setlength{\parindent}{1cm}
%================================
% THEOREM BOX
%================================

\tcbuselibrary{theorems,skins,hooks}
\newtcbtheorem[number within=section]{Theorem}{Teorema}
{%
	enhanced,
	breakable,
	colback = mytheorembg,
	frame hidden,
	boxrule = 0sp,
	borderline west = {2pt}{0pt}{mytheoremfr},
	sharp corners,
	detach title,
	before upper = \tcbtitle\par\smallskip,
	coltitle = mytheoremfr,
	fonttitle = \bfseries\sffamily,
	description font = \mdseries,
	separator sign none,
	segmentation style={solid, mytheoremfr},
}
{th}

\tcbuselibrary{theorems,skins,hooks}
\newtcbtheorem[number within=chapter]{theorem}{Teorema}
{%
	enhanced,
	breakable,
	colback = mytheorembg,
	frame hidden,
	boxrule = 0sp,
	borderline west = {2pt}{0pt}{mytheoremfr},
	sharp corners,
	detach title,
	before upper = \tcbtitle\par\smallskip,
	coltitle = mytheoremfr,
	fonttitle = \bfseries\sffamily,
	description font = \mdseries,
	separator sign none,
	segmentation style={solid, mytheoremfr},
}
{th}


\tcbuselibrary{theorems,skins,hooks}
\newtcolorbox{Theoremcon}
{%
	enhanced
	,breakable
	,colback = mytheorembg
	,frame hidden
	,boxrule = 0sp
	,borderline west = {2pt}{0pt}{mytheoremfr}
	,sharp corners
	,description font = \mdseries
	,separator sign none
}

%================================
% Corollery
%================================
\tcbuselibrary{theorems,skins,hooks}
\newtcbtheorem[number within=section]{Corollary}{Corollario}
{%
	enhanced
	,breakable
	,colback = myp!10
	,frame hidden
	,boxrule = 0sp
	,borderline west = {2pt}{0pt}{myp!85!black}
	,sharp corners
	,detach title
	,before upper = \tcbtitle\par\smallskip
	,coltitle = myp!85!black
	,fonttitle = \bfseries\sffamily
	,description font = \mdseries
	,separator sign none
	,segmentation style={solid, myp!85!black}
}
{th}
\tcbuselibrary{theorems,skins,hooks}
\newtcbtheorem[number within=chapter]{corollary}{Corollario}
{%
	enhanced
	,breakable
	,colback = myp!10
	,frame hidden
	,boxrule = 0sp
	,borderline west = {2pt}{0pt}{myp!85!black}
	,sharp corners
	,detach title
	,before upper = \tcbtitle\par\smallskip
	,coltitle = myp!85!black
	,fonttitle = \bfseries\sffamily
	,description font = \mdseries
	,separator sign none
	,segmentation style={solid, myp!85!black}
}
{th}


%================================
% LENMA
%================================

\tcbuselibrary{theorems,skins,hooks}
\newtcbtheorem[number within=section]{Lenma}{Lemma}
{%
	enhanced,
	breakable,
	colback = mylenmabg,
	frame hidden,
	boxrule = 0sp,
	borderline west = {2pt}{0pt}{mylenmafr},
	sharp corners,
	detach title,
	before upper = \tcbtitle\par\smallskip,
	coltitle = mylenmafr,
	fonttitle = \bfseries\sffamily,
	description font = \mdseries,
	separator sign none,
	segmentation style={solid, mylenmafr},
}
{th}

\tcbuselibrary{theorems,skins,hooks}
\newtcbtheorem[number within=chapter]{lenma}{Lemma}
{%
	enhanced,
	breakable,
	colback = mylenmabg,
	frame hidden,
	boxrule = 0sp,
	borderline west = {2pt}{0pt}{mylenmafr},
	sharp corners,
	detach title,
	before upper = \tcbtitle\par\smallskip,
	coltitle = mylenmafr,
	fonttitle = \bfseries\sffamily,
	description font = \mdseries,
	separator sign none,
	segmentation style={solid, mylenmafr},
}
{th}


%================================
% PROPOSITION
%================================

\tcbuselibrary{theorems,skins,hooks}
\newtcbtheorem[number within=section]{Prop}{Proposizione}
{%
	enhanced,
	breakable,
	colback = mypropbg,
	frame hidden,
	boxrule = 0sp,
	borderline west = {2pt}{0pt}{mypropfr},
	sharp corners,
	detach title,
	before upper = \tcbtitle\par\smallskip,
	coltitle = mypropfr,
	fonttitle = \bfseries\sffamily,
	description font = \mdseries,
	separator sign none,
	segmentation style={solid, mypropfr},
}
{th}

\tcbuselibrary{theorems,skins,hooks}
\newtcbtheorem[number within=chapter]{prop}{Proposizione}
{%
	enhanced,
	breakable,
	colback = mypropbg,
	frame hidden,
	boxrule = 0sp,
	borderline west = {2pt}{0pt}{mypropfr},
	sharp corners,
	detach title,
	before upper = \tcbtitle\par\smallskip,
	coltitle = mypropfr,
	fonttitle = \bfseries\sffamily,
	description font = \mdseries,
	separator sign none,
	segmentation style={solid, mypropfr},
}
{th}


%================================
% CLAIM
%================================

\tcbuselibrary{theorems,skins,hooks}
\newtcbtheorem[number within=section]{claim}{Claim}
{%
	enhanced
	,breakable
	,colback = myg!10
	,frame hidden
	,boxrule = 0sp
	,borderline west = {2pt}{0pt}{myg}
	,sharp corners
	,detach title
	,before upper = \tcbtitle\par\smallskip
	,coltitle = myg!85!black
	,fonttitle = \bfseries\sffamily
	,description font = \mdseries
	,separator sign none
	,segmentation style={solid, myg!85!black}
}
{th}



%================================
% Exercise
%================================

\tcbuselibrary{theorems,skins,hooks}
\newtcbtheorem[number within=section]{Exercise}{Esercizio}
{%
	enhanced,
	breakable,
	colback = myexercisebg,
	frame hidden,
	boxrule = 0sp,
	borderline west = {2pt}{0pt}{myexercisefg},
	sharp corners,
	detach title,
	before upper = \tcbtitle\par\smallskip,
	coltitle = myexercisefg,
	fonttitle = \bfseries\sffamily,
	description font = \mdseries,
	separator sign none,
	segmentation style={solid, myexercisefg},
}
{th}

\tcbuselibrary{theorems,skins,hooks}
\newtcbtheorem[number within=chapter]{exercise}{Esercizio}
{%
	enhanced,
	breakable,
	colback = myexercisebg,
	frame hidden,
	boxrule = 0sp,
	borderline west = {2pt}{0pt}{myexercisefg},
	sharp corners,
	detach title,
	before upper = \tcbtitle\par\smallskip,
	coltitle = myexercisefg,
	fonttitle = \bfseries\sffamily,
	description font = \mdseries,
	separator sign none,
	segmentation style={solid, myexercisefg},
}
{th}

%================================
% EXAMPLE BOX
%================================

\newtcbtheorem[number within=section]{Example}{Esempio}
{%
	colback = myexamplebg
	,breakable
	,colframe = myexamplefr
	,coltitle = myexampleti
	,boxrule = 1pt
	,sharp corners
	,detach title
	,before upper=\tcbtitle\par\smallskip
	,fonttitle = \bfseries
	,description font = \mdseries
	,separator sign none
	,description delimiters parenthesis
}
{ex}

\newtcbtheorem[number within=chapter]{example}{Esempio}
{%
	colback = myexamplebg
	,breakable
	,colframe = myexamplefr
	,coltitle = myexampleti
	,boxrule = 1pt
	,sharp corners
	,detach title
	,before upper=\tcbtitle\par\smallskip
	,fonttitle = \bfseries
	,description font = \mdseries
	,separator sign none
	,description delimiters parenthesis
}
{ex}

%================================
% DEFINITION BOX
%================================

\newtcbtheorem[number within=section]{Definition}{Definizione}{enhanced,
	before skip=2mm,after skip=2mm, colback=red!5,colframe=red!80!black,boxrule=0.5mm,
	attach boxed title to top left={xshift=1cm,yshift*=1mm-\tcboxedtitleheight}, varwidth boxed title*=-3cm,
	boxed title style={frame code={
					\path[fill=tcbcolback]
					([yshift=-1mm,xshift=-1mm]frame.north west)
					arc[start angle=0,end angle=180,radius=1mm]
					([yshift=-1mm,xshift=1mm]frame.north east)
					arc[start angle=180,end angle=0,radius=1mm];
					\path[left color=tcbcolback!60!black,right color=tcbcolback!60!black,
						middle color=tcbcolback!80!black]
					([xshift=-2mm]frame.north west) -- ([xshift=2mm]frame.north east)
					[rounded corners=1mm]-- ([xshift=1mm,yshift=-1mm]frame.north east)
					-- (frame.south east) -- (frame.south west)
					-- ([xshift=-1mm,yshift=-1mm]frame.north west)
					[sharp corners]-- cycle;
				},interior engine=empty,
		},
	fonttitle=\bfseries,
	title={#2},#1}{def}
\newtcbtheorem[number within=chapter]{definition}{Definizione}{enhanced,
	before skip=2mm,after skip=2mm, colback=red!5,colframe=red!80!black,boxrule=0.5mm,
	attach boxed title to top left={xshift=1cm,yshift*=1mm-\tcboxedtitleheight}, varwidth boxed title*=-3cm,
	boxed title style={frame code={
					\path[fill=tcbcolback]
					([yshift=-1mm,xshift=-1mm]frame.north west)
					arc[start angle=0,end angle=180,radius=1mm]
					([yshift=-1mm,xshift=1mm]frame.north east)
					arc[start angle=180,end angle=0,radius=1mm];
					\path[left color=tcbcolback!60!black,right color=tcbcolback!60!black,
						middle color=tcbcolback!80!black]
					([xshift=-2mm]frame.north west) -- ([xshift=2mm]frame.north east)
					[rounded corners=1mm]-- ([xshift=1mm,yshift=-1mm]frame.north east)
					-- (frame.south east) -- (frame.south west)
					-- ([xshift=-1mm,yshift=-1mm]frame.north west)
					[sharp corners]-- cycle;
				},interior engine=empty,
		},
	fonttitle=\bfseries,
	title={#2},#1}{def}



%================================
% Solution BOX
%================================

\makeatletter
\newtcbtheorem{question}{Problema}{enhanced,
	breakable,
	colback=white,
	colframe=myb!80!black,
	attach boxed title to top left={yshift*=-\tcboxedtitleheight},
	fonttitle=\bfseries,
	title={#2},
	boxed title size=title,
	boxed title style={%
			sharp corners,
			rounded corners=northwest,
			colback=tcbcolframe,
			boxrule=0pt,
		},
	underlay boxed title={%
			\path[fill=tcbcolframe] (title.south west)--(title.south east)
			to[out=0, in=180] ([xshift=5mm]title.east)--
			(title.center-|frame.east)
			[rounded corners=\kvtcb@arc] |-
			(frame.north) -| cycle;
		},
	#1
}{def}
\makeatother

%================================
% SOLUTION BOX
%================================

\makeatletter
\newtcolorbox{solution}{enhanced,
	breakable,
	colback=white,
	colframe=myg!80!black,
	attach boxed title to top left={yshift*=-\tcboxedtitleheight},
	title=Solution,
	boxed title size=title,
	boxed title style={%
			sharp corners,
			rounded corners=northwest,
			colback=tcbcolframe,
			boxrule=0pt,
		},
	underlay boxed title={%
			\path[fill=tcbcolframe] (title.south west)--(title.south east)
			to[out=0, in=180] ([xshift=5mm]title.east)--
			(title.center-|frame.east)
			[rounded corners=\kvtcb@arc] |-
			(frame.north) -| cycle;
		},
}
\makeatother

%================================
% Question BOX
%================================

\makeatletter
\newtcbtheorem{qstion}{Question}{enhanced,
	breakable,
	colback=white,
	colframe=mygr,
	attach boxed title to top left={yshift*=-\tcboxedtitleheight},
	fonttitle=\bfseries,
	title={#2},
	boxed title size=title,
	boxed title style={%
			sharp corners,
			rounded corners=northwest,
			colback=tcbcolframe,
			boxrule=0pt,
		},
	underlay boxed title={%
			\path[fill=tcbcolframe] (title.south west)--(title.south east)
			to[out=0, in=180] ([xshift=5mm]title.east)--
			(title.center-|frame.east)
			[rounded corners=\kvtcb@arc] |-
			(frame.north) -| cycle;
		},
	#1
}{def}
\makeatother

\newtcbtheorem[number within=chapter]{wconc}{Wrong Concept}{
	breakable,
	enhanced,
	colback=white,
	colframe=myr,
	arc=0pt,
	outer arc=0pt,
	fonttitle=\bfseries\sffamily\large,
	colbacktitle=myr,
	attach boxed title to top left={},
	boxed title style={
			enhanced,
			skin=enhancedfirst jigsaw,
			arc=3pt,
			bottom=0pt,
			interior style={fill=myr}
		},
	#1
}{def}



%================================
% NOTE BOX
%================================


\usetikzlibrary{arrows,calc,shadows.blur}
\tcbuselibrary{skins}
\newtcolorbox{note}[1][]{%
	enhanced jigsaw,
	colback=gray!20!white,%
	colframe=gray!80!black,
	size=small,
	boxrule=1pt,
	title=\textbf{N.B.},
	halign title=flush center,
	coltitle=black,
	breakable,
	drop shadow=black!50!white,
	attach boxed title to top left={xshift=1cm,yshift=-\tcboxedtitleheight/2,yshifttext=-\tcboxedtitleheight/2},
	minipage boxed title=1.5cm,
	boxed title style={%
			colback=white,
			size=fbox,
			boxrule=1pt,
			boxsep=2pt,
			underlay={%
					\coordinate (dotA) at ($(interior.west) + (-0.5pt,0)$);
					\coordinate (dotB) at ($(interior.east) + (0.5pt,0)$);
					\begin{scope}
						\clip (interior.north west) rectangle ([xshift=3ex]interior.east);
						\filldraw [white, blur shadow={shadow opacity=60, shadow yshift=-.75ex}, rounded corners=2pt] (interior.north west) rectangle (interior.south east);
					\end{scope}
					\begin{scope}[gray!80!black]
						\fill (dotA) circle (2pt);
						\fill (dotB) circle (2pt);
					\end{scope}
				},
		},
	#1,
}

%%%%%%%%%%%%%%%%%%%%%%%%%%%%%%
% SELF MADE COMMANDS
%%%%%%%%%%%%%%%%%%%%%%%%%%%%%%


\newcommand{\thm}[2]{\begin{Theorem}{#1}{}#2\end{Theorem}}
\newcommand{\cor}[2]{\begin{Corollary}{#1}{}#2\end{Corollary}}
\newcommand{\mlenma}[2]{\begin{Lenma}{#1}{}#2\end{Lenma}}
\newcommand{\mprop}[2]{\begin{Prop}{#1}{}#2\end{Prop}}
\newcommand{\clm}[3]{\begin{claim}{#1}{#2}#3\end{claim}}
\newcommand{\wc}[2]{\begin{wconc}{#1}{}\setlength{\parindent}{1cm}#2\end{wconc}}
\newcommand{\thmcon}[1]{\begin{Theoremcon}{#1}\end{Theoremcon}}
\newcommand{\ex}[2]{\begin{Example}{#1}{}#2\end{Example}}
\newcommand{\dfn}[2]{\begin{Definition}[colbacktitle=red!75!black]{#1}{}#2\end{Definition}}
\newcommand{\dfnc}[2]{\begin{definition}[colbacktitle=red!75!black]{#1}{}#2\end{definition}}
\newcommand{\qs}[2]{\begin{question}{#1}{}#2\end{question}}
\newcommand{\pf}[2]{\begin{myproof}[#1]#2\end{myproof}}
\newcommand{\nt}[1]{\begin{note}#1\end{note}}

\newcommand*\circled[1]{\tikz[baseline=(char.base)]{
		\node[shape=circle,draw,inner sep=1pt] (char) {#1};}}
\newcommand\getcurrentref[1]{%
	\ifnumequal{\value{#1}}{0}
	{??}
	{\the\value{#1}}%
}
\newcommand{\getCurrentSectionNumber}{\getcurrentref{section}}
\newenvironment{myproof}[1][\proofname]{%
	\proof[\bfseries #1: ]%
}{\endproof}

\newcommand{\mclm}[2]{\begin{myclaim}[#1]#2\end{myclaim}}
\newenvironment{myclaim}[1][\claimname]{\proof[\bfseries #1: ]}{}

\newcounter{mylabelcounter}

\makeatletter
\newcommand{\setword}[2]{%
	\phantomsection
	#1\def\@currentlabel{\unexpanded{#1}}\label{#2}%
}
\makeatother




\tikzset{
	symbol/.style={
			draw=none,
			every to/.append style={
					edge node={node [sloped, allow upside down, auto=false]{$#1$}}}
		}
}


% deliminators
\DeclarePairedDelimiter{\abs}{\lvert}{\rvert}
\DeclarePairedDelimiter{\norm}{\lVert}{\rVert}

\DeclarePairedDelimiter{\ceil}{\lceil}{\rceil}
\DeclarePairedDelimiter{\floor}{\lfloor}{\rfloor}
\DeclarePairedDelimiter{\round}{\lfloor}{\rceil}

\newsavebox\diffdbox
\newcommand{\slantedromand}{{\mathpalette\makesl{d}}}
\newcommand{\makesl}[2]{%
\begingroup
\sbox{\diffdbox}{$\mathsurround=0pt#1\mathrm{#2}$}%
\pdfsave
\pdfsetmatrix{1 0 0.2 1}%
\rlap{\usebox{\diffdbox}}%
\pdfrestore
\hskip\wd\diffdbox
\endgroup
}
\newcommand{\dd}[1][]{\ensuremath{\mathop{}\!\ifstrempty{#1}{%
\slantedromand\@ifnextchar^{\hspace{0.2ex}}{\hspace{0.1ex}}}%
{\slantedromand\hspace{0.2ex}^{#1}}}}
\ProvideDocumentCommand\dv{o m g}{%
  \ensuremath{%
    \IfValueTF{#3}{%
      \IfNoValueTF{#1}{%
        \frac{\dd #2}{\dd #3}%
      }{%
        \frac{\dd^{#1} #2}{\dd #3^{#1}}%
      }%
    }{%
      \IfNoValueTF{#1}{%
        \frac{\dd}{\dd #2}%
      }{%
        \frac{\dd^{#1}}{\dd #2^{#1}}%
      }%
    }%
  }%
}
\providecommand*{\pdv}[3][]{\frac{\partial^{#1}#2}{\partial#3^{#1}}}
%  - others
\DeclareMathOperator{\Lap}{\mathcal{L}}
\DeclareMathOperator{\Var}{Var} % varience
\DeclareMathOperator{\Cov}{Cov} % covarience
\DeclareMathOperator{\E}{E} % expected

% Since the amsthm package isn't loaded

% I prefer the slanted \leq
\let\oldleq\leq % save them in case they're every wanted
\let\oldgeq\geq
\renewcommand{\leq}{\leqslant}
\renewcommand{\geq}{\geqslant}

% % redefine matrix env to allow for alignment, use r as default
% \renewcommand*\env@matrix[1][r]{\hskip -\arraycolsep
%     \let\@ifnextchar\new@ifnextchar
%     \array{*\c@MaxMatrixCols #1}}


%\usepackage{framed}
%\usepackage{titletoc}
%\usepackage{etoolbox}
%\usepackage{lmodern}


%\patchcmd{\tableofcontents}{\contentsname}{\sffamily\contentsname}{}{}

%\renewenvironment{leftbar}
%{\def\FrameCommand{\hspace{6em}%
%		{\color{myyellow}\vrule width 2pt depth 6pt}\hspace{1em}}%
%	\MakeFramed{\parshape 1 0cm \dimexpr\textwidth-6em\relax\FrameRestore}\vskip2pt%
%}
%{\endMakeFramed}

%\titlecontents{chapter}
%[0em]{\vspace*{2\baselineskip}}
%{\parbox{4.5em}{%
%		\hfill\Huge\sffamily\bfseries\color{myred}\thecontentspage}%
%	\vspace*{-2.3\baselineskip}\leftbar\textsc{\small\chaptername~\thecontentslabel}\\\sffamily}
%{}{\endleftbar}
%\titlecontents{section}
%[8.4em]
%{\sffamily\contentslabel{3em}}{}{}
%{\hspace{0.5em}\nobreak\itshape\color{myred}\contentspage}
%\titlecontents{subsection}
%[8.4em]
%{\sffamily\contentslabel{3em}}{}{}  
%{\hspace{0.5em}\nobreak\itshape\color{myred}\contentspage}



%%%%%%%%%%%%%%%%%%%%%%%%%%%%%%%%%%%%%%%%%%%
% TABLE OF CONTENTS
%%%%%%%%%%%%%%%%%%%%%%%%%%%%%%%%%%%%%%%%%%%

\usepackage{tikz}
\definecolor{doc}{RGB}{0,60,110}
\usepackage{titletoc}
\contentsmargin{0cm}
\titlecontents{chapter}[3.7pc]
{\addvspace{30pt}%
	\begin{tikzpicture}[remember picture, overlay]%
		\draw[fill=doc!60,draw=doc!60] (-7,-.1) rectangle (-0.9,.5);%
		\pgftext[left,x=-3.5cm,y=0.2cm]{\color{white}\Large\sc\bfseries Capitolo\ \thecontentslabel};%
	\end{tikzpicture}\color{doc!60}\large\sc\bfseries}%
{}
{}
{\;\titlerule\;\large\sc\bfseries Pagina \thecontentspage
	\begin{tikzpicture}[remember picture, overlay]
		\draw[fill=doc!60,draw=doc!60] (2pt,0) rectangle (4,0.1pt);
	\end{tikzpicture}}%
\titlecontents{section}[3.7pc]
{\addvspace{2pt}}
{\contentslabel[\thecontentslabel]{2pc}}
{}
{\hfill\small \thecontentspage}
[]
\titlecontents*{subsection}[3.7pc]
{\addvspace{-1pt}\small}
{}
{}
{\ --- \small\thecontentspage}
[ \textbullet\ ][]

\makeatletter
\renewcommand{\tableofcontents}{%
	\chapter*{%
	  \vspace*{-20\p@}%
	  \begin{tikzpicture}[remember picture, overlay]%
		  \pgftext[right,x=15cm,y=0.2cm]{\color{doc!60}\Huge\sc\bfseries \contentsname};%
		  \draw[fill=doc!60,draw=doc!60] (13,-.75) rectangle (20,1);%
		  \clip (13,-.75) rectangle (20,1);
		  \pgftext[right,x=15cm,y=0.2cm]{\color{white}\Huge\sc\bfseries \contentsname};%
	  \end{tikzpicture}}%
	\@starttoc{toc}}
\makeatother


%From M275 "Topology" at SJSU
\newcommand{\id}{\mathrm{id}}
\newcommand{\taking}[1]{\xrightarrow{#1}}
\newcommand{\inv}{^{-1}}

%From M170 "Introduction to Graph Theory" at SJSU
\DeclareMathOperator{\diam}{diam}
\DeclareMathOperator{\ord}{ord}
\newcommand{\defeq}{\overset{\mathrm{def}}{=}}

%From the USAMO .tex files
\newcommand{\ts}{\textsuperscript}
\newcommand{\dg}{^\circ}
\newcommand{\ii}{\item}

% % From Math 55 and Math 145 at Harvard
% \newenvironment{subproof}[1][Proof]{%
% \begin{proof}[#1] \renewcommand{\qedsymbol}{$\blacksquare$}}%
% {\end{proof}}

\newcommand{\liff}{\leftrightarrow}
\newcommand{\lthen}{\rightarrow}
\newcommand{\opname}{\operatorname}
\newcommand{\surjto}{\twoheadrightarrow}
\newcommand{\injto}{\hookrightarrow}
\newcommand{\On}{\mathrm{On}} % ordinals
\DeclareMathOperator{\img}{im} % Image
\DeclareMathOperator{\Img}{Im} % Image
\DeclareMathOperator{\coker}{coker} % Cokernel
\DeclareMathOperator{\Coker}{Coker} % Cokernel
\DeclareMathOperator{\Ker}{Ker} % Kernel
\DeclareMathOperator{\rank}{rank}
\DeclareMathOperator{\Spec}{Spec} % spectrum
\DeclareMathOperator{\Tr}{Tr} % trace
\DeclareMathOperator{\pr}{pr} % projection
\DeclareMathOperator{\ext}{ext} % extension
\DeclareMathOperator{\pred}{pred} % predecessor
\DeclareMathOperator{\dom}{dom} % domain
\DeclareMathOperator{\ran}{ran} % range
\DeclareMathOperator{\Hom}{Hom} % homomorphism
\DeclareMathOperator{\Mor}{Mor} % morphisms
\DeclareMathOperator{\End}{End} % endomorphism

\newcommand{\eps}{\epsilon}
\newcommand{\veps}{\varepsilon}
\newcommand{\ol}{\overline}
\newcommand{\ul}{\underline}
\newcommand{\wt}{\widetilde}
\newcommand{\wh}{\widehat}
\newcommand{\vocab}[1]{\textbf{\color{blue} #1}}
\providecommand{\half}{\frac{1}{2}}
\newcommand{\dang}{\measuredangle} %% Directed angle
\newcommand{\ray}[1]{\overrightarrow{#1}}
\newcommand{\seg}[1]{\overline{#1}}
\newcommand{\arc}[1]{\wideparen{#1}}
\DeclareMathOperator{\cis}{cis}
\DeclareMathOperator*{\lcm}{lcm}
\DeclareMathOperator*{\argmin}{arg min}
\DeclareMathOperator*{\argmax}{arg max}
\newcommand{\cycsum}{\sum_{\mathrm{cyc}}}
\newcommand{\symsum}{\sum_{\mathrm{sym}}}
\newcommand{\cycprod}{\prod_{\mathrm{cyc}}}
\newcommand{\symprod}{\prod_{\mathrm{sym}}}
\newcommand{\Qed}{\begin{flushright}\qed\end{flushright}}
\newcommand{\parinn}{\setlength{\parindent}{1cm}}
\newcommand{\parinf}{\setlength{\parindent}{0cm}}
% \newcommand{\norm}{\|\cdot\|}
\newcommand{\inorm}{\norm_{\infty}}
\newcommand{\opensets}{\{V_{\alpha}\}_{\alpha\in I}}
\newcommand{\oset}{V_{\alpha}}
\newcommand{\opset}[1]{V_{\alpha_{#1}}}
\newcommand{\lub}{\text{lub}}
\newcommand{\del}[2]{\frac{\partial #1}{\partial #2}}
\newcommand{\Del}[3]{\frac{\partial^{#1} #2}{\partial^{#1} #3}}
\newcommand{\deld}[2]{\dfrac{\partial #1}{\partial #2}}
\newcommand{\Deld}[3]{\dfrac{\partial^{#1} #2}{\partial^{#1} #3}}
\newcommand{\lm}{\lambda}
\newcommand{\uin}{\mathbin{\rotatebox[origin=c]{90}{$\in$}}}
\newcommand{\usubset}{\mathbin{\rotatebox[origin=c]{90}{$\subset$}}}
\newcommand{\lt}{\left}
\newcommand{\rt}{\right}
\newcommand{\bs}[1]{\boldsymbol{#1}}
\newcommand{\exs}{\exists}
\newcommand{\st}{\strut}
\newcommand{\dps}[1]{\displaystyle{#1}}

\newcommand{\sol}{\setlength{\parindent}{0cm}\textbf{\textit{Solution:}}\setlength{\parindent}{1cm} }
\newcommand{\solve}[1]{\setlength{\parindent}{0cm}\textbf{\textit{Solution: }}\setlength{\parindent}{1cm}#1 \Qed}

% Things Lie
\newcommand{\kb}{\mathfrak b}
\newcommand{\kg}{\mathfrak g}
\newcommand{\kh}{\mathfrak h}
\newcommand{\kn}{\mathfrak n}
\newcommand{\ku}{\mathfrak u}
\newcommand{\kz}{\mathfrak z}
\DeclareMathOperator{\Ext}{Ext} % Ext functor
\DeclareMathOperator{\Tor}{Tor} % Tor functor
\newcommand{\gl}{\opname{\mathfrak{gl}}} % frak gl group
\renewcommand{\sl}{\opname{\mathfrak{sl}}} % frak sl group chktex 6

% More script letters etc.
\newcommand{\SA}{\mathcal A}
\newcommand{\SB}{\mathcal B}
\newcommand{\SC}{\mathcal C}
\newcommand{\SF}{\mathcal F}
\newcommand{\SG}{\mathcal G}
\newcommand{\SH}{\mathcal H}
\newcommand{\OO}{\mathcal O}

\newcommand{\SCA}{\mathscr A}
\newcommand{\SCB}{\mathscr B}
\newcommand{\SCC}{\mathscr C}
\newcommand{\SCD}{\mathscr D}
\newcommand{\SCE}{\mathscr E}
\newcommand{\SCF}{\mathscr F}
\newcommand{\SCG}{\mathscr G}
\newcommand{\SCH}{\mathscr H}

% Mathfrak primes
\newcommand{\km}{\mathfrak m}
\newcommand{\kp}{\mathfrak p}
\newcommand{\kq}{\mathfrak q}

% number sets
\newcommand{\RR}[1][]{\ensuremath{\ifstrempty{#1}{\mathbb{R}}{\mathbb{R}^{#1}}}}
\newcommand{\NN}[1][]{\ensuremath{\ifstrempty{#1}{\mathbb{N}}{\mathbb{N}^{#1}}}}
\newcommand{\ZZ}[1][]{\ensuremath{\ifstrempty{#1}{\mathbb{Z}}{\mathbb{Z}^{#1}}}}
\newcommand{\QQ}[1][]{\ensuremath{\ifstrempty{#1}{\mathbb{Q}}{\mathbb{Q}^{#1}}}}
\newcommand{\CC}[1][]{\ensuremath{\ifstrempty{#1}{\mathbb{C}}{\mathbb{C}^{#1}}}}
\newcommand{\PP}[1][]{\ensuremath{\ifstrempty{#1}{\mathbb{P}}{\mathbb{P}^{#1}}}}
\newcommand{\HH}[1][]{\ensuremath{\ifstrempty{#1}{\mathbb{H}}{\mathbb{H}^{#1}}}}
\newcommand{\FF}[1][]{\ensuremath{\ifstrempty{#1}{\mathbb{F}}{\mathbb{F}^{#1}}}}
% expected value
\newcommand{\EE}{\ensuremath{\mathbb{E}}}
\newcommand{\charin}{\text{ char }}
\DeclareMathOperator{\sign}{sign}
\DeclareMathOperator{\Aut}{Aut}
\DeclareMathOperator{\Inn}{Inn}
\DeclareMathOperator{\Syl}{Syl}
\DeclareMathOperator{\Gal}{Gal}
\DeclareMathOperator{\GL}{GL} % General linear group
\DeclareMathOperator{\SL}{SL} % Special linear group

%---------------------------------------
% BlackBoard Math Fonts :-
%---------------------------------------

%Captital Letters
\newcommand{\bbA}{\mathbb{A}}	\newcommand{\bbB}{\mathbb{B}}
\newcommand{\bbC}{\mathbb{C}}	\newcommand{\bbD}{\mathbb{D}}
\newcommand{\bbE}{\mathbb{E}}	\newcommand{\bbF}{\mathbb{F}}
\newcommand{\bbG}{\mathbb{G}}	\newcommand{\bbH}{\mathbb{H}}
\newcommand{\bbI}{\mathbb{I}}	\newcommand{\bbJ}{\mathbb{J}}
\newcommand{\bbK}{\mathbb{K}}	\newcommand{\bbL}{\mathbb{L}}
\newcommand{\bbM}{\mathbb{M}}	\newcommand{\bbN}{\mathbb{N}}
\newcommand{\bbO}{\mathbb{O}}	\newcommand{\bbP}{\mathbb{P}}
\newcommand{\bbQ}{\mathbb{Q}}	\newcommand{\bbR}{\mathbb{R}}
\newcommand{\bbS}{\mathbb{S}}	\newcommand{\bbT}{\mathbb{T}}
\newcommand{\bbU}{\mathbb{U}}	\newcommand{\bbV}{\mathbb{V}}
\newcommand{\bbW}{\mathbb{W}}	\newcommand{\bbX}{\mathbb{X}}
\newcommand{\bbY}{\mathbb{Y}}	\newcommand{\bbZ}{\mathbb{Z}}

%---------------------------------------
% MathCal Fonts :-
%---------------------------------------

%Captital Letters
\newcommand{\mcA}{\mathcal{A}}	\newcommand{\mcB}{\mathcal{B}}
\newcommand{\mcC}{\mathcal{C}}	\newcommand{\mcD}{\mathcal{D}}
\newcommand{\mcE}{\mathcal{E}}	\newcommand{\mcF}{\mathcal{F}}
\newcommand{\mcG}{\mathcal{G}}	\newcommand{\mcH}{\mathcal{H}}
\newcommand{\mcI}{\mathcal{I}}	\newcommand{\mcJ}{\mathcal{J}}
\newcommand{\mcK}{\mathcal{K}}	\newcommand{\mcL}{\mathcal{L}}
\newcommand{\mcM}{\mathcal{M}}	\newcommand{\mcN}{\mathcal{N}}
\newcommand{\mcO}{\mathcal{O}}	\newcommand{\mcP}{\mathcal{P}}
\newcommand{\mcQ}{\mathcal{Q}}	\newcommand{\mcR}{\mathcal{R}}
\newcommand{\mcS}{\mathcal{S}}	\newcommand{\mcT}{\mathcal{T}}
\newcommand{\mcU}{\mathcal{U}}	\newcommand{\mcV}{\mathcal{V}}
\newcommand{\mcW}{\mathcal{W}}	\newcommand{\mcX}{\mathcal{X}}
\newcommand{\mcY}{\mathcal{Y}}	\newcommand{\mcZ}{\mathcal{Z}}


%---------------------------------------
% Bold Math Fonts :-
%---------------------------------------

%Captital Letters
\newcommand{\bmA}{\boldsymbol{A}}	\newcommand{\bmB}{\boldsymbol{B}}
\newcommand{\bmC}{\boldsymbol{C}}	\newcommand{\bmD}{\boldsymbol{D}}
\newcommand{\bmE}{\boldsymbol{E}}	\newcommand{\bmF}{\boldsymbol{F}}
\newcommand{\bmG}{\boldsymbol{G}}	\newcommand{\bmH}{\boldsymbol{H}}
\newcommand{\bmI}{\boldsymbol{I}}	\newcommand{\bmJ}{\boldsymbol{J}}
\newcommand{\bmK}{\boldsymbol{K}}	\newcommand{\bmL}{\boldsymbol{L}}
\newcommand{\bmM}{\boldsymbol{M}}	\newcommand{\bmN}{\boldsymbol{N}}
\newcommand{\bmO}{\boldsymbol{O}}	\newcommand{\bmP}{\boldsymbol{P}}
\newcommand{\bmQ}{\boldsymbol{Q}}	\newcommand{\bmR}{\boldsymbol{R}}
\newcommand{\bmS}{\boldsymbol{S}}	\newcommand{\bmT}{\boldsymbol{T}}
\newcommand{\bmU}{\boldsymbol{U}}	\newcommand{\bmV}{\boldsymbol{V}}
\newcommand{\bmW}{\boldsymbol{W}}	\newcommand{\bmX}{\boldsymbol{X}}
\newcommand{\bmY}{\boldsymbol{Y}}	\newcommand{\bmZ}{\boldsymbol{Z}}
%Small Letters
\newcommand{\bma}{\boldsymbol{a}}	\newcommand{\bmb}{\boldsymbol{b}}
\newcommand{\bmc}{\boldsymbol{c}}	\newcommand{\bmd}{\boldsymbol{d}}
\newcommand{\bme}{\boldsymbol{e}}	\newcommand{\bmf}{\boldsymbol{f}}
\newcommand{\bmg}{\boldsymbol{g}}	\newcommand{\bmh}{\boldsymbol{h}}
\newcommand{\bmi}{\boldsymbol{i}}	\newcommand{\bmj}{\boldsymbol{j}}
\newcommand{\bmk}{\boldsymbol{k}}	\newcommand{\bml}{\boldsymbol{l}}
\newcommand{\bmm}{\boldsymbol{m}}	\newcommand{\bmn}{\boldsymbol{n}}
\newcommand{\bmo}{\boldsymbol{o}}	\newcommand{\bmp}{\boldsymbol{p}}
\newcommand{\bmq}{\boldsymbol{q}}	\newcommand{\bmr}{\boldsymbol{r}}
\newcommand{\bms}{\boldsymbol{s}}	\newcommand{\bmt}{\boldsymbol{t}}
\newcommand{\bmu}{\boldsymbol{u}}	\newcommand{\bmv}{\boldsymbol{v}}
\newcommand{\bmw}{\boldsymbol{w}}	\newcommand{\bmx}{\boldsymbol{x}}
\newcommand{\bmy}{\boldsymbol{y}}	\newcommand{\bmz}{\boldsymbol{z}}

%---------------------------------------
% Scr Math Fonts :-
%---------------------------------------

\newcommand{\sA}{{\mathscr{A}}}   \newcommand{\sB}{{\mathscr{B}}}
\newcommand{\sC}{{\mathscr{C}}}   \newcommand{\sD}{{\mathscr{D}}}
\newcommand{\sE}{{\mathscr{E}}}   \newcommand{\sF}{{\mathscr{F}}}
\newcommand{\sG}{{\mathscr{G}}}   \newcommand{\sH}{{\mathscr{H}}}
\newcommand{\sI}{{\mathscr{I}}}   \newcommand{\sJ}{{\mathscr{J}}}
\newcommand{\sK}{{\mathscr{K}}}   \newcommand{\sL}{{\mathscr{L}}}
\newcommand{\sM}{{\mathscr{M}}}   \newcommand{\sN}{{\mathscr{N}}}
\newcommand{\sO}{{\mathscr{O}}}   \newcommand{\sP}{{\mathscr{P}}}
\newcommand{\sQ}{{\mathscr{Q}}}   \newcommand{\sR}{{\mathscr{R}}}
\newcommand{\sS}{{\mathscr{S}}}   \newcommand{\sT}{{\mathscr{T}}}
\newcommand{\sU}{{\mathscr{U}}}   \newcommand{\sV}{{\mathscr{V}}}
\newcommand{\sW}{{\mathscr{W}}}   \newcommand{\sX}{{\mathscr{X}}}
\newcommand{\sY}{{\mathscr{Y}}}   \newcommand{\sZ}{{\mathscr{Z}}}


%---------------------------------------
% Math Fraktur Font
%---------------------------------------

%Captital Letters
\newcommand{\mfA}{\mathfrak{A}}	\newcommand{\mfB}{\mathfrak{B}}
\newcommand{\mfC}{\mathfrak{C}}	\newcommand{\mfD}{\mathfrak{D}}
\newcommand{\mfE}{\mathfrak{E}}	\newcommand{\mfF}{\mathfrak{F}}
\newcommand{\mfG}{\mathfrak{G}}	\newcommand{\mfH}{\mathfrak{H}}
\newcommand{\mfI}{\mathfrak{I}}	\newcommand{\mfJ}{\mathfrak{J}}
\newcommand{\mfK}{\mathfrak{K}}	\newcommand{\mfL}{\mathfrak{L}}
\newcommand{\mfM}{\mathfrak{M}}	\newcommand{\mfN}{\mathfrak{N}}
\newcommand{\mfO}{\mathfrak{O}}	\newcommand{\mfP}{\mathfrak{P}}
\newcommand{\mfQ}{\mathfrak{Q}}	\newcommand{\mfR}{\mathfrak{R}}
\newcommand{\mfS}{\mathfrak{S}}	\newcommand{\mfT}{\mathfrak{T}}
\newcommand{\mfU}{\mathfrak{U}}	\newcommand{\mfV}{\mathfrak{V}}
\newcommand{\mfW}{\mathfrak{W}}	\newcommand{\mfX}{\mathfrak{X}}
\newcommand{\mfY}{\mathfrak{Y}}	\newcommand{\mfZ}{\mathfrak{Z}}
%Small Letters
\newcommand{\mfa}{\mathfrak{a}}	\newcommand{\mfb}{\mathfrak{b}}
\newcommand{\mfc}{\mathfrak{c}}	\newcommand{\mfd}{\mathfrak{d}}
\newcommand{\mfe}{\mathfrak{e}}	\newcommand{\mff}{\mathfrak{f}}
\newcommand{\mfg}{\mathfrak{g}}	\newcommand{\mfh}{\mathfrak{h}}
\newcommand{\mfi}{\mathfrak{i}}	\newcommand{\mfj}{\mathfrak{j}}
\newcommand{\mfk}{\mathfrak{k}}	\newcommand{\mfl}{\mathfrak{l}}
\newcommand{\mfm}{\mathfrak{m}}	\newcommand{\mfn}{\mathfrak{n}}
\newcommand{\mfo}{\mathfrak{o}}	\newcommand{\mfp}{\mathfrak{p}}
\newcommand{\mfq}{\mathfrak{q}}	\newcommand{\mfr}{\mathfrak{r}}
\newcommand{\mfs}{\mathfrak{s}}	\newcommand{\mft}{\mathfrak{t}}
\newcommand{\mfu}{\mathfrak{u}}	\newcommand{\mfv}{\mathfrak{v}}
\newcommand{\mfw}{\mathfrak{w}}	\newcommand{\mfx}{\mathfrak{x}}
\newcommand{\mfy}{\mathfrak{y}}	\newcommand{\mfz}{\mathfrak{z}}


\title{\Huge{Appunti di}\\Algebra Lineare e Geometria Analitica}
\author{\Large{Ayman Marpicati}}
\date{\normalsize{Università degli Studi di Brescia}\\A.A. 2022/2023}
\setlength\parindent{0pt}

\begin{document}


\maketitle
\cleardoublepage
% \pdfbookmark[<level>]{<title>}{<dest>}
\pdfbookmark[section]{\contentsname}{toc}
\tableofcontents
\null\newpage

\setlength{\headheight}{15pt}

\pagestyle{fancy}
%... Then configure it.
\fancyhead{} % clear all header fields
\fancyhead[LO]{\rightmark}
\fancyhead[RO]{\thepage}
\fancyhead[RE]{\leftmark}
\fancyhead[LE]{\thepage}
\fancyfoot{} % clear all footer fields


\chapter{Nozioni preliminari}
\section{Relazioni su un insieme}
\dfn{Relazione su un insieme}{Una \textbf{relazione} su un insieme \textit{A} è un qualunque sottoinsieme di \(\mcR\) del prodotto cartesiano \(A \times A\).

Una relazione \(\mcR\) su un insieme \textit{A} si dice:
\begin{itemize}
    \item \textbf{riflessiva} se, per ogni \(a \in A, \ a\mcR a\);
    \item \textbf{simmetrica} se, per ogni \(a,b \in A, \ a\mcR b \ \text{allora} \ a = b\);
    \item \textbf{antisimmetrica} se, per ogni \(a,b \in A, \ a\mcR b \text{ e } b\mcR a \text{ allora } a = b\);
    \item \textbf{transitiva} se, per ogni \(a,b,c \in A, \ a\mcR b \text{ e } b\mcR c \text{ allora } a \mcR c\);
\end{itemize}}

\dfn{Relazione d'ordine totale}{Una relazione d'ordine \(\mcR\) su un insieme \textit{A} si dice \textbf{relazione d'ordine} se è riflessiva, antisimmetrica e transitiva. Se inoltre, gli elementi di \textit{A} sono a due a due confrontabili, cioè, per ogni \(a, b \in A\), risulta \(a \mcR b\) oppure  \(b \mcR a\), la relazione \(\mcR\) si dice \textbf{relazione d'ordine totale}.}

\section{Strutture algebriche}
\dfn{Gruppo}{Sia \((G, \star)\) un insieme con un'operazione \(\star\). La struttura \((G, \star)\) si dice \textbf{gruppo} se:
\begin{itemize}
    \item l'operazione \(\star\)  è associativa;
    \item esiste in \textit{G} l'elemento neutro;
    \item ogni elemento di \(g \in G\) è simmetrizzabile.  
\end{itemize}
Se l'operazione \(\star\) soddisfa anche la proprietà commutativa, il gruppo si dice \textbf{abeliano}.}

\dfn{Campo}{Sia \textit{A} un insieme sul quale sono definite due operazioni che indichiamo con i simboli "\(+\)" e "\(\cdot\)" e che chiamiamo somma e prodotto rispettivamente. La struttura \((A, +, \cdot)\) è un \textbf{campo} se sussistono le condizioni seguenti:
\begin{itemize}
    \item \((A, +)\) è un gruppo abeliano il cui elemento neutro è indicato con 0;
    \item \((A\backslash\{0\}, \cdot)\) è un gruppo abeliano con elemento neutro \(e \neq 0\);
    \item valgono le proprietà distributive (sinistra e destra) del prodotto rispetto alla somma, cioè per ogni \(a,b,c \in A\) \[
        a \cdot (b + c) = a \cdot b + a \cdot c; \ (a + b) \cdot c = a \cdot c + b \cdot c
    \]
\end{itemize}}

\section{Matrici}

\dfn{Matrice}{Dato un campo K si dice \textbf{matrice} di tipo \(m \times n\) su \(K\) una tabella del tipo: \[ A=
\left( \; \begin{matrix}
    a _{11} & a _{12} & \hdots & a _{1n} \\
    a _{21} & a _{22} & \hdots & a _{2n} \\
    \vdots & \vdots & \ddots & \vdots \\
    a _{m1} & a _{m2} & \hdots & a _{mn} \\
\end{matrix} \; \right)
\] avente \(m\) righe ed \(n\) colonne, i cui elementi \(a _{ij}\) sono elementi di \(K.\) }

\dfn{Matrice quadrata}{Una matrice di tipo \(n \times n\) è detta \textbf{matrice quadrata} di ordine \(n\). Queste vengono indicate con \(M_n(K)\).}

\dfn{Prodotto righe per colonne}{Date le matrici \(A = (a _{ih}) \in K ^{m,n}(K)\) con \(i \in I_m, h \in I_n\) e \(B = (b _{hj}) \in K ^{n,p}\) con \(h \in I_n, j \in I_p\), si dice \textbf{prodotto righe per colonne} di \(A\) per \(B\) la matrice \[
    A \cdot B = (c _{ij}) \text{ con } i \in I_m, \ j \in I_p \qquad \text{ove}
\] \[
    c _{ij} = a _{i1} b _{1j} + a _{i2} b _{2j} + ... + a _{in} b _{nj}= \sum_{h \in I_n} a _{ih} b _{hj}
\]    }

\ex{}{Prendiamo per esempio le due matrici: \[A=
\left( \; \begin{matrix}
    -3 & 0 & 2 \\
    -4 & 7 & 1 \\
\end{matrix} \; \right) \quad B=
\left( \; \begin{matrix}
    -5 & -1 & 2 \\
    0 & 1 & -2 \\
    1 & 1 & 3 \\
\end{matrix} \; \right)\]
 Il loro prodotto è \[
\left( \; \begin{matrix}
    -3 \cdot (-5) + 0 \cdot 0 + 2 \cdot 1 & -3 \cdot (-1) + 0 \cdot 1 + 2 \cdot 1 & -3 \cdot 2 + 0 \cdot (-2) + 2 \cdot 3 \\
    -4 \cdot (-5) + 7 \cdot 0 + 1 \cdot 1 & -4 \cdot (-1) + 7 \cdot 1 + 1 \cdot 1 & -4 \cdot 2 + 7 \cdot (-2) + 1 \cdot 3 \\
\end{matrix} \; \right)
\] Quindi \[
    A \cdot B =
\left( \; \begin{matrix}
    17 & 5 & 0 \\
    21 & 12 & -19 \\
\end{matrix} \; \right)
\]}
\dfn{Matrice identica}{L'elemento neutro delle matrici quadrate di ordine \(n\) è la \textbf{matrice identica}, cioè la matrice: \[
\left( \; \begin{matrix}
    1 & 0 & \hdots & 0 \\
    0 & 1 & \hdots & 0 \\
    \vdots & \vdots & \ddots & \vdots \\
    0 & 0 & \hdots & 1 \\
\end{matrix} \; \right)
\]}

\dfn{Trasposta di una matrice}{Sia \(A = (a _{ij})\) una matrice di \(K ^{m,n}\). Si dice \textbf{trasposta} di \(A\) la matrice \(K^{n,m}\) ottenuta scambiando tra loro le righe con le colonne, cioè \(^{t}A = (b _{ji})\) ove \(b _{ji} = a _{ij}\) per ogni \(i \in I_n\) e \(j \in I_m.\)  }


\chapter{Spazi vettoriali}
\section{Generalità}
\dfn{Spazio vettoriale}{Siano \(K\)  un campo e \(V\) un insieme. Si dice che \(V\) è uno \textbf{spazio vettoriale} sul campo \(K\), se sono definite due operazioni: un'operazione interna binaria su \(V\), detta somma, \(+: V \times V \rightarrow V\) e un'operazione estrema detta prodotto esterno o prodotto per scalari, \(\cdot:K \times V \rightarrow V\), tali che
\begin{itemize}
    \item \((V, +)\) sia un gruppo abeliano;
    \item il prodotto esterno \(\cdot\) soddisfi le seguenti proprietà:
        \begin{itemize}
            \item \( (h \cdot k) \cdot v = h \cdot (k \cdot v) \quad \forall h, k \in K \quad e \quad \forall v \in V \) 
            \item \( (h + k) \cdot v = h \cdot v + k \cdot v \quad \forall h, k \in K \quad e \quad \forall v \in V\)
            \item \(h \cdot (v + w) = h \cdot v + h \cdot w \quad \forall h,k \in K \quad e \quad \forall v, w \in V\)
            \item \(1 \cdot v = v \quad \forall v \in V\)
        \end{itemize}
\end{itemize}}
Gli elementi dell'insieme \textit{V} sono detti \textbf{vettori}, gli elementi del campo \textit{K} sono chiamati \textbf{scalari}. L'elemento neutro di \((V, +)\) è detto \textbf{vettor nullo} e indicato \ul{0} per distinguerlo da 0, zero del campo \textit{K}. L'opposto di ogni vettore \textbf{v} viene indicato con \textbf{-v}.

\thm{}{Sia \textit{V} uno spazio vettoriale sul campo \textit{K}, siano \(k \in K\)  e \(v \in V\). Allora \[
    kv = \ul{0} \iff k = 0 \text{ oppure } v = \ul{0}
\]}

\section{Sottospazi di uno spazio vettoriale}
\dfn{Sottospazio vettoriale}{Sia \(\emptyset \neq U \subseteq V\), diremo che \(U\) è \textbf{sottospazio vettoriale} di \(V\) se è esso stesso uno spazio vettoriale rispetto alla restrizione delle stesse operazioni.}

\mprop{Primo criterio di riconoscimento}{Sia \(V(K)\) uno spazio vettoriale e sia \(\emptyset \neq  U \subseteq V\) un suo sottoinsieme. Il sottoinsieme \(U\) è uno spazio vettoriale di \(V\) se, e soltanto se, sono verificate le seguenti condizioni:
\begin{enumerate}
    \item \(\forall u, u' \in U \quad u + u' \in U\) 
    \item \(\forall k \in K, \ \forall u \in U \quad ku \in U\) 
\end{enumerate}}

\mprop{Secondo criterio di riconoscimento}{Sia \(V(K)\) uno spazio vettoriale sul campo \(K\) e sia \(\emptyset \neq U \subseteq V\), \(U\) è sottospazio di \(V(K)\) se e soltanto se 
   \[
       hv_{1} + kv_{2} \in U \quad \forall v_{1}, v_{2} \in U \quad e \quad h, k \in K
   \] 
}

\section{Indipendenza e dipendenza lineare}
\dfn{Combinazione lineare}{Siano \(v _{1}, v_2, ..., v_n \in V(K)\) si dice combinazione lineare di vettori \(v_1, v_2, ..., v_n\) ogni vettore \(v\): \[
    v = k_1 \cdot v_1 + k_2 \cdot v_2 + ... + k_n \cdot v_n \quad \text{con} \ k_1, k_2, ..., k_n \in K
\] }
\dfn{Sistema di vettori libero}{Sia \(V(K)\) e sia \(A\) un sistema di vettori di \(V(K)\), \(A=[v_1, v_2, ..., v_n]\), allora \(A\) si dice \textbf{libero} se l'unica combinazione lineare di vettori di \(A\) che dà il vettore nullo è a coefficienti tutti nulli \[
    \ul{0} = k_1 \cdot v_1 + k_2 \cdot v_2 + ... + k_n \cdot v_n \implies k_1 = k_2 = ... = k_n = \ul{0}
\]
Se \(A\) è libero i suoi vettori si dicono \textbf{linearmente indipendenti}.}


\dfn{Sistema di vettori legato}{Sia \(V(K)\) e sia \(A\) un sistema di vettori di \(V(K)\), \(A=[v_1, v_2, ..., v_n]\), allora \(A\) si dice \textbf{legato} se \textbf{non} è libero. Quindi:\[
        \exists k_1, k_2, ..., k_n \ \text{non tutti nulli} : \ \ul{0}=k_1 \cdot v_1 + k_2 \cdot v_2 + ... + k_n \cdot v_n
\]
Se \(A\) è legato i suoi vettori si dicono \textbf{linearmente dipendenti}.}
Qui di seguito daremo delle proposizioni riguardo ai sistemi liberi e legati:
\mprop{}{Sia \(A=[v_1, v_2, ..., v_n]\) un sistema di generatori di \(V(K)\). Se \(\ul{0}\) appartiene ad \(A\), il sistema \(A\) è legato.}
\pf{Dimostrazione}{Sia \(\ul{0} \in A\), senza perdita di generalità, possiamo supporre che \(\ul{0} = v_1\) quindi: \[
    1 \cdot v_1 + 0 \cdot v_2 + ... + 0 \cdot v_n = 1 \cdot \ul{0} + \ul{0} = \ul{0} \implies \ \text{A è legato}
\]}
\mprop{}{Sia \(A=[v_1, v_2, ..., v_n]\) un sistema di generatori di \(V(K)\). Se in \(A\) appaiono due vettori proporzionali allora A è legato.}
\pf{Dimostrazione}{Senza perdita di generalità possiamo supporre che \(v_1 = k v_2\) e quindi: \[
    1v_1 + k v_2 + 0v_3 + ... + 0 v_n = v_1 - kv_2 + \ul{0} = \ul{0} \implies \ \text{A è legato}
\]}
\mprop{}{Sia \(A=[v_1, v_2, ..., v_n]\) un sistema di generatori di \(V(K)\). A è legato se e solo se almeno uno dei vettori si può riscrivere come combinazione lineare degli altri.}
\pf{Dimostrazione}{\( \implies \): Per ipotesi \(A\) è legato e quindi: \[
    \ul{0}=k_1v_1 + k_2 v_2 + ... + k_n v_n \ \text{con almeno un } k_i = 0
\] Senza perdita di generalità supponiamo che \(k_1 \neq 0\)
    \begin{gather*} 
    -k_1 v_1 = k_2 v_2 + ... + k_n v_n \qquad v_1 = \frac{1}{k_1} (-k_2 v_2 - ... -k_n v_n) \\
   v_1 = -\frac{k_2}{k_1}v_2 - \frac{k_3}{k_1}v_3 - ... - \frac{k_n}{k1}v_n
\end{gather*}
e quindi \(v_1\) è combinazione lineare di \(v_1, ..., v_n\).\\
\( \impliedby \): Per ipotesi uno dei vettori di \(A\) è combinazione lineare degli altri e senza perdita di generalità: \[
    v_1 = k_2 v_2 + k_3 v_3 + ... + k_n v_n \qquad \ul{0}= -1v_1 + k_2 v_2 + ... + k_n v_n
\] siccome \(-1 \neq 0\) \(A\) è legato. }
\mprop{}{Sia \(A=[v_1, v_2, ..., v_n]\) un sistema di generatori di \(V(K)\) e sia \(u \in V(K)\). Se \(A \cup \{u\}\) è legato, allora \(u\) è combinazione lineare dei vettori di \(A\).}
\pf{Dimostrazione}{Per ipotesi \(A \cup \{u\}\) è legato, cioè: \[
    \exists k_1, k_2, ..., k_n, b \in K \ \text{ non tutti nulli } : \ \ul{0} = k_1v_1 + k_2v_2 +...+k_nv_n + bu
\] sia per assurdo \(b = 0\) \[
    \ul{0}=k_1v_1 + k_2v_2 + ... + k_nv_n \text{ con } k_1 \neq 0 \ \implies \ A \text{ è legato, \textbf{assurdo!} } \implies \ b \neq 0
\] \[
    -bu = k_1v_1 + k_2v_2 + ... + k_nv_n \quad u = -\frac{k_1}{b}v_1 - \frac{k_2}{b}v_2 - ... - \frac{k_n}{b}v_n
\] \( \implies \) \(u\) è combinazione lineare dei vettori \(v_1, v_2, ..., v_n\)  }
\mprop{}{Sia \(A=[v_1, v_2, ..., v_n]\) un sistema di generatori di \(V(K)\) e sia \(B \supseteq A\) sistema di vettori di \(V(K)\). Se \(A\) è legato allora anche \(B\) è legato.}
\pf{Dimostrazione}{\[
    \exists k_1, k_2, ..., k_n \in K \ \text{ non tutti nulli } : \ \ul{0} = k_1v_1 + k_2v_2 +...+k_nv_n
\] Se \(B=[v_1, v_2, ..., v_n, w_1, w_2, ..., w_m]\) allora \[
    \ul{0} = k_1v_1 + k_2v_2 +...+k_nv_n + 0w_1 + 0w_2 + ... + 0w_m
\]\( \implies \ B\) è legato.}
\mprop{}{Sia \(A=[v_1, v_2, ..., v_n]\) un sistema di generatori di \(V(K)\) e sia \(B \subseteq A\) sistema di vettori di \(V(K)\), se \(A\) è libero, allora \(B\) è libero.}
\pf{Dimostrazione}{Sia, per assurdo, \(B\) legato, allora per la proposizione precedente anche \(A\) è legato. \textbf{Assurdo!} Quindi \(B\)  è libero.} 

\section{Sistemi di generatori di uno spazio vettoriale}
\dfn{Sistema di generatori}{Sia \(A\) sistema di vettori di \(V(K)\). \(A\) si dice sistema di generatori di \(V(K)\) se ogni \(v \in V(K)\) si può scrivener come combinazione lineare di un numero finito di vettori di A.}

\dfn{Copertura lineare}{Sia \(A\) un sistema di vettori di \(V(K)\) si dice copertura (o chiusura) lineare di \(A\) l'insieme \(\mcL(A)\) di tutte le combinazioni lineari di sottoinsiemi finiti di A.}
\nt{Dato \(A\) sistema di vettori di \(V(K)\) \begin{enumerate}
    \item \(\mcL(A)\) è il più piccolo sottospazio di \(V(K)\) che contiene \(A\) 
    \item \(\mcL(A) \le V(K)\) 
    \item \(\mcL(\mcL(A)) = \mcL(A)\)
\end{enumerate}}
Ogni spazio vettoriale ammette un sistema di generatori e:
\begin{itemize}
    \item se \(V(K)\) ammette un sistema di generatori finito \( \implies\) \(V(K)\) si dice finitamente generato.
    \item se ogni sistema di generatori di \(V(K)\) ha cardinalità infinita \( \implies\) \(V(K)\) non è finitamente generato.
\end{itemize}

\section{Basi e dimensione}
\mlenma{}{Sia \(S=[v_1, v_2, ..., v_n]\) un sistema di generatori per uno spazio vettoriale \(V(K)\), e sia \(v \in S\) combinazione lineare degli altri vettori (linearmente dipendente dagli altri) \( \implies\) \(S\backslash \{v\} \) è sistema di generatori per \(V(K)\)}
\pf{Dimostrazione}{Sia, senza perdere di generalità, \(v_1\) combinazione lineare di \(v_2, v_3, ..., v_n\) \[
    v_1 = k_2v_2 + k_3v_3 + ... + k_nv_n
\] sia \(v \in V(K)\) \[
    v = h_1v_1 + h_2 v_2 + ... + h_nv_n = h_1(k_2v_2 + ... + k_nv_n) + h_2v_2 + ... + h_nv_n
\] \[
v = \underbrace{(h_1k_2 + h_2)}_{\in K}v_2 + ... + \underbrace{(h_1k_n + h_n)}_{\in K}v_n \ \in \mcL([v_2, v_3, ..., v_n]) = \mcL(S \backslash \{v_1\} )
\] \( \implies \ S \backslash \{v_1\} \) è un sistema di generatori.}

\thm{}{Sia \(V(K)\) uno spazio vettoriale finitamente generato, non banale (\(V(K) \neq \{\ul{0}\} \)), allora esso ammette un sistema libero di generatori.}
\pf{Dimostrazione}{sia \(A = [v_1, v_2, ..., v_n]\) un sistema di generatori per \(V(K)\), abbiamo due possibilità: \begin{enumerate}
    \item A è libero \( \implies \) A è un sistema di generatori libero;
    \item A è legato \( \implies \ \exists v \in A\) combinazione lineare degli altri, senza perdita di generalità possiamo porre \(v = v_1 \ \implies \ A\backslash \{v_1\} = A_1 \) è sistema di generatori.
\end{enumerate} 
Se ci troviamo nel secondo caso possiamo reiterare il procedimento e trovare \(A_2 \rightarrow A_3 \rightarrow ...\) finché non arriviamo ad un sistema libero di generatori.

Osserviamo che \(A\) contiene almeno un \(v \in A: \ v \neq \ul{0}\), questo perché \(A_n = [0]\) e \(v_n \neq \ul{0}\) perché \(A \neq \{\ul{0} \} \ \implies \ A_n\) è necessariamente libero.  }

\dfn{Base}{Sia \(S = (v_1, v_2, ..., v_n)\) sequenza libera di vettori di \(V(K)\). \(S\) è detta base se e solo se \(S\) è una sequenza libera di generatori.}

\dfn{Base canonica di \(\RR^{n} \) }{\(((1,0,0,...,0)(0,1,0,...,0),...,(0,0,0,...,1))\) è una base canonica per \(\RR^{n} \).}

\mlenma{Lemma di Steinitz}{Sia \(V(K)\) uno spazio vettoriale finitamente generato. Sia \(B = [v_1, v_2, ..., v_n]\) sistema di generatori e \(A = [u_1, u_2, ..., u_m]\) sistema libero. Allora la cardinalità di A sarà sempre minore o uguale a quella del sistema di generatori. \((m \le n)\)  }

\pf{Dimostrazione}{Sia per assurdo \(m >n\), poiché \(B\) genera \(V(K)\) \(u_1\) si scrive come: \[
    u_1 = k_1v_1 + k_2 v_2 + ... + k_n v_n
\] Essendo \(A\) libero \(u_1 \neq \ul{0} \implies k_1, k_2, ..., k_n\) non sono tutti nulli \( \implies\) senza perdita di generalità \(k_1 \neq 0\) \[
    -k_1v_1 = -u_1 + k_2v_2 + ... + k_nv_n \qquad v_1 = \frac{1}{k_1}(u_1 - k_2v_2 - ... - k_nv_n)
\] \[
\implies v_1 \in \mcL([u_1, v_2, v_3, ..., v_n])
\] B è sistema di generatori, \(B \cup \{u_1\} \) è sistema di generatori, di conseguenza \((B \cup \{u_1\} \backslash \{v_1\} ) = B_1 = [u_1, v_2, ..., v_n]\) è ancora sistema di generatori per \(V(K)\).

Allo stesso modo posso riscrivere \[
    u_2 = \alpha u_1 + h_2v_2 + h_3 v_3 + ... + h_n v_n \quad \text{con} \ \alpha, h_2, h_3, ..., h_n \in K
\] Se avessimo \(h_2 = h_3 = ... = h_n = 0 \ u_2 = \alpha\) ma ciò non può succedere perché \(A\) è libero \( \implies \exists h_i \neq 0\) e senza perdita di generalità supporremo \(h_2 \neq 0\) quindi: \[
    -h_2 v_2 = \alpha u_1 - u_2 + h_3 v_3 + ... + h_n v_n \qquad v_2 = \frac{1}{h_2} (-\alpha u_1 + u_2 - h_3 v_3 - ... - h_n v_n)
\] \(v_2\) è linearmente dipendente da \(B_2 = [u_1, u_2, v_3, ..., v_n]\) e \(B_2\), per lo stesso motivo di \(B_1\) è ancora sistema di generatori.

Ora immaginiamoci di reiterare il procedimento \(n\) volte fino a trovare un sistema \(B_n = [u_1, u_2, ..., u_n]\). Siccome avevamo supposto che \(m > n\) essendo \(B_n\) sistema di generatori dovremo essere in grado di scrivere anche \(u _{n+1}\) come combinazione lineare dei vettori di \(B_n\), cioè: \[
    u _{n+1} \in \mcL(B_n) \qquad u _{n+1} = \alpha_1 u_1 + \alpha_2 u_2 + ... + \alpha_n u_n
\] questo comporta che \(A\) sia legato, ma questo è \textbf{assurdo!} \( \implies \ m \le n\).}

\thm{}{Sia \(V(K)\) uno spazio vettoriale finitamente generato, e siano \(B_1\) e \(B_2\) due sue basi le loro cardinalità sono uguali: \[
    B_1 = (v_1, v_2, ..., v_n) \qquad B_2 = (u_1, u_2, ..., u_n) \qquad m=n
\]} 
\pf{Dimostrazione}{Per dimostrarlo è sufficiente applicare il lemma di Steinitz \begin{itemize}
    \item \(B_1\) sistema di generatori, \(B_2\) sistema libero \( \implies \ n \ge m\); 
    \item \(B_2\) sistema di generatori, \(B_1\) sistema libero \( \implies \ m \ge n\). 
\end{itemize} \(m \ge n \text{ e } n \ge m \iff n = m\). }

\dfn{Dimensione}{Dato uno spazio vettoriale finitamente generato, non banale, chiamiamo \textbf{dimensione} di \(V\) la cardinalità di una qualsiasi delle sue basi. Inoltre se \(V = \{0\} \) poniamo la \(\dim(V)=0\) }

Qui di seguito enunciamo una serie di conseguenze del lemma di Steinitz.
\mprop{}{Sia \(V_n(K)\) uno spazio vettoriale di dimensione \(n\) su \(K\) e sia \(S=[v_1, v_2, ..., v_n]\) un sistema di generatori. Allora \(S\) è libero.}
\pf{Dimostrazione}{Sia \(B=[w_1, w_2, ..., w_n]\) una base di \(V_n(K)\). Sia per assurdo \(S\) legato.
Senza perdita di generalità \(v_1 = k_2v_2 + k_3v_3 + ... + k_nv_n\). Allora \(S' = S \backslash \{v_1\} \) è ancora sistema di generatori. \(|S'| = n-1 \ge |B|\) perché \(B\) è libero per il lemma di Steinitz. \textbf{Assurdo!}. Quindi \(S\) è libero.}

\mprop{}{Sia \(V(K)\) uno spazio vettoriale di dimensione \(n\) sul campo \(K\). Sia \(S=[v_1, v_2, ..., v_n]\) un sistema libero. Allora \(S\) è anche un sistema di generatori.}
\pf{Dimostrazione}{Sia \(B=[w_1, w_2, ..., w_n]\) una base di \(V(K)\), supponiamo per assurdo che \(S\) non generi. \[
    \implies \ \exists v \in V \text{ con } v \neq \ul{0} 
\] \(S' = S \cup \{u\} \) è ancora libero, supponiamo per assurdo che non lo sia: \[
    \text{ sia } \ul{0} = k_1v_1 + k_2v_2 + ... + k_nv_n + \alpha v \text{ con } \alpha \neq 0
\] \[
    \text{ altrimenti avremmo: } \ul{0} = k_1v_1 + k_2v_2 + ... + k_nv_n
\] \[
    v = \frac{1}{\alpha}(-k_1v_1 - k_2 v_2 - ... - k_nv_n) \in \mcL(S)
\] \( \implies v \in \mcL(S)\) \textbf{assurdo!} Contro l'ipotesi che \(v \notin \mcL(S) \ \implies \ S'\) è libero. \[
\underbrace{|S'| = n+1}_{\text{sistema libero}} \le \underbrace{|B| = n}_{\text{sequenza di generatori}} \rightarrow \text{ per il lemma di Steinitz }
\] \textbf{Assurdo!} \( \implies\) \(S\) è un sistema di generatori.}

\mprop{}{\(m\) vettori in \(V_n(K)\) con \(m > n\) sono sempre linearmente dipendenti.}
\pf{Dimostrazione}{Siano per assurdo [\(v_1, v_2, ..., v_m\)], \(m\) vettori linearmente indipendenti con \(m > n\). Sia \(B\) una base di \(V_n(K)\). \(m = |S=[v_1, v_2, ..., v_m]| \le |B| = n\) per il lemma di Steinitz. Ma per ipotesi \(m > n\), \textbf{assurdo!} }

\mprop{}{\(m\) vettori in \(V_n(K)\) con \(m < n \implies\) non possono generare. }
\pf{Dimostrazione}{siano \(v_1, v_2, ..., v_m\) per assurdo \(m\) vettori che generano \(V_n(K)\) con \(m < n\) allora: \[m = |S=[v_1, v_2, ..., v_n]| \ge |B|=n \ \text{ con } \ m \ge n \quad \text{per il lemma di Steinitz}\] \textbf{Assurdo!} Va contro all'ipotesi.}

\thm{Teorema di caratterizzazione delle basi}{Sia \(B=(v_1, v_2,...,v_n)\) una sequenza di vettori di \(V(K)\). \(B\) è una base se e solo se ogni vettore di \(V\) si può scrivere in maniera univoca come combinazione lineare dei vettori di \(B\). \[
    \forall v \in V,\ \exists ! \ v = k_1v_1 + k_2 v_2 + ... + k_n v_n \quad k_i \in K
\]}
\pf{Dimostrazione}{
\( \implies\) sia \(B\) una base di \(V\). Per ogni \(v\) si ha che \(v \in \mcL(B)\) perché \(B\) è una sequenza di generatori. Supponiamo per assurdo che esista \(v \in V\): \[
    v = v = k_1v_1 + k_2 v_2 + ... + k_n v_n = h_1 v_1 + h_2 v_2 + ... + h_n v_n \quad \text{ con almeno un  } k_i \neq h_i 
\] \[
(k_1 - h_1) v_1 + (k_2 - h_2) v_2 + ... + (k_n - h_n) v_n = \ul{0} 
\] \(B\) è una sequenza libera, quindi \((k_i - h_i) = 0 \implies k_i = h_i\) perché l'unica combinazione lineare che dà il vettore nullo è quella a coefficienti tutti nulli.
Ma avevamo supposto che \(k_i \neq h_i \implies\) \textbf{assurdo!} \( \implies \exists!\) la combinazione lineare dei vettori di \(B\) che dà \(v \ (\forall v \in V)\).

\( \impliedby\) per ipotesi \(\forall v \in V \ \exists !\) combinazione lineare dei vettori di \(B\) che dà \(v\). \(B\) è una sequenza di generatori, cioè \(\forall v \in V \implies v \in \mcL(B)\). Supponiamo per assurdo che \(B\) sia legato \(\implies \exists k_i \in K\) non nullo: \[
    \ul{0} = k_1v_1 + k_2 v_2 + ... + k_n v_n \quad \ul{0} = 0v_1 + 0v_2 + ... + 0v_n
\] quindi esistono almeno due combinazioni lineari di \(B\) che danno \(\ul{0} \). Dato che \(\ul{0} \in V\) per ipotesi esiste un unica combinazione lineare dei vettori di \(B\) che dà \(\ul{0} \). \textbf{Assurdo!} Quindi \(B\) è una sequenza libera e \(B\) è una base per \(V\).}

\dfn{Componenti di un vettore rispetto ad una base}{Sia \(B=(v_1, v_2, ..., v_n)\) una base di \(V_n(K)\) e sia \(v \in V\). Chiameremo componenti di \(v\) rispetto alla base \(B\) la sequenza \((k_1, k_2, ..., k_n)\): \[
    v = k_1v_1 + k_2 v_2 + ... + k_nv_n
\]}

\mprop{}{Sia \(V_n(K)\) uno spazio vettoriale di dimensione \(n\) sul campo \(K\), allora \(V_n(K)\) ammette almeno un sottospazio di dimensione \(m\) \(\forall 0 \le m \le n\). }
\pf{Dimostrazione}{sia  \(B=(v_1, v_2,...,v_n)\) una base di \(V_n(K)\) e sia \(0 \le m \le n\), ci sono due possibilità: \begin{enumerate}
    \item \(m=0 \implies\) \{\ul{0} \} è il sottospazio voluto;
    \item \(0 < m \le n\) e quindi \(S=(v_1, v_2,...,v_m)\)
\end{enumerate} \(\mcL(S)\) ha dimensione \(m\) perché \(S\) è libero \((S \subseteq B)\) e genera, per definizione \(\mcL(S)\).}

\mprop{}{Siano \(U, W \le V_n(K)\) e sia \(U \le W\), allora: \begin{enumerate}
    \item \(\dim(U) \le \dim(W)\)
    \item \(U = W \iff \dim(U) = \dim(W)\)
\end{enumerate}}
\pf{Dimostrazione}{Dimostriamo i due punti:\begin{enumerate}
        \item Sia \(B\) base per \(U\) e \(B'\) base per \(W\), se per assurdo \[\underbrace{\dim(U) = |B|}_{\text{sequenza libera di \(W\) }} > \underbrace{\dim(W) = |B'|}_{\text{genera \(W\)} } \] contro il lemma di Steinitz. 
        \item \( \implies\) è banale; \\
        \( \impliedby\) sia per assurdo \(U < W\) e sia \(B\) base di \(U\), allora \[
            |B| = \dim(U) = \dim(W)
        \] quindi \(B\) è una base anche per \(W\) \( \implies \mcL(B) = W \implies W = U\) \textbf{Assurdo!}
\end{enumerate}}

\thm{Teorema del completamento ad una base}{Sia \(V_n(K)\) uno spazio vettoriale di dimensione \(n\) e sia \(A=(v_1, v_2,...,v_p)\), ove \(p \le n\), una sequenza libera di vettori in \(V_n(K)\). Allora, in una qualunque base di \(B\) di \(V_n(K)\), esiste una sequenza \(B'\) di vettori, tale che \(A \cup B'\) è una base di \(V_n(K)\).}

\section{Intersezione e somma di sottospazi}
\mprop{}{Sia \(V_n(K)\) uno spazio vettoriale di dimensione \(n\) sul campo \(K\) e siano \(U, V \le V \implies U\cap W\) è sottospazio di \(V\).}
\pf{Dimostrazione}{Richiamo il secondo criterio di riconoscimento dei sottospazi. \(U \cap W\) è un sottospazio di \(V\) \(\iff\) è sottoinsieme non vuoto di \(V\): \[
    \forall v_1, v_2 \in U \cap W, \ \forall k_1, k_2 \in K, \ k_1 v_1 + k_2v_2 \in U \cap W
\] \(U \cap W\) è sottoinsieme non vuoto di \(V\), perché \(U \subseteq V\), \(W \subseteq V\) e \(\ul{0} \in U \cap W\). Siano ora \(v_1, v_2 \in U \cap W\) e \(k_1, k_2 \in K\), osserviamo per il secondo criterio di riconoscimento che \(k_1 v_1 + k_2 v_2 \in U\) e per lo stesso motivo \(k_1 v_1 + k_2 v_2 \in W\) \( \implies k_1 v_1 + k_2 v_2 \in U \cap W \implies U \cap W\) è un sottospazio vettoriale.}

\nt{Sotto le stesse ipotesi della proposizione precedente abbiamo che \(U \cup W\) non è un sottospazio a meno che \(U \subseteq W\) oppure \(W \subseteq U\).}

\dfn{Spazio di somma}{Dati \(U\) e \(W \le V\) spazio vettoriale di dimensione \(n\) su \(K\) definiamo lo \textbf{spazio di somma} come: \[U + W := \{u + w \ | \ u \in U \ e \ w \in W\} \]}

\mprop{}{Dati \(U\) e \(W \le V\) spazio vettoriale di dimensione \(n\) su \(K\) abbiamo che: \(U + W \le V\) }
\pf{Dimostrazione}{Osserviamo che \(U + W \subseteq V\) perché dato \(u \in U \) e \(w \in W\), \(u \in V\) e \(w \in V \implies u + w \in V\), il quale non è vuoto perché \(\ul{0} \in U + W\). Siano \(v_1, v_2 \in U + W\) e siano \(k_1, k_2 \in K\) \[
    k_1 \cdot \underbrace{v_1}_{\text{\(=u_1 + w_1\) }}  + k_2 \cdot \underbrace{v_2}_{\text{\(= u_2 + w_2\) }}  = k_1(u_1 + w_1) + k_2(u_2 + w_2) = \underbrace{(k_1 u_1 + k_1 w_1)}_{\text{\(u_3 \in U\) per il 2° criterio}} + \underbrace{ (k_2 u_2 + k_2 w_2)}_{\text{\(w_3 \in W\) per il 2° criterio}}  
\] \[
    \implies u_3 + w_3 \in U + W \implies \text{ per il 2° criterio } \ U + W \le V
\]}

\newpage
\mprop{}{Siano \(U, W \le V_n(K)\) allora \(U + W\) è il più piccolo sottospazio di \(V\) che cotiene \(U \cup W\); equivalentemente \[\mcL(U \cup W) = U + W\]}
\dfn{Somma diretta}{Dati \(U, W \le V_n(K)\) diremo che \(U + W\) è somma diretta se \(\forall v \in U + W\) può essere scritto come unico modo come \(u + w\). Equivalentemente \[
    \forall v \in U + W \quad \exists ! \ u \in U \ e \ w \in W : \quad v = u + w
\]Se \(U + W\) è una somma diretta allora la indicheremo con \(U \oplus W\). }

\mprop{}{Siano \(U, W \le V_n(K)\) allora \(U \oplus W \iff U \cap W = \{\ul{0} \} \).}
\pf{Dimostrazione}{\( \implies\) Siano \(U, W\) in somma diretta e sia, per assurdo: \(x \in U \cap W\) con \(x \neq \ul{0} \). Sia \(v = u + w\) con \(u \in U \ e \ w \in W\). Consideriamo \[
    v + x - x = v \implies v = u + w + x - x = \underbrace{u + x}_{\in U} + \underbrace{w - x}_{\in W} = u_1 + w_1
\] \[
    u = u + x \quad e \quad w = w - x \text{ poiché la somma è diretta } \implies x = \ul{0} \implies \text{\textbf{Assurdo!}} \implies U \cap W = \{\ul{0} \} 
\] \( \impliedby\) Siano \(U, W: \ U \cap W = \{\ul{0} \}\) e supponiamo per assurdo che esista \(v \in U + W:\) \[
    v = u_1 + w_1 \quad e \quad v = u_2 + w_2 \qquad \text{ con } \ u_1, u_2 \in U \quad e \quad w_1, w_2 \in W \quad e \quad (u_1, w_1) \neq (u_2, w_2)
\] \[
    u_1 + w_1 = u_2 + w_2 \quad v_2 = \underbrace{u_1 - u_2}_{\in U} = \underbrace{w_2 - w_1}_{\in W} \in U \cap W
\] \[
    \implies u_1 - u_2 = \ul{0} \quad e \quad w_2 - w_1 = \ul{0}
\] \[
    \implies u_1 = u_2 \quad e \quad w_1 = w_2
\] che è \textbf{assurdo!} Questo perché avevamo supposto che \(v\) avesse due scritture distinte come somma i elementi di \(U \ e \ W.\) \[
    \implies \exists ! \ (u_1, w_1): \quad u, \in U \quad e \quad w_1 \in W: \quad v=u_1 + w_1 \ e \ U \oplus W 
\]}

\cor{}{Siano \(U, W \le V_n(K)\) allora \(V = U \oplus W \iff U + W = V \ e \ U \cap W = \{\ul{0} \} \).}
\nt{Siano \(U, W \le V_n(K)\) e sia \(B_1\) una base di \(V\) e \(B_2\) una base di \(W\) \( \implies B_1 \cup B_2\) è sequenza di generatori per lo spazio \(U + W\). In generale l'unione di due basi, non è a sua volta una base per \(U + W.\) }

\mprop{}{Siano \(U, M \le V_n(K): U \oplus W\) e sia \(A\) una sequenza libera di vettori di \(U\) e \(B\) una sequenza libera di vettori di \(U\). Allora \(A \cup B\) è una sequenza libera di vettori della \(U \oplus W\).}
\pf{Dimostrazione}{Siano \(A = (u_1, u_2, ..., u_k)\) e \(B = (w_1, w_2, ..., w_h)\) e supponiamo per assurdo che \(a_1, a_2, ..., a_k \in K\) e \( b_1, b_2, ..., b_h \in K\), quindi per assurdo sia legata la combinazione lineare: \[
    \ul{0} = a_1 u_1 + a_2 u_2 + ... + a_k u_k + b_1 w_1 + b_2 w_2 + ... + b_h w_h \ \text{ non tutti nulli }
\] \[
    \underbrace{-(a_1 u_1 + a_2 u_2 + ... + a_k u_k)}_{ \in U } = \underbrace{b_1 w_1 + b_2 w_2 + ... + b_h w_h}_{\in W}
\] \[
    \implies \ul{0} = b_1 w_1 + b_2 w_2 + ... + b_h w_h \quad e \quad \ul{0} = a_1 u_1 + a_2 u_2 + ... + a_k w_k
\]ma \(A\) e \(B\) sono sequenze libere quindi \(a_1 = a_2 = ... = a_k = 0 \quad e \quad b_1 = b_2 = ... = b_h = 0\) 
\[ \implies \nexists a_1, a_2, ..., a_k, b_1, b_2, ..., b_h \text{ non tutti nulli: } \] 
\[
    \ul{0} = a_1 u_1 + a_2 u_2 + ... + a_k u_k + b_1 w_1 + b_2 w_2 + ... + b_h w_h \implies \text{\textbf{Assurdo!}}
\] \( \implies A \cup B\) è una sequenza libera. 
}

\cor{}{Siano \(U, W \in V_n(K): U \oplus W\) e siano \(B_U\) e \(B_W\) basi di \(U\) e \(W\) \( \implies B_U \cup B_W\) è una base per \(U \oplus W\). }

\mprop{Formula di Grassmann}{Dati \(U, W \le V_n(K)\) abbiamo che: \[
    \dim(U + W) + \dim(U \cap W) = \dim(U) + \dim(W)
\]}
\dfn{Complemento diretto}{Sia \(W \le V_n(K)\) si dice \textbf{complemento diretto} di \(W\) in \(V\) uno spazio \(U \le V: U \oplus W = V.\) }
\nt{Un complemento diretto di \(W\) in \(V\) esiste sempre e si trova estendendo una base di \(W\) a una base di \(V\). In generale questo non è unico.}


\chapter{Sistemi lineari}
\section{Determinante di una matrice quadrata}
\dfn{Determinante}{Sia \(A = (a _{ij})\) una matrice quadrata, di ordine \(n\), a elementi in un campo \(K.\) Si dice \textbf{determinante} di \(A\), e si scrive \(|A|\) oppure \(\det(A)\), l'elemento di \(K\) definito ricorsivamente come segue: \begin{enumerate}
    \item se \(n = 1 \qquad A = (a _{11}) \qquad \det(A) = |A| = a _{11}\) 
    \item se \(n > 1 \qquad A = a _{ij} \qquad \det(A)=(-1)^{1+1} a _{11} \det A _{11} + (-1)^{1+2} a _{12} \det A _{12} + ... + (-1)^{1 + n} a _{1n} \det A _{1n}\) 
\end{enumerate}}

Se \(A = \left( \; \begin{matrix} a _{11} & a _{12} \\ a _{21} & a _{22} \end{matrix} \; \right) \), il suo determinante è \(|A| = a _{11} a _{22} - a _{12} a _{21}\).

Mentre se \[A = 
\left( \; \begin{matrix}
    a_{11} & a _{12} & a _{13} \\
    a _{21} & a _{22} & a _{23} \\
    a _{31} & a _{32} & a _{33} \\
\end{matrix} \; \right)
\]
Allora la il determinante di \(A\) è \[
    |A| = a _{11} a _{22} a_{33} + a_{13} a_{21} a_{32} + a_{12} a_{23} a_{31} - a_{13} a_{22} a_{32} - a_{11} a_{23} a_{32} - a_{12} a_{21} a_{33}  
\]
\dfn{Complemento algebrico}{Sia \(A = (a_{ij} )\) una matrice quadrata di ordine \(n\), a elementi in campo \(K\). Si dice \textbf{complemento algebrico} dell'elemento \(a_{hk} \), e si indica \(\Gamma_{hk} \), il determinante della matrice quadrata di ordine \(n -1\), ottenuta da \(A\) sopprimendo la h-esima riga e la k-esima colonna, preso con il segno \((-1)^{h+k} \). }
\thm{Primo teorema di Laplace}{Data la matrice quadrata di ordine \(n\), la somma dei prodotti degli elementi di una sua riga (o colonna), per i rispettivi complementi algebrici, è il determinante di \(A.\) }
Pertanto, la formula per il calcolo del determinante di \(A = (a_{ij} )\) rispetto alla a i-esima riga è \[
    |A| = \sum_{j = 1}^{n} a_{ij} \Gamma _{ij} \qquad \forall i = 1,2,..., n
\] rispetto alla j-esima colonna è \[
    |A| = \sum_{i = 1}^{n} a_{ij} \Gamma _{ij} \qquad \forall j = 1,2,..., n
\]

\thm{Secondo teorema di Laplace}{Sia \(A\) una matrice quadrata di ordine \(n\). La somma dei prodotti degli elementi di una sua riga (o colonna) per i complementi algebrici degli elementi di un'altra riga (o colonna) vale zero. Quindi \[
    A \in M_n(K) \implies
\begin{cases}
    a_{i1} \Gamma_{j1} + a_{i2} \Gamma_{j2} + ... + a_{in} \Gamma_{jn} = 0 \quad i \neq j \\
    a_{1i} \Gamma_{1j} + a_{2i} \Gamma_{2j} + ... + a_{ni} \Gamma_{nj} = 0 \quad i \neq j \\
\end{cases}
\]}

\thm{Teorema di Bidet}{Date due matrici quadrate di ordine \(n\), \(A\) e \(B\), il determinante della matrice prodotto \(A \cdot B\) è uguale al prodotto dei determinanti di \(A\) e \(B\), cioè \[
    |A \cdot B| = |A| |B| 
\] }

\section{Matrici invertibili}
\dfn{Matrice invertibile}{Una matrice quadrata, di ordine \(n\), si dice \textbf{invertibile} quando esiste una matrice \(B\), quadrata e dello stesso ordine, tale che \(A \cdot B = B \cdot A = I_n\), dove \(I_n\) è la matrice identica di ordine \(n\). La matrice \(B\) si dice \textbf{inversa} di \(A\) e si indica \(A^{-1} \).}

\thm{}{Sia \(A \in M_n(K)\); allora \(A\) è invertibile \( \iff |A| \neq 0\) e in tal caso \[
    A^{-1} = \frac{1}{|A| }\left.^tA_a\right.
\] dove \(A_a\) si chiama \textbf{matrice aggiunta} di \(A\) ed è la matrice ottenuta da \(A\) sostituendo ogni elemento con il suo complemento algebrico \(\Gamma\). }

\section{Dipendenza lineare e determinanti}
\dfn{Minore}{Sia \(A \in K^{m,n} \). Si chiama \textbf{minore di ordine \(p\)} estratto da \(A\), con \(p \in \mathbb{N}\), \(p \neq 0\), \(p \le \min \{m,n\} \), una matrice quadrata di ordine \(p\) ottenuta cancellando \(m-p\) righe e \(n-p\) colonna da \(A\). }

\thm{}{ Una sequenza \(S=(v_1,v_2, \ldots,v_n)\) di \(n\) vettori dello spazio vettoriale \(V_n(K)\) è libera se, e soltanto se, la matrice \(A\), che ha nelle proprie righe (o colonne) le componenti dei vettori di \(S\) in una base di \(V_n(K)\), ha determinante non nullo ed è legata se, e soltanto se, tale matrice \(A\) ha determinante nullo.}

\dfn{Rango di una matrice}{Sia \(A\) una matrice di \(K^{m,n}(K)\). Si dice \textbf{rango} della matrice \(A\), e si scrive \(\rho (A)\), l'ordine massimo di un minore estraibile da \(A\) con determinante non nullo.}

\newpage
\paragraph{Osservazione:} Data la matrice \(A\) di \(K^{m,n}(K)\)
\begin{enumerate}
    \item \(\rho (A)=0 \iff A\) è la matrice nulla;
    \item \(\rho (A) = \rho (^{t}A)\) ;
    \item \(\rho (A) \le \min(m,n)\).
\end{enumerate}

\dfn{Spazio delle righe e delle colonne}{Data una matrice \(A\), avente \(m\) righe ed \(n\) colonne, si dice \textbf{spazio delle righe} di \(A\), e si indica \(\mcL(R) \), il sottospazio \(K^{n}(K)\) generato dalle righe di \(A\). Si dice  \textbf{spazio delle colonne} di \(A\), e si indica \(\mcL(C) \), il sottospazio vettoriale di \(K^{m}(K)\) generato dalle colonne di \(A\).}

\thm{Teorema di Kronecker}{Gli spazi vettoriali \(\mcL(R) \) ed \(\mcL(C)\), di una matrice \(A \in K^{m,n}(K)\), hanno la stessa dimensione e tale dimensione coincide con il rango di \(A\). Cioè: \[
\dim(\mcL(R) ) = \dim(\mcL(C) ) = \rho (A)
.\] }
\pf{Dimostrazione}{Dimostriamo che \(\dim(\mcL(R) ) = \rho (A)\). La dimostrazione per quanto riguarda le colonne è completamente analoga. Sia \(s = \dim(\mcL(R) )\implies\) abbiamo \(s\) righe linearmente indipendenti nella matrice \(A\) e quindi per il teorema precedente esiste un minore in \(A\) di ordine \(s\) a determinante non nullo. Pertanto \(\rho (A) \ge s\). Sia per assurdo \(\rho (A) = r > s\), dovrebbe esistere in \(A\) un minore di ordine \(r\) a determinante non nullo. Se chiamiamo ora \(S = (R_1, R_2, \ldots, R_r)\) la sequenza di righe nella matrice \(A\), la matrice \(A\) ha un minore di ordine \(r\) non singolare e di conseguenza è libera. Quindi \[
\dim \mcL(R) \ge \dim \mcL(S) = r > s = \dim \mcL(R) 
.\] Ma questo è un \textbf{assurdo!} Quindi \[
\rho (A) = r \le s = \dim \mcL(R) \implies r = s 
.\]}

\cor{}{Se \(A\) è una matrice quadrata di ordine \(n\), con elementi in un campo \(K\), le sequent condizioni sono equivalenti:
\begin{enumerate}
    \item \(|A| \neq 0\) ;
    \item \(A\) è invertibile;
    \item \(\rho (A) = n\) ;
    \item le righe sono linearmente indipendenti e, quindi, sono base di \(K^{n}\);
    \item le colonne sono linearmente indipendenti e, quindi, sono base di \(K^{n}\).
\end{enumerate}}

\thm{Teorema degli orlati}{Una matrice \(A \in K^{m,n}(K)\) ha rango \(p\) se, e solo se, esiste un minore \(M\) di ordine \(p\) a determinante non nullo e tutti i minori di ordine \(p + 1\), che contengono \(M\), hanno determinante nullo.}

\section{Sistemi lineari}
\dfn{Sistema lineare}{Un \textbf{sistema lineare} è un insieme di \(m\) equazioni lineari in \(n\) incognite a coefficienti in campo \(K\).} Un sistema lineare si può, quindi, indicare nel modo seguente: \[
\begin{cases}
    \ a_{11}x_1+a_{12}x_2+\ldots +a_{1n}x_n = b_1 \\
    \ a_{21}x_1+a_{22}x_2+\ldots +a_{2n}x_n = b_2 \\
    \ \ldots \ldots \ldots \ldots \ldots \ldots \ldots \ldots \ldots  \\
    \ a_{m_1}x_1+a_{m_2}x_2+\ldots +a_{mn}x_n = b_m \\
\end{cases}
\] con \(a_{ij}, b_l \in K\). Gli elementi \(a_{ij}\) si chiamano coefficienti delle incognite, gli elementi \(b_l\) si dicono termini noti.

La matrice \(m \times n\) \[
A =
\left( \; \begin{matrix}
    a_{11} & a_{12} & \ldots  & a_{1n} \\
    a_{21} & a_{22} & \ldots  & a_{2n} \\
    \vdots & \vdots & \ddots & \vdots \\
    a_{m1} & a_{m 2} & \ldots  & a_{mn} \\
\end{matrix} \; \right)
\] è detta matrice dei coefficienti o \textbf{matrice incompleta}, la matrice \(n \times 1\) \[
X =
\left( \; \begin{matrix}
    x_1 \\
    x_2 \\
    \vdots \\
    x_n \\
\end{matrix} \; \right)
\] è detta delle matrice colonna delle incognite, mentre la matrice \(m\times 1\) \[
B = 
\left( \; \begin{matrix}
    b_1 \\
    b_2 \\
    \vdots \\
    b_m \\
\end{matrix} \; \right)
\] è detta matrice colonna dei termini noti. La matrice \(m \times (n+1)\) \[
A | B = 
\left( \; \begin{matrix}
    a_{11} & \ldots  & a_{1n} & b_1 \\
    a_{21} & \ldots  & a_{2n} & b_2 \\
    \vdots & \ddots & \vdots & \vdots \\
    a_{m1} & \ldots  & a_{mn} & b_m \\
\end{matrix} \; \right)
\] è detta \textbf{matrice completa}.
Infine, il sistema iniziale si può riscrivere come: \(A \cdot X = B\), cioè \[
\left( \; \begin{matrix}
    a_{11} & a_{12} & \ldots  & a_{1n} \\
    a_{21} & a_{22} & \ldots  & a_{2n} \\
    \vdots & \vdots & \ddots & \vdots \\
    a_{m1} & a_{m 2} & \ldots  & a_{mn} \\
\end{matrix} \; \right)
\left( \; \begin{matrix}
    x_1 \\
    x_2 \\
    \vdots \\
    x_n \\
\end{matrix} \; \right)
=
\left( \; \begin{matrix}
    b_1 \\
    b_2 \\
    \vdots \\
    b_m \\
\end{matrix} \; \right)
\] 

\dfn{Sistema omogeneo}{Un sistema lineare si dice \textbf{omogeneo} quando tutti i termini noti sono nulli. \[
AX = \ul{0} 
\] }

\paragraph{Osservazione:} Data \(A \in K^{m,n} \quad A = 
\left( \; \begin{matrix}
    C_1 & C_2 & \ldots & C_n \\
\end{matrix} \; \right)
\) ove le colonne \(C_j\) sono vettori di \(K^{m,1}\) e quindi utilizziamo utilizzando questa notazione il sistema si può scrivere come \[
x_1C_1 + x_2C_2+\ldots +x_nC_n = B
\] 

\dfn{Sistema compatibile}{Un sistema lineare in \(m\) equazioni ed \(n\) incognite ha soluzione, ovvero si dice che il sistema è \textbf{compatibile}, se esiste almeno una n-upla  \(\alpha _1, \alpha _2, \ldots , \alpha _n\) di elementi di \(K\) che risolve tutte le equazioni del sistema. Tale n-upla è detta \textbf{soluzione}.}

\paragraph{Osservazione:} Posto \(A = (C_1, C_2, \ldots , C_n)\) \[
A 
\left( \; \begin{matrix}
    \alpha _1 \\
    \alpha _2 \\
    \vdots \\
    \alpha _n \\
\end{matrix} \; \right)=B \iff 
\alpha _1 C_1 + \alpha _2 C_2 + \ldots + \alpha _n C_n = B
\] che è equivalente a dire che \(B\) è combinazione lineare delle colonne di \(A\). Quindi il sistema è risolubile se, e soltanto se, \(B \in \mcL(C_1, C_2, \ldots ,C_n) \).

\thm{Teorema di Rouché-Capelli}{Un sistema lineare \(A X = B\)
 è compatibile se, e soltanto se, \(\rho (A) = \rho (A|B)\).}

 \pf{Dimostrazione}{"\(\implies \)" Sia \(AX = B\) risolubile, \(\implies \exists \ (\alpha _1, \alpha _2, \ldots ,\alpha _n) : \ \alpha _1 C_1 + \alpha _2 C_2 + \ldots + \alpha _n C_n = B\) quindi 
\[ B \in \mcL(C_1, C_2, \ldots , C_n) \implies \underbrace{\dim \mcL(C_1, C_2, \ldots ,C_n, B)}_{= \rho (A|B)} = \underbrace{\dim \mcL(C_1, C_2, \ldots ,C_n)   }_{= \rho (A)} \]
\[\implies \rho (A|B) = \rho (A)\] 
 "\(\impliedby \)" Per ipotesi abbiamo che \(\rho (A|B) = \rho (A)\). Quindi 
\[
\dim \mcL(C_1, C_2, \ldots ,C_n, B) = \dim \mcL(C_1, C_2, \ldots ,C_n) \implies \mcL(C_1, C_2, \ldots ,C_n, B) = \mcL(C_1, C_2, \ldots ,C_n)\]
\[\implies B \in \mcL(C_1, C_2, \ldots ,C_n)  
    \]\[\implies  \exists (k_1,k_2, \ldots ,k_n): \ k_1C_1+k_2C_2+\ldots +k_nC_n = B\]
Quindi la n-upla \((k_1, k_2, \ldots , k_n)\) è soluzione di \(AX = B\) e di conseguenza il sistema è compatibile.}

\thm{Teorema di Cramer}{Sia \(AX = B\) un sistema lineare in \(n\) equazioni ed \(n\) incognite. Se \(\det(A) \neq 0\) allora \(AX = B\) ammette un'unica soluzione.}
\pf{Dimostrazione}{Sia \(|A| \neq 0 \iff n = \rho(A) = \rho(A|B) \) perché \(A|B\) ha \(n\) righe, quindi per il teorema di Rouché-Capelli il sistema è compatibile e ammette almeno una soluzione. Supponiamo ora per assurdo che non ammetta soluzione unica, siano \(X_1\) e \(X_2\) due soluzioni distinte di \(AX=B\). Avremo che sia \(AX_1=B\) e sia \(AX_2 = B\), quindi \(AX_1= AX_2\). Ora ricordiamo che \(|A| \neq 0\), quindi \(A\) è invertibile, perciò \[
\exists A^{-1} : \quad A^{-1}A = I
\] Quindi possiamo giustificare la seguente equazione \[
A^{-1} (AX_1) = A^{-1}(AX_2) \iff (A^{-1}A)X_1 = (A^{-1}A)X_2 \iff IX_1 = IX_2 \iff X_1=X_2
\] ma questo è un \textbf{assurdo}! Poiché avevamo supposto che \(X_1\neq X_2\), quindi esiste un'unica soluzione.}
Indichiamo con \(B_1\), la matrice ottenuta sostituendo a \(C_i\) la colonna dei termini noti (\(B\)).
\[
A = (C_1, C_2, \ldots , C_n) \quad B_1 = (C_1, C_2, \ldots , C_{i-1}, B, C_{i+1}, \ldots , C_n)
\] Se \(\det (A) \neq 0\) allora (\(X_1, X_2, \ldots , X_n\)) è data da: \[
X_1 = \frac{|B_1| }{|A| }=\frac{\det(B_1)}{\det(A)}
\]

\dfn{Sistema principale equivalente}{Sia \(AX = B\) un sistema compatibile, si dice sistema principale equivalente un sistema \(A'X = B'\) ottenuto eliminando \(m-p\) equazioni da \(AX=B\) tale che \(\rho (A'|B') = \rho (A') = p\).}
\thm{}{Un sistema \(AX = B\) compatibile ha le stesse soluzioni di un suo sistema principale equivalente.}

\paragraph{Osservazione:}\(\rho (A)=\rho (A|B)\) se il sistema lineare è omogeneo e quindi è sempre compatibile. In particolare \(X = 
\left( \; \begin{matrix}
    0 \\
    \vdots \\
    0 \\
\end{matrix} \; \right)
\) è sempre soluzione di \(AX = \ul{0} \).

\dfn{Autosoluzioni}{Le soluzioni di un sistema lineare omogeneo diverse dalla soluzione nulla si dicono  \textbf{autosoluzioni}.}
\newpage
\nt{Non è detto che un sistema lineare omogeneo ammetta autosoluzioni.}

\mprop{}{Un sistema lineare omogeneo \(AX = B = \ul{0} \) ammette autosoluzioni se, e solo se, \(\rho (A) < n\) (con \(n\) numero di incognite).}

\cor{}{Un sistema lineare omogeneo \(AX = B = \ul{0} \) con \(A \in M_n(K)\) ammette autosoluzioni se, e soltanto se,  \(\det (A) = 0\).}

\thm{}{Sia \(AX = \ul{0} \) un sistema lineare omogeneo con \(A \in K^{m,n}\) e sia \(S\) l'insieme delle sue soluzioni, allora \(S\) è un sottospazio di \(K^{n}\) di dimensione \(n-\rho (A)\).}

\paragraph{Osservazioni:} 
\begin{enumerate}
    \item \(\ul{0} \in S\) 
    \item se \(n-\rho (A) > 0\) abbiamo autosoluzioni 
    \item Se \(B \neq \ul{0} \) l'insieme delle soluzioni di \(AX = B\) non è un sottospazio di \(K^{n}\) perché \(A\ul{0} = \ul{0} \neq B \implies \{\ul{0} \} \notin S\).
\end{enumerate}

\mprop{}{Sia \(AX = B\) un sistema lineare in \(m\) equazioni ed \(n\) incognite, detto \(S\) l'insieme delle soluzioni abbiamo che \[
S =
\begin{cases}
    \ \{x_0+z : \ x_0 \in S, \ z \in S\}\text{ se }AX = B\text{ è compatibile } \\
    \ \emptyset \text{ se } AX = B \text{ non è compatibile} \\
\end{cases}
\] }

\dfn{Sistema lineare omogeneo associato}{Dato \(AX = B\) sistema lineare in \(m\) equazioni ed \(n\) incognite diciamo che \(AX = \ul{0} \) è il \textbf{sistema lineare omogeneo associato} a \(AX = B\).}

\mprop{}{Le soluzioni di un sistema lineare compatibile \(AX=B\) sono tutte e sole del tipo \(\overline{X}= X_0 + Z\), ove \(X_0\) è una soluzione particolare di \(AX = B\) e \(Z\) è la soluzione di \(AX = \ul{0} \), sistema omogeneo associato ad \(AX = B\).}

\pf{Dimostrazione}{Sia \(\overline{X}\) soluzione di \(AX=B\), poniamo \(Z = \overline{X}-X_0 \iff \overline{X}=X_0+Z\) \[
AZ = A(\overline{X}-X_0) = A\overline{X}- AX_0 = B- B = \ul{0} 
\] Quindi \(Z\) è soluzione del sistema lineare omogeneo associato ad \(A\). Di conseguenza \(\overline{X}= X_0 + Z\) }

Dato \(AX=B\) sistema lineare in \(m\) equazioni ed \(n\) incognite compatibile, le sue soluzioni sono tante quante quelle del sistema lineare omogeneo associato che costituiscono uno spazio vettoriale di dimensione \(n- \rho (A)\). Se il campo è infinito, posto \(\rho (A) = p\), si dice che le soluzioni sono \(\infty^{n-p}\) (cioè che l'insieme delle soluzioni dipende da \(n-\rho(A)\) parametri).

\newpage
\thm{}{Sia \(AX = \ul{0} \) un sistema lineare omogeneo in \(n\) incognite e sia \(\rho(A) = n-1\). Se si indica con \(A'X = \ul{0} \) un sistema principale equivalente ad \(AX = \ul{0} \) e si indicano con \(\Gamma_1, \Gamma_2, \ldots , \Gamma_n\) i determinanti dei minori di ordine \(n-1\), ottenuti eliminando in \(A'\) successivamente la prima, la seconda, \ldots , la n-esima colonna, allora le soluzioni del sistema sono, al variare di \(\lambda \in K\), \[
S = (\lambda \Gamma _1, -\lambda \Gamma _2, \ldots , (-1)^{n-1} \lambda \Gamma _n)
\] }

\section{Cambiamenti di base}
in uno spazio vettoriale \(V_n(K)\), di dimensione \(n\), siano \(B = (e_1,e_2, \ldots ,e_n)\) e \(B' = (e'_1,e'_2, \ldots ,e'_n)\) due basi assegnate. Ogni vettore della base \(B'\) si può esprimere come combinazione lineare dei vettori della base \(B\), cioè \[
\begin{cases}
    \ e_1' = a_{11}e_1+ a_{12}e_2+\ldots + a_{1n}e_n \\
    \ e_2' = a_{21}e_1+ a_{22}e_2+\ldots +a_{2n}e_n \\
    \ \ldots \ldots \ldots \ldots \ldots \ldots \ldots \ldots \ldots  \\
    \ e_n'= a_{n 1}e_1+ a_{n 2}e_2 + \ldots + a_{nn}e_n \\
\end{cases}
\]con le seguenti posizioni \[
A = 
\left( \; \begin{matrix}
    a_{11} & a_{12} & \ldots  & a_{1n} \\
    a_{21} & a_{22} & \ldots  & a_{2n} \\
    \vdots & \vdots & \ddots & \vdots \\
    a_{n1} & a_{n 2} & \ldots  & a_{nn} \\
\end{matrix} \; \right), \ E= 
\left( \; \begin{matrix}
    e_1 \\
    e_2 \\
    \vdots \\
    e_n \\
\end{matrix} \; \right) \text{ ed } E'=
\left( \; \begin{matrix}
    e_1' \\
    e_2' \\
    \vdots \\
    e_n' \\
\end{matrix} \; \right)
\] 
il sistema si può scrivere in forma compatta \[
E' = AE
\] 
\dfn{Matrice del cambiamento di base}{La matrice A si dice \textbf{matrice del cambiamento di base} da \(B\) a \(B'\).}

\mprop{}{La matrice \(A\) del cambiamento di base da \(B\) a \(B'\) è invertibile e \(A^{-1}=A'\).}
\pf{Dimostrazione}{\[
    E = A'E' = A'(AE) = (A'A)E \implies A'A=I_n \] \[
E' = AE = A(A'E') = (AA')E' \implies AA'=I_n
\] }

Stabiliamo il legame tra le componenti di uno stesso vettore \(v\), rispetto a due basi diverse \(B\) e \(B'\). Poniamo \[
X = \left( \; \begin{matrix} x_1\\ \vdots\\ x_n \end{matrix} \; \right) \text{ e } X'=\left( \; \begin{matrix} x'_1\\ \vdots\\ x'_n \end{matrix} \; \right)
\] Possiamo scrivere il generico vettore \(v \in V_n(K)\) \[
v = x_1e_1+x_2e_2 + \ldots + x_n e_n = (x_1,x_2, \ldots , x_n)E= {^{t}X}E \] \[
v = x_1'e_1'+x_2'e_2' + \ldots + x_n' e_n' = (x_1',x_2', \ldots , x_n')E= {^{t}X'}E'
\] \[
v = {^{t}X}E = {^{t}X'}E
\] Sostituendo si ha \({^{t}X}E = {^tX'}AE\), ove \(A\) è la matrice del cambiamento di base da \(B\) a \(B'\), quindi, dato che le componenti dei vettori sono univocamente determinate \[
X = {^tA}X'
\] \[
X' = {^tA^{-1}}X
\] Possiamo dire quindi che le componenti di uno stesso vettore rispetto a due basi \(B\) e \(B'\) sono legate dalla matrice del cambiamento di base da \(B\) a \(B'\).


\chapter{Autovalori, autovettori e diagonalizzabilità}
\section{Ricerca di autovalori, polinomio caratteristico}
\dfn{Polinomio ed equazione caratteristica}{Se \(A\) è una matrice quadrata di ordine \(n\), si dice \textbf{polinomio caratteristico} di \(A\), e si indica \(p_A(\lambda )\), il determinante della matrice \(A-\lambda I_n\), cioè \[
p_A(\lambda ) = |A-\lambda I_n| 
\] L'equazione \(p_A(\lambda) = |A -\lambda I_n| \) è detta \textbf{equazione caratteristica} di \(A\).}

\dfn{Autovalori}{Le radici del polinomio caratteristico si chiamano \textbf{autovalori} di \(A\).}
\dfn{Autospazio}{Lo spazio delle soluzioni del sistema \((A-\overline{\lambda }I_n)X=0\), dove \(\overline{\lambda }\) è un autovalore, si chiama \textbf{autospazio} associato a \(\overline{\lambda }\) e si indica con \(V_{\overline{\lambda }}\).}
\dfn{Autovettori}{I vettori non nulli dell'autospazio \(V_{\overline{\lambda }}\) si chiamano  \textbf{autovettori} relativi a \(\overline{\lambda }\).}
\paragraph{Osservazione:} Si potrebbe dimostrare che se il polinomio caratteristico di \(A \in M_n(K)\) ha grado \(n\) allora gli autovalori di \(A\) sono al massimo \(n\).
\dfn{Matrici simili}{Due matrici \(A,B \in M_n(K)\) si dicono \textbf{simili} se esiste \(P \in M_n(K)\) con \(|P| \neq 0\) tale che  \[
B = P^{-1}AP \quad PB = AP
\] }

\mprop{}{Due matrici simili \(A,B\) hanno lo stesso determinante e lo stesso polinomio caratteristico (e di conseguenza gli stessi autovalori).}
\pf{Dimostrazione}{Per ipotesi le due matrici \(A,B\) sono simili quindi:\[
\exists P \in M_n(K), \ |P| \neq 0 : \ B = P^{-1}AP
\] \[
|B| = |P^{-1}AP| = |P^{-1}| |A| |P| = \frac{1}{|P| }|A| |P| =|A| \implies |B| = |A| 
\] \[
p_B(\lambda ) = |B - \lambda I_n| = |P^{-1}AP - \lambda P^{-1}I_n P| = |P^{-1}(A - \lambda I_n)P| = \frac{1}{|P| }|A-\lambda I_n| |P| = |A - \lambda I_n| = p_A(\lambda )
\] e attraverso questa serie di passaggi abbiamo potuto dimostrare che se due matrici sono simili allora avranno sia lo stesso determinante che lo stesso polinomio caratteristico. }

\section{Matrici diagonalizzabili}
\dfn{Matrice diagonalizzabile}{Una matrice \(A \in M_n(K)\) si dice \textbf{diagonalizzabile} se è simile ad una matrice diagonale, ovvero esistono \(D, P \in M_n(K)\) con \(D\) matrice diagonale, \(|P| \neq 0\) e \(D = P^{-1}AP\).}

\thm{Primo criterio di diagonalizzabilità}{Una matrice \(A \in M_n(K)\) è diagonalizzabile se, e soltanto se, \(K^{n}\) ammette una base costituita da autovettori di \(A\).}
\pf{Dimostrazione}{\("\implies "\) Per ipotesi \(A\) è diagonalizzabile quindi \(\exists \ D,P \in M_n(K): D\) è diagonale \(|P| \neq 0\) e \(PD = AP\). Per semplicità denotiamo le colonne di \(P= 
\left( \; \begin{matrix}
    P_1 & P_2 & \ldots  & P_n \\
\end{matrix} \; \right)
\). \[
AP = A 
\left( \; \begin{matrix}
    P_1 & P_2 & \ldots  & P_n \\
\end{matrix} \; \right) =
\left( \; \begin{matrix}
    AP_1 & AP_2 & \ldots  & AP_n \\
\end{matrix} \; \right)
\] \[
PD = 
\left( \; \begin{matrix}
    P_1 & P_2 & \ldots  & P_n \\
\end{matrix} \; \right) 
\left( \; \begin{matrix}
    d_1 & 0 & \ldots  & 0 \\
    0 & d_2 & \ldots  & 0 \\
    \vdots & \vdots & \ddots & \vdots \\
    0 & 0 & \ldots  & d_n \\
\end{matrix} \; \right) = 
\left( \; \begin{matrix}
    d_1P_1 & d_2P_2 & \ldots  & d_nP_n \\
\end{matrix} \; \right)
\] Quindi \[
\left( \; \begin{matrix}
    AP_1 & AP_2 & \ldots  & AP_n \\
\end{matrix} \; \right) = 
\left( \; \begin{matrix}
    d_1P_1 & d_2P_2 & \ldots  & d_nP_n \\
\end{matrix} \; \right)
\iff 
AP_1=d_1P_1, \ AP_2 = d_2P_2, \ \ldots, \ AP_n = d_n P_n
\] \[
\implies AX = \lambda X \quad \lambda =d_i \quad X = P_i
\] dove \(d_i\) è un autovalore, \(P_i\) è un autovettore di \(A\) e \(
\left( \; \begin{matrix}
    P_1 & P_2 & \ldots  & P_n \\
\end{matrix} \; \right) 
\) è una sequenza di \(n\) autovettori. Poiché \(\dim K^{n}=n\) e la sequenza è composta da \(n\) vettori, è sufficiente controllare la lineare indipendenza di \(P\). Ma siccome avevamo supposto per ipotesi che \(|P| \neq 0\) le sue \(n\) colonne sono linearmente indipendenti. Quindi \(B = (P_1, P_2, \ldots , P_n)\) è una base di \(K^{n}\) costituita da autovettori di \(A\).

"\(\impliedby \)" è analogo, basta ripercorrere il ragionamento a ritroso.}

\paragraph{Osservazione:} Se \(A \in M_n(K)\) è diagonalizzabile allora:
\begin{itemize}
    \item \(D\) ha sulla diagonale principale gli autovalori di \(A\);
    \item \(P\), cioè la matrice diagonalizzante, ha nelle colonne gli autovettori della base di \(K^{n}\).
\end{itemize}

\dfn{Molteplicità algebrica e geometrica}{Sia \(\overline{\lambda }\) un autovalore di \(A \in M_n(K)\); si chiama:
\begin{itemize}
    \item \textbf{molteplicità algebrica} di \(\overline{\lambda }\) il numero di volte che \(\overline{\lambda }\) è radice del polinomio caratteristico, e si indica con \(a_{\overline{\lambda }}\) 
    \item \textbf{molteplicità geometrica} di \(\overline{\lambda }\) la dimensione dell'autospazio \(V_{\overline{\lambda }}\) associato a \(\overline{\lambda }\), e si indica con \(g_{\overline{\lambda }}\).
\end{itemize}}

\mprop{}{Sia \(\overline{\lambda }\) un autovalore di \(A \in M_n(K)\). Allora \[
1 \le g_{\overline{\lambda }} \le a_{\overline{\lambda }}
\] }

\mprop{}{Sia \(A \in M_n(K)\) e siano \(\lambda _1, \lambda _2, \ldots ,\lambda_n\) \(t\) autovalori di \(A\) distinti tra loro, allora la somma dei relativi autospazi è diretta. \[
V_{\lambda _1} \oplus V_{\lambda _2} \oplus \ldots \oplus V_{\lambda _t}
\] }
\paragraph{Osservazioni:} 
\begin{enumerate}
    \item \(A \in M_n(K) \implies \deg(p_A(\lambda )) = n\), quindi ho al massimo \(n\) autovalori;
    \item \(\sum a_{\lambda _i}\le n\);
    \item \(\sum a_{\lambda _i}= n \iff\) tutti gli autovalori di \(A\) sono in \(K\);
    \item \(S =V_{\lambda _1} \oplus V_{\lambda _2} \oplus \ldots \oplus V_{\lambda _t} \implies \dim S = \sum \dim V_{\lambda _i} = \sum g_{\lambda _i}\)
    \item Autovettori provenienti da autospazi diversi sono tra loro linearmente indipendenti (perché la somma è diretta).
\end{enumerate}

\thm{Secondo criterio di diagonalizzabilità}{Sia \(A \in M_n(K)\) e siano \(\lambda _1, \lambda _2, \ldots , \lambda _n\) gli autovalori distinti di \(A\). Allora \(A\) è diagonalizzabile se, e soltanto se:
\begin{enumerate}
    \item tutti gli autovalori di \(A\) sono in \(K\);
    \item Per ogni autovalore vale \(a_{\lambda _i} = g_{\lambda _i}\)(e allora si dice che l'autovalore è regolare).
\end{enumerate}}

\pf{Dimostrazione}{"\(\implies\)" Per ipotesi \(A\) è diagonalizzabile. Per il primo criterio di diagonalizzabilità \(K^{n}\) ammette una base \(B\) formata da autovettori, cioè tale che  \(\mcL(B) = K^{n}\) e \(B \subseteq V_{\lambda _1} \oplus V_{\lambda _2} \oplus \ldots \oplus V_{\lambda _t} \le K^{n}\). Quindi \[
K^{n} = \mcL(B) \le \mcL(V_{\lambda _1} \oplus V_{\lambda _2} \oplus \ldots \oplus V_{\lambda _t}) = V_{\lambda _1} \oplus V_{\lambda _2} \oplus \ldots \oplus V_{\lambda _t} \le K^{n} 
\] \[
\implies V_{\lambda _1} \oplus V_{\lambda _2} \oplus \ldots \oplus V_{\lambda _t} = K^{n} \]
\[
\implies n = \dim K^{n} = \dim (V_{\lambda _1} \oplus V_{\lambda _2} \oplus \ldots \oplus V_{\lambda _t}) = \sum g_{\lambda _i}\le \sum a_{\lambda _i} \le n
\]
Siccome  \(\sum a_{\lambda _i}=n\) tutti gli autovalori di \(A\) sono in \(K\). Inoltre \(\sum g_{\lambda _i} = \sum a_{\lambda _i}\) e \(g_{\lambda _i}\le a_{\lambda _i}\implies a_{\lambda _i}=g_{\lambda _i}\).

"\(\impliedby \)" Per ipotesi abbiamo che tutti gli autovalori di \(A\) soni in \(K\) e per ogni autovalore vale \(a_{\lambda _i}= g_{\lambda _i}\). Per ogni autovalore \(\overline{\lambda }\) avremo un relativo autospazio a cui corrisponde una relativa base di autovettori  \(B_1, B_2, \ldots , B_t\). Chiamiamo \(B = \bigcup_{i = 1}^t B_i \), cioè l'unione di tutte le basi. Certamente \(B\) è libera perché la somma di sottospazi distinti è diretta. \[
|B| = |\bigcup B_i | = \sum |B_i| = \sum \dim V_{\lambda _i}= \sum g_{\lambda _i}= \sum a_{\lambda _i}=n
\] Quindi \(B\) è una base di \(K^{n}\) costituita da autovettori e per il primo criterio di diagonalizzabilità \(A\) è diagonalizable.}


\chapter{Forme bilineari e prodotti scalari}
\section{Forme bilineari}
\dfn{Forma bilineare e prodotto scalare}{Sia \(V_n(K)\) uno spazio vettoriale. Una \textbf{forma bilineare} in \(V\) è una funzione \(*: \ V \times V \to K: \)
\begin{itemize}
    \item \((u+v) * w = u * w + v * w \qquad \forall u, v, w \in V \ \forall k \in K\)
    \item \(u * (v + w) = u * v + u * w \qquad \forall u, v, w \in V \ \forall k \in K\)
    \item \((ku) * v = u * (kv) = k (u * v) \qquad \forall u, v, w \in V \ \forall k \in K\) 
\end{itemize} 
Se poi \(*\) verifica anche l'ulteriore proprietà
\begin{itemize}
    \item \(v * w = w * v \qquad \forall u, v, w \in V \ \forall k \in K\)
\end{itemize}
Allora si chiama \textbf{prodotto scalare} (o forma bilineare simmetrica).}
\paragraph{Osservazione:} Si deduce chiaramente che \(\forall v \in V \quad \ul{0} * v = 0 = v * \ul{0} \).

\ex{Prodotto scalare euclideo e standard}{
\begin{enumerate}
    \item Definiamo il \textbf{prodotto scalare euclideo} come una funzione \(*:\RR^{n} \times \RR^{n} \to \RR: \) \[
     (x_1, x_2, \ldots , x_n)*(x_1', x_2', \ldots , x_n') = x_1x_1'+x_2x_2'+\ldots +x_nx_n'
    \] 
    \item Definiamo il \textbf{prodotto scalare standard} come la funzione \(*: M_n(\RR) \times M_n(\RR) \to \RR:\) \[
\left( \; \begin{matrix}
    x_{11} & x_{12} & \ldots  & x_{1n} \\
    x_{21} & x_{22} & \ldots  & x_{2n} \\
    \vdots & \vdots & \ddots & \vdots \\
    x_{n 1} & x_{n 2} & \ldots  & x_{n n} \\
\end{matrix} \; \right) \ * \
\left( \; \begin{matrix}
    x_1' & x'_{12} & \ldots  & x'_{1n} \\
    x'_{21} & x'_{22} & \ldots  & x'_{2n} \\
    \vdots & \vdots & \ddots & \vdots \\
    x'_{n 1} & x'_{n 2} & \ldots  & x'_{n n} \\
\end{matrix} \; \right) = x_{11}x_{11}'+x_{12}x_{12}'+ \ldots + x_{n n}x_{n n}'
    \] 
\end{enumerate}
}

\section{Prodotti scalari e ortogonalità}
\dfn{Ortogonalità}{In uno spazio vettoriale \(V(K)\), con prodotto scalare "\(\cdot \)", due vettori \(v\) e \(w\) di \(V\) si dicono \textbf{ortogonali}, e si scrive \(v \perp w\), se \(v \cdot w = 0\).}
\dfn{Complemento ortogonale}{Sia \(V(K)\) uno spazio vettoriale e "\(\cdot \)" un prodotto scalare. Sia \(\emptyset \neq A \subseteq V\); si chiama  \textbf{complemento ortogonale} (o più semplicemente ortogonale) di \(A\) l'insieme \[
A^{\perp} = \{v \in V : \ v \perp w, \ \forall w \in A\} \qquad \ul{0} \in A^{\perp} \neq \emptyset
\] }

\mprop{}{Sia \(V(K)\) uno spazio vettoriale con prodotto scalare "\(\cdot \)". Sia \(\emptyset \neq A \subseteq V\). Allora \(A^{\perp}\) è un sottospazio vettoriale.}
\pf{Dimostrazione}{Sappiamo che \(\ul{0} \in A^{\perp} \neq \emptyset\) \\ 
Dobbiamo dimostrare che \[
\forall u_1, u_2 \in A^{\perp}, \ \forall k_1,k_2 \in K \qquad k_1u_1+k_2u_2 \in A^{\perp}
\] Possiamo scrivere per la proprietà di ortogonalità che\[
\forall w \in A \quad u_1\cdot w = 0 \quad u_2 \cdot w = 0
\] Quindi \[
(k_1u_1+k_2u_2) \cdot w = (k_1u_1) \cdot w + (k_2u_2) \cdot w = k_1 (\underbrace{u_1 \cdot w}_{=0}) + k_2 (\underbrace{u_2 \cdot w}_{=0} )
\] \[
\implies k_1u_1+k_2u_2 \in A^{\perp} \implies A^{\perp} \ \text{è un sottospazio.}
\] 
}

\paragraph{Osservazioni:} 
\begin{enumerate}
    \item \(A \subseteq B \implies A^{\perp} \supseteq B^{\perp}\)
    \item \(A^{\perp} = [\mcL(A) ]^{\perp}\)
    \item Generalmente se \(A \le V(K) \implies A \neq (A^{\perp})^{\perp}\), ma \(A \subseteq (A^{\perp})^{\perp}\)
\end{enumerate}

\mprop{}{Sia \(V_n(K)\) uno spazio vettoriale con prodotto scalare "\(\cdot \)" e siano \(v, w \in V(K)\) con \(w \cdot w \neq \ul{0} \). Allora \[
\exists \ v_1, v_2 \in V: \ v = v_1+v_2, \ v_1 = kw, \ v_2 \perp w
\] }

\pf{Dimostrazione}{ \[
k = \frac{v \cdot w}{w \cdot w} \qquad v_1= kw = \left( \frac{v \cdot w}{w \cdot w} \right) \cdot w
\] \[
v_2=v-v_1 \iff v_1+v_2=v
\] Ora verifichiamo che \(v_2 \perp w\) \[
v_2 \perp w \iff  (v-v_1) \cdot w = \left( v - \frac{v \cdot  w}{w \cdot w} \right) \cdot  w = v - w - \frac{v \cdot w}{w \cdot w}\cdot w \cdot w = v \cdot w - v \cdot w = 0
\] }
\dfn{Coefficiente di Fourier e proiezione}{Sia \(V_n(K)\) uno spazio vettoriale con prodotto scalare "\(\cdot \)" e siano \(v, w \in V(K)\) con \(w \cdot w \neq \ul{0} \). Allora \[
    k = \frac{v \cdot w}{w \cdot w}
\] si chiama \textbf{coefficiente di Fourier} di \(v\) lungo \(w\) e \[
v_1= \frac{v \cdot w}{w \cdot w}\cdot w
\] si chiama \textbf{proiezione} di \(v\) lungo \(w\).}

\dfn{Forma quadratica}{Sia \(V_n(K)\) uno spazio vettoriale con prodotto scalare "\(\cdot \)" e sia \(v \in V(K)\). Si chiama \textbf{forma quadratica} associata a "\(\cdot \)" la funzione \[
q: \ 
\begin{cases}
    \ V \to K  \\
    \ v \mapsto q(v) = v \cdot v \\
\end{cases}
\] }

\section{Spazi con prodotto scalare definito positivo}
\dfn{Prodotto scalare definito positivo}{Sia \(V(K)\) uno spazio vettoriale su campo \(K\) \ul{ordinato}. Un prodotto scalare "\(\cdot \)" in \(V\) si dice \textbf{definito positivo} se \(\) \[
\forall v \in V \quad v \cdot v \ge 0 \quad e \quad v \cdot v = 0 \iff v = \ul{0} 
\] Per chiarezza da qui in avanti quando si parla di prodotti scalari definiti positivi \(K = \RR \) in modo tale che esso sia ordinato. Di conseguenza denotiamo con \textbf{spazio vettoriale metrico reale} \(V_n^{\circ}(\RR )\), cioè uno spazio vettoriale dotato di un prodotto scalare definito positivo.}

\dfn{}{Dato \(V^{\circ}_n(\RR)\) si chiama  \textbf{norma} la funzione \[
\|\cdot \|: \ 
\begin{cases}
    \ V \to \RR \\
    \ v \mapsto \|v\|=\sqrt{v \cdot v} = \sqrt{q(v)} \\
\end{cases}
\] }
\ex{Vettori geometrici}{\[
\vec{v} \cdot \vec{w} = |\vec{v} | |\vec{w}| \cos \alpha 
\] \[
\|\vec{v}\|=\sqrt{\vec{v} \cdot \vec{v}} = \sqrt{|\vec{v}| |\vec{v}| \cos 0} = \sqrt{|\vec{v}| ^2} = |\vec{v}| 
\] }

\paragraph{Osservazioni:}

\begin{enumerate}
    \item La norma generalizza la nozione di "lunghezza" di un vettore.
    \item \(\|v\|= \ul{0} \iff v \cdot v = 0 \iff v = \ul{0} \) 
\end{enumerate}
\mprop{}{In \(V^{\circ}_n(\RR )\) valgono i seguenti fatti
\begin{enumerate}
    \item \(\|v\|\ge 0 \quad e\quad \|v\|=0 \iff v = \ul{0} \) 
    \item \(\|kv\|=|k|\|v\| \) 
    \item \(|v \cdot w| \le \|v\|\cdot \|w\|\) (disuguaglianza di Schwarz)
    \item \(\|v+w\|\le \|v\|+\|w\|\) (disuguaglianza triangolare)
\end{enumerate}}

\paragraph{Osservazioni:} Sia "\(\cdot \)" un prodotto scalare euclideo definito su \(\RR ^{n}(\RR )\). La sua base canonica è \[B_c = ((1,0,\ldots , 0), (0,1,0, \ldots , 0), \ldots , (0,0, \ldots , 0, 1)) = (e_1, e_2, \ldots , e_n)\] 
\begin{enumerate}
    \item \(\|e_i\|=\sqrt{e_i \cdot e_i} = 1\) 
    \item \(e_i \cdot e_j = 0 \quad \forall i \neq j \implies e_i \perp e_j\) 
    \item \(\forall \underbrace{(x_1, x_2, \ldots , x_n)}_{=v}  = x_1(1, 0, \ldots , 0) + x_2 (0,1, 0 , \ldots , 0) + \ldots + x_n(0,0, \ldots , 0, 1)\) \\ \(\implies v \cdot e_i = x_i =\) i-esima componente di \(v\) rispetto a \(B_c\)
\end{enumerate}

\dfn{Base ortogonale e ortonormale}{I vettori \(v_1, v_2, \ldots , v_n\) di uno spazio vettoriale \(V^{\circ}_n(\RR )\) formano un insieme \textbf{ortogonale} se \(v_i \cdot v_j = 0, \ i \neq j\). Se inoltre ciascuno dei \(v_i\) ha norma unitaria, allora parleremo di insieme  \textbf{ortonormale}. Se poi tali vettori costituiscono una base di \(V^{\circ}_n(\RR )\) parleremo di base ortogonale o ortonormale.}

\mprop{}{Se \(\emptyset\neq A \subseteq V^{\circ}_n(\RR )\) e costituito da vettori tutti non nulli. Allora \(A\) è libero.}
\pf{Dimostrazione}{\[A = \{v_1,v_2, \ldots , v_n\} \quad v_i \cdot v_j = 0 \quad \forall i \neq j. \quad  \text{Siano} \ \alpha _1, \alpha _2, \ldots , \alpha _n \in \RR : \ \alpha _1 v_1 + \alpha _2v_2 + \ldots + \alpha _n v_n = \ul{0} \]
\[
0 = 0 \cdot v_1=(\alpha _1, \alpha _2, \ldots , \alpha _n) \cdot v_1 = \alpha _1 \underbrace{v_1 \cdot v_1}_{\neq 0 \implies v_1 \neq \ul{0} \implies \|v_1\|^2 \neq 0}  + \alpha _2 \underbrace{v_2 \cdot v_2}_{=0}  + \ldots + \alpha _n \underbrace{v_n \cdot v_n}_{= 0}  = \underbrace{\|v_1\|^2}_{\neq 0} \underbrace{\alpha _1}_{\implies \alpha _1=0} 
\] Ripeto il ragionamento per ciascuno dei \(v_i\) e ottengo che gli unici \(\alpha \) che mi danno il vettore nullo sono quelli tutti nulli. Quindi se \(\alpha _1 = \alpha _2 = \ldots = \alpha _n = 0 \implies A\) è libero.
}

\paragraph{Osservazione:} In \(V^{\circ}_n(\RR )\) se \(A\) è un insieme ortogonale di \(n\) vettori tutti diversi dal vettore nullo allora \(A\) è libero. Dunque fissato un ordine abbiamo una base ortogonale.

\thm{Processo di ortogonalizzazione di Gram-Schmidt}{Siano \(V^{\circ}_n(\RR )\) e \(B = (e_1, e_2, \ldots , e_n)\) una base. La sequenza \(B' = (e_1', e_2', \ldots , e_n')\) così definita \[
e_1'=e_1
\] \[
e_2'=e_2-\frac{e_2\cdot e_1'}{e_1'\cdot e_1'}\cdot e_1'
\] \[
e_3'=e_3-\frac{e_3\cdot e_1'}{e_1'\cdot e_1'}\cdot e_1'-\frac{e_3\cdot e_2'}{e_2'\cdot e_2'}\cdot e_2'
\] \[
\vdots
\] \[
e_n' = e_n - \frac{e_n \cdot e_1'}{e_1'\cdot e_1'}\cdot e_1'- \ldots - \frac{e_n \cdot e_{n-1}'}{e_{n-1}'\cdot e_{n-1}'}\cdot e'_{n-1}
\] è una base ortogonale di  \(V^{\circ}_n(\RR )\).}
\paragraph{Osservazione:} Se i primi \(p\) vettori di \(B\) sono già ortogonali tra loro il metodo di Gram-Schmidt non li cambia.

\thm{}{Se \(A\) è un sottoinsieme non vuoto di \(V^{\circ}_n(\RR )\), la cui copertura non coincide con \(V^{\circ}_n(\RR )\), allora \[
V^{\circ}_n(\RR ) = \mcL(A) \oplus A^{\perp}
\] }
\pf{Dimostrazione}{Prima di tutto dimostriamo che \(\mcL(A) \cap A^{\perp} = \{\ul{0} \} \) infatti: \(v \in \mcL(A) \cap A^{\perp}\) e se \(v \in A^{\perp}=[\mcL(A) ]^{\perp} \quad v \cdot v = 0 \implies v = \ul{0} \) poiché ci troviamo in un prodotto scalare definito positivo. Quindi la somma è diretta. Ora si può dimostrare che \(\mcL(A) \oplus A^{\perp}=V^{\circ}_n(\RR )\). Sia \(\dim \mcL(A) = p\) e sia \(B = (v_1, v_2, \ldots , v_p)\) una base ortogonale di \(\mcL(A) \); per il teorema di completamento ad una base possiamo completare \(B\) ad una base di \(V^{\circ}_n(\RR )\). Aggiungiamo a \(B\) \(n-p\) vettori. Ora applichiamo a tale base il processo di ortogonalizzazione di Gram-Schmidt.  \(B' = (v_1 , \ldots , v_p , v_{p+1}', \ldots , v_n')\) è una base ortogonale di \(V_n^{\circ}(\RR )\). Quindi \(\mcL(B') = V_n^{\circ}(\RR )\). Ora tutti i vettori aggiunti sono ortogonali ai vettori originali, cioè \(v_{p+1}', \ldots , v_n' \in \mcL(A)^{\perp} = A^{\perp} \implies \mcL(A) \oplus A^{\perp} = V^{\circ}_n(\RR ) \).}

\paragraph{Osservazioni:} 
\begin{enumerate}
    \item \(A^{\perp}\) è un complemento diretto di \(\mcL(A) \) 
    \item Per la formula di Grassmann abbiamo che \[
    n = \dim (\mcL(A) \oplus A^{\perp}) = \dim \mcL(A) + \dim A^{\perp} \implies \dim A^{\perp} = n-\dim \mcL(A) 
    \] 
    \item Per il punto precedente possiamo affermare che se il prodotto scalare è definito positivo allora \(U \le V_n^{\circ}(\RR ) \implies U = (U^{\perp})^{\perp}\)
\end{enumerate}

\thm{}{L'insieme delle soluzioni di un sistema lineare omogeneo è un sottospazio vettoriale di \(\dim : \ n- \rho(A) \)}
\pf{Dimostrazione}{ In \(\RR ^{n}\) con prodotto scalare euclideo \[
(x_1,x_2, \ldots , x_n) \cdot (x_1', x_2', \ldots , x_n') = x_1x_1' + x_2x_2' + \ldots + x_n x_n'
\] Quindi possiamo riscrivere il sistema come
\[
\begin{cases}
    \ a_{11}x_1+\ldots +a_{1n}x_n = 0 \\
    \ \ldots \ldots \ldots \ldots \ldots  \ldots \ldots \ldots  \\
    \ a_{m 1}x_1+\ldots +a_{mn}x_n = 0 \\
\end{cases} \iff
\begin{cases}
    \ (a_{11}, \ldots , a_{1n}) \cdot (x_1, \ldots , x_n) = 0 \\
    \ \ldots \ldots \ldots \ldots \ldots \ldots \ldots \ldots \ldots \ldots  \\
    \ (a_{m1}, \ldots , a_{mn}) \cdot (x_1, \ldots , x_n) = 0 \\
\end{cases}
\] Pensando alle righe di \(A\) come vettori di \(\RR ^{n}\) le equazioni del sistema esprimono il fatto che il prodotto scalare di tali righe per il generico vettore \((x_1, x_2, \ldots , x_n)\) è uguale a zero. Quindi il generico vettore è ortogonale a tutte le righe di \(A\). Chiamando \(\mcL(R) \) lo spazio generato dalle righe di \(A\). L'insieme \(S\) delle soluzioni di \(AX = \ul{0} \) coincide con \(\mcL(R) ^{\perp}\). E quindi per il teorema di Kronecker \(\dim S = n - \dim \mcL(R) = n - \rho(A) \).}

\section{Matrici di forme bilineari}

\dfn{Matrice di forma bilineare}{Sia \(V_n(K)\) uno spazio vettoriale, "\(*\)" una forma bilineare e \(B=(e_1,e_2, \ldots , e_n)\) base di \(V_n(K)\). Si chiama \textbf{matrice della forma bilineare} "\(*\)" rispetto a \(B\) \[
A^{*}_B =
\left( \; \begin{matrix}
    e_1*e_1 & e_1*e_2 & \ldots  & e_1*e_n \\
    e_2*e_1 & e_2*e_2 & \ldots  & e_2*e_n \\
    \vdots & \vdots & \ddots & \vdots \\
    e_n*e_1 & e_n*e_2 & \ldots  & e_n*e_n \\
\end{matrix} \; \right) \in M_n(K)
\] Si può indicare in modo più compatto con \[
A^{*}_B = (e_i * e_j)
\] }
\nt{La matrice di una forma bilineare dipende dalla base fissata.}

\mprop{}{La matrice che rappresenta un prodotto scalare rispetto a una base qualsiasi è simmetrica.}

\pf{Dimostrazione}{\(B = (e_1, e_2, \ldots , e_n)\) e "\(\cdot \)" è il prodotto scalare. Allora \(A^{\cdot }_B = (e_i \cdot e_j) = (e_j \cdot e_i) = {^tA^{\cdot }_B}\).}

\mprop{}{Sia "\(\cdot \)" un prodotto scalare su \(V_n(K)\) e sia \(B\) una sua base. Sia \(A^{\cdot }_B\) una matrice associata a "\(\cdot \)" rispetto alla base \(B\). Allora 
\begin{itemize}
    \item \(B\) è ortogonale \(\iff A^{\cdot }_B\) è diagonale \[
    e_i \cdot e_j = 0 \quad \forall i \neq j \iff a_{ij} = 0 \quad \forall i \neq j
    \] 
    \item \(B\) è ortonormale \(\iff A^{\cdot }_B = I_n \in M_n(K)\) \[
    e_i \cdot e_j = 0 \quad \forall i \neq j \quad e \quad e_i \cdot  e_i = 1 \quad \forall 1 \le i \le n \iff a_{ij} = 0 \quad \forall i \neq j \quad e\quad a_{ii} = 1 \quad \forall 1 \le i \le n
    \] 
\end{itemize}}

\paragraph{Osservazione:} Utilizzando la matrice associata ad una forma bilineare "\(*\)" è possibile calcolare \[v * w \quad \forall v, w \in V_n(K)\].

\mprop{}{Sia \(B\) una base di \(V_n(K)\) e sia "\(*\)" una forma bilineare su \(V\). Dette \[
X = \left( \; \begin{matrix} x_1\\ \vdots\\ x_n \end{matrix} \; \right) \quad e \quad  Y = \left( \; \begin{matrix} y_1\\ \vdots\\ y_n \end{matrix} \; \right)
\] le matrici colonne delle componenti rispettivamente di \(v\) e di \(w \in V\) risulta: \[
v * w = {^tX} A^{*}_B Y
\] }

\section{Matrici ortogonali e basi ortonormali}
\dfn{Matrice ortogonale}{Sia \(A \in M_n(K)\) diciamo che \(A\) è \textbf{ortogonale} se \({^tA}= A ^{-1}\). Quindi \[A {^tA}= {^tA}A = I_n\]}

\mprop{}{Sia \(A \in M_n(K)\) una matrice ortogonale. Allora \(|A|  \in \{-1, 1\} \)}

\pf{Dimostrazione}{\[
|I_n| = 1 = |A A ^{-1}| = |A {^tA}| = |A| |{^tA}| = |A| |A| = |A| ^2
\] \[
|A| ^2 = 1 \iff |A| = \pm 1
\] }

\mprop{}{Sia \(A \in M_n(K)\). \(A\) è ortogonale se, e soltanto se, le sue righe (o colonne) costituiscono una base ortonormale di \(\RR ^{n}(\RR )\) rispetto al prodotto scalare euclideo (dello spazio euclideo \(\RR ^{n}(\RR )\)).}

\pf{Dimostrazione}{"\(\implies \)" \[
\left( \; \begin{matrix} R_1\\ \vdots\\ R_n \end{matrix} \; \right) \iff  {^tA}= ({^tR_1}, \ldots , {^tR_n})
\] \[
A {^tA} = I_n = \left( \; \begin{matrix} R_1\\ \vdots\\ R_n \end{matrix} \; \right)({^tR_1}, \ldots , {^tR_n}) = 
\left( \; \begin{matrix}
    R_1 \cdot R_1 & R_1 \cdot R_2 & \ldots  & R_1 \cdot R_n \\
    R_2 \cdot R_1 & R_2 \cdot R_2 & \ldots  & R_2 \cdot R_n \\
    \vdots & \vdots & \ddots & \vdots \\
    R_n \cdot R_1 & R_n \cdot R_2 & \ldots  & R_n \cdot R_n \\
\end{matrix} \; \right)
\] \[
R_i \cdot R_j = 0 \quad \text{se} \quad i \neq j, \quad R_i \cdot R_i = 1 \quad \forall 1 \le i \le n
\] Quindi le righe di \(A\) sono una base ortonormale. Il ragionamento è completamente analogo per le colonne. \\ "\(\impliedby \)" Si può dimostrare ripercorrendo le implicazioni al contrario.} 

\section{Matrici reali simmetriche}
\thm{}{Sia \(A \in M_n(\RR )\) simmetrica allora 
\begin{enumerate}
    \item Gli autovalori di \(A\) sono tutti reali (teorema spettrale)
    \item Gli autovettori di \(A\) relativi ad autospazi distinti sono ortogonali tra loro
\end{enumerate}}

\pf{Dimostrazione del punto 2}{Siano \(x\) e \(y\) autovettori relativi ad autovalori \(\lambda \) e \(\mu\) distinti. Quindi \(AX = \lambda x\) e \(AX = \mu y\). Sia \(\lambda \neq 0\). Quindi \[
        ({^tx}{^ty}) \lambda = (\lambda {^tx})y = {^t(x\lambda )y = {^t(Ax)y = \underbrace{({^tx}{^tA})y = ({^tx}A)y}_{\text{per la simmetria di \(A\)}}} = {^tx}(Ay)}
\] \[
= {^tx}\mu y = \mu ({^tx}y) = \mu ({^tx}{^ty}) \implies ({^tx}{^ty})\lambda = ({^tx}{^ty}) \mu
\] \[
\lambda k = \mu k \iff (\lambda - \mu) k = 0 \iff \mu = \lambda \quad \text{oppure}\quad {^tx}{^ty}=0
\] ma \(\mu \neq \lambda \) perché \(x\) e \(y\) stanno in autospazi distinti \(\implies {^tx}{^ty}=0 \implies x\) e \(y\) sono ortogonali.}

\cor{}{Una matrice reale e simmetrica di ordine \(n\) ammette \(n\) autovalori contati con la loro molteplicità algebrica.}

\dfn{Matrice ortogonalmente diagonalizzabile}{Data \(A \in M_n(K)\) è detta \textbf{ortogonalmente diagonalizzabile} se esistono \(D\), matrice diagonale di ordine \(n\), e \(P\) matrice ortogonale di ordine \(n\) tali che \[
D = P^{-1}AP = {^tP}AP
\] }
\thm{}{I seguenti fatti sono equivalenti
\begin{enumerate}
    \item \(A\in M_n(\RR )\) è ortogonalmente diagonalizzabile;
    \item \(\RR ^{n}\) ammette una base ortonormale di autovettori di \(A\);
    \item \(A\) è una matrice reale e simmetrica.
\end{enumerate}}

\chapter{Spazi affini}
\section{\(A_n(K)\), spazio affine di dimensione \(n\)}
\dfn{Spazio affine}{Si dice \textbf{spazio affine} di dimensione \(n\) sul campo \(K\), e si indica \(\AA_{n}(K) \), la struttura costituita da
\begin{enumerate}
    \item un insieme non vuoto \(A \), detto insieme dei punti
    \item uno spazio vettoriale \(V_n(K)\) 
    \item un'applicazione \[
    f: \quad A \times A \to V_n(K)
    \] con le seguenti proprietà
    \begin{enumerate}
        \item \(\forall P \in A \ e \ \forall v \in V \quad \exists ! \ Q \in A : \quad f(P, Q) = \vec{PQ} = v\)
        \item \(\vec{PQ} + \vec{QR} = \vec{PR} \quad \forall P, Q, R \in A\)
    \end{enumerate}
\end{enumerate}}

\mprop{}{In \(A_n(K)\), per ogni \(P,Q\) e \(R \in A\)
\begin{enumerate}
    \item il vettore \(\vec{RR} = \ul{0} \)
    \item \(\vec{PQ} = \vec{PR} \iff Q = R\)
    \item \(\vec{PQ} = \ul{0}  \iff P=Q\)
    \item \(v = \vec{PQ} \implies -v = \vec{QP}\)
    \item \(\forall P_1, P_2, Q_1, Q_2 \in A\) risulta \(\vec{P_1P_2} = \vec{Q_1Q_2} \iff \vec{P_1Q_1} = \vec{P_2Q_2}\)
\end{enumerate}}

\pf{Dimostrazione}{ Dimostriamo ogni punto separatamente
\begin{enumerate}
    \item \(\vec{R R} + \vec{RR} = \vec{RR}\) perciò \(2 \vec{RR} = \vec{RR} \iff \vec{RR} = \ul{0} \)
    \item posto \(v = \vec{PQ}\) allora \(v = \vec{PR}\), ma \(\exists ! \ Q : \ \vec{PQ}=v \implies R = Q\) 
    \item per la proprietà 1 \(\vec{RR} = \ul{0} \implies \) per l'unicità di \(Q: \vec{PQ}=\ul{0} \implies Q=P\)
    \item \(\vec{PQ} + \vec{QP} = \vec{PP} = \ul{0} \implies \vec{PQ} = - \vec{QP}\)
    \item ovvio, essendo \(\vec{P_1P_2} + \vec{P_2Q_2} = \vec{P_1Q_2} = \vec{P_1Q_1} + \vec{Q_1Q_2}\)
        \begin{center}
            \begin{tikzpicture}
                \coordinate (P_1) at (0, 0);
                \coordinate (P_2) at (2, 0);
                \coordinate (Q_1) at (1, 1);
                \coordinate (Q_2) at (3, 1);
                \draw[thick, -stealth] (P_1) -- (P_2);
                \draw[thick, -stealth] (P_1) -- (Q_1);
                \draw[thick, -stealth] (Q_1) -- (Q_2);
                \draw[thick, -stealth] (P_2) -- (Q_2);
                \draw[red, thick, ->] (P_1) -- (Q_2);
                \node[below left] at (P_1) {\(P_1\)};
                \node[below right] at (P_2) {\(P_2\)};
                \node[above right] at (Q_2) {\(Q_2\)};
                \node[above left] at (Q_1) {\(Q_1\)};
            \end{tikzpicture}
        \end{center} 
\end{enumerate}
}

\dfn{Sottospazio affine}{Sia \(A_n(K)\) uno spazio affine. Si dice \textbf{sottospazio affine} di dimensione \(m \le n\) una struttura data da
\begin{enumerate}
    \item \(\emptyset \neq A' \subseteq A\), detto \textbf{sostegno del sottospazio affine} 
    \item \(V_m(K)\) sottospazio di \(V_n(K)\) 
    \item la restrizione dell'applicazione \(f\) ad \(A' \times A'\) troncata a \(V_m(K)\), purché questa sia ancora un'applicazione che gode delle proprietà elencate nella definizione di spazio affine
\end{enumerate}}

\dfn{Traslazione}{Fissato un vettore \(v \in V_n(K)\) si dice \textbf{traslazione}, individuata da \(v\), la corrispondenza \[
t_v: A \to A \quad e \quad P \to Q
\] che associa a un punto \(P \in A\) il punto \(Q\) traslato di \(P\) mediante il vettore \(v\).}

\paragraph{Osservazione:} \(\forall v \in V_n(K)\) la mappa \(t_v\) è una biiezione di \(A\), insieme di punti di \((A, V_n(K), f)\). E l'inversa di \(t_v\) è \(t_{-v}\).

\dfn{Sottospazio lineare}{Sia \(A_n(K)\) uno spazio affine. Si dice \textbf{sottospazio lineare} l'insieme dei traslati di un punto \(P\), detto \textbf{origine}, mediante i vettori \(v \in V_h(K) \le V_n(K)\), con \(h\) detta dimensione del sottospazio lineare. Inoltre si denota con \(S_h = [P, V_h(K)]\) il sottospazio lineare dato dal punto \(P\) e dallo spazio di traslazione \(V_h\).}

\dfn{Punti, rette, piani e iperpiani}{Sia \(A_n(K)\) uno spazio affine. Si dicono
\begin{itemize}
    \item \textbf{punti} i sottospazi lineari di dimensione 0 \[
            S_0 = [P, \{\ul{0} \} ] = \{P\} 
    \]
    \item \textbf{rette} i sottospazi lineari di dimensione 1 \[
            S_1 = [P, \mcL(v) ] \quad \text{con} \ v \neq \ul{0} \quad e \quad v \in V_n(K)
    \] 
    \item \textbf{piani} i sottospazi lineari di dimensione 2 \[
            S_2 = [P, \mcL(v_1, v_2) ] \quad \text{con} \ v_1, v_2 \neq \ul{0} \quad e \quad v_1, v_2 \in V_n(K)
    \] 
    \item \textbf{iperpiani} sono i sottospazi di dimensione \(n-1\)
\end{itemize}}
\mprop{}{Sia \(S_h = [P,V_h(K)]\) un sottospazio lineare di dimensione \(h\) sottospazio di \(A_n(K)\).
\begin{enumerate}
    \item siano \(Q, R \in S_h \implies \vec{QR} \in V_h(K)\) 
    \item se \(Q \in S_h\) e \(v \in V_h\), allora \(R = t_v (Q) \in S_h\)
\end{enumerate}}

\pf{Dimostrazione}{Dimostriamo entrambi i punti separatamente
\begin{enumerate}
    \item Per ipotesi \(Q \in S_h\), quindi \(Q = t_v(P)\) con \(v \in V_h(K)\). \(v = \vec{PQ} \in V_h\) e analogamente \(\vec{PR} \in V_h\). Ma allora \(\vec{QR} = \vec{QP} + \vec{PR} = - \vec{PQ} + \vec{PR} \in V_h\).
\begin{center}
        \begin{tikzpicture}
            \coordinate (P) at (0, 0);
            \coordinate (Q) at (2, 0);
            \coordinate (R) at (1.5, 1);
            \draw[thick, -stealth] (P) -- (R);
            \draw[thick, -stealth] (P) -- (Q);
            \draw[red, thick, -stealth] (Q) -- (R);
            \node[above right] at (R) {\(R\)};
            \node[below right] at (Q) {\(Q\)};
            \node[below left] at (P) {\(P\)};
        \end{tikzpicture}
\end{center}
    \item Poiché \(Q \in S_h, \ \vec{PQ} \in V_h\). Allora \(\vec{PR} + \vec{QR} = \vec{PQ} + \vec{v} \in V_h \implies \vec{PR} \in V_h\). Posto \(w = \vec{PR}\), \(t_w(P) = R\) con \(w \in V_h \implies R \in S_h\).
\begin{center}
        \begin{tikzpicture}
            \coordinate (P) at (0, 0);
            \coordinate (Q) at (2, 0);
            \coordinate (R) at (1.5, 1);
            \draw[red, thick, -stealth] (P) -- (R);
            \draw[thick, -stealth] (P) -- (Q);
            \draw[red, thick, -stealth] (R) -- (Q);
            \node[above right] at (R) {\(Q\)};
            \node[below right] at (Q) {\(R\)};
            \node[below left] at (P) {\(P\)};
            \node[red] at (2, 0.6) {\(v\)};
        \end{tikzpicture}
\end{center}
\end{enumerate}}

\mprop{}{Sia \(S_h=[P, V_h(K)]\) un sottospazio lineare di \(A_n(K)\). Ogni punto di \(S_h\) può essere scelto come origine di \(S_h\). Cioè dato \(Q \in S_h\) abbiamo che \([Q, V_h(K)] = S_h\).}

\pf{Dimostrazione}{Sia \(R \in S_h\). Allora \(\vec{PR} \in V_n\) e \(\vec{PQ}\in V_n\). Quindi \(\vec{QR} = \vec{QP} + \vec{PR} = - \vec{PQ} + \vec{PR} \in V_h \implies \vec{QR} \in V_h\).
\begin{center}
        \begin{tikzpicture}
            \coordinate (P) at (0, 0);
            \coordinate (Q) at (2, 0);
            \coordinate (R) at (1.5, 1);
            \draw[thick, -stealth] (P) -- (R);
            \draw[thick, -stealth] (P) -- (Q);
            \draw[red, thick, -stealth] (Q) -- (R);
            \node[above right] at (R) {\(R\)};
            \node[below right] at (Q) {\(Q\)};
            \node[below left] at (P) {\(P\)};
        \end{tikzpicture}
\end{center}
Detto \(w = \vec{QR}\) abbiamo che \(R = t_v(Q)\). \(R\) è traslato di \(Q\) tramite il vettore \(w \in V_h \implies R \in [Q, V_h]\), quindi \[
    S_h \subseteq [Q, V_h]
\] con lo stesso ragionamento scambiamo \(P\) e \(Q\) si dimostra che  \[
[Q, V_h] \subseteq [P, V_h] = S_h
\] e ciò vale solo se \(S_h = [Q, V_h]\).}

\mprop{}{Siano \(S_h\) e \(S_k\) due sottospazi lineari di \(A_n(K)\). Allora \(S_h \subseteq S_k \iff S_h \cap S_k \neq \emptyset\) e \(V_h \le V_k\).}
\pf{Dimostrazione}{"\(\implies \)" Ovviamente \(S_h \cap S_k \neq \emptyset\) e sia \(P \in S_h \cap S_k\). Potremo scrivere \(S_h = [P, V_h]\) e \(S_k = [P, V_k]\). Sia \(v \in V_h\) e sia \(Q = t_v(P) \in S_h \subseteq S_k \implies Q \in S_k\) e sia \(Q = t_v(P)\) ovvero \(\vec{PQ}=v \in V_k \implies V_h \le V_k\).

"\(\impliedby \)" Sia \(P \in S_h \implies [P, V_h] \subseteq [P, V_k]\) (poiché per ipotesi \(V_h \subseteq V_k\)) \([P, V_h] = S_h\) e \([P, V_k]= S_k \implies S_h \subseteq S_k\).}

\mprop{}{Siano \(S_h\) e \(S_k\) sottospazi lineari di \(A_n(K)\). Sia \(S_h \cap S_k \neq \emptyset\) e sia \(P \in S_h \cap S_k\). Allora \[
        S_h \cap S_k = [P, V_h \cap V_k]
\] }

\pf{Dimostrazione}{Sia \(Q \in S_h \cap S_k\). Osserviamo che \(S_h = [P, V_h]\) e \(S_k = [P, V_k]\). \(Q = t_v(P)\) con \(v \in V_h\) (perché \(Q \in S_h\)). Ma \(Q = t_v(P)\) con \(v \in V_k\) (perché \(Q \in S_k\)). Quindi \(Q \in [P, V_h \cap V_k]\) perché \(v \in V_h \cap V_k\), cioè \[
        S_h \cap S_k \subseteq [P, V_h \cap V_k]
\] Viceversa dato \(Q = t_v(P)\) con \(v \in V_h \cap V_k \implies Q \) appartiene sia a \(S_h\) che ad \(S_k\), quindi \(Q \in S_h \cap S_k\), ovvero  \[
[P, V_h \cap V_k] \subseteq S_h \cap S_k
\] \[
\implies [P, V_h \cap V_k] = S_h \cap S_k
\] }

\dfn{Parallelismo tra sottospazi}{Due sottospazi lineari, \(S_p = [P, V_p]\) ed \(S_q = [Q, V_q]\), di \(A_n(K)\) si dicono \textbf{paralleli}, e si scrive \(S_p || S_q\), se i rispettivi spazi di traslazione sono confrontabili, ovvero quando \(V_p \subseteq V_q\), oppure \(V_q \subseteq V_p\).}

\paragraph{Osservazione 1:} La relazione di parallelismo non è transitiva. E' invece riflessiva e simmetrica. Non è quindi una relazione d'equivalenza.
\paragraph{Osservazione 2:} Due sottospazi lineari della stessa dimensione sono paralleli se, e soltanto se, hanno lo stesso spazio di traslazione. Quindi la relazione di parallelismo considerata tra spazi della stessa dimensione è una relazione d'equivalenza.

\mprop{}{Due sottospazi lineari paralleli e di uguale dimensione o coincidono oppure hanno intersezione vuota.}
\dfn{}{
\begin{itemize}
    \item Sia \(S=[P, V_1]\) una retta. Lo spazio \(V_1\) si dice \textbf{direzione} della retta \(S\). Quindi due rette sono parallele se, e soltanto se, hanno la stessa direzione
    \item Sia \(\pi = [P, V_2]\subseteq A_n(K)\) con \(n \ge 2\). Lo spazio \(V_2\) è detto \textbf{giacitura} di \(\pi\). Quindi due piani sono paralleli se, e soltanto se, hanno la stessa giacitura.
    \item Tre o più punti si dicono \textbf{allineati} se esiste una retta che li contiene tutti.
    \item Due o più rette si dicono \textbf{complanari} se esiste un piano che le contiene tutte.
\end{itemize}
}

\section{Proprietà di punti, rette e piani}
\mprop{}{In \(A_n(k)\), con \(n \ge 2\) 
\begin{enumerate}
    \item per ogni due punti distinti passa un'unica retta
    \item per due rette distinte, parallele o incidenti, passa un unico piano
    \item due rette complanari, aventi intersezione vuota, sono parallele
    \item per un punto passa un'unica retta parallela a una retta data (V Postulato di Euclide)
    \item per un punto passa un unico piano, parallelo ad un piano dato
    \item per tre punti, non allineati, passa un unico piano
    \item una retta, avente due punti distinti in un piano, giace nel piano
    \item per un punto passano almeno due rette distinte
\end{enumerate}}

\mprop{}{In \(A_3(K)\),
\begin{enumerate}
    \item una retta e un piano, aventi intersezione vuota, sono paralleli
    \item due piani, aventi intersezione vuota, sono paralleli
    \item due piani distinti, aventi in comune un punto, hanno in comune una retta per quel punto
    \item per una retta passano almeno due piani distinti
\end{enumerate}}

\dfn{Rette sghembe}{In \(A_n(K)\), con \(n \ge 3\), due rette non complanari si dicono \textbf{sghembe}.}
\mprop{}{In \(A_n(K)\), con \(n \ge 3\), esistono due rette \(r_1\) e \(r_2\) sghembe tra loro. Inoltre due rette sghembe \(r_1\) e \(r_2\), sono contenute su due piani \(\pi_1\) e \(\pi_2\) paralleli tra loro e distinti.}

\pf{Dimostrazione}{ Per ipotesi, \(A_n(K)\) ha dimensione almeno 3, quindi esistono nello spazio vettoriale  \(V_n(K)\) almeno 3 vettori linearmente indipendenti. Siano  essi \(u, v, w\). Siano inoltre, \(P\) un punto di \(A\) e \(Q\) il traslato di \(P\) mediante il vettore \(u\) (\(Q = t_u(P)\)). Dimostriamo che le rette \(r = [P, \mcL(v) ]\) ed \(s = [Q, \mcL(w) ]\) sono sghembe. Se infatti, esistesse un piano \(\pi = [P, V_2]\) che le contiene entrambe, lo spazio di traslazione di \(\pi \) conterrebbe 3 vettori linearmente indipendenti, cioè \(v, w\) e \(u = \vec{PQ}\) e ciò è un \textbf{assurdo!}
\begin{figure}[ht]
    \centering
    \def\svgwidth{0.3\columnwidth}
    \incfig{due-rette-complanari-con-3-vettori-linearmente-indipendenti}
    \label{fig:due-rette-complanari-con-3-vettori-linearmente-indipendenti}
\end{figure}
Siano ora \(t = [T, \mcL(v) ]\) e \(t' = [T', \mcL(v') ]\) due rette sghembe. I vettori \(v\) e \(v'\) generano uno spazio vettoriale \(V_2\) di dimensione 2. Pertanto, i piani  \(\pi = [T, V_2]\) e \(\pi' = [T', V_2]\), che risultano paralleli, sono distinti e contengono, rispettivamente le rette \(t\) e \(t'\).}

\section{Geometria analitica in \(A_n(\RR )\)}
\dfn{Riferimento affine}{Si dice \textbf{riferimento affine} di \(A_n(\RR )\) una coppia \(RA = [O,B]\) costituita da un punto \(O\) fissato, detto origine, e da una base \(B\) dello spazio vettoriale \(V_n(\RR )\).}

\dfn{Coordinate}{Fissato, in \(A_n(\RR )\), un riferimento affine \(RA = [O,B]\), si dicono \textbf{coordinate} del punto \(P\) in \(RA\) le componenti, in \(B\), del vettore \(\vec{OP}\) e si scrive \(P=(x_i)_{i \in I_n}\).}

\begin{enumerate}
\item In \(A_1(\RR )\), un riferimento affine è una coppia \(RA = [O,B]\), ove \(O\) è un punto fissato e \(B = (e_1)\) è una base di \(V_1(\RR )\). Se \(\vec{OP}=xe_1\), si scrive \(P=(x)\) e si dice che \(x\) è l'\textbf{ascissa} del punto \(P\) in \(RA\). 
\begin{figure}[ht]
    \centering
    \def\svgwidth{0.5\columnwidth}
    \incfig{ascissa}
    \label{fig:ascissa}
\end{figure}
\item In \(A_2(\RR )\), un riferimento affine è una coppia \(RA = [O,B]\), ove \(O\) è un punto fissato e \(B = (e_1, e_2)\) è una base di \(V_2(\RR )\). La retta \([O, \mcL(e_1) ]\) è detta \textbf{asse delle ascisse} e la retta \([O, \mcL(e_2) ]\) è detta \textbf{asse delle ordinate}. Se \(\vec{OP} = xe_1 + y e_2\), si scrive \(P = (x, y)\) e si dice che \((x,y)\) è la coppia delle coordinate di \(P\) in \(RA\), dette rispettivamente \textbf{ascissa} e \textbf{ordinata} del punto \(P\).
\begin{figure}[ht]
    \centering
    \def\svgwidth{100pt}
    \incfig{ascissa-e-ordinata}
    \label{fig:ascissa-e-ordinata}
\end{figure}
\item In \(A_3(\RR )\), un riferimento affine è una coppia \(RA = [O,B]\), ove \(O\) è un punto fissato e \(B = (e_1, e_2, e_3)\) è una base di \(V_3(\RR )\). La retta \([O, \mcL(e_1) ]\) è detta \textbf{asse delle ascisse}, la retta \([O, \mcL(e_2) ]\) è detta \textbf{asse delle ordinate} e la retta \([O, \mcL(e_3) ]\) è detta \textbf{asse delle quote}. Sono detti \textbf{piani coordinati} i piani \(xy = [O, \mcL(e_1, e_2) ], xz = [O, \mcL(e_1, e_3) ]\) e \(yz = [O, \mcL(e_2, e_3) ]\). Inoltre, se \(\vec{OP}=xe_1+ ye_2 + ze_3\), si scrive \(P = (x, y,z)\) e si dice che \((x, y, z)\) è la terna delle coordinate di \(P\) in \(RA\), dette rispettivamente \textbf{ascissa, ordinata} e \textbf{quota} del punto \(P\).
\begin{figure}[ht]
    \centering
    \def\svgwidth{200pt}
    \incfig{terne-di-coordinate}
    \label{fig:terne-di-coordinate}
\end{figure}
\end{enumerate}

\thm{}{In \(A_n(K)\), con \(RA = [O, B]\), siano \(P = (x_1', x_2', \ldots , x_n')\) e \(Q = (x_1 '', x_2 '', \ldots  , x_n '')\) due punti di \(A\). Allora le componenti di \(\vec{PQ}\) rispetto a \(B\) sono \[
(x_1 '' - x_1', x_2 '' - x_2' , \ldots, x_n '' - x_n')
\] }

\pf{Dimostrazione}{Posti due vettori \[
\vec{OP} : \ x_1' e_1+ x_2' e_2 + \ldots + x_n' e_n
\] \[
\vec{OQ}: \ x_1 '' e_1 + x_2 '' e_2 + \ldots + x_n '' e_n
\]Per la proprietà della definizione di spazio affine possiamo dire che \[
\vec{PQ} = \vec{PO} + \vec{OQ} = \vec{OQ} - \vec{OP} = \sum_{i \in I_n} (x_i '' - x_i') e_i
\] }
Posti \[
 X '' = \left( \; \begin{matrix} x ''_1 \\ x ''_2\\ \vdots \\ x ''_n \end{matrix} \; \right), X' = \left( \; \begin{matrix} x'_1 \\ x'_2\\ \vdots \\ x'_n \end{matrix} \; \right) \text{ e } T = \left( \; \begin{matrix} t_1 \\ t_2\\ \vdots \\ t_n \end{matrix} \; \right)
\] si ottiene l'equivalente, ma spesso più agevole, forma matriciale: \[
X '' - X' = T
\] che può essere riscritta come \[
X '' = X' + T
\] Da quest'ultima equazione si vede che le coordinate del traslato del punto \(P = (x_1', x_2', \ldots , x_n')\), attraverso il vettore \(v\) di componenti \((t_1, t_2, \ldots , t_n)\), si ottengono sommando, ordinatamente, alle coordinate di \(P\) le componenti del vettore di traslazione. Per questo le relazioni che compaiono nell'equazione sono anche dette \textbf{equazioni della traslazione individuata da \(v\)}.

\dfn{Punto medio}{Dato \(P\) e \(Q \in A\) (insieme dei punti di \(A_n(\RR )\)), definiamo il punto medio del segmento \([PQ]\) come \[
M = t_{1/2 \vec{PQ}} (P)
\] 
\begin{center}
    \begin{tikzpicture}
        \coordinate (P) at (0, 0);
        \coordinate (M) at (2, 0);
        \coordinate (R) at (4, 0);
        \draw[thick, -stealth] (P) -- (M);
        \draw[thin, -stealth] (P) -- (R);
        \node[above] at (R) {\(R\)};
        \node[above] at (M) {\(M\)};
        \node[above] at (P) {\(P\)};
    \end{tikzpicture}
\end{center}}

\mprop{}{Dati \(P, Q \in A\) e dato un riferimento affine \(RA = [O, B]\) abbiamo che le coordinate del punto medio di \(P\) e \(Q\) sono le semisomme delle coordinate omonime di \(P\) e di \(Q\).}

\dfn{Punto simmetrico}{In \(A_n(\RR )\) dati i punti \(P\) e \(C\) diremo che \(S\) è il \textbf{punto simmetrico} di \(P\) rispetto a \(C\) se \(C\) è il punto medio di \([P, S]\).}

\section{Rappresentazioni analitiche}
\dfn{Equazioni parametriche di una retta in \(A_n(\RR )\)}{Sia \(RA = [O,B]\) un riferimento fissato in \(A_n(\RR )\), ove \(B = (e_1, e_2, \ldots , e_n)\). Sia \(r = [P, V_1 = \mcL(v) ]\) la retta di origine il punto \(P = (x_1', x_2', \ldots , x_n')\) e spazio di traslazione generato da \(v = (l_1, l_2, \ldots , l_n)\). Il generico vettore \(w\) di \(\mcL(v) \) è proporzionale al vettore \(v\), cioè \(w = tv\), con \(t \in \RR \), quindi, \(w = (tl_1, tl_2, \ldots , tl_n)\). Dato che la retta \(r\) è il luogo dei traslati di \(P\) attraverso i vettori di \(\mcL(v) \), applicando le equazioni del teorema precedente si ottengono le coordinate del generico punto di \(r\) \[
\begin{cases}
    \ x_1 = x_1' + l_1t \\
    \ x_2 = x_2' + l_2t \\
    \ \ldots \ldots \ldots \ldots \\
    \ x_n = x_n' + l_nt \\
\end{cases} \quad \text{con} \quad t \in \RR , \quad (l_1, l_2, \ldots , l_n) \neq \ul{0} 
\] tali equazioni sono dette \textbf{equazioni parametriche} di \(r\) in \(A_n(\RR )\). Al variare di \(t \in \RR \), si ottengono le coordinate di tutti i punti di una retta e, quindi, tutti i punti di una retta sono \(\infty^{1}\).}

\dfn{Parametri direttori}{Si dicono \textbf{parametri direttori} di \(r = [P, V_1]\), le componenti di un qualunque vettore nullo di \(V_1\).}
\paragraph{Osservazione:} I parametri direttori di una retta sono, quindi, determinati a meno di un fattore non nullo di proporzionalità. Definiamo la classe dei parametri direttori di \(r\) come \(p.d.r = [(l_1, l_2, \ldots , l_n)]\) con \((l_1, l_2, \ldots , l_n)\) un qualsiasi vettore appartenente a \(V_1\).

\subsubsection{Equazioni parametriche di una retta in \(A_2(\RR )\)} 
In \(A_2(\RR )\), sia fissato un riferimento \(RA = [O, B]\), ove \(B = (e_1, e_2)\). Una retta \(r = [P, V_1]\) è il luogo dei traslati di un punto \(P\) mediante i vettori di \(V_1 \subset V_2\). Se \(P\) ha coordinate \((x_0, y_0)\) e \(V_1 = \mcL(v) \), ove \(v = le_1 + me_2\), le equazioni della definizione diventano \[
\begin{cases}
    \ x = x_0 + lt \\
    \ y = y_0 + mt \\
\end{cases}
\quad \text{ove}\quad t \in \RR , \quad (l,m) \neq (0,0)\] 
e sono dette \textbf{equazioni parametriche} di \(r\) in \(A_2(\RR )\).

\subsubsection{Equazioni parametriche di una retta in \(A_3(\RR )\)}
In \(A_3(\RR )\), sia fissato un riferimento \(RA = [O, B]\), ove \(B = (e_1, e_2, e_3)\). Una retta \(r = [P, V_1]\) è il luogo dei traslati di un punto \(P\) mediante i vettori di \(V_1 \subset V_3\). Se \(P\) ha coordinate \((x_0, y_0, z_0)\) e \(V_1 = \mcL(v) \), ove \(v = le_1 + me_2 + ne_3\), le equazioni della definizione diventano \[
\begin{cases}
    \ x = x_0 + lt \\
    \ y = y_0 + mt \\
    \ z = z_0 + nt \\
\end{cases}
\quad \text{ove}\quad t \in \RR , \quad (l,m,n) \neq (0,0,0)\] 
e sono dette \textbf{equazioni parametriche} di \(r\) in \(A_3(\RR )\).

\paragraph{Osservazione:} In modo del tutto analogo possiamo determinare le equazioni parametriche di sottospazi lineari di dimensione \(n\), che quindi dipenderanno da \(n\) parametri.

\subsubsection{Equazione cartesiana di una retta in \(A_2(\RR )\)}
In \(A_2(\RR )\) una retta si può rappresentare attraverso le sue equazioni parametriche in questo modo \[
\begin{cases}
    \ x = x_p + lt \\
    \ y = y_p + mt \\
\end{cases}
\] possiamo convertire questo sistema lineare in forma matriciale e quindi \[
\left( \;
\begin{matrix}
    x \\
    y \\
\end{matrix} \;
\right) = 
\left( \; \begin{matrix}
    x_p \\
    y_p \\
\end{matrix} \; \right) +
t  
\left( \; \begin{matrix}
    l \\
    m \\
\end{matrix} \; \right) \iff 
\left( \; \begin{matrix}
    x - x_p \\
    y - y_p \\
\end{matrix} \; \right) = t
\left( \; \begin{matrix}
    l \\
    m \\
\end{matrix} \; \right) \iff 
\left| \;
\begin{matrix}
    x - x_p & y - y_p \\
    l & m \\
\end{matrix} \;
\right| = 0
\] 
Quindi vale la relazione \[
    ((x-x_p) m) (l (y-y_p)) = mx - ly -mx_p + ly_p = 0
\] Possiamo raggruppare i termini noti \(-mx_p + ly_p\) in un generico termine \(c\) e quindi l'equazione cartesiana della retta diventa \[
ax + by + c = 0 \quad \text{con}\quad (a,b) \neq (0,0)
\] Quindi i parametri direttori della generica retta \(r\) saranno \(p.d.r = [(l,m)] = [(-b, a)]\).

\subsubsection{Mutua posizione di due rette in \(A_2(\RR )\)}
Siano due rette \[
r: ax + by + c = 0 \quad (a,b) \neq (0,0)
\] \[
s: a'x + b'y + c' = 0 \quad (a', b') \neq (0,0)
\]
La loro intersezione può essere \[ r \cap s = \
\begin{cases}
    \ \text{un unico punto se \(r\) e \(s\) sono incidenti} \\
    \ \emptyset \text{ se \(r\) e \(s\) sono parallele e distinte} \\
    \ r \equiv s \text{ se sono coincidenti } \\
\end{cases}
\]
Consideriamo il sistema \[r \cap s=
\begin{cases}
    \ ax + by + c = 0 \\
    \ a'x + b'y + c' = 0 \\
\end{cases}
\]
Le coordinate dei punti di \(r \cap s\) sono le soluzioni del sistema. Posti \[
A =
\left( \; \\
 \begin{matrix}
    a & b \\
    a' & b' \\
\end{matrix} \; \\
 \right) \quad \text{la matrice incompleta del sistema,}
 \quad A|B =
\left( \; \\
 \begin{matrix}
    a & b & -c \\
    a' & b' & -c' \\
\end{matrix} \; \\
 \right) \quad \text{la matrice completa del sistema}
\] possiamo dire che \(\rho(A) \ge 1\) poiché abbiamo richiesto che \((a,b) \neq (0,0)\) e \(\rho(A) \le 2\). Quindi abbiamo due casi possibili
\begin{enumerate}
    \item se \(\rho(A) = 2 \implies \rho(A) = \rho(A|B) =2\), quindi il sistema è compatibile e ha \(\infty^{2-2}\) soluzioni \(\implies \exists !\) soluzione del sistema \(\implies r \cap s = \{P\} \implies r \cap s\) sono \textbf{incidenti}.
    \item se \(\rho(A) = 1\) allora \(r || s\), ma non sappiamo se esse siano parallele e distinte o se esse coincidano. Perciò dobbiamo suddividere in due sottocasi
        \begin{enumerate}
            \item se fossero parallele e distinte il sistema non sarebbe compatibile, perciò \(2= \rho(A|B) > \rho(A) = 1\)
            \item se invece \(\rho(A) = 1\) e \(\rho(B) = 1\) il sistema ammette \(\infty^{2-1}\) soluzioni, perciò \(r \equiv s \implies r||s\) se \(\rho(A) = 1\)
        \end{enumerate}
\end{enumerate}
\subsubsection{Fasci di rette in \(A_2(\RR )\)}
\dfn{Fascio improprio di rette}{Si dice \textbf{fascio improprio di rette} l'insieme di tutte e sole le rette del piano \(A_2(\RR )\) parallele ad una retta data.}
\newpage
\mprop{}{Una retta appartiene al fascio improprio di rette parallele alla retta \(r = [P, V_1]: ax + by + c = 0, \ (a,b) \neq (0,0)\), se, e soltanto se, ha un'equazione del tipo \[
ax + by + k = 0 \quad \text{ove} \quad k \in \RR 
\] detta \textbf{equazione del fascio improprio di rette}. Da cui si deduce che le rette di un fascio improprio di rette sono \(\infty^{1}\)}
\paragraph{Osservazione:} Tutte e sole le rette parallele ad \(r\) hanno parametri direttori \([(-b, a)]\) e quindi \(r\) e \(s\) sono la stessa retta \(\iff (a,b,c) \sim (a',b', c')\).

\dfn{Fascio proprio di rette}{Si dice \textbf{fascio proprio di rette} l'insieme di tutte le rette di \(A_2(\RR )\) passanti per un punto \(P\) dato, detto \textbf{centro} o \textbf{sostegno} del fascio.}

\mprop{}{Siano \(r: ax + by + c = 0\) e \(r': a'x + b'y + c' = 0\), con \((a,b) \neq (0,0)\) e \((a', b') \neq (0,0)\), due distinte rette incidenti in un punto \(P\). Una retta \(s\) appartiene al fascio di centro \(P\) se, e soltanto se, ha un'equazione di tipo \[
\lambda (ax + by + c) + \mu(a'x+ b'y + c') = 0 \quad \text{ove} \quad \lambda , \mu \in \RR \quad e \quad (\lambda , \mu) \neq (0,0)
\] detta \textbf{equazione del fascio proprio di rette}. Se nell'equazione risulta \(\lambda  \neq 0\), posto \(k = \mu / \lambda \), si ottiene \[
ax + by + c + k (a'x + b'y + c') = 0 \quad \text{ove} \quad k \in \RR 
\] detta \textbf{equazione ridotta del fascio proprio di rette}, in cui, ovviamente, la retta \(r' : a'x + b'y + c' = 0\) non è rappresentata. Quindi possiamo dire che le rette di un fascio proprio di rette sono \(\infty^{1}\).}

\subsubsection{Simmetrie in \(A_2(\RR )\)}
\dfn{Simmetria rispetto ad una retta}{Il punto \(T\) si dice \textbf{simmetrico} del punto \(H\), rispetto alla retta \(r = [P, V_1]\), detta \textbf{asse di simmetria}, nella direzione \(W_1 \neq V_1\), se lo è nella simmetria di centro \(C = r \cap s\), dove \(s = [H, W_1]\). Tale simmetria si dice anche \textbf{simmetria rispetto ad una retta in una direzione assegnata}.}

\begin{figure}[ht]
    \centering
    \def\svgwidth{160pt}
    \incfig{simmetria-punto-retta}
    \label{fig:simmetria-punto-retta}
\end{figure}

\subsubsection{Equazione cartesiana di un piano in \(A_3(\RR )\)}
In \(A_3(\RR )\) dato il \(RA = [O, B]\), con \(B = (e_1, e_2, e_3)\). Sia \(\alpha = [P, V_2]\) un piano con \(P = (x_p, y_p, z_p)\) e \(V_2 = \mcL(v, v') \) (con \(v \neq kv'\)), tali che  \[
v = le_1 + me_2 + ne_3 \qquad 
v' = l'e_1 + m'e_2 + n'e_3
\] 
Il generico vettore \(w \in V_2\) si scrive come \(w = tv + t' v'\). Quindi \(t_w(P)\) è il generico punto appartenente a \(\alpha \). Di conseguenza possiamo dire che \[
\begin{cases}
    \ x = x_p + tl + t'l' \\
    \ y = y_p + tm + t'm' \\
    \ z = z_p + tn + t'n' \\
\end{cases} \implies 
\left( \;
 \begin{matrix}
    x \\
    y \\
    z \\
\end{matrix} \;
 \right) =
\left( \;
 \begin{matrix}
    x_p \\
    y_p \\
    z_p \\
\end{matrix} \;
 \right) +
\left( \;
 \begin{matrix}
    tl + t'l' \\
    tm + t'm' \\
    tn + t'm' \\
\end{matrix} \;
 \right) 
\] cioè, per l'equazione della traslazione \(\left( \; \begin{matrix} x \\ y\\ z \end{matrix} \; \right) \) sono le coordinate del generico punto di \(\alpha \) date dalla somma di \(\left( \; \begin{matrix} x_p \\ y_p\\ z_p \end{matrix} \; \right) \), cioè le coordinate di \(P\) con \(\left( \; \begin{matrix} tl + t'l' \\ tm + t'm'\\ tn + t'n' \end{matrix} \; \right) \), cioè le componenti di \(w\).

Seguendo un ragionamento analogo a quello fatto per le rette in \(A_2(\RR )\) possiamo descrivere un piano in \(A_3(\RR )\) come \[
\left| \; \begin{matrix}
    x-x_p & y-y_p & z-z_p \\
    l & m & n \\
    l' & m' & n' \\
\end{matrix} \; \right| = 0
\] e da questa ne ricaviamo la seguente equazione \[
ax + by + cz + d = 0 \quad \text{con} \quad  (a,b,c) \neq (0,0,0)
\] detta \textbf{equazione cartesiana} del piano in \(A_3(\RR )\). Tale equazione è definita a meno di un fattore di proporzionalità non nullo.

\subsubsection{Equazioni cartesiane delle rette in \(A_3(\RR )\)}

Fissiamo un \(RA = [O, B]\) con \(B = (e_1, e_2, e_3)\) e data una retta \(r = [P, V_1=\mcL(l,m,n) ]\) possiamo scrivere l'equazione parametrica della retta \[
r : \
\begin{cases}
    \ x = x_p + tl \\
    \ y = y_p + tm \\
    \ z = z_p + tn \\
\end{cases} \quad \text{con}\quad (l,m,n) \neq (0,0,0)
\] Da cui deriva la seguente relazione \[
\frac{x-x_p}{l} = \frac{y-y_p}{m} = \frac{z - z_p}{n}
\] in particolare, se poniamo ad esempio \(l \neq 0\), otteniamo il seguente sistema \[
\begin{cases}
    \ y = \frac{m}{l} (x-x_p) + y_p \\
    \ z = \frac{n}{l} (x-x_p) + z_p \\
\end{cases} \implies 
\begin{cases}
    \ y = \frac{m}{l} x + k \\
    \ z = \frac{n}{l} x + h \\
\end{cases} \ \text{ove} \quad h, k \in \RR 
\] esistono, ovviamente le equazioni relative ai casi \(m \neq 0\) e \(n \neq 0\) e, dato che la terna \((l,m,n)\) è non nulla, ogni retta ammette sempre, almeno, una rappresentazione simile. In ogni caso, qualunque essa sia, possiamo concludere che una retta si rappresenta con un sistema di due equazioni lineari nelle incognite \(x, y\) e \(z\), in cui il rango della matrice incompleta è uguale a 2. E infatti sussiste anche il viceversa, cioè \[
\begin{cases}
    \ ax + by + cz + d = 0 \\
    \ a'x + b'y + c'z + d' = 0 \\
\end{cases} \ \text{con} \quad \rho
\left( \; \begin{matrix}
    a & b & c \\
    a' & b' & c' \\
\end{matrix} \; \right) = 2
\] rappresenta una retta. Infatti per il teorema di Rouché-Capelli il sistema è compatibile e ammette \(\infty^{1}\) soluzioni, cioè le sue soluzioni dipendono da un solo parametro.  

Analogamente a quanto già osservato in \(A_2(\RR )\), dalla precedente equazione deriva che le componenti, dei vettori dello spazio di traslazione della retta \(r\), sono le soluzioni del sistema omogeneo associato a una rappresentazione cartesiana di \(r\) stessa. Quindi possiamo dedurre la classe dei parametri direttori della retta \(r\) attraverso la regola dei minori. L'insieme delle \(\infty^{1}\) soluzioni del sistema omogeneo \[
\begin{cases}
    \ ax + by + cz + d = 0 \\
    \ a'x + b'y + c'z + d' = 0 \\
\end{cases} \ \text{con} \quad \rho
\left( \; \begin{matrix}
    a & b & c \\
    a' & b' & c' \\
\end{matrix} \; \right) = 2
\] è \[
\left\{ \left( t
\left| \; \begin{matrix}
    b & c \\
    b' & c' \\
\end{matrix} \; \right|,
-t
\left| \; \begin{matrix}
    a & c \\
    a' & c' \\
\end{matrix} \; \right|, t
\left| \; \begin{matrix}
    a & b \\
    a' & b' \\
\end{matrix} \; \right| \right)
: \ t \in R \right\}
\] 
\subsubsection{Mutua posizione di due piani in \(A_3(\RR )\)}
Fissato un \(RA\) e dati due piani in \(A_3(\RR )\)\[
\alpha : ax + by + cz + d = 0 \qquad \alpha' : a'x + b'y + c'z + d' = 0
\] la loro intersezione è data dal sistema \[
\alpha \cap \alpha' : \
\begin{cases}
    \ ax + by + cz + d = 0 \\
    \ a'x + b'y + c'z + d' = 0 \\
\end{cases}
\] Quindi possiamo distinguere in 3 casi:
\begin{enumerate}
    \item \(\rho(A) = 2 \implies \rho(A) = \rho(A|B) = 2 \) quindi il sistema è compatibile e ammette \(\infty^{3-2} = \infty^{1}\) soluzioni \(\implies \alpha \cap \alpha' = r\), quindi \(\alpha \) e \(\alpha'\) sono due piani \textbf{incidenti}.
    \item Nel caso in cui \(\rho(A) = 1\) dobbiamo distinguere in due sottocasi
        \begin{enumerate}
            \item \(\rho(A|B) = 2 \) e \(\rho(A) = 1\), il sistema non è compatibile, quindi \(\alpha \cap \alpha' = \emptyset\) e \(\alpha \) è parallelo e distinto da \(\alpha'\). \(\alpha \) e \(\alpha'\) sono detti \textbf{paralleli e distinti}.
            \item \(\rho(A) = 1\) e \(\rho(A|B) = 1\), il sistema è compatibile e ammette \(\infty^{3-1} = \infty^{2}\) soluzioni. Quindi l'insieme delle soluzioni dipende da due parametri \(\implies \alpha \equiv \alpha'\).
        \end{enumerate}
\end{enumerate}

\mprop{Condizione di parallelismo tra piani}{\(\alpha || \alpha' \iff \rho(A) = 1 \iff a = ka' \ b = kb' \ c = kc' \iff [(a,b,c)] = [(ka',kb',kc')] = [(a', b', c')]\). Questa viene denominata condizione analitica di parallelismo tra piani.}

\subsubsection{Fasci di piani in \(A_3(\RR )\)}
\dfn{Fascio improprio di piani}{Si dice \textbf{fascio improprio di piani} l'insieme di tutti e soli i piani di \(A_3(\RR )\) paralleli a un piano dato.}
\mprop{}{Un piano appartiene al fascio improprio di piani paralleli ad \(\alpha =[P, V_2]: ax + by + cz + d = 0\), con \((a,b,c) \neq (0,0,0)\), se, e soltanto se, ha un'equazione del tipo \[
ax + by + cz + k = 0 \quad \text{ove} \quad k \in \RR 
\] detta \textbf{equazione del fascio improprio di piani}. I piani di un fascio improprio sono \(\infty^{1}\).}

\dfn{Fascio proprio di piani}{Si dice \textbf{fascio proprio di piani}, l'insieme di tutti e soli i piani di \(A_3(\RR )\) passanti per una retta data \(r\), detta \textbf{asse} o \textbf{sostegno} del fascio.}
\mprop{}{Siano \(r\) una retta, \(\alpha : ax + by + cz + d = 0\) e \(\alpha' : a'x + b'y + c'z + d' = 0\), con \((a,b,c) \neq (0,0,0)\) e \((a',b',c') \neq (0,0,0)\), due distinti piani per \(r\). Un piano \(\beta \) appartiene al fascio di sostegno \(r\) se, e soltanto se, ha un'equazione del tipo \[
\lambda (ax+by+cz+d) + \mu (a'x + b'y +c'z+d') = 0 \quad \text{ove}\quad \lambda , \mu \in \RR \quad e \quad (\lambda , \mu) \neq (0,0)
\] detta \textbf{equazione del fascio proprio di piani}. Se nell'equazione risulta \(\lambda \neq 0\), posto \(h = \mu / \lambda \), si ottiene \[
ax + by + cz + d + h(a'x + b'y + c'z + d') = 0 \quad \text{ove} \quad h \in \RR 
\] detta \textbf{equazione ridotta del fascio proprio di piani}, in cui ovviamente il piano \(\beta : a'x + b'y + c'z + d' = 0\) non è rappresentato. Dalla rappresentazione ridotta del fascio si deduce che i piani di un fascio proprio sono \(\infty^{1}\).}

\subsubsection{Mutua posizione di due rette in \(A_3(\RR )\)}
Siano assegnate le rette \[
r :
\begin{cases}
    \ ax + by + cz + d = 0 \\
    \ a'x + b'y + c'z + d' = 0 \\
\end{cases} \quad \rho
\left( \; \begin{matrix}
    a & b & c \\
    a' & b' & c' \\
\end{matrix} \; \right) = 2
\] \[
s :
\begin{cases}
    \ a ''x + b ''y + c ''z + d '' = 0 \\
    \ a'''x + b'''y + c'''z + d''' = 0 \\
\end{cases} \quad \rho
\left( \; \begin{matrix}
    a ''& b '' & c ''\\
    a''' & b''' & c''' \\
\end{matrix} \; \right) = 2
\] Sia \[
r \cap s:
\begin{cases}
    \ ax + by + cz + d = 0 \\
    \ a'x + b'y + c'z + d' = 0 \\
    \ a ''x + b ''y + c ''z + d '' = 0 \\
    \ a'''x + b'''y + c'''z + d''' = 0 \\
\end{cases}
\] il sistema costituito dalle loro equazioni e siano \(A\) e \(A|B\) le matrici incompleta e completa associate al sistema. \[
AX = B \quad \text{con} \quad B =
\left( \; \begin{matrix}
    -d \\
    -d' \\
    -d'' \\
    -d''' \\
\end{matrix} \; \right) 
\quad X = 
\left( \; \begin{matrix}
    x \\
    y \\
    z \\
\end{matrix} \; \right) 
\quad A =
\left( \; \begin{matrix}
    a & b & c \\
    a' & b' & c' \\
    a '' & b '' & c '' \\
    a''' & b''' & c''' \\
\end{matrix} \; \right) 
\] Esaminiamo i 4 casi possibili:

\begin{enumerate}
    \item \(\rho(A|B) = 4 \implies \rho(A) = 3\) poiché \(A|B\) è ottenuta aggiungendo una colonna ad \(A\), quindi \(\rho(A|B) \le \rho(A) + 1 \implies \rho(A) =3 \). Il sistema non è compatibile per il teorema di Rouché-Capelli \(\implies r  \) e \(s\) sono o parallele e disgiunte, oppure sghembe. Ma siccome \(\rho(A) = 3 \implies r = [P,V_1] \quad s =[P', V_1'] \quad V_1 \neq V_1' \implies r\) non è parallela ad \(s\). Quindi \(r\) e \(s\) sono \textbf{sghembe}.
    \item \(\rho(A|B) = 3\) e \(\rho(A) =3\). Il sistema è compatibile e per il teorema di Rouché-Capelli  esiste un'unica soluzione \(r \cap s = \{P\} \implies r\) e \(s\) si dicono \textbf{incidenti}.
    \item \(\rho(A|B) = 3\) e \(\rho(A) = 2\). Il sistema non è compatibile per il teorema di R.C. Siccome \(\rho(A) =2 \implies V_1= V_1' \implies r\) è parallela a \(s\) e \(r \neq s\). Si dice che \(r\) e \(s\) sono \textbf{parallele e distinte}.
    \item \(\rho(A|B) = \rho(A) = 2\) il sistema è compatibile e ammette \(\infty^{1}\) soluzioni. Si dice che  le rette \(r\) e \(s\) sono \textbf{coincidenti}.
\end{enumerate}

\dfn{Stella propria di rette}{In \(A_3(\RR )\) si dice \textbf{stella propria} di rette, l'insieme di tutte e sole le rette passanti per un punto assegnato.}
\paragraph{Osservazione:} Possiamo scrivere la rappresentazione di tutte e sole le rette della stella passanti per \(P = (x_0, y_0, z_0)\) come \[
\alpha :
\begin{cases}
    \ x = x_0 + tl \\
    \ y = y_0 + tm \\
    \ z = z_0 + tn \\
\end{cases}
\] e da qui abbiamo che \[
t = \frac{x-x_0}{l}= \frac{y-y_0}{m} = \frac{z-z_0}{n} = 
\begin{cases}
    \ m(x-x_0) = l(y-y_0) \\
    \ n (x-x_0) = l(z-z_0) \\
\end{cases}
\] dividendo per \(l\) (supponendo \(l \neq 0\)) si ottiene che abbiamo solo due parametri liberi e quindi abbiamo \(\infty^{2}\) rette nella stella di rette per \(P\).

\dfn{Stella impropria di rette}{In \(A_3(\RR )\) si dice \textbf{stella impropria} di rette, l'insieme di tutte e sole le rette parallele ad una retta data.}

\paragraph{Osservazione:} Una rappresentazione analitica di tutte le rette parallele a una retta assegnata, di parametri direttori \((l,m,n)\), è \[
\beta : 
\begin{cases}
    \ x = x' + tl \\
    \ y = y' + tm \\
    \ z = z' + tn \\
\end{cases} \iff  
\begin{cases}
    \ m(x-x') = l(y-y') \\
    \ n(x-x') = l(z-z') \\
\end{cases}
\] Questa volta non sono i parametri direttori ad essere i parametri, ma i punti di \(P = (x', y', z')\). Quest'ultima è detta \textbf{equazione cartesiana della stella impropria} di \(r\). Abbiamo \(\infty^2\) rette in \(A_3(\RR )\) parallele ad una retta data.

\subsubsection{Mutua posizione di un piano e una retta in \(A_3(\RR )\)}
Siano \[
\alpha = [P, V_2]: ax + by + cz + d = 0, \quad \text{con}\quad (a,b,c) \neq (0,0,0) \]
\[
r=[Q, V_1]:
\begin{cases}
    \ a'x + b'y + c'z + d' = 0 \\
    \ a ''x + b ''y + c ''z + d '' = 0 \\
\end{cases} \text{con} \quad 
\rho
\left( \; \begin{matrix}
    a' & b' & c' \\
    a '' & b '' & c '' \\
\end{matrix} \; \right) = 2
\] e sia \(r \cap \alpha \) rappresentato dal sistema lineare \(AX = B\), dove \[
A =
\left( \; \begin{matrix}
    a & b & c \\
    a' & b' & c' \\
    a '' & b '' & c '' \\
\end{matrix} \; \right) \quad
B =
\left( \; \begin{matrix}
    -d \\
    -d' \\
    -d '' \\
\end{matrix} \; \right) \quad X =
\left( \; \begin{matrix}
    x \\
    y \\
    z \\
\end{matrix} \; \right) 
\] Sono 3 i casi possibili
\begin{enumerate}
    \item sia \(\rho(A|B) = \rho(A) =3\), il sistema è compatibile e, per il teorema di R.C. ammette un'unica soluzione \(r \cap \alpha  = \{P\} \implies r\) e \(\alpha \) si dicono \textbf{incidenti}
    \item \(\rho(A|B) = 3\) e \(\rho(A) =2\), il sistema non è compatibile, quindi \(r || \alpha \) e \(r \notin \alpha \) 
    \item \(\rho(A|B) = 2\) e \(\rho(A) = 2\), il sistema è compatibile e ammette \(\infty^{1}\) soluzioni, quindi \(r\) è contenuto in \(\alpha \)(\(r\) è anche chiaramente parallelo ad \(\alpha \))
\end{enumerate}

\paragraph{Osservazione:} \(\rho(A) = 2 \iff r || \alpha \) ovvero \[
\left| \; \begin{matrix}
    a & b & c \\
    a' & b' & c' \\
    a '' & b '' & c '' \\
\end{matrix} \; \right| = 0 \iff 
a \underbrace{
\left| \; \begin{matrix}
    b' & c' \\
    b '' & c '' \\
\end{matrix} \; \right|
}_{ \Gamma_1 \to l } 
- b 
\underbrace{
\left| \; \begin{matrix}
    a' & b' \\
    a '' & c '' \\
\end{matrix} \; \right| 
}_{\Gamma_2 \to m} 
+ c
\underbrace{
\left| \; \begin{matrix}
    a' & b' \\
    a '' & b '' \\
\end{matrix} \; \right|
}_{\Gamma_3 \to n} = 0
\] Quindi posti i parametri direttori \([(l,m,n)]\) possiamo dare la
\mprop{Condizione di parallelismo tra retta e piano}{La condizione di parallelismo tra retta e piano si esprime come \[
al + bm + cn = 0
\] dove \([(l,m,n)]\) sono i parametri direttori della retta e il piano è \(ax + by + cz + d = 0\).}

\dfn{Stella impropria di piani}{Si dice \textbf{stella impropria di piani} l'insieme di tutti e soli i piani di \(A_3(\RR )\) paralleli ad una retta data.}
\paragraph{Osservazione:} Chiaramente dalla proposizione precedente segue che dati parametri direttori \([(l,m,n)]\) abbiamo che esistono \(\infty^{2}\) piani appartenenti alla stella impropria di piani.

\dfn{Stella propria di piani}{Si dice \textbf{stella propria di piani} l'insieme di tutti e soli i piani di \(A_3(\RR )\) passanti per un punto  assegnato detto \textbf{centro} o \textbf{sostegno} della stella.}
\paragraph{Osservazione:} Analogamente abbiamo \(\infty^{2}\) piani nella stella propria di piani.

\subsubsection{Simmetrie in \(A_3(\RR )\)}
\dfn{Simmetrico rispetto a una retta e una giacitura assegnata}{Il punto \(T\) si dice \textbf{simmetrico} del punto \(H\), rispetto alla retta \(r=[Q, V_1]\), nella giacitura \(V_2 \not\supseteq V_1\), se è simmetrico di \(H\) rispetto al punto \(C = \alpha  \cap r\), dove \(\alpha = [H, V_2]\).}

\begin{figure}[ht]
    \centering
    \def\svgwidth{150pt}
    \incfig{simmetria-retta-giacitura}
    \label{fig:simmetria-retta-giacitura}
\end{figure}

\dfn{Simmetria rispetto a un piano in una direzione assegnata}{Un punto \(T\) si dice \textbf{simmetrico} del punto \(H\), rispetto al piano \(\alpha = [Q, V_2]\), nella direzione \(V_1 \not\supseteq V_2\), se è simmetrico di \(H\) rispetto al punto \(C = \alpha \cap r\), dove \(r = [H, V_1]\).}
\begin{figure}[ht]
    \centering
    \def\svgwidth{150pt}
    \incfig{simmetria-direzione-piano}
    \label{fig:simmetria-direzione-piano}
\end{figure}

\section{Curve e superfici algebriche}
\dfn{Curva algebrica reale}{Si dice \textbf{curva algebrica reale} di \(A_2(\RR )\) l'insieme dei punti del piano \(A_2(\RR )\) le cui coordinate soddisfano un'equazione del tipo \(f(x,y) = 0\), dove \(f\) è un polinomio a coefficienti reali e non costante nelle variabili \(x\) e \(y\).}
\dfn{Superficie algebrica reale}{Si dice \textbf{superficie algebrica reale} di \(A_3(\RR )\) l'insieme dei punti di \(A_3(\RR )\) le cui coordinate soddisfano un'equazione del tipo \(f(x,y,z) = 0\) dove \(f\) è un polinomio a coefficienti reali e non costante nelle variabili \(x,y,z\).}
\dfn{Curva algebrica reale}{Si dice \textbf{curva algebrica reale} di \(A_3(\RR )\) l'insieme dei punti di \(A_3(\RR )\) le cui coordinate soddisfano un sistema delle equazioni di due superfici algebriche reali che in essa si intersecano.}

\chapter{Spazi euclidei}
\section{\(E_n(\RR )\), spazio euclideo di dimensione \(n\)}
\dfn{Spazio euclideo}{Si dice \textbf{spazio euclideo} di dimensione \(n\) sul campo \(\RR \) la struttura costituita da uno spazio affine \(A_n(\RR )\) il cui spazio vettoriale \(V_n^{\circ}(\RR )\) sia dotato di un prodotto scalare "\(\cdot \)" definito positivo.}
\dfn{Ortogonalità tra sottospazi}{Siano \(S_h = [P, V_h]\) e \(S_k = [Q, V_k]\) due sottospazi lineari di \(E_{n}(\RR) \). Diremo che \(S_h\) è \textbf{ortogonale} a \(S_k\) se \[
V_h \subseteq V_k^{\perp} \quad \text{oppure} \quad V_h \supseteq V_k^{\perp}
\] }

\paragraph{Osservazione:} La relazione di ortogonalità è simmetrica. Infatti se \(S_h \perp S_k\) allora
\begin{enumerate}
    \item \(V_h \subseteq V_k^{\perp} \implies V_h^{\perp} \supseteq \left( V_k^{\perp} \right) ^{\perp} = V_k \implies V_k \subseteq V_h^{\perp} \implies S_k \perp S_h\)
    \item \(V_h \supseteq V_k^{\perp} \implies V_h^{\perp} \subseteq \left( V_k^{\perp} \right) ^{\perp} = V_k \implies V_k \supseteq V_h^{\perp} \implies S_h \perp S_k\)
\end{enumerate}
In entrambi i casi \(S_h \perp S_k \iff S_k \perp S_h\). Quindi diremo semplicemente che \(S_h\) e \(S_k\) sono ortogonali.

\mprop{}{In \(E_2(\RR )\), dati la retta \(r\) e il punto \(H\), esiste un'unica retta passante per \(H\) e ortogonale a \(r\).}
\pf{Dimostrazione}{Dimostriamo prima di tutto l'esistenza della retta, successivamente ci occuperemo dell'unicità. Poniamo \(r : [P, V_1]\) e definiamo una \(s : [H, V_1^{\perp}]\). \(s\) è una retta poiché \(\RR ^{2} = V_1 \oplus V_1^{\perp}\), per la formula di Grassmann \(V_1^{\perp}\) ha dimensione 1, quindi \(s\) è una retta. \(H \in s\) per costruzione e \(r \perp s\) perché \(V_1^{\perp} \subseteq V_1^{\perp}\), cioè lo spazio di traslazione della retta \(s\) contiene la direzione ortogonale a \(V_1\). Ora l'unicità della retta segue dall'unicità dello spazio di traslazione e poiché esso ha dimensione 1, anche la retta è unica.}

\mprop{}{In \(E_3(\RR )\), siano assegnati una retta \(r\) e un piano \(\alpha \). Dato un punto \(H\) 
\begin{enumerate}
    \item esiste un'unica retta \(s\) passante per \(H\) e ortogonale al piano \(\alpha \)
    \item esiste un unico piano \(\beta \) passante per \(H\) e ortogonale alla retta \(r\)
\end{enumerate}}
\newpage
\pf{Dimostrazione}{Dimostriamo i 2 punti separatamente
\begin{enumerate}
    \item poniamo \(\alpha = [P, V_2]\) e \(s = [H, V_2^{\perp}]\). \(s\) è una retta perché \(\dim(V_2^{\perp}) = 1\), poiché \(\RR^{3} = V_2 \oplus V_2^{\perp}\) per la formula di Grassmann. \(H \in s\) e \(s \perp\alpha \) valgono per costruzione.
    \item poniamo \(r = [Q, V_1]\) e definiamo \(\beta = [H, V_1^{\perp}]\). Verifichiamo che \(\beta \) sia un piano. Osserviamo che dato che  \[
    \underbrace{\RR ^{3}}_{3} = \underbrace{V_1}_{1} \oplus \underbrace{V_1^{\perp}}_{2}  \implies \dim(V_1^{\perp}) = 2
    \] quindi \(\beta \) è un piano. \(H \in \beta \) e \(\beta \perp r\) valgono per costruzione. L'unicità del piano segue dall'unicità di \(V_2\) di dimensione 2 e perpendicolare a \(V_1\).
\end{enumerate}}

\mprop{}{Siano \(r: [P, V_1]\) e \(\alpha = [Q, V_2]\) rispettivamente una retta e un piano di \(E_3(\RR )\). Se \(r \perp \alpha \) abbiamo che
\begin{enumerate}
    \item \(r \perp s \quad \forall s \subseteq \alpha \), cioè \(r\) è perpendicolare a ogni retta \(s\) contenuta nel piano \(\alpha \)
    \item \(\alpha \perp \beta \quad \forall \beta \supseteq r\), cioè \(\alpha \) è perpendicolare a ogni piano \(\beta \) contenente \(r\)
\end{enumerate}}

\pf{Dimostrazione}{Dimostriamo i 2 punti separatamente
\begin{enumerate}
    \item Sia \(s \subseteq \alpha \) con \(s = [H, V_1']\), allora  \[
    \underbrace{V_1' \subseteq V_2}_{\text{poiché } s \subseteq \alpha } = \underbrace{V_1^{\perp}}_{\text{poiché } r \perp s} \implies r \perp s
    \] 
    \item Sia \(\beta \subseteq \alpha \) con \(\beta = [H, V_2']\), allora  \[
    \underbrace{V_2' \supseteq V_1}_{\text{poiché } \beta \supseteq r } = \underbrace{V_2^{\perp}}_{\text{poiché } r \perp \alpha } \implies \alpha \perp \beta
    \] 
\end{enumerate}}

\mprop{}{Siano \(\alpha\) e \(r\) rispettivamente un piano e una retta di \(E_3(\RR)\), con \(\alpha \) non ortogonale a \(r\). Allora esiste un unico piano \(\beta\) ortogonale ad \(\alpha\) e contenente la retta \(r\).}

\pf{Dimostrazione}{Sia \(\beta = [P, V_1 \oplus V_2^{\perp}]\) dove \(r = [P, V_1]\) e \(\alpha = [Q, V_2]\). \(\beta\) è un piano perché \(\dim(V_1) = 1, \ \dim(V_2^{\perp})= 1\) e \(V_1 \neq V_2^{\perp}\) (poiché \(\alpha \not\perp r\) ) \( \implies \dim(V_1 \oplus V_2^{\perp}) = 2 \implies \beta\) è un piano. Per costruzione abbiamo che \(\beta \perp \alpha\), infatti lo spazio di traslazione di \(\beta\) è: \[
        V_1 \oplus V_2^{\perp} \supseteq V_2^{\perp} \text{  }
    \] e \(V_2\) è lo spazio di traslazione di \(\alpha\). Inoltre \(\beta \) contiene \(r\) per le proposizioni precedenti ed è ovviamente ortogonale a \(\alpha \). Per costruzione \(\beta \) è l'unico piano che soddisfa queste condizioni.}

\section{Geometria analitica in \(E_n(\RR)\) }
\dfn{}{In \(E_n(\RR)\) si dice \textbf{riferimento cartesiano ortogonale monometrico} la coppia \(RC = [O, \mcB]\) dove \(O\) è un punto di \(E_n(\RR)\) e \(\mcB=(e_1, e_2, ..., e_n)\) è una base ortonormale.}
\newpage
\nt{\begin{enumerate}
    \item In \(E_2(\RR)\) si conviene indicare la base ortonormale come \(\mcB = (i,j)\) 
    \item In \(E_3(\RR)\) si conviene indicare la base ortonormale come \(\mcB = (i,j, k)\) 
\end{enumerate}}

\section{Ortogonalità}
\subsubsection{Ortogonalità fra rette}
Siano \(r_1, r_2\) due rette di \(E_2(\RR)\) e sia \(r_1 = [P, f(v)]\) con \(v = li + mj\), analogamente \(r_2 = [P, f(v')]\) con \( v' = l'i + m'j\) \[
    v \perp v' \iff ll' + mm' = 0
    \] se \(r_1\) ha equazione \(ax + by + c = 0\) e \(r_2\) ha equazione \(a'x + b'y + c' = 0\) allora \(P.d.r_1 = [(-b,a)]\), e \(P.d.r_2 = [(-b', a')]\) quindi \[
    r_1 \perp r_2 \iff -b (-b') + aa' = bb' + aa' = 0
\]
Se abbiamo due rette \(r_1, r_2\) in \(E_3(\RR)\) con \(p.d.r_1 = [(l,m,n)]\), \(p.d.r_2 = [(l',m',n')]\) allor a \(r_1 \perp r_2 \iff v_1\), cioè il generatore della direzione della retta \(r_1\), è ortogonale a \(v_2\), che è generatore della direzione della retta \(r_2\). \[
    v_1 = li + mj + nk \qquad v_2 = l'i + m'j + n'k
\] \[
    v_1 \perp v_2 \iff r_1 \perp r_2 \iff ll' + mm' + nn' = 0
\] Analogamente se \(r_1, r_2\) sono rette in \(E_n(\RR)\) con \(p.d.r_1 = [(x_1, x_2, ..., x_n)]\), \(p.d.r_2 = [(x_1', x_2', ..., x_n')]\) \[
r_1 \perp r_2 \iff x_1x_1' + x_2 x_2' + \ldots  + x_n x_n' = 0
\] 

\subsubsection{Direzione ortogonale a un iperpiano}
\mprop{}{Sia \(r: ax + by + c = 0\) una retta di \(E_2(\RR)\), allora \([(a, b)]\) è la classe dei parametri direttori della direzione ortogonale a \(r\).}
\pf{Dimostrazione}{Per ipotesi \(p.d.r = [(-b,a)]\) e abbiamo che per essere ortogonale la direzione \((a, b)(-b, a)= 0\) oppure equivalentemente \((ai + bj)( -b i + aj ) = 0 \implies [(a, b)] \perp r\).  }
\mprop{}{Sia \(\pi: ax + by + cz + d = 0\) un piano in \(E_3(\RR)\), allora \([(a, b, c)]\) è la classe dei parametri direttori della direzione ortogonale a \(\pi\).}
\pf{Dimostrazione}{Sia \(v \in V_2\) tale che \(pi\) ha spazio di traslazione \(V_2\). Se \(v = (x, y, z) \implies ax + by + cz = 0 \iff (x, y, z)(a, b, c) = 0 \implies (a,b,c) \perp v \ \forall v \in V_2 \) }
\mprop{}{Più in generale: sia \(S _{n-1}\) un iperpiano in \(E_n(\RR)\) di equazione cartesiana \(a_1 x_1 + a_2 x_2 + ... + a_n x_n + a_0 = 0 \implies [(a_1, a_2, ..., a_n)]\) è la classe dei parametri direttori della direzione ortogonale a \(S _{n-1}\).}

\subsubsection{Ortogonalità fra piani in \(E_3(\RR )\)}
\mprop{}{Siano \(\alpha: ax + by + cz + d = 0\) e \(\beta: a'x + b'y + c'z + d' = 0\) due piani in \(E_3(\RR)\), con \((a,b,c) \neq (0,0,0)\), allora \(\alpha \perp \beta \iff a a' + b b' + c c' = 0\) }
\pf{Dimostrazione}{\(\alpha \perp \beta \iff V_2 \supseteq V_2'^{\perp}\) dove \(V_2\) è la giacitura di \(\alpha\) e \(V_2'\) è la giacitura di \(\beta\). \[V_2'^{\perp}= [\mcL((a', b',c')) ] \iff (a', b', c') \in V_2\]
\((x, y, z) \in V_2 \iff ax + by + cz = 0\) e quindi \((a', b', c') \in V_2 \iff a a' + b b' + c c' = 0\) }
\subsubsection{Ortogonalità fra retta e piano in \(E_3(\RR )\)}
\mprop{}{In \(E_3(\RR)\), sia \(r\) una retta con \(p.d.r = [(l,m,n)]\) e sia \(\alpha\) un piano di equazione \(ax + by + cz + d= 0\), allora \(r \perp \alpha\) se, e soltanto se, \([(a,b,c)] = [(l,m,n)]\) }
\pf{Dimostrazione}{\(r \perp \alpha \iff V_1=V_2^{\perp}\) dove \(V_1\) è la direzione della retta e \(V_2\) è la giacitura di \(\alpha\). \[
        V_1 = \mcL((l,m,n)) = V_2 ^{\perp} = \mcL((a,b,c)) \iff [(a,b,c)] = [(l,m,n)]  
\]}

\section{Distanza}
\subsubsection{Distanza fra due punti in \(E_n(\RR )\)}
Siano \(P = (x_{1} , x_{2}, ..., x_{n} )\) e \(Q = (x_{1} ' x_2 ', ..., x_n ')\). La distanza tra \(P\) e \(Q\) è la norma del vettore \(\vec{PQ}\), quindi\[
        d(P,Q) = ||\vec{PQ}|| = \sqrt{\vec{PQ} \cdot \vec{PQ}}
\]     \[
    \vec{{PQ}} = (x_1'-x_1) e_1 + ... + (x_n' - x_n) e_n
\] \[
    d(P,Q) = ||\vec{{PQ}}|| = \sqrt{(x_1' - x_1)^{2} + ... + (x_n'-x_n)^{2}  }
\]
\begin{enumerate}
    \item In \(E_{2}(\RR)\), dati \(P = (x, y) \) e \( Q = (x', y')\)\[
    \vec{{PQ}} = (x'- x) i + (y' - y)j
\] \[
    d (P, Q) = \sqrt{(x' - x)^{2} + (y'-y)^{2} }
\] 
    \item In \(E_{3}(\RR)\), dati \(P = (x, y,z) \) e \( Q = (x', y',z')\)\[
    \vec{{PQ}} = (x'- x) i + (y' - y)j + (z' - z)k
\] \[
    d (P, Q) = \sqrt{(x' - x)^{2} + (y'-y)^{2}  + (z' - z ) ^2}
\] 
\end{enumerate}
\subsubsection{Distanza punto retta in \(E_2(\RR )\)}
Siano \(P = (x_0, y_0)\) e \(r = [Q, V_1]\) rispettivamente un punto e una retta in \(E_{2}(\RR)\). Definiamo la distanza tra il punto \(P\) e la retta \(r\) come la distanza tra \(P\) e il punto \(H\), piede della perpendicolare per \(P\) a \(r\) (cioè l'intersezione tra \(r\) e la retta perpendicolare a \(r\) passante per \(P\)). 
\begin{center}
    \begin{tikzpicture}[scale = 0.5]
        \coordinate (P) at (0, 0);
        \coordinate (H) at (0, -2);
        \coordinate (P') at (3, -2);
        \draw[thick, -stealth] (P) -- (H); 
        \draw[thick, -stealth] (P) -- (0, 1); 
        \draw[thick, -stealth] (P) -- (P'); 
        \draw[thick, -stealth] (H) -- (P'); 
        \draw[thin] (-3.5, -2) -- (3.5, -2);
        \draw[thin] (0, 2) -- (0, -2);
        \node[below] at (P') {\(P'\)};
        \node[below] at (H) {\(H\)};
        \node[below] at (-2, -2) {\(r\)};
        \node[left] at (P) {\(P\)};
        \node[left] at (0, 1.5) {\(n\)};
        \node[right] at (0, 0.4) {\(v\)};
    \end{tikzpicture}
\end{center}
Determiniamo \(||\vec{{PH}}||\). Se \(r\) ha equazione \(ax + by +c = 0\) allora \(V_1^{\perp} = \mcL(a i + b j)  \). Posta \[n = [P, V_1^{\perp}] \implies n = [P, \mcL(a i + bj) ]\]
\(H = n \cap r \) è la proiezione di \(P\) su \(r\) (cioè l'intersezione tra \(r\) e la retta per \(P^{\perp} \)). Sia \(P' = (x', y')\) un generico punti su \(r\) di equazione \(ax + by + c = 0\). \(PH\) è la componente di \(PP'\) lungo \(v\). \[PP' = (x'-x_0) i + (y' - y_0)j\]\[
    \vec{{PH}} = \frac{\vec{PP'} \cdot v}{ v \cdot v } v
\] \[
d(P,r) = d (P, H) = || \vec{{PH}} || = \left\| \left( \frac{\vec{{PP'}} \cdot v}{v \cdot v} v \right)  \right\| = \frac{|\vec{PP'} \cdot v| }{\|v\|} = \frac{|(x'-x_0) a + (y' - y_0) b| }{\sqrt{a^2 + b ^2} } = \frac{|x'a + y'b - x_0a - y_0b| }{\sqrt{a^2 + b ^2} }
\] e, dato che \(P'\) appartiene a \(r\) e che, quindi, \(ax' + by' = -c\), si ha \[
d(P, r) = \frac{|ax_0 + by_0 + c|}{\sqrt{a^{2} + b^{2} }}
\] 
\subsubsection{Distanza punto piano in \(E_3(\RR )\)}
Siano \(P = (x_0, y_0, x_0)\) e \(\alpha : ax + by + cz + d = 0\) rispettivamente un punto e un piano di \(E_{3}(\RR)\). Definiamo la distanza \(d(P, \alpha)\) come la distanza tra \(P\) e il punto \(H\), intersezione tra \(\alpha\) e la retta per \(p \perp \alpha \). Infatti \(d(P, \alpha )= d (P, H) = ||\vec{{PH}}||\). Analogamente al caso precedente abbiamo che \[
    d(P, \alpha ) = \frac{|ax_0 + by_0 + cz_0 + d|}{\sqrt{a^{2} + b^{2} + c^{2} }}
\]

\subsubsection{Distanza punto retta in \(E_{3}(\RR)\) }
Siano \(P\) e \(r = [Q, V_1]\) rispettivamente un punto e una retta in \(E_{3}(\RR)\). Sia \(\alpha \) il piano per \(P\) ortogonale a \(r\) e sia \(H\) l'intersezione tra \(r\) e \(\alpha \). Definiamo \(d(P, r) = d(P, H) = ||\vec{{PH}}||\).

\ex{}{In \(E_{3}(\RR)\) determiniamo la distanza di \(P = (3, 0, 1)\) da \(r : 
\begin{cases}
    x + y = 1 \\
    z = 2 \\
\end{cases}
\)  \[
\begin{cases}
    x = 1-t \\
    y = t \\
    z = 2 \\
\end{cases} \quad P.d.r = [(-1, 1, 0)]=[(a,b,c)] \qquad \alpha: -x + y + 0 \cdot z + d = 0
\] \[
    \text{ Imponiamo il passaggio per \(P\)}: \quad -3 + 0 + d = 0 \quad d = 3 \quad \alpha : -x + y + 3 = 0
\] \[
    \alpha \cap r :
\begin{cases}
    x + y = 1 \\
    -x + y + 3 = 0 \\
    z = 2 \\
\end{cases} \quad
\begin{cases}
    x+y=1 \\
    0x + 2y= -2 \\
    z=2 \\
\end{cases} \implies x = 2; \ y = -1
\]\[
    H: (2, -1, 2) \quad d(P,r) = ||\vec{{PH}}|| = \vec{{PH}} = (-1) i + (-1) j + k = -1 -j + k
\]}

\subsubsection{Distanza tra due rette sghembe in \(E_3(\RR )\)}
\dfn{Retta di minima distanza}{Si dice \textbf{retta di minima distanza} tra due rette \(r\) e \(s\) sghembe in \(E_{3}(\RR)\) una retta ortogonale e incidente sia ad \(r\) che ad \(s\).}

\dfn{Distanza tra due rette sghembe in \(E_{3}(\RR)\)}{Definiamo \textbf{la distanza tra due rette \(r\) e \(s\) sghembe} in \(E_{3}(\RR)\) come la distanza tra i punti \(R\) e \(S\) ottenuti intersecando la retta \(t\) di minima distanza tra \(r\) e \(s\) con \(r\) e \(s\).}

\mprop{}{La retta di minima distanza tra \(r\) e \(s\) esiste ed è unica.}

\subsubsection{Assi e piani assiali}
\dfn{Asse}{In \(E_{2}(\RR)\) dati due punti \(P,Q\), si dice \textbf{asse} del segmento \([P,Q]\) la retta passante per il punto medio di \(P\) e \(Q\) e ortogonale alla retta per \(P\) e \(Q\).}
\mprop{}{L'asse di un segmento \([P,Q]\) è il luogo dei punti equidistanti da \(P\) e da \(Q\).}
\pf{Dimostrazione}{Dobbiamo dimostrare che \(||\vec{PH} || = ||\vec{QH} || \quad \forall H \in a\) (asse di \([P,Q]\)). \[
    \vec{PH} = \vec{PM} + \vec{MH} \quad e \quad \vec{QH} = \vec{QM} + \vec{MH} 
\] \[
    ||\vec{PH} || = \sqrt{||PM || ^{2} + ||MH||^{2}  } \quad ||\vec{QH} || = \sqrt{||QM|| ^{2} + ||MH|| ^{2} } \quad \text{ ma } \quad ||PM||= ||QM|| 
\] \[
    ||\vec{PH} || = \sqrt{||PM|| ^{2} + ||MH|| ^{2} } = \sqrt{||QM|| ^{2} + ||MH|| ^{2} } = ||\vec{QH} || 
\]}

\ex{}{Determiniamo l'asse di \(P=(1,1)\) e \(Q=(2, -4)\). Il punto \(M = (\frac{3}{2}, -\frac{3}{2})\) \[
    \vec{PQ} = (2-1) i + (-4-1) j = 1 - 5 j = (1, -5)
\] \(r \perp \vec{PQ} \) per \(M\)  è del tipo \[
    x -5y + c = 0 \quad \text{e passa per \(M\) }
\] \[
    \frac{3}{2} + \frac{15}{2} + c = 0 \quad c = -9 \implies r : \ x - 5y -9 = 0
\]Alternativamente \[
    r: \ H \in r \iff d(H,P) = d(H, Q)
\]se \(H = (x, y)\) \[
    \sqrt{(x-1)^{2} + (y-1) ^{2} } = \sqrt{(x-2)^{2} + (y + 4)^{2} }
\] \[
    x^{2} - 2x + 1 + y^{2}  -2y + 1 = x^{2} - 4x + 4 + y^{2} + 8y + 16 \implies r: \ 2x -10y -18 = 0
\]}

\dfn{Piano assiale}{In \(E_{3}(\RR)\) si dice \textbf{piano assiale} del segmento \([P,Q]\) il piano \(\alpha \) passante per il punto medio di \(P\) e \(Q\) e ortogonale al segmento \([P,Q]\).}

\mprop{}{Il piano assiale del segmento \([P,Q]\) è il luogo dei punti equidistanti tra \(P\) e \(Q\).}

\section{Circonferenza e sfera}
\dfn{Circonferenza}{Dato un punto \(C = (x_0, y_0)\) in \(E_{2}(\RR)\) e dato \(r\), numero reale positivo, si dice \textbf{circonferenza} di centro \(C\) e raggio \(r\) il luogo dei punti aventi distanza \(r\) da \(C\). }
\dfn{Sfera}{Sia \(C = (x_{0}, y_{0}, z_{0} )\) e sia \(r\) un numero reale positivo. Si dice \textbf{sfera} di raggio \(C\) e di centro \(r\) il luogo dei punti aventi distanza \(r\) da \(C\).}
\paragraph{Osservazione:} La circonferenza è una curva algebrica reale, mentre la sfera è una superficie algebrica reale.
\subsubsection{Rappresentazione analitica di una circonferenza in \(E_2(\RR )\)}
Sia il generico punto \(P=(x, y)\) appartenente alla circonferenza di centro \(C = (x_0, y_0)\) e raggio \(r\). \[
d(P,C) = \sqrt{(x-x_0)^{2} + (y-y_0)^2} = \sqrt{x^{2} + y^{2} + 2ax + 2by + x_0^2 + y_0^2} = r \iff (x-x_0)^{2} + (y-y_0)^{2} = r^{2}  
\] \[
    x^{2} + y^{2} + 2ax + 2by + c = 0 \iff x^{2} + y^{2} - 2x_0x - 2y_0y + (x_0^{2} + y_0^{2} - r^{2} ) = 0
\] 

\mprop{Equazione cartesiana di una circonferenza}{Tutte e sole le circonferenze si rappresentano come \[x^{2} + y^{2} + 2ax + 2by + c = 0 \quad \text{con} \quad a^{2} + b^{2} - c > 0\] e avremo che \(C = (-a, -b)\) e \(r = \sqrt{a^{2} + b^{2} -c}\)  }

\nt{Se \(r\) fosse 0 allora \(a^{2} + b^{2} - c = 0\) e quindi \(x^{2} + y^{2} + 2ax + 2by + c = 0\) rappresenta il solo punto \(C = (-a, -b)\).}
\mprop{}{Per tre punti non allineati in \(E_{2}(\RR)\) passa un unica circonferenza.}

\subsubsection{Rappresentazione analitica di una sfera in \(E_3(\RR )\)}
Sia il generico punto \(P = (x,y,z)\) appartenente alla sfera, allora \[
    d(P,C) = \sqrt{(x-x_0)^{2} + (y-y_0)^{2} + (z-z_0)^{2}} = r \iff (x-x_0)^{2} + (y-y_0)^{2} + (z-z_0)^{2} = r^{2}
\]
\mprop{Equazione cartesiana di una sfera}{Tutte e sole le sfere si rappresentano come \[x^{2} + y^{2} + z^{2} + 2ax + 2by + 2cz + d = 0 \quad \text{con} \quad a^{2} + b^{2} + c^{2} > 0\] e avremo che \(C = (-a, -b, -c)\) e \( r = \sqrt{a^{2} + b^{2} + c^{2} - d}\)  }
\nt{Se \(a^{2} + b^{2} + c^{2} - d = 0\) allora \(x^{2} + y^{2} + z^{2} + 2ax + 2by + 2cz + d = 0\) è realizzata dal solo punto \(C = (-a, -b, -c)\). }

\mprop{}{Per quattro punti non complanari di \(E_{3}(\RR)\) passa un'unica sfera.}

\subsubsection{Circonferenze in \(E_{3}(\RR)\) }
\dfn{Circonferenza in \(E_{3}(\RR)\) }{In \(E_3(\RR )\) dati un piano \(\alpha \), un suo punto \(C\) e un numero reale positivo \(r\), si dice \textbf{circonferenza} di centro \(C\) e raggio \(r\) il luogo dei punti di \(\alpha \) aventi distanza \(r\) da \(C\).}

\paragraph{Osservazione:} Una circonferenza appartiene a infinite sfere. Quindi per tre punti non allineati passano infinite sfere.

\mprop{}{Tutte e sole le circonferenze di \(E_{3}(\RR)\) ammettono una rappresentazione del tipo \[
\begin{cases}
    \ ax + by + cz + d = 0 \quad \rightarrow \quad \text{piano \(\alpha \) } \\
    \ (x- x_0)^{2} + (y - y_0)^{2} + (z-z_0)^{2} = R^{2} \\ 
\end{cases}
\] \[
    d(C', \alpha ) < R \quad \text{ove} \quad C' = (x_0, y_0, z_0) \quad e \quad 
    \frac{|ax_0 + by_0 + cz_0 + d| }{\sqrt{a^{2} + b^{2} + c^{2} }} < R
\]}
\paragraph{Osservazione:} Vi sono infinite sfere che intersecano la circonferenza, ma solo in una di esse il centro \(C'\) della sfera coincide con il centro \(C\) della circonferenza.
Il centro della circonferenza \(C\) si trova intersecando il piano \(\alpha \) con la retta per il centro della sfera \(C'\) perpendicolarmente ad \(\alpha \). Per determinare il raggio della circonferenza utilizziamo il teorema di Pitagora. Conosciamo sia \([C,C'] = h\) che il raggio \(R\) della sfera. Quindi \[
    r = \sqrt{R^{2} - h^{2}}
\]
\begin{figure}[ht]
    \centering
    \def\svgwidth{130pt}
    \incfig{circonferenza-tridimensionale}
    \label{fig:circonferenza-tridimensionale}
\end{figure}
\nt{Una circonferenza in \(E_3(\RR )\) si può ottenere anche intersecando anche altre superfici superfici con un piano, non solo una sfera.}

\ex{}{Determinare se la seguente è una circonferenza \[
\begin{cases}
    x^{2} + y^{2} = 7 \\
    z = 3 \ \rightarrow \ \alpha \\
\end{cases}
\] \[
    x^{2} + y^{2} + z^{2} - z^{2} = 7 \quad  \text{ e siccome \(z = 3\)} \quad 
\begin{cases}
    x^{2} + y^{2} + z^{2} = 16 \\
    z = 3 \\
\end{cases} \text{ che descrive una circonferenza.}
\]}



\chapter{Ampliamento e complessificazione}
Il concetto di ampliamento dello spazio affine, e di conseguenza anche di quello euclideo, si basa sulla relazione di parallelismo. Abbiamo visto che la relazione di parallelismo tra sottospazi lineari di uno spazio affine \(A_n(K)\) è una relazione di equivalenza. La classe delle delle rette parallele è costituita da tutte le rette che hanno lo stesso spazio di traslazione \(V_1\) e che ora diciamo avere la stessa \textbf{direzione}. Allo stesso modo abbiamo definito la classe dei piani paralleli come tutti i piani aventi lo stesso spazio di traslazione \(V_2\) e che ora diciamo avere la stessa \textbf{giacitura}. Qui avviene il passo fondamentale che è necessario assimilare al meglio per capire tutto ciò che seguirà. Dobbiamo \textbf{liberarci della nozione di parallelismo}, da ora in poi quando si parla di spazi ampliati non esisteranno più rette che non si incontrano mai o piani che non si intersecano. Possiamo ora considerare ad esempio lo spazio di traslazione \(V_1\) di una retta \(r = [P, V_1]\) come un \textbf{punto}, di natura particolare, che chiameremo \textbf{punto improprio}, a essa appartenente. La direzione della retta \(r\) accomuna anche tutte le rette parallele ad essa e quindi, essendo essa il punto improprio, appartiene a tutte le rette parallele a \(r\) e di conseguenza tutte le rette con la stessa direzione si intersecano nel loro punto improprio. Allo stesso modo daremo definizioni di ulteriori enti geometrici impropri, ma il concetto rimane invariato. Rette complanari risultano sempre incidenti, piani paralleli si intersecano nella loro retta impropria. Questo, una volta capito, è il modo più semplice e intuitivo per avvicinarci alla \textbf{geometria proiettiva} e, come vedremo in seguito, costituisce l'ambiente migliore per studiare curve, superfici e più in particolare coniche e quadriche.

\section{Ampliamento proiettivo di \(A_{2}(\RR)\)}
\dfn{Piano affine ampliato \(\tilde{A}_2(\RR)\) }{
Il \textbf{piano affine ampliato} \(\tilde{A}_{2}(\RR ) \), indotto da \(A_2(\RR )\), è la struttura algebrica così definita
\begin{enumerate}
    \item l'insieme dei punti che possono essere
        \begin{itemize}
            \item \textbf{propri} cioè l'insieme dei punti di \(A\) di \(A_2(\RR )\) 
            \item \textbf{impropri} cioè l'insieme dei punti di \(A_\infty\), che sono le direzioni delle rette, ovvero gli spazi di traslazione di dimensione 1
        \end{itemize}
    \item l'insieme delle rette che possono essere
        \begin{itemize}
            \item \textbf{proprie} cioè l'insieme delle rette esistenti nello spazio affine, ciascuna arricchita del proprio punto improprio
            \item \textbf{la retta impropria} cioè il luogo degli \(\infty^{1}\) punti impropri del piano, tale retta viene indicata con \(r_\infty\)
        \end{itemize}
    \item l'applicazione \(f\) dello spazio affine, la quale rimane inalterata, mantiene cioè lo stesso dominio, lo stesso codominio e le stesse proprietà
\end{enumerate}}

\mprop{}{Due rette distinte di \(\tilde{A}_{2}(\RR)\) sono sempre incidenti.}
\pf{Dimostrazione}{La dimostrazione segue banalmente dalla definizione, ma la diamo per esteso per consolidare meglio le idee. Siano \(r\) e \(s\) due rette distinte di \(\tilde{A}_{2}(\RR)\), allora abbiamo 3 possibili casi
\begin{enumerate}
    \item \(r\) e \(s\) sono proprie e non parallele tra loro, ciò significa che \(r\) è incidente a \(s\) in \(A_{2}(\RR)\) \(\subseteq \) \(\tilde{A}_{2}(\RR)\) e il punto improprio di \(r\) è diverso da quello di \(s\).
    \item \(r\) e \(s\) sono proprie ma sono fra loro parallele. \(r \cap s = \emptyset\) in \(A_{2}(\RR)\), ma \(r\) e \(s\) hanno la stessa direzione, quindi si intersecano nello stesso punto improprio in \(\tilde{A}_{2}(\RR ) \).
    \item \(r\) è propria e \(s = r_{\infty} \), cioè la retta impropria. Quindi \(r \cap s = r \cap r_{\infty} \), alla retta impropria appartiene per definizione il punto improprio di \(r\) e quindi si intersecano nel punto improprio di \(r\).
\end{enumerate}}

\mprop{}{Per due punti distinti di \(\tilde{A}_{2}(\RR)\) passa un'unica retta. }
\pf{Dimostrazione}{Siano \(A\) e \(B\) i due punti distinti considerati, abbiamo 3 casi possibili
\begin{enumerate}
    \item \(A\) e \(B\) sono entrambi propri, quindi esiste un'unica retta in \(A_2(\RR )\) passante per \(A\) e \(B\). Inoltre la retta impropria non li contiene essendo essi punti propri e quindi esiste un'unica retta passante per \(A\) e \(B\).
    \item \(A\) è proprio e \(B\) è improprio (o viceversa). Poniamo \(B\) come direzione \(V_1\), ciò implica che esiste un'unica retta passante per \(A\) e avente come direzione \(V_1 = B\). 
    \item \(A\) e \(B\) sono entrambi impropri. Nessuna retta propria li contiene entrambi (ogni retta propria ha un unico punto improprio), tuttavia \(A, B \in r_{\infty} \) che è l'unica che li contiene entrambi.
\end{enumerate}}

\section{Geometria analitica in \(\tilde{A}_{2}(\RR ) \)}

Indichiamo con \[ \frac{\RR ^{3} \backslash  \{(0,0,0)\}}{\rho } \] l'insieme delle terne definite a meno di un fattore di proporzionalità reale e non nullo. In cui \(\rho \) indica la relazione di equivalenza data dalla proporzionalità. Quindi consideriamo due terne equivalenti se sono proporzionali.

\mprop{}{Sia \(RA = [O, B]\) un riferimento affine di \(A_{2}(\RR) \) e sia \[
\phi : A \cup A_\infty \quad  \to \quad \frac{\RR ^{3} \backslash  \{(0,0,0)\}}{\rho }
\] sia \(P \in A\) di coordinate \((x,y)\) \[
\phi(P) = [(x,y,1)]
\] sia \(P \in A_\infty\) corrispondente alla direzione \([(l,m)]\) \[
\phi (P) = [(l,m,0)]
\] la mappa \(\phi\) è una biiezione e le coordinate indotte da \(\phi\) sono chiamate \textbf{coordinate omogenee}.}

\paragraph{Osservazione:} Sia \(P\) di coordinate omogenee \([(x_1, x_2, x_3)]\), con \(x_3 \neq 0\), quindi punto proprio. Allora le sue coordinate omogenee sono \[
    \left[ \left( \frac{x_1}{x_3}, \frac{x_2}{x_3}, 1 \right)  \right] 
\] quindi scritto in coordinate affini \[
P = (x,y) = \left[ \left( \frac{x_1}{x_3}, \frac{x_2}{x_3} \right) \right]
\] Se invece \(P\) è improprio, quindi \(x_3 = 0\), allora \[
P = [(x_1, x_2, 0)] \quad [(l,m)] = [(x_1, x_2)]
\] quindi \(P\) non ha coordinate affini e le sue coordiante omogenee sono date dai parametri direttori della retta.

\subsubsection{Rappresentazione delle rette in \(\tilde{A}_{2}(\RR) \)}
Sia \(RA [O, B]\) un riferimento affine di \(A_{2}(\RR) \). In \(A_{2}(\RR) \) l'equazione cartesiana di una retta è 
\[ax+by+c = 0 \quad \text{con} \quad  (a,b) \neq (0,0)\]
per i sui punti propri \(P = \left[ \left( \frac{x_1}{x_3}, \frac{x_2}{x_3}, 1 \right)  \right] \) dovrà valere l'equazione \(ax+by+c = 0\), quindi \[
a \left( \frac{x_1}{x_3} \right) + b \left( \frac{x_2}{x_3} \right) + c = 0 \]
quindi, moltiplicando tutto per \(x_3\), che si suppone non nullo, otteniamo
\[ ax_1+ bx_2+ cx_3=0 \quad \text{con}\quad (a,b) \neq (0,0)
\] Il punto improprio di \(ax+by+c=0\) è \([(-b, a, 0)]\). Sostituiamo in \(ax_1+ bx_2+ cx_3=0\) le coordinate omogenee \([(-b, a, 0)]\) e otteniamo la seguente \[
a(-b) + b a + 0 = 0
\] che è sempre verificata, quindi \(ax_1+ bx_2+ cx_3=0\) è l'\textbf{equazione omogenea di una retta} \(r\) in \(\tilde{A}_{2}(\RR) \). \\
Siano ora \((a,b) = (0,0)\), allora \(ax_1+ bx_2+ cx_3=0\) si riduce a \(0 x_1+0x_2+cx_3=0\) con \(c \neq 0, \ cx_3 = 0, \ x_3 = 0\) è la \(r_{\infty}\) perché rispettata da tutti e soli i punti impropri. 
L'equazione \(ax_1+bx_2+cx_3= 0\) con \((a,b,c) \neq (0,0,0)\) rappresenta in ogni caso, anche quello della \(r_\infty\), una retta di \(\tilde{A}_{2}(\RR) \). Di conseguenza è l'equazione cartesiana di una retta di \(\tilde{A}_{2}(\RR)\).

\section{Complessificazione di \(\tilde{A}_{2}(\RR) \)}

Utilizzare il campo complesso, anziché quello reale, ci consente di dimostrare i teoremi dell'ordine per le curve e le superfici, il cui utilizzo agevola in maniera determinante lo studio delle proprietà geometriche. Definiamo \(\tilde{A}_{2}(\CC) \) il piano affine ampliato e complessificato, in cui
\begin{itemize}
    \item i \textbf{punti} sono le terne, non nulle, di numeri complessi determinati a meno di un fattore di proporzionalità complesso e non nullo.
\[
\frac{\CC^{3}\backslash \{(0,0,0)\} }{\rho }
\] 
    \item le \textbf{rette} sono il luogo delle autosoluzioni, non nulle, di un'equazione del tipo \[
    ax_1+bx_2+cx_3= 0 \quad \text{con} \quad (a,b,c) \neq (0,0,0) \quad e \quad a,b,c \in \CC
    \] 
\end{itemize}

\dfn{Punti e rette in \(\tilde{A}_{2}(\CC)\)}{In \(\tilde{A}_{2}(\CC)\) si dicono:
\begin{itemize}
    \item \textbf{punti e rette reali} tutti i punti e le rette che ammettono una rappresentazione reale
    \item \textbf{punti e rette immaginari} tutti i punti e le rette che ammettono solo rappresentazioni immaginarie
\end{itemize}}

\dfn{Coniugati}{Si dicono \textbf{coniugati} due enti (punti, rette ecc\ldots) che ammettono rappresentazioni coniugate. La funzione di coniugio è quella che ad ogni numero complesso \(z = x + iy \in \CC\), associa il suo complesso coniugato \[
\overline{z} = x - iy = \text{Re}(z) - i \text{Im}(z)
\] }

\mprop{}{Un ente geometrico (punto, retta, curva ecc\ldots ) è reale se, e soltanto se, coincide con il proprio coniugato.}

\paragraph{Osservazione:} Una retta reale ha infiniti punti immaginari
\paragraph{Osservazione:} Se un'equazione reale è realizzata da un punto \(P\) allora \(\overline{P}\) è soluzione se, e soltanto se, \(P \in r\) è reale. Quindi \(\overline{P} \in \overline{r}\) e \(r = \overline{r}\).

\mprop{}{In \(\tilde{A}_{2}(\CC) \)
\begin{enumerate}
    \item la retta che congiunge due punti \(P\) e \(\overline{P}\) immaginari e coniugati è reale.
    \item per un punto \(P\) immaginario \((P \neq \overline{P})\) passa un'unica retta reale.
    \item due rette immaginarie e coniugate si intersecano in un punto reale di \(\tilde{A}_{2}(\CC) \).
    \item ogni retta \(r\) immaginaria ha un unico punto reale in \(\tilde{A}_{2}(\CC)\).
\end{enumerate}}
\pf{Dimostrazione}{Dimostriamo ogni punto separatemente
\begin{enumerate}
    \item Siano \(P\) e \(\overline{P}\) due punti immaginari e coniugati. Sia \(r\) la retta che li congiunge. La retta \(\overline{r}\), coniugata di \(r\), rimane individuata da \(P\) e \(\overline{P}\), quindi, per l'unicità della retta che congiunge due punti, \(r = \overline{r}\), che pertanto è anche reale.
    \item La retta \(rt(P, \overline{P})\) è reale per la proposizione precedente. Supponiamo per assurdo che esista un \(s \neq rt(P, \overline{P})\) retta reale per \(P\). Ciò implica che \(\overline{P} \in s\) poiché \(s\) è reale. Quindi \(s = rt(P, \overline{P})\) che è \textbf{assurdo!} Poiché avevamo supposto che \(s\) fosse distinta dalla congiungente fra \(P\) e \(\overline{P}\). Quindi esiste ed è unica la retta \(r\) reale per \(P\).
    \item Sia \(r\) una retta immaginaria e \(\overline{r}\) la sua coniugata. Ovviamente, \(r \neq \overline{r}\), altrimenti \(r\) sarebbe reale, quindi \(r \cap \overline{r}\) è un punto \(P\). Dato che \(P\) appartiene ad \(r\) e a \(\overline{r}\), il suo coniugato \(\overline{P}\) appartiene sia a \(\overline{r}\) che a \(\overline{\overline{r}}\), che coincide con \(r\). Quindi \(P\) coincide con \(\overline{P}\) e di conseguenza è reale.
    \item Per ipotesi \(r \neq \overline{r}\) quindi esiste un punto \(P\) intersezione di \(r\) e \(\overline{r}\). Per la proposizione precedente \(P\) è reale. Sia \(S \in r\) un punto reale. Essendo reale \(S\) coincide con il proprio coniugato \(\overline{S}\) e inoltre \(S \in r \cap \overline{r}\). Ma per l'unicità del punto di intersezione, \(S = P\).
\end{enumerate}}

\section{Curve algebriche reali in \(\tilde{A}_{2}(\CC)\)}
\dfn{Curve algebriche reali in \(\tilde{A}_{2}(\CC) \)}{Una \textbf{curva algebrica reale di \(\tilde{A}_{2}(\CC)\)} è il luogo delle autosoluzioni di un'equazione del tipo \[
F(x_1, x_2, x_3) = 0
\] dove \(F(x_1, x_2, x_3) = 0\) è un polinomio omogeneo a coefficienti reali nelle variabili \(x_1, x_2, x_3\).}

\paragraph{Osservazione:} Ogni curva algebrica reale di \(\tilde{A}_{2}(\CC)\) che contiene un punto  \(P\) contiene anche \(\overline{P}\).

\dfn{Curva riducibile}{In \(\tilde{A}_{2}(\CC) \) una curva algebrica reale \(C : F(x_1, x_2, x_3)\) si dice \textbf{riducibile} se \(F\) è il prodotto di polinomi di grado più basso. Altrimenti la curva si dice \textbf{irriducibile}.}
Se \(C\) è riducibile risulta \[F(x_1, x_2, x_3) = [F_1(x_1, x_2, x_3)]^{n_1} \cdot [F_2(x_1, x_2, x_3)]^{n_2} \cdot \ldots \cdot [F_t(x_1, x_2, x_3)]^{n_t}\] dove i polinomi \(F_i(x_1, x_2, x_3)\) sono polinomi irriducibili di grado positivo. Quindi avremo che \[
\deg(F) = n_1 \deg(F_1) + \ldots + n_t \deg (F_t)
\] 
\paragraph{Osservazione:} Geometricamente una curva riducibile si riduce in componenti ottenute uguagliando a zero i vari fattori. \[
C = C_1 \cup C_2 \cup \ldots \cup C_t
\] 

\dfn{Ordine}{Si dice \textbf{ordine} di una curva algebrica reale in \(\tilde{A}_{2}(\CC) \) il grado del polinomio \(F\) che la definisce.}

\thm{Teorema dell'ordine}{L'ordine di una curva algebrica reale è uguale al numero di intersezioni in comune con una qualsiasi retta \(r\) di \(\tilde{A}_{2}(\CC) \) a patto che \(r\) non sia componente della curva e che le intersezioni siano contate con la dovuta molteplicità.}

\dfn{Punti semplici ed r-upli}{Sia \(C\) una curva algebrica di \(\tilde{A}_{2}(\CC) \) e sia \(P \in C\)
\begin{itemize}
    \item \(P\) si dice \textbf{semplice} se la generica retta per \(P\) interseca \(C\) in \(P\) con molteplicità unitaria ed esiste un'unica retta, chiamata retta tangente, con molteplicità di intersezione in \(P\) maggiore di 1.
    \item \(P\) si dice \textbf{r-uplo} (doppio, triplo, ecc\ldots ) se la generica retta per \(P\) interseca \(C\) in \(P\) con molteplicità \(r\), ed esistono \(r\) (contate con la loro molteplicità) rette con molteplicità di intersezione in \(P\) maggiore di \(r\) (rette tangenti).
\end{itemize}}

\mprop{}{Sia \(C\) una curva algebrica reale di \(\tilde{A}_{2}(\CC) \). Se una retta \(r\) ha più di \(n\) intersezioni con \(C\), con \(n\) l'ordine di \(C\), allora \(r\) è componente di \(C\).}
\pf{Dimostrazione}{Per il teorema dell'ordine se \(r\) non fosse componente della curva \(C\) avrebbe esattamente \(n\) intersezioni con \(C\) (a patto di contarle con la dovuta molteplicità).}

\mprop{}{Sia \(C\) una curva algebrica reale di \(\tilde{A}_{2}(\CC) \) di ordine \(n\). Allora \(C\) non possiede punti (\(n + 1\))-upli.}
\pf{Dimostrazione}{Dato che \(C\) è di ordine \(n\) questo significa che esiste una retta \(r \in \tilde{A}_{2}(\CC) \) non componente di \(C\) passante per un punto dato di \(C\). Sia, per assurdo \(P\) un punto \((n+1)\)-uplo. \[
|r \cap C| \ge n+1 \quad  \text{perché passa per \(P\)}
\] ma dato che \(r\) non è componente per il teorema dell'ordine \[
|r \cap C| = n < n+1
\] \textbf{Assurdo!} }

\mprop{}{Sia \(C\) una curva algebrica reale di \(\tilde{A}_{2}(\CC)\) di ordine \(n\). \(C\) ha un punto \(n\)-uplo \(P\) se, e soltanto se, \(C\) è unione di \(n\) rette (contate con la dovuta molteplicità) per \(P\).}
\pf{Dimostrazione}{"\(\implies \)" Sia \(P \neq Q \in C\) e sia \(r\) la retta \(rt(P,Q)\). Supponiamo per assurdo \(r\) non sia componente, allora per il teorema dell'ordine \[
n = |r \cap C| \ge \underbrace{n}_{\in P}  + \underbrace{1}_{\in Q}
\] \textbf{Assurdo!} Quindi per ogni punto \(Q \in C\) la retta \(PQ\) è componente. Di conseguenza \(C\) è unione di rette per \(P\). Quindi queste rette sono \(n = \deg(F) = \) ordine di \(C\). \\
"\(\impliedby \)" Sia \(C\) unione di \(n\) rette per \(P\). Allora la generica retta per \(P\) non componente di \(C\) interseca \(C\) solo in \(P\), quindi \(P\) è punto \(n\)-uplo.
}

\dfn{Punto multiplo}{Sia \(C\) una curva algebrica reale di \(\tilde{A}_{2}(\CC)\) e sia \(P \in C\). Se \(P\) non è un punto semplice allora si dice \textbf{punto multiplo}.}

\thm{}{Sia \(C\) una curva algebrica reale di \(\tilde{A}_{2}(\CC)\) di ordine \(n\) e sia \(F(x_1, x_2, x_3) = 0\) il polinomio omogeneo che la definisce. I punti multipli di \(C\) sono le classi di autosoluzioni del sistema associato alle derivate: \[
\begin{cases}
    \ \frac{dF}{dx_1}= 0 \\
    \ \frac{dF}{dx_2}=0 \\
    \ \frac{dF}{dx_3}=0 \\
\end{cases}
\] }

\ex{}{Ad esempio prendiamo una curva algebrica reale\[
x_1^2+2x_2^2+3x_1x_3-3x_2x_3=0
\] I punti multipli di \(C\) sono le classi di autosoluzioni del seguente sistema \[
\begin{cases}
    \ \frac{dF}{dx_1}= 2x_1+3x_3=0 \\
    \ \frac{dF}{dx_2}= 4x_2-3x_3=0 \\
    \ \frac{dF}{dx_3} = 3x_1-3x_2 = 0 \\
\end{cases}
\] da cui ricaviamo la seguente matrice, con determinante non nullo \[
A = 
\left( \; \begin{matrix}
    2 & 0 & 3 \\
    0 & 4 & -3 \\
    3 & -3 & 0 \\
\end{matrix} \; \right) \qquad |A| \neq 0
\] }

\chapter{Coniche}
\section{Coniche in \(\tilde{A}_{2}(\CC)\)}
\dfn{Conica}{Si dice \textbf{conica} una curva algebrica reale di \(\tilde{A}_{2}(\CC)\) (curva piana) del secondo ordine. Una conica si rappresenta eguagliando a \(0\) un polinomio omogeneo \(F\) di secondo grado nelle variabili  \(x_1, x_2, x_3\), a coefficienti reali. La generica equazione della conica è \[
C: a_{11}x_1^2+ 2a_{12}x_1x_2+2a_{13}x_1x_3+a_{22}x_2^2+2a_{23}x_2x_3+a_{33}x_3^2=0
\] Se chiamiamo \[
X = \left( \; \begin{matrix} x_1\\ x_2 \\ x_3 \end{matrix} \; \right) \quad A =
\left( \; \begin{matrix}
    a_{11} & a_{12} & a_{13} \\
    a_{12} & a_{22} & a_{23} \\
    a_{13} & a_{23} & a_{33} \\
\end{matrix} \; \right)
\] Possiamo riscrivere l'equazione come prodotto righe per colonne \[
C : \ {^tX}AX = \ul{0} 
\] \(A\) è una matrice reale e simmetrica ed è detta \textbf{matrice della conica}.} 

\ex{}{Consideriamo la conica \[
-x_1^2+ax_1x_2+5x_2^2-3x_2x_3+6x_3^2=0
\] \[
A = 
\left( \; \begin{matrix}
    -1 & 2 & 0 \\
    2 & 5 & -\frac{3}{2} \\
    0 & -\frac{3}{2} & 6 \\
\end{matrix} \; \right)
\] Ora facciamo il prodotto  \[
\left( \; \begin{matrix}
    x_1 & x_2 & x_3 \\
\end{matrix} \; \right) \cdot 
\left( \; \begin{matrix}
    -1 & 2 & 0 \\
    2 & 5 & -\frac{3}{2} \\
    0 & -\frac{3}{2} & 6 \\
\end{matrix} \; \right) \cdot 
\left( \; \begin{matrix} x_1\\ x_2\\ x_3 \end{matrix} \; \right) = 0
\] \[
\left( \; \begin{matrix}
    -x_1+2x_2 & 2x_1+5x_2-\frac{3}{2}x_3 & -\frac{3}{2}x_2+6x_3 \\
\end{matrix} \; \right) \cdot \left( \; \begin{matrix} x_1\\ x_2\\ x_3 \end{matrix} \; \right) = 0
\] \[
x_1(-x_1+2x_2) + x_2\left(2x_1+5x_2- \frac{3}{2}x_3\right) + x_3\left(-\frac{3}{2}x_2 + 6x_3\right) = 0
\] \[
-x_1^2+4x_1x_2+5x_2^2-3x_2x_3+6x_3^2=0
\] che è uguale all'equazione di partenza.}

\paragraph{Osservazione:} L'equazione della generica conica in \(\tilde{A}_{2}(\CC)\) dipende da 6 coefficienti definiti a meno di un fattore di proporzionalità. Quindi le coniche di \(\tilde{A}_{2}(\CC)\) sono \(\infty^{5}\).

\mprop{}{Sia \(C\) una conica di \(\tilde{A}_{2}(\CC)\) riducibile. Allora \(C\) è unione di 2 rette che possono essere
\begin{itemize}
    \item reali e distinte
    \item reali e coincidenti
    \item immaginarie e coniugate
\end{itemize}}

\pf{Dimostrazione}{Sia \(C\) la conica associata al polinomio \(F=(x_1,x_2, x_3) = 0\). Se \(C\) è riducibile \(F=(x_1,x_2, x_3) = F_1=(x_1,x_2, x_3) \cdot F_2=(x_1,x_2, x_3)\) dove \(F_1\) e \(F_2\) hanno grado 1, quindi rappresentano delle rette e di conseguenza \(C\) è unione di due rette \(r_1\) e \(r_2\). Se \(r_1\) e \(r_2\) sono entrambe reali siamo nei casi \(1\) o \(2\). Se invece \(r_1\) è immaginaria allora \(\overline{r_1}\) è ancora componente di \(C\) (per ogni \(P \in r_1, \ \overline{P}\in C\)), ma \(r_1 \neq \overline{r_1} \implies \overline{r_1}=r_2\implies C\) si riduce in due rette immaginarie e coniugate.}

\subsubsection{Punti multipli di una conica}
\mprop{}{In \(\tilde{A}_{2}(\CC) \) una conica
\begin{enumerate}
    \item non ha punti tripli
    \item ha un punto doppio se, e soltanto se, è riducibile. E abbiamo due possibilità
        \begin{enumerate}
            \item ha solo un punto doppio \(P\) e si riduce in due rette distinte per \(P\)
            \item ha almeno due punti doppi allora ne ha \(\infty^{1}\) e si fattorizza in una retta reale contata due volte
        \end{enumerate}
\end{enumerate}}

\pf{Dimostrazione}{Dimostriamo il secondo punto \\ "\(\implies \)" Per ipotesi \(C\) ha un punto doppio \(P\). Sia \(R \in C\) e consideriamo la retta \(r = rt(P,R)\), se non fosse componente avrebbe \[
|r \cap C| \ge 2 + 1 = 3 \quad \text{intersezioni con \(C\)}
\] \textbf{Assurdo!} Questo è in contraddizione con il teorema dell'ordine. \\
"\(\impliedby \)" Sia \(C\) per ipotesi riducibile. Allora  \(C = r_1 \cup r_2\). Sia \(P \in r_1 \cap  r_2\) e sia \(r\) una retta per \(P\) diversa da \(r_1\) e da \(r_2\). Quindi \(r \cap C = P\). Per il teorema dell'ordine \(P\) ha molteplicità doppia e abbiamo due casi possibili
\begin{enumerate}
    \item se \(r_1=r_2\) abbiamo \(\infty^{1}\) punti doppi e \(C = r_1 \cup  r_1\)
    \item altrimenti abbiamo un \textbf{solo} \(P\) punto doppio che è \(r_1\cap r_2\)
\end{enumerate}
Dobbiamo dimostrare che esiste un solo punto doppio. Siano per assurdo \(P_1\) e \(P_2\) punti doppi distinti e sia \(C=r_1 \cup r_2\) con \(r_1 \neq r_2\). Sia \(Q \in r_2\) e \(P_2 \in r_1\), allora \[
|rt(P_2, Q) \cap C| \ge \underbrace{2}_{P_2} + \underbrace{1}_{Q} 
\] Per il teorema dell'ordine \(rt(P_2, Q)\) è componente. \textbf{Assurdo!} Perché avremmo 3 componenti \((r_1, r_2, rt(P_2, Q))\).} 

\dfn{Coniche generali o degeneri}{Una conica si dice
 \begin{itemize}
    \item \textbf{generale}, se è priva di punti doppi \(\implies \) quindi non è riducibile
    \item \textbf{semplicemente degenere} se ha un solo punto doppio \(\implies C = r_1\cup r_2\) con \(r_1\neq r_2\) 
    \item \textbf{doppiamente degenere} se ha \(\infty^{1}\) punti doppi \(\implies C = r \cup r\)
\end{itemize}}

\thm{}{In \(\tilde{A}_{2}(\CC) \) i punti doppi di una conica \(C\) si trovano considerando le classi di autosoluzioni del sistema omogeneo \[
AX = \ul{0}
\] dove \(A\) è la matrice associata a \(C\).}

\pf{Dimostrazione}{\[
C: F(x_1, x_2, x_3) = 0 \quad \text{dove \(F\) è:} 
\]
\[
a_{11}x_1^2+ 2a_{12}x_1x_2+2a_{13}x_1x_3+a_{22}x_2^2+2a_{23}x_2x_3+a_{33}x_3^2=0
\] i punti doppi si trovano risolvendo 
\[
\begin{cases}
    \ \frac{\partial F}{\partial x_1}= 2a_{11}x_1+2a_{12}x_2+2a_{13}x_3=0\\
    \ \frac{\partial F}{\partial x_2}= 2a_{12}x_1+2a_{22}x_2+2a_{23}x_3=0\\
    \ \frac{\partial F}{\partial x_3} = 2a_{13}x_1+2a_{23}x_2+2a_{33}x_3=0\\
\end{cases}
\] Possiamo dividere tutti i fattori per 2 \[
\left( \; \begin{matrix}
    a_{11} & a_{12} & a_{13} \\
    a_{12} & a_{22} & a_{23} \\
    a_{13} & a_{23} & a_{33} \\
\end{matrix} \; \right) \cdot \left( \; \begin{matrix} x_1\\ x_2\\ x_3 \end{matrix} \; \right) = \left( \; \begin{matrix} 0\\ 0\\ 0\\ \end{matrix} \; \right)
\] \[
\implies AX = \ul{0} 
\]  
}

\thm{}{In \(\tilde{A}_{2}(\CC) \) una conica \(C : {^tX}A X = \ul{0} \) risulta
\begin{enumerate}
    \item generale se, e soltanto se, \(\rho(A) = 3\) 
    \item semplicemente degenere se, e soltanto se, \(\rho(A) = 2\) 
    \item doppiamente degenere se, e soltanto se, \(\rho(A) = 1\)
\end{enumerate}}

\pf{Dimostrazione}{  
Dimostriamo tutti i casi singolarmente:
\begin{enumerate}
    \item \(C\) è generale se, e soltanto se, non ha punti doppi. Se \(AX = \ul{0} \) ha solo la soluzione nulla \( \iff \rho(A) =3\).
    \item \(C\) è semplicemente degenere se ha un solo punto doppio. \( \iff AX = \ul{0} \) ha \(\infty^{1}\) soluzioni \(\iff \rho(A) = 2\)
    \item \(C\) è doppiamente degenere se ha \(\infty^{1}\) punti doppi \(\iff AX = \ul{0} \) ha \(\infty^{2}\) soluzioni (se \([(x_1, x_2, x_3)]\) è soluzione \([(2x_1, 2x_2, 2x_3)]\) è lo stesso punto doppio) \(\iff \rho(A) =1\)
\end{enumerate}
}

\subsubsection{Classificazione affine di una conica generale}
Sia \(C\) una conica di \(\tilde{A}_{2}(\CC) \) e \(r\) una qualsiasi retta, osserviamo che \(r \cap C\) può essere
\begin{enumerate}
    \item due punti reali e distinti
    \item un punto reale con molteplicità doppia
    \item due punti immaginari e coniugati
\end{enumerate}
Se consideriamo come retta la \(r_{\infty}\) questa serie di casistiche ci dà la classificazione affine delle coniche generali.

\dfn{Ellisse, iperbole e parabola}{Sia \(C\) una conica generale di \(\tilde{A}_{2}(\CC) \). Allora \(C \cap r_{\infty}\) è data da due punti \(P, Q\) (non necessariamente distinti) e \(C\) si dice:
\begin{enumerate}
    \item \textbf{ellisse}, se \(P\) e \(Q\) sono immaginari e coniugati
    \item \textbf{iperbole}, se \(P\) e \(Q\) sono reali e distinti
    \item \textbf{parabola}, se \(P\) e \(Q\) sono reali e coincidenti
\end{enumerate}}

\subsubsection{Condizioni analitiche}
Sia \(C\) una conica generale di equazione \[
a_{11}x_1^2+ 2a_{12}x_1x_2+2a_{13}x_1x_3+a_{22}x_2^2+2a_{23}x_2x_3+a_{33}x_3^2=0
\] 
La \(r_{\infty}\) ha equazione \(x_{3}=0\) \[
\begin{cases}
    \ a_{11}x_1^2+2a_{12}x_1x_2+a_{22}x_2^2= 0 = C \cap r_{\infty} \\
    \ x_3 = 0 \\
\end{cases}
\]
Almeno uno fra \(x_1, x_2 \neq 0\). Supponiamo \(x_2 \neq 0\) e dividiamo per \(x_2^2\) \[
a_{11} \left( \frac{x_1}{x_2} \right) ^2 + 2a_{12} \left( \frac{x_1}{x_2} \right)  + a_{22} = 0
\] 
La risolviamo in \(\left( \frac{x_1}{x_2} \right) \). Se 
\begin{enumerate}
    \item \(\frac{\Delta}{4} > 0\) abbiamo due soluzioni reali e distinte \(\implies \) \textbf{iperbole};
    \item \(\frac{\Delta}{4} = 0\) abbiamo due soluzioni coincidenti \(\implies \) \textbf{parabola};
    \item \(\frac{\Delta}{4} < 0\) abbiamo due soluzioni immaginarie e coniugate \(\implies \) \textbf{ellisse}.
\end{enumerate}

\[
\frac{\Delta}{4} = \left( \frac{b}{2} \right) ^2 - ac = \left( \frac{2a_{12}}{2} \right) ^2 - a_{11} a_{22} = a_{12}^2 - a_{11} a_{22}
\]Per semplificare le cose, data la matrice della conica \[
A = 
\left( \; \begin{matrix}
    a_{11} & a_{12} & a_{13} \\
    a_{12} & a_{22} & a_{23} \\
    a_{13} & a_{23} & a_{33} \\
\end{matrix} \; \right)
\quad \text{poniamo} \quad A^* =
\left( \; \begin{matrix}
    a_{11} & a_{12} \\
    a_{12} & a_{22} \\
\end{matrix} \; \right) \]
Per classificare la conica basta studiare il determinante di \(A^{*}\)
\[
|A^{*}| = a_{11}a_{22}-a_{12}^2= - \frac{\Delta}{4}
\] Se \(C\) è una conica generale \((|A| = 0)\) allora si applicano le casistiche precedentemente elencate.

\section{Polarità associata a una conica}

\dfn{Coniugato rispetto ad una conica}{Data una conica \(C: {^tX}AX = 0\) e dati due punti di \(\tilde{A}_{2}(\CC) \) \[
        P' = [(x_1', x_2', x_3')] \quad e \quad P '' = [(x_1 '', x_2 '', x_3 '')]
\] si dice che \(P'\) è coniugato a \( P ''\) rispetto a \(C\) se \[
{^tX'AX '' = 0 \quad con \quad X' = \left( \; \begin{matrix} x'_1\\ x_2'\\ x'_3 \end{matrix} \; \right)} \quad  e \quad X '' = \left( \; \begin{matrix} x''_1\\ x_2 ''\\ x''_3 \end{matrix} \; \right)
\]}

\paragraph{Osservazione:} Sia \(P'\) coniugato a \(P ''\), ovvero \[
    {^tX'} A X '' = 0 \implies {^t({^tX'}AX '')} = 0 = {^tX ''} {^tA}{^t({^tX'})} = {^tX ''} A X' = 0 \implies P '' \text{ è coniugato a } P' 
\] Quindi la relazione di coniugio è simmetrica, perciò potremo dire semplicemente che \(P'\) e \(P ''\) sono coniugati.

\dfn{Polare}{Sia \(C\) una conica e \(P'\) un punto di \(\tilde{A}_{2}(\CC) \). Si dice \textbf{polare} di \(P'\) rispetto a \(C\), il luogo dei coniugati di \(P'\) rispetto a \(C\). Il punto \(P'\) prende il nome di \textbf{polo} di tale luogo.}

\mprop{}{In \(\tilde{A}_{2}(\CC) \) la polare di un punto \(P\) rispetto ad una conica generale è una retta.}
\pf{Dimostrazione}{Sia \(P = [(x_1', x_2', x_3')]\) allora \(Q = [(x_1, x_2, x_3)]\) appartiene alla polare di \(P\) se, e soltanto se, \[
        (x_1', x_2', x_3') \  A \left( \; \begin{matrix} x_1\\ x_2\\ x_3 \end{matrix} \; \right) = \ul{0} \quad \text{poniamo} \quad (x_1', x_2', x_3') \ A = (a,b,c)
\] \[
(a, b, c) \left( \; \begin{matrix} x_1\\ x_2\\ x_3 \end{matrix} \; \right) = ax_1+bx_2+cx_3=0
\] che è l'equazione di una retta. A meno che \((a,b,c) = (0,0,0)\). Sia per assurdo \((a,b,c) = (0,0,0) \) ciò significa che \((x_1', x_2', x_3') \ A = (0,0,0) \) e questo avviene se, e soltanto se, \[
{^tA} \left( \; \begin{matrix} x'_1\\ x_2'\\ x'_3 \end{matrix} \; \right) = A \left( \; \begin{matrix} x'_1\\ x_2'\\ x'_3 \end{matrix} \; \right) = \ul{0} 
\] quindi \(\left( \; \begin{matrix} x_1'\\ x_2'\\ x_3' \end{matrix} \; \right)\) sono le coordinate di un punto doppio e di conseguenza \(P\) è un punto doppio di \(C\), ma per ipotesi \(C\) è generale. \textbf{Assurdo!} Quindi \((a,b,c) \neq (0,0,0) \implies ax_1 + bx_2+ cx_3 = 0\) è una retta. Essa è detta \textbf{retta polare} di \(P\) rispetto a \(C\).}

\dfn{Polarità}{Si dice \textbf{polarità} associata a una conica generale, la corrispondenza che associa a ogni punto, detto polo, la sua polare \[
\text{polo} \leftrightarrow \text{polare}
\] è facile dimostrare che questa relazione è una biiezione.}

\mprop{Principio di reciprocità}{
Sia \(C\) una conica generale di \(\tilde{A}_{2}(\CC) \), sia \(P \in \tilde{A}_{2}(\CC) \) e sia \(p\) la polare di \(P\), allora
\begin{enumerate}
    \item le polari dei punti di \(p\) passano per \(P\)
    \item i poli delle rette per \(P\) appartengono a \(p\)
\end{enumerate}
}
\pf{Dimostrazione}{Dimostriamo i due punti separatamente
\begin{enumerate}
    \item Sia \(Q \in p \implies Q, P\) sono coniugati \(\implies P \in q\), polare di \(Q\)
    \item Sia \(q\) una retta per \(P\). Il polo \(Q\) di \(q\) è coniugato a tutti i punti di \(q\) di conseguenza \(Q\) è coniugato a \(P\), quindi \(Q \in p\).
\end{enumerate}}

\mprop{}{Sia \(C\) una conica generale di \(\tilde{A}_{2}(\CC) \). Allora
\begin{enumerate}
    \item sia \(P \in C\), questo implica che la polare \(p\) di \(P\) è la retta tangente a \(C\) in \(P\)
    \item Sia \(P \notin C\), la polare di \(P\) è la congiungente dei due punti \(T_1\) e \(T_2\) ottenuti intersecando le tangenti \(t_1\) e \(t_2\) alla conica per \(P\).
\end{enumerate}}
\pf{Dimostrazione}{Dimostriamo i due punti separatamente
\begin{enumerate}
    \item Sia \(P\), di coordinate \(X_P = \left( \; \begin{matrix} x'_1\\ x_2'\\ x'_3 \end{matrix} \; \right)\), appartenente alla conica, allora la polare di \(P\) ha equazione \({^tX_P}A \left( \; \begin{matrix} x_1\\ x_2\\ x_3 \end{matrix} \; \right) = \ul{0} \) che è la formula della retta tangente a \(C\) in \(P\).
    \item \(T_1 \in C\) implica che la polare di \(T_1\) rispetto a \(C\) è \(t_1\). \(P \in t_1\) quindi \(P\) appartiene alla polare di \(T_1\). Perciò per il principio di reciprocità \(T_1\) appartiene alla polare di \(P\) e di conseguenza \(T_1 \in p\). Analogamente \(T_2 \in C\) significa che la polare di \(T_2 \in t_2\) e \(P \in t_2\) significa che \(T_2 \in p\). Quindi infine \(T_1, T_2 \in p \implies p\) è la congiungente di \(T_1\) e \(T_2\).
\end{enumerate}}

\paragraph{Osservazione:} Equivalentemente il punto 2 si può riscrivere nel seguente modo
 \mprop{}{Se \(P \notin C\) la sua polare \(p\) si ottiene congiungendo i punti \(T_1\) e \(T_2\) di tangenza delle tangenti per passanti \(P\).}

 \dfn{Centro e diametri di una conica}{Si dice \textbf{centro} di una conica generale di \(\tilde{A}_{2}(\CC) \) il polo della retta impropria. Si dicono \textbf{diametri} di una conica generale le rette polari dei punti impropri.}
\paragraph{Osservazione:} Per il principio di reciprocità i diametri passano per il centro della conica. Quindi sono il fascio proprio (se c'è proprio) di rette per \(C\).
\end{document}
