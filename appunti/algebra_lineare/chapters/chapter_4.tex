
\chapter{Autovalori, autovettori e diagonalizzabilità}
\section{Ricerca di autovalori, polinomio caratteristico}
\dfn{Polinomio ed equazione caratteristica}{Se \(A\) è una matrice quadrata di ordine \(n\), si dice \textbf{polinomio caratteristico} di \(A\), e si indica \(p_A(\lambda )\), il determinante della matrice \(A-\lambda I_n\), cioè \[
p_A(\lambda ) = |A-\lambda I_n| 
\] L'equazione \(p_A(\lambda) = |A -\lambda I_n| \) è detta \textbf{equazione caratteristica} di \(A\).}

\dfn{Autovalori}{Le radici del polinomio caratteristico si chiamano \textbf{autovalori} di \(A\).}
\dfn{Autospazio}{Lo spazio delle soluzioni del sistema \((A-\overline{\lambda }I_n)X=0\), dove \(\overline{\lambda }\) è un autovalore, si chiama \textbf{autospazio} associato a \(\overline{\lambda }\) e si indica con \(V_{\overline{\lambda }}\).}
\dfn{Autovettori}{I vettori non nulli dell'autospazio \(V_{\overline{\lambda }}\) si chiamano  \textbf{autovettori} relativi a \(\overline{\lambda }\).}
\paragraph{Osservazione:} Si potrebbe dimostrare che se il polinomio caratteristico di \(A \in M_n(K)\) ha grado \(n\) allora gli autovalori di \(A\) sono al massimo \(n\).
\dfn{Matrici simili}{Due matrici \(A,B \in M_n(K)\) si dicono \textbf{simili} se esiste \(P \in M_n(K)\) con \(|P| \neq 0\) tale che  \[
B = P^{-1}AP \quad PB = AP
\] }

\mprop{}{Due matrici simili \(A,B\) hanno lo stesso determinante e lo stesso polinomio caratteristico (e di conseguenza gli stessi autovalori).}
\pf{Dimostrazione}{Per ipotesi le due matrici \(A,B\) sono simili quindi:\[
\exists P \in M_n(K), \ |P| \neq 0 : \ B = P^{-1}AP
\] \[
|B| = |P^{-1}AP| = |P^{-1}| |A| |P| = \frac{1}{|P| }|A| |P| =|A| \implies |B| = |A| 
\] \[
p_B(\lambda ) = |B - \lambda I_n| = |P^{-1}AP - \lambda P^{-1}I_n P| = |P^{-1}(A - \lambda I_n)P| = \frac{1}{|P| }|A-\lambda I_n| |P| = |A - \lambda I_n| = p_A(\lambda )
\] e attraverso questa serie di passaggi abbiamo potuto dimostrare che se due matrici sono simili allora avranno sia lo stesso determinante che lo stesso polinomio caratteristico. }

\section{Matrici diagonalizzabili}
\dfn{Matrice diagonalizzabile}{Una matrice \(A \in M_n(K)\) si dice \textbf{diagonalizzabile} se è simile ad una matrice diagonale, ovvero esistono \(D, P \in M_n(K)\) con \(D\) matrice diagonale, \(|P| \neq 0\) e \(D = P^{-1}AP\).}

\thm{Primo criterio di diagonalizzabilità}{Una matrice \(A \in M_n(K)\) è diagonalizzabile se, e soltanto se, \(K^{n}\) ammette una base costituita da autovettori di \(A\).}
\pf{Dimostrazione}{\("\implies "\) Per ipotesi \(A\) è diagonalizzabile quindi \(\exists \ D,P \in M_n(K): D\) è diagonale \(|P| \neq 0\) e \(PD = AP\). Per semplicità denotiamo le colonne di \(P= 
\begin{pmatrix}
    P_1 & P_2 & \ldots  & P_n \\
\end{pmatrix}
\). \[
AP = A 
\begin{pmatrix}
    P_1 & P_2 & \ldots  & P_n \\
\end{pmatrix} =
\begin{pmatrix}
    AP_1 & AP_2 & \ldots  & AP_n \\
\end{pmatrix}
\] \[
PD = 
\begin{pmatrix}
    P_1 & P_2 & \ldots  & P_n \\
\end{pmatrix} 
\begin{pmatrix}
    d_1 & 0 & \ldots  & 0 \\
    0 & d_2 & \ldots  & 0 \\
    \vdots & \vdots & \ddots & \vdots \\
    0 & 0 & \ldots  & d_n \\
\end{pmatrix} = 
\begin{pmatrix}
    d_1P_1 & d_2P_2 & \ldots  & d_nP_n \\
\end{pmatrix}
\] Quindi \[
\begin{pmatrix}
    AP_1 & AP_2 & \ldots  & AP_n \\
\end{pmatrix} = 
\begin{pmatrix}
    d_1P_1 & d_2P_2 & \ldots  & d_nP_n \\
\end{pmatrix}
\iff 
AP_1=d_1P_1, \ AP_2 = d_2P_2, \ \ldots, \ AP_n = d_n P_n
\] \[
\implies AX = \lambda X \quad \lambda =d_i \quad X = P_i
\] dove \(d_i\) è un autovalore, \(P_i\) è un autovettore di \(A\) e \(
\begin{pmatrix}
    P_1 & P_2 & \ldots  & P_n \\
\end{pmatrix} 
\) è una sequenza di \(n\) autovettori. Poiché \(\dim K^{n}=n\) e la sequenza è composta da \(n\) vettori, è sufficiente controllare la lineare indipendenza di \(P\). Ma siccome avevamo supposto per ipotesi che \(|P| \neq 0\) le sue \(n\) colonne sono linearmente indipendenti. Quindi \(B = (P_1, P_2, \ldots , P_n)\) è una base di \(K^{n}\) costituita da autovettori di \(A\).

"\(\impliedby \)" è analogo, basta ripercorrere il ragionamento a ritroso.}

\paragraph{Osservazione:} Se \(A \in M_n(K)\) è diagonalizzabile allora:
\begin{itemize}
    \item \(D\) ha sulla diagonale principale gli autovalori di \(A\);
    \item \(P\), cioè la matrice diagonalizzante, ha nelle colonne gli autovettori della base di \(K^{n}\).
\end{itemize}

\dfn{Molteplicità algebrica e geometrica}{Sia \(\overline{\lambda }\) un autovalore di \(A \in M_n(K)\); si chiama:
\begin{itemize}
    \item \textbf{molteplicità algebrica} di \(\overline{\lambda }\) il numero di volte che \(\overline{\lambda }\) è radice del polinomio caratteristico, e si indica con \(a_{\overline{\lambda }}\) 
    \item \textbf{molteplicità geometrica} di \(\overline{\lambda }\) la dimensione dell'autospazio \(V_{\overline{\lambda }}\) associato a \(\overline{\lambda }\), e si indica con \(g_{\overline{\lambda }}\).
\end{itemize}}

\mprop{}{Sia \(\overline{\lambda }\) un autovalore di \(A \in M_n(K)\). Allora \[
1 \le g_{\overline{\lambda }} \le a_{\overline{\lambda }}
\] }

\mprop{}{Sia \(A \in M_n(K)\) e siano \(\lambda _1, \lambda _2, \ldots ,\lambda_n\) \(t\) autovalori di \(A\) distinti tra loro, allora la somma dei relativi autospazi è diretta. \[
V_{\lambda _1} \oplus V_{\lambda _2} \oplus \ldots \oplus V_{\lambda _t}
\] }
\paragraph{Osservazioni:} 
\begin{enumerate}
    \item \(A \in M_n(K) \implies \deg(p_A(\lambda )) = n\), quindi ho al massimo \(n\) autovalori;
    \item \(\sum a_{\lambda _i}\le n\);
    \item \(\sum a_{\lambda _i}= n \iff\) tutti gli autovalori di \(A\) sono in \(K\);
    \item \(S =V_{\lambda _1} \oplus V_{\lambda _2} \oplus \ldots \oplus V_{\lambda _t} \implies \dim S = \sum \dim V_{\lambda _i} = \sum g_{\lambda _i}\)
    \item Autovettori provenienti da autospazi diversi sono tra loro linearmente indipendenti (perché la somma è diretta).
\end{enumerate}

\thm{Secondo criterio di diagonalizzabilità}{Sia \(A \in M_n(K)\) e siano \(\lambda _1, \lambda _2, \ldots , \lambda _n\) gli autovalori distinti di \(A\). Allora \(A\) è diagonalizzabile se, e soltanto se:
\begin{enumerate}
    \item tutti gli autovalori di \(A\) sono in \(K\);
    \item Per ogni autovalore vale \(a_{\lambda _i} = g_{\lambda _i}\)(e allora si dice che l'autovalore è regolare).
\end{enumerate}}

\pf{Dimostrazione}{"\(\implies\)" Per ipotesi \(A\) è diagonalizzabile. Per il primo criterio di diagonalizzabilità \(K^{n}\) ammette una base \(B\) formata da autovettori, cioè tale che  \(\mcL(B) = K^{n}\) e \(B \subseteq V_{\lambda _1} \oplus V_{\lambda _2} \oplus \ldots \oplus V_{\lambda _t} \le K^{n}\). Quindi \[
K^{n} = \mcL(B) \le \mcL(V_{\lambda _1} \oplus V_{\lambda _2} \oplus \ldots \oplus V_{\lambda _t}) = V_{\lambda _1} \oplus V_{\lambda _2} \oplus \ldots \oplus V_{\lambda _t} \le K^{n} 
\] \[
\implies V_{\lambda _1} \oplus V_{\lambda _2} \oplus \ldots \oplus V_{\lambda _t} = K^{n} \]
\[
\implies n = \dim K^{n} = \dim (V_{\lambda _1} \oplus V_{\lambda _2} \oplus \ldots \oplus V_{\lambda _t}) = \sum g_{\lambda _i}\le \sum a_{\lambda _i} \le n
\]
Siccome  \(\sum a_{\lambda _i}=n\) tutti gli autovalori di \(A\) sono in \(K\). Inoltre \(\sum g_{\lambda _i} = \sum a_{\lambda _i}\) e \(g_{\lambda _i}\le a_{\lambda _i}\implies a_{\lambda _i}=g_{\lambda _i}\).

"\(\impliedby \)" Per ipotesi abbiamo che tutti gli autovalori di \(A\) soni in \(K\) e per ogni autovalore vale \(a_{\lambda _i}= g_{\lambda _i}\). Per ogni autovalore \(\overline{\lambda }\) avremo un relativo autospazio a cui corrisponde una relativa base di autovettori  \(B_1, B_2, \ldots , B_t\). Chiamiamo \(B = \bigcup_{i = 1}^t B_i \), cioè l'unione di tutte le basi. Certamente \(B\) è libera perché la somma di sottospazi distinti è diretta. \[
|B| = |\bigcup B_i | = \sum |B_i| = \sum \dim V_{\lambda _i}= \sum g_{\lambda _i}= \sum a_{\lambda _i}=n
\] Quindi \(B\) è una base di \(K^{n}\) costituita da autovettori e per il primo criterio di diagonalizzabilità \(A\) è diagonalizable.}
