\chapter{Coniche}
\section{Coniche in \(\tilde{A}_{2}(\CC)\)}
\dfn{Conica}{Si dice \textbf{conica} una curva algebrica reale di \(\tilde{A}_{2}(\CC)\) (curva piana) del secondo ordine. Una conica si rappresenta eguagliando a \(0\) un polinomio omogeneo \(F\) di secondo grado nelle variabili  \(x_1, x_2, x_3\), a coefficienti reali. La generica equazione della conica è \[
C: a_{11}x_1^2+ 2a_{12}x_1x_2+2a_{13}x_1x_3+a_{22}x_2^2+2a_{23}x_2x_3+a_{33}x_3^2=0
\] Se chiamiamo \[
X = \left( \; \begin{matrix} x_1\\ x_2 \\ x_3 \end{matrix} \; \right) \quad A =
\left( \; \begin{matrix}
    a_{11} & a_{12} & a_{13} \\
    a_{12} & a_{22} & a_{23} \\
    a_{13} & a_{23} & a_{33} \\
\end{matrix} \; \right)
\] Possiamo riscrivere l'equazione come prodotto righe per colonne \[
C : \ {^tX}AX = \ul{0} 
\] \(A\) è una matrice reale e simmetrica ed è detta \textbf{matrice della conica}.} 

\ex{}{Consideriamo la conica \[
-x_1^2+ax_1x_2+5x_2^2-3x_2x_3+6x_3^2=0
\] \[
A = 
\left( \; \begin{matrix}
    -1 & 2 & 0 \\
    2 & 5 & -\frac{3}{2} \\
    0 & -\frac{3}{2} & 6 \\
\end{matrix} \; \right)
\] Ora facciamo il prodotto  \[
\left( \; \begin{matrix}
    x_1 & x_2 & x_3 \\
\end{matrix} \; \right) \cdot 
\left( \; \begin{matrix}
    -1 & 2 & 0 \\
    2 & 5 & -\frac{3}{2} \\
    0 & -\frac{3}{2} & 6 \\
\end{matrix} \; \right) \cdot 
\left( \; \begin{matrix} x_1\\ x_2\\ x_3 \end{matrix} \; \right) = 0
\] \[
\left( \; \begin{matrix}
    -x_1+2x_2 & 2x_1+5x_2-\frac{3}{2}x_3 & -\frac{3}{2}x_2+6x_3 \\
\end{matrix} \; \right) \cdot \left( \; \begin{matrix} x_1\\ x_2\\ x_3 \end{matrix} \; \right) = 0
\] \[
x_1(-x_1+2x_2) + x_2\left(2x_1+5x_2- \frac{3}{2}x_3\right) + x_3\left(-\frac{3}{2}x_2 + 6x_3\right) = 0
\] \[
-x_1^2+4x_1x_2+5x_2^2-3x_2x_3+6x_3^2=0
\] che è uguale all'equazione di partenza.}

\paragraph{Osservazione:} L'equazione della generica conica in \(\tilde{A}_{2}(\CC)\) dipende da 6 coefficienti definiti a meno di un fattore di proporzionalità. Quindi le coniche di \(\tilde{A}_{2}(\CC)\) sono \(\infty^{5}\).

\mprop{}{Sia \(C\) una conica di \(\tilde{A}_{2}(\CC)\) riducibile. Allora \(C\) è unione di 2 rette che possono essere reali e distinte, reali e coincidenti oppure immaginarie e coniugate.}

\pf{Dimostrazione}{Sia \(C\) la conica associata al polinomio \(F=(x_1,x_2, x_3) = 0\). Se \(C\) è riducibile \(F=(x_1,x_2, x_3) = F_1=(x_1,x_2, x_3) \cdot F_2=(x_1,x_2, x_3)\) dove \(F_1\) e \(F_2\) hanno grado 1, quindi rappresentano delle rette e di conseguenza \(C\) è unione di due rette \(r_1\) e \(r_2\). Se \(r_1\) e \(r_2\) sono entrambe reali siamo nei casi \(1\) o \(2\). Se invece \(r_1\) è immaginaria allora \(\overline{r_1}\) è ancora componente di \(C\) (per ogni \(P \in r_1, \ \overline{P}\in C\)), ma \(r_1 \neq \overline{r_1} \implies \overline{r_1}=r_2\implies C\) si riduce in due rette immaginarie e coniugate.}

\subsubsection{Punti multipli di una conica}
\mprop{}{In \(\tilde{A}_{2}(\CC) \) una conica
\begin{enumerate}
    \item non ha punti tripli
    \item ha un punto doppio se, e soltanto se, è riducibile. E abbiamo due possibilità
        \begin{enumerate}
            \item ha solo un punto doppio \(P\) e si riduce in due rette distinte per \(P\)
            \item ha almeno due punti doppi allora ne ha \(\infty^{1}\) e si fattorizza in una retta reale contata due volte
        \end{enumerate}
\end{enumerate}}

\pf{Dimostrazione}{Dimostriamo il secondo punto \\ "\(\implies \)" Per ipotesi \(C\) ha un punto doppio \(P\). Sia \(R \in C\) e consideriamo la retta \(r = rt(P,R)\), se non fosse componente avrebbe \[
|r \cap C| \ge 2 + 1 = 3 \quad \text{intersezioni con \(C\)}
\] \textbf{Assurdo!} Questo è in contraddizione con il teorema dell'ordine. \\
"\(\impliedby \)" Sia \(C\) per ipotesi riducibile. Allora  \(C = r_1 \cup r_2\). Sia \(P \in r_1 \cap  r_2\) e sia \(r\) una retta per \(P\) diversa da \(r_1\) e da \(r_2\). Quindi \(r \cap C = P\). Per il teorema dell'ordine \(P\) ha molteplicità doppia e abbiamo due casi possibili
\begin{enumerate}
    \item se \(r_1=r_2\) abbiamo \(\infty^{1}\) punti doppi e \(C = r_1 \cup  r_1\)
    \item altrimenti abbiamo un \textbf{solo} \(P\) punto doppio che è \(r_1\cap r_2\)
\end{enumerate}
Dobbiamo dimostrare che esiste un solo punto doppio. Siano per assurdo \(P_1\) e \(P_2\) punti doppi distinti e sia \(C=r_1 \cup r_2\) con \(r_1 \neq r_2\). Sia \(Q \in r_2\) e \(P_2 \in r_1\), allora \[
|rt(P_2, Q) \cap C| \ge \underbrace{2}_{P_2} + \underbrace{1}_{Q} 
\] Per il teorema dell'ordine \(rt(P_2, Q)\) è componente. \textbf{Assurdo!} Perché avremmo 3 componenti \((r_1, r_2, rt(P_2, Q))\).} 

\dfn{Coniche generali o degeneri}{Una conica si dice
 \begin{itemize}
    \item \textbf{generale}, se è priva di punti doppi \(\implies \) quindi non è riducibile
    \item \textbf{semplicemente degenere} se ha un solo punto doppio \(\implies C = r_1\cup r_2\) con \(r_1\neq r_2\) 
    \item \textbf{doppiamente degenere} se ha \(\infty^{1}\) punti doppi \(\implies C = r \cup r\)
\end{itemize}}

\thm{}{In \(\tilde{A}_{2}(\CC) \) i punti doppi di una conica \(C\) si trovano considerando le classi di autosoluzioni del sistema omogeneo \[
AX = \ul{0}
\] dove \(A\) è la matrice associata a \(C\).}

\pf{Dimostrazione}{\[
C: F(x_1, x_2, x_3) = 0 \quad \text{dove \(F\) è:} 
\]
\[
a_{11}x_1^2+ 2a_{12}x_1x_2+2a_{13}x_1x_3+a_{22}x_2^2+2a_{23}x_2x_3+a_{33}x_3^2=0
\] i punti doppi si trovano risolvendo 
\[
\begin{cases}
    \ \frac{\partial F}{\partial x_1}= 2a_{11}x_1+2a_{12}x_2+2a_{13}x_3=0\\
    \ \frac{\partial F}{\partial x_2}= 2a_{12}x_1+2a_{22}x_2+2a_{23}x_3=0\\
    \ \frac{\partial F}{\partial x_3} = 2a_{13}x_1+2a_{23}x_2+2a_{33}x_3=0\\
\end{cases}
\] Possiamo dividere tutti i fattori per 2 \[
\left( \; \begin{matrix}
    a_{11} & a_{12} & a_{13} \\
    a_{12} & a_{22} & a_{23} \\
    a_{13} & a_{23} & a_{33} \\
\end{matrix} \; \right) \cdot \left( \; \begin{matrix} x_1\\ x_2\\ x_3 \end{matrix} \; \right) = \left( \; \begin{matrix} 0\\ 0\\ 0\\ \end{matrix} \; \right)
\] \[
\implies AX = \ul{0} 
\]  
}

\thm{}{In \(\tilde{A}_{2}(\CC) \) una conica \(C : {^tX}A X = \ul{0} \) risulta
\begin{enumerate}
    \item generale se, e soltanto se, \(\rho(A) = 3\) 
    \item semplicemente degenere se, e soltanto se, \(\rho(A) = 2\) 
    \item doppiamente degenere se, e soltanto se, \(\rho(A) = 1\)
\end{enumerate}}

\pf{Dimostrazione}{  
Dimostriamo tutti i casi singolarmente:
\begin{enumerate}
    \item \(C\) è generale se, e soltanto se, non ha punti doppi. Se \(AX = \ul{0} \) ha solo la soluzione nulla \( \iff \rho(A) =3\).
    \item \(C\) è semplicemente degenere se ha un solo punto doppio. \( \iff AX = \ul{0} \) ha \(\infty^{1}\) soluzioni \(\iff \rho(A) = 2\)
    \item \(C\) è doppiamente degenere se ha \(\infty^{1}\) punti doppi \(\iff AX = \ul{0} \) ha \(\infty^{2}\) soluzioni (se \([(x_1, x_2, x_3)]\) è soluzione \([(2x_1, 2x_2, 2x_3)]\) è lo stesso punto doppio) \(\iff \rho(A) =1\)
\end{enumerate}
}

\subsubsection{Classificazione affine di una conica generale}
Sia \(C\) una conica di \(\tilde{A}_{2}(\CC) \) e \(r\) una qualsiasi retta, osserviamo che \(r \cap C\) può essere
\begin{enumerate}
    \item due punti reali e distinti
    \item un punto reale con molteplicità doppia
    \item due punti immaginari e coniugati
\end{enumerate}
Se consideriamo come retta la \(r_{\infty}\) questa serie di casistiche ci dà la classificazione affine delle coniche generali.

\dfn{Ellisse, iperbole e parabola}{Sia \(C\) una conica generale di \(\tilde{A}_{2}(\CC) \). Allora \(C \cap r_{\infty}\) è data da due punti \(P, Q\) (non necessariamente distinti) e \(C\) si dice:
\begin{enumerate}
    \item \textbf{ellisse}, se \(P\) e \(Q\) sono immaginari e coniugati
    \item \textbf{iperbole}, se \(P\) e \(Q\) sono reali e distinti
    \item \textbf{parabola}, se \(P\) e \(Q\) sono reali e coincidenti
\end{enumerate}}

\subsubsection{Condizioni analitiche}
Sia \(C\) una conica generale di equazione \[
a_{11}x_1^2+ 2a_{12}x_1x_2+2a_{13}x_1x_3+a_{22}x_2^2+2a_{23}x_2x_3+a_{33}x_3^2=0
\] 
La \(r_{\infty}\) ha equazione \(x_{3}=0\) \[
\begin{cases}
    \ a_{11}x_1^2+2a_{12}x_1x_2+a_{22}x_2^2= 0 = C \cap r_{\infty} \\
    \ x_3 = 0 \\
\end{cases}
\]
Almeno uno fra \(x_1, x_2 \neq 0\). Supponiamo \(x_2 \neq 0\) e dividiamo per \(x_2^2\) \[
a_{11} \left( \frac{x_1}{x_2} \right) ^2 + 2a_{12} \left( \frac{x_1}{x_2} \right)  + a_{22} = 0
\] 
La risolviamo in \(\left( \frac{x_1}{x_2} \right) \). Se 
\begin{enumerate}
    \item \(\frac{\Delta}{4} > 0\) abbiamo due soluzioni reali e distinte \(\implies \) \textbf{iperbole};
    \item \(\frac{\Delta}{4} = 0\) abbiamo due soluzioni coincidenti \(\implies \) \textbf{parabola};
    \item \(\frac{\Delta}{4} < 0\) abbiamo due soluzioni immaginarie e coniugate \(\implies \) \textbf{ellisse}.
\end{enumerate}

\[
\frac{\Delta}{4} = \left( \frac{b}{2} \right) ^2 - ac = \left( \frac{2a_{12}}{2} \right) ^2 - a_{11} a_{22} = a_{12}^2 - a_{11} a_{22}
\]Per semplificare le cose, data la matrice della conica \[
A = 
\left( \; \begin{matrix}
    a_{11} & a_{12} & a_{13} \\
    a_{12} & a_{22} & a_{23} \\
    a_{13} & a_{23} & a_{33} \\
\end{matrix} \; \right)
\quad \text{poniamo} \quad A^* =
\left( \; \begin{matrix}
    a_{11} & a_{12} \\
    a_{12} & a_{22} \\
\end{matrix} \; \right) \]
Per classificare la conica basta studiare il determinante di \(A^{*}\)
\[
|A^{*}| = a_{11}a_{22}-a_{12}^2= - \frac{\Delta}{4}
\] Se \(C\) è una conica generale \((|A| = 0)\) allora si applicano le casistiche precedentemente elencate.

\section{Polarità associata a una conica}

\dfn{Coniugato rispetto ad una conica}{Data una conica \(C: {^tX}AX = 0\) e dati due punti di \(\tilde{A}_{2}(\CC) \) \[
        P' = [(x_1', x_2', x_3')] \quad e \quad P '' = [(x_1 '', x_2 '', x_3 '')]
\] si dice che \(P'\) è coniugato a \( P ''\) rispetto a \(C\) se \[
{^tX'AX '' = 0 \quad con \quad X' = \left( \; \begin{matrix} x'_1\\ x_2'\\ x'_3 \end{matrix} \; \right)} \quad  e \quad X '' = \left( \; \begin{matrix} x''_1\\ x_2 ''\\ x''_3 \end{matrix} \; \right)
\]}

\paragraph{Osservazione:} Sia \(P'\) coniugato a \(P ''\), ovvero \[
    {^tX'} A X '' = 0 \implies {^t({^tX'}AX '')} = 0 = {^tX ''} {^tA}{^t({^tX'})} = {^tX ''} A X' = 0 \implies P '' \text{ è coniugato a } P' 
\] Quindi la relazione di coniugio è simmetrica, perciò potremo dire semplicemente che \(P'\) e \(P ''\) sono coniugati.

\dfn{Polare}{Sia \(C\) una conica e \(P'\) un punto di \(\tilde{A}_{2}(\CC) \). Si dice \textbf{polare} di \(P'\) rispetto a \(C\), il luogo dei coniugati di \(P'\) rispetto a \(C\). Il punto \(P'\) prende il nome di \textbf{polo} di tale luogo.}

\mprop{}{In \(\tilde{A}_{2}(\CC) \) la polare di un punto \(P\) rispetto ad una conica generale è una retta.}
\pf{Dimostrazione}{Sia \(P = [(x_1', x_2', x_3')]\) allora \(Q = [(x_1, x_2, x_3)]\) appartiene alla polare di \(P\) se, e soltanto se, \[
        (x_1', x_2', x_3') \  A \left( \; \begin{matrix} x_1\\ x_2\\ x_3 \end{matrix} \; \right) = \ul{0} \quad \text{poniamo} \quad (x_1', x_2', x_3') \ A = (a,b,c)
\] \[
(a, b, c) \left( \; \begin{matrix} x_1\\ x_2\\ x_3 \end{matrix} \; \right) = ax_1+bx_2+cx_3=0
\] che è l'equazione di una retta. A meno che \((a,b,c) = (0,0,0)\). Sia per assurdo \((a,b,c) = (0,0,0) \) ciò significa che \((x_1', x_2', x_3') \ A = (0,0,0) \) e questo avviene se, e soltanto se, \[
{^tA} \left( \; \begin{matrix} x'_1\\ x_2'\\ x'_3 \end{matrix} \; \right) = A \left( \; \begin{matrix} x'_1\\ x_2'\\ x'_3 \end{matrix} \; \right) = \ul{0} 
\] quindi \(\left( \; \begin{matrix} x_1'\\ x_2'\\ x_3' \end{matrix} \; \right)\) sono le coordinate di un punto doppio e di conseguenza \(P\) è un punto doppio di \(C\), ma per ipotesi \(C\) è generale. \textbf{Assurdo!} Quindi \((a,b,c) \neq (0,0,0) \implies ax_1 + bx_2+ cx_3 = 0\) è una retta. Essa è detta \textbf{retta polare} di \(P\) rispetto a \(C\).}

\dfn{Polarità}{Si dice \textbf{polarità} associata a una conica generale, la corrispondenza che associa a ogni punto, detto polo, la sua polare \[
\text{polo} \leftrightarrow \text{polare}
\] è facile dimostrare che questa relazione è una biiezione.}

\mprop{Principio di reciprocità}{
Sia \(C\) una conica generale di \(\tilde{A}_{2}(\CC) \), sia \(P \in \tilde{A}_{2}(\CC) \) e sia \(p\) la polare di \(P\), allora
\begin{enumerate}
    \item le polari dei punti di \(p\) passano per \(P\)
    \item i poli delle rette per \(P\) appartengono a \(p\)
\end{enumerate}
}
\pf{Dimostrazione}{Dimostriamo i due punti separatamente
\begin{enumerate}
    \item Sia \(Q \in p \implies Q, P\) sono coniugati \(\implies P \in q\), polare di \(Q\)
    \item Sia \(q\) una retta per \(P\). Il polo \(Q\) di \(q\) è coniugato a tutti i punti di \(q\) di conseguenza \(Q\) è coniugato a \(P\), quindi \(Q \in p\).
\end{enumerate}}

\mprop{}{Sia \(C\) una conica generale di \(\tilde{A}_{2}(\CC) \). Allora
\begin{enumerate}
    \item sia \(P \in C\), questo implica che la polare \(p\) di \(P\) è la retta tangente a \(C\) in \(P\)
    \item Sia \(P \notin C\), la polare di \(P\) è la congiungente dei due punti \(T_1\) e \(T_2\) ottenuti intersecando le tangenti \(t_1\) e \(t_2\) alla conica per \(P\).
\end{enumerate}}
\pf{Dimostrazione}{Dimostriamo i due punti separatamente
\begin{enumerate}
    \item Sia \(P\), di coordinate \(X_P = \left( \; \begin{matrix} x'_1\\ x_2'\\ x'_3 \end{matrix} \; \right)\), appartenente alla conica, allora la polare di \(P\) ha equazione \({^tX_P}A \left( \; \begin{matrix} x_1\\ x_2\\ x_3 \end{matrix} \; \right) = \ul{0} \) che è la formula della retta tangente a \(C\) in \(P\).
    \item \(T_1 \in C\) implica che la polare di \(T_1\) rispetto a \(C\) è \(t_1\). \(P \in t_1\) quindi \(P\) appartiene alla polare di \(T_1\). Perciò per il principio di reciprocità \(T_1\) appartiene alla polare di \(P\) e di conseguenza \(T_1 \in p\). Analogamente \(T_2 \in C\) significa che la polare di \(T_2 \in t_2\) e \(P \in t_2\) significa che \(T_2 \in p\). Quindi infine \(T_1, T_2 \in p \implies p\) è la congiungente di \(T_1\) e \(T_2\).
\end{enumerate}}

\paragraph{Osservazione:} Equivalentemente il punto 2 si può riscrivere nel seguente modo
 \mprop{}{Se \(P \notin C\) la sua polare \(p\) si ottiene congiungendo i punti \(T_1\) e \(T_2\) di tangenza delle tangenti per passanti \(P\).}

 \dfn{Centro e diametri di una conica}{Si dice \textbf{centro} di una conica generale di \(\tilde{A}_{2}(\CC) \) il polo della retta impropria. Si dicono \textbf{diametri} di una conica generale le rette polari dei punti impropri.}
\paragraph{Osservazione:} Per il principio di reciprocità i diametri passano per il centro della conica. Quindi sono il fascio proprio (se c'è proprio) di rette per \(C\).

Per determinare le coordinate del centro dobbiamo scegliere due punti \(X_{\infty} = [(1, 0, 0)]\), punto improprio dell'asse \(x\), e \(Y_{\infty} = [(0,1,0)]\), punto improprio dell'asse \(y\). La polare di \(X_{\infty}\) è \[
    (1,0,0)
\left( \; \begin{matrix}
    a_{11} & a_{12} & a_{13} \\
    a_{12} & a_{22} & a_{23} \\
    a_{13} & a_{23} & a_{33} \\
\end{matrix} \; \right)
\left( \; \begin{matrix} x_1 \\ x_2\\ x_3 \end{matrix} \; \right) = 0 \qquad (a_{11}, a_{12}, a_{13}) \left( \; \begin{matrix} x_1 \\ x_2\\ x_3 \end{matrix} \; \right) = a_{11}x_1+a_{12}x_2 + a_{13} x_3 = 0
\] Analogamente la polare di \(Y_{\infty}\) è  \[
a_{12}x_1 + a_{22}x_2 + a_{23}x_3 = 0
\] \[
\begin{cases}
    \ a_{11}x_1+a_{12}x_2+a_{13}x_3 = 0 \qquad P_1 \\
    \ a_{12}x_1+a_{22}x_2+a_{23}x_3=0 \qquad P_2 \\
\end{cases}
\] 
Il centro \(C\) è proprio se \(P_1\) e \(P_2\) non sono paralleli. Se \[
\left| \; \begin{matrix}
    a_{11} & a_{12} \\
    a_{12} & a_{22} \\
\end{matrix} \; \right| = |A^{*}| \neq 0
\] il centro è un punto proprio. Ma il centro è un punto proprio se \(C\) è un ellisse o un'iperbole. Quindi in questo caso i diametri sono un fascio proprio di rette di centro \(C\). \[
F: \ \lambda (a_{11}x_1+a_{12}x_2+a_{13}x_3) + \mu (a_{12}x_1 + a_{22}x_2 + a_{23}x_3) = 0
\] \paragraph{Equazione del fascio dei diametri} Se \(C\) è una parabola \(\implies |A^{*}| = 0 \implies P_1 \) parallelo a \(P_2 \implies  \) il centro è un punto improprio. \(\implies \) i diametri formano un fascio improprio di equazione \[
a_{11}x_1+a_{12}x_2 + kx_3 = 0 \quad \text{con} \quad k \in \CC
\] fascio improprio dei diametri della parabola.

\subsubsection{Asintoti di una conica}
\dfn{Asintoti}{Si dicono \textbf{asintoti} di una conica le rette proprie tangenti alla conica nei suoi punti impropri.}
\paragraph{Osservazione:} Gli asintoti di una conica sono quindi le rette polari nei suoi punti impropri. Gli asintoti sono quindi dei diametri e passano per il centro. Se il centro è proprio (cioè se \(C\) è un'ellisse o un'iperbole) gli asintoti sono le rette che congiungono il centro con i punti impropri di \(C\).

\mprop{}{La parabola è una conica con centro improprio e priva di asintoti.}
\pf{Dimostrazione}{Sia \(C\) una parabola \(\implies C\) è tangente alla retta impropria in un punto che chiamiamo \(P_{\infty}\). Quindi la retta polare di \(P_{\infty}\) è \(r_{\infty} \implies \) il polo della \(r_{\infty}\) è \(P_{\infty} \implies \) il punto \(P_\infty\) è il centro della parabola. Osserviamo che \(C\) ha solo un punto improprio \(P_\infty \implies \) ammette solo una tangente nel suo punto improprio. Ma \(t\) è la \(r_\infty \implies \) la \(r_{\infty}\) non è un asintoto.}

\dfn{Coniche a centro}{Diremo che l'iperbole e l'ellisse sono coniche \textbf{a centro}, mentre la parabola è detta conica \textbf{non a centro}.}

\section{Proprietà metriche di una conica}
\dfn{Iperbole equilatera}{Un'iperbole si dice \textbf{equilatera} se i suoi asintoti sono ortogonali.}

\mprop{}{Una conica generale è un'iperbole equilatera se, e soltanto se, \[a_{11} + a_{22} = 0\]}

\ex{}{Si stabiliscano i valori di \(k \in \RR\) tali che \[
C: \ 2kx^2 + 2 (k-2) xy - 4 y ^2 + 2x + 1 = 0
\] sia un'iperbole equilatera. \\ 
\begin{enumerate}
    \item \(2k = -(-4) \to k = 2\)
    \item Sostituiamo dentro all'equazione e scriviamola in forma omogenea \[ 4x_1 ^2 + 0 x_1 x_2 - 4 x_2 ^2 + 2 x_1 x_3 + x_3 ^2 = 0 \quad 
        A = 
\left| \; \begin{matrix}
    4 & 0 & 1 \\
    0 & -4 & 0 \\
    1 & 0 & 1 \\
\end{matrix} \; \right| \neq 0
\] \(k = 2\) dà luogo ad un'iperbole equilatera.
\end{enumerate}}

\dfn{Ortogonale al punto improprio}{Diremo che la retta \(p\) di parametri direttori \([(l', m')]\) è \textbf{ortogonale al punto improprio} \(P: [(l,m,0)]\) se \[ll' + mm' = 0\]}

\dfn{Asse di una conica}{Si dice \textbf{asse}, di una conica generale, ogni diametro ortogonale al proprio polo.}

\dfn{Vertici}{Si dicono \textbf{vertici} le intersezioni proprie della conica con i propri assi.}

\subsubsection{Condizioni analitiche}
\mprop{}{Gli assi di una conica a centro (ellisse o iperbole) sono due e sono ortogonali tra loro, a meno che non si tratti di una circonferenza generalizzata, in tal caso tutti i diametri sono assi.}

\pf{Dimostrazione}{Per definizione i diametri sono le polari dei punti impropri. Dato \(P_\infty : [(l,m,0)]\) \[
\left( \; \begin{matrix}
    l & m & 0 \\
\end{matrix} \; \right)
\left( \; \begin{matrix}
    a_{11} & a_{12} & a_{13} \\
    a_{12} & a_{22} & a_{23} \\
    a_{13} & a_{23} & a_{33} \\
\end{matrix} \; \right)
\left( \; \begin{matrix} x_1 \\ x_2\\ x_3 \end{matrix} \; \right) = 0
\] Il generico diametro è: \[
\left( \; \begin{matrix}
    la_{11} + m a_{12} & l a_{12} + m a_{22} & l a_{13} + m a_{23} \\
\end{matrix} \; \right)
\left( \; \begin{matrix} x_1 \\ x_2\\ x_3 \end{matrix} \; \right) = 0
\] \[
(la_{11} + m a_{12}) x_1 + (la_{12} + ma_{22}) x_2 + (la_{13} + ma_{23}) x_3 = 0
\] I parametri direttori del diametro \(d\) sono\[
p.d.d : \ [(-la_{12} - m a_{22}, l a_{11} + ma_{12})]
\] Il polo di \(d\) è \(P_\infty: [(l,m,0)]\). Inoltre \(d\) è un asse se è ortogonale a \(P_\infty\), ovvero se \[
l ( -la_{12}-ma_{22} ) + m(la_{11} + ma_{12}) = 0
\] \[
-l^2 a_{12} + ml (-a_{22}+a_{11}) + m^2 a_{12} = 0 \iff l^2a_{12} + ml(a_{22}- a_{11}) - m^2 a_{12} = 0 
\] moltiplicando tutti i fattori per \(\left( \frac{1}{m^2} \right) \) otteniamo\[
a_{12} \left( \frac{l}{m} \right) ^2 + \frac{l}{m} (a_{22} - a_{11}) - a_{12} = 0
\] Sia \(a_{12}\neq 0\), abbiamo un'equazione di secondo grado che ammette due soluzioni \[
[(l', m')] \quad e\quad [(l '', m '')]
\]consideriamo \([(l', m')]\). La direzione ortogonale ad essa è \([(-m', l')]\). Sostituendo la direzione ortogonale nell'equazione e otteniamo \[
a_{12}\left( \frac{m}{l} \right) ^2 - \frac{m}{l}(a_{22}-a_{11}) - a_{12} = 0
\] moltiplico tutti i fattori per \(l^2\) e ottengo l'equazione  \[
m^2a_{12} - ml (a_{22}-a_{11}) - l^2 a_{12} = 0 \iff m(ma_{12} + la_{11}) - l(ma_{22} - la_{12})
\] che è nuovamente la direzione ortogonale. Quindi gli assi sono a due a due ortogonali fra di loro.}

\mprop{}{La parabola ha un unico asse e un solo vertice \(v\). Inoltre la tangente alla parabola in \(v\) è ortogonale all'asse.}
\pf{Dimostrazione}{Il punto \(P_\infty\) di una parabola è \([(-a_{12}, a_{11}, 0)]\). I \(p.d.d = [(-a_{12}, a_{11})]\). La direzione ortogonale è data da \([(a_{11}, a_{12})]\), quindi il punto \(P_\infty\) è \([(a_{11}, a_{12}, 0)]\). Da cui segue che l'asse è unico ed è la polare di \((a_{11}, a_{12}, 0)\). Sostituendo nell'equazione del fascio improprio dei diametri abbiamo che l'asse ha equazione: \[
a_{11}(a_{11}x_1+ a_{12}x_2 + a_{13}x_3) + a_{12}(a_{12}x_1 + a_{22}x_2 + a_{23}x_3) = 0
\] Per il teorema dell'ordine \(a\) interseca la parabola \(C\) in due punti, ma uno è \(P_\infty\) quindi l'altro punto sarà l'unico vertice della parabola. \\ Ora dimostriamo la seconda parte del teorema. \(v \in a\) che è il polo di \(t\). Per il principio di reciprocità \(t\) contiene il polo di \(a\), ovvero \(P_{\infty} \in t\). Ma \(P_\infty\) è ortogonale ad \(a \implies t \perp a\).}
