\documentclass[twoside]{report}

\input{preamble}
\input{macros}
\input{letterfonts}

\title{\Huge{Analisi Matematica 1}\\Ingegneria dell'Automazione Industriale}
\author{\huge{Ayman Marpicati}}
\date{A.A. 2022/2023}
\setlength\parindent{0pt}

\begin{document}


\maketitle
\cleardoublepage
% \pdfbookmark[<level>]{<title>}{<dest>}
\pdfbookmark[section]{\contentsname}{toc}
\tableofcontents
\null\newpage

\setlength{\headheight}{15pt}

\pagestyle{fancy}
%... then configure it.
\fancyhead{} % clear all header fields
\fancyhead[LO]{\rightmark}
\fancyhead[RO]{\thepage}
\fancyhead[RE]{\leftmark}
\fancyhead[LE]{\thepage}
\fancyfoot{} % clear all footer fields

\chapter{Derivate}
\dfn{Derivate unilatere}{Sia \(I\) intervallo , \(f: \ I \to  \RR \) e \(x_0\) \textbf{un punto interno ad \(I\)}.
\begin{enumerate}
    \item Se esiste il limite \[
    \lim_{x \to x_0^{-}} \frac{f(x) - f(x_0)}{x- x_0}=     \lim_{h \to 0^{-}} \frac{f(x_0+h) - f(x_0)}{h} \in \overline{\RR }
    \] (cioè, il limite \textbf{sinistro} del rapporto incrementale), esso viene detto \textbf{derivata sinistra} di \(f\) in \(x_0\) e si indica con \(f'_{-}(x_0)\) 
    \item Analogamente per la derivata destra
\end{enumerate}}

\thm{Legame fra la derivata e le derivate unilatere}{Siano \(f : \ I \to  \RR \) e \(x'\) \textbf{ punto interno} ad \(I\). Allora \[
\exists  f'(x_0) \in \overline{\RR } \quad \text{se e solo se }
\] \[
\exists f'_{+}(x_0) \in \overline{\RR }, \ \exists f'_{-}(x_0) \in \overline{\RR } \quad e \quad f'(x_0)_{-} = f'(x_0)_{+}
\] }
\pf{Dimostrazione}{Inserire dimostrazione}

\section{Classificazione dei punti di non derivabilità}
Siano \(I\) intervallo , \(f: \ I \to  \RR \) e \(x_0\) \textbf{un punto interno ad \(I\)} tale che
\begin{itemize}
    \item \(f\) è continua in \(x_0\) 
    \item \(f\) non è derivabile in \(x_0\)
\end{itemize}
Allora si presentano questi casi
\begin{enumerate}
    \item Punto angoloso;
    \item Punto a tangente verticale;
    \item Cuspide
\end{enumerate}

\dfn{Punto angoloso}{Diciamo che \(x_0\) è un \textbf{punto angoloso} se \[
\exists f'_{-}(x_0), \ f'_{+}(x_0) \in \overline{\RR }, \quad (f'_{-}(x_0) \neq f'_{+}(x_0))
\] e almeno una delle due è finita}
\begin{figure}[]
    \centering
    \begin{tikzpicture}
        \begin{axis}[
            xmin= -2, xmax= 2,
            ymin= -2, ymax = 2,
            axis lines = middle,
        ]
            \addplot[domain=-2:2, samples=100]{x};
        \end{axis}
    \end{tikzpicture}
    \caption{}
    \label{Modulo da cambiare poi}
\end{figure}

\paragraph{Osservazione:} Se \(x_0\) è un punto angoloso \(\not\exists f'(x_0)\) perché \(f'_{+}(x_0) \neq f'_{-}(x_0)\)

\dfn{Punto di flesso a tangente verticale}{Diciamo che \(x_0\) è un \textbf{punto di flesso a tangente verticale} se \[
\exists f'(x_0) \in \{-\infty,+\infty\} 
\] }

\dfn{Cuspide}{Diciamo che \(x_0\) è un \textbf{punto di cuspide} se \[
\exists f'_{-}(x_0),f'_{+}(x_0) \in \{-\infty, +\infty\} , \quad f'_{-}(x_0) \neq f'_{+}(x_0)
\] }
\paragraph{Osservazione:} Dei tre tipi di punti di non derivabilità \[
f'(x_0)
\begin{cases}
    \ \text{esiste se \(x_0\) è punto di flesso a tangente verticale} \\
    \ \text{non esiste se \(x_0\) è punto di cuspide o punto angoloso} \\
\end{cases}
\] 

\paragraph{Osservazione:} In tutti i casi classificati una, fra derivata destra e derivata sinistra della funzione del punto, esiste (finita o infinita). Nel prossimo esempio, \(f\) è continua in \(x_0\), ma \[
\not\exists f'_{-}(x_0), \quad f'_{+}(x_0)
\] 
Il caso in cui ho continuità e non esiste almeno una fra le due non viene classificato
\ex{}{\[
f(x) = 
\begin{cases}
    \ x \sin \frac{1}{x} \quad \text{se} \ x \neq 0 \\
    \ 0 \quad \text{se} \ x = 0 \\
\end{cases}
\] \(f\) è continua in \(x_0=0:\) \[
\lim_{x \to 0} f(x) = \lim_{x \to 0} x \sin \frac{1}{x}= 0
\] 
\nt{\(|x \sin \frac{1}{x}| = |x| |\sin \frac{1}{x}| \le |x| \quad \forall x \in \RR \implies graf (f)\) è compreso fra \(y = x\) e \(y = -x\)}}

\begin{figure}[]
    \centering
    \begin{tikzpicture}
        \begin{axis}[
            xmin= -10, xmax= 10,
            ymin= -10, ymax = 10,
            axis lines = middle,
        ]
            \addplot[domain=-10:10, samples=100]{x * sin(1/x)};
        \end{axis}
    \end{tikzpicture}
    \caption{Non esiste né la derivata destra né quella sinistra}
    \label{}
\end{figure}

Diamo uno strumento per lo studio delle forme indeterminate \[
\frac{0}{0}, \ \frac{\infty}{\infty}
\] 
o riconducibili ad esse.

\section{Teorema di de l'Hopital}
\thm{Il teorema di de l'Hopital per le f.i. \(\frac{0}{0}\)}{Siano \(f, g : \ (a, b) \to  \RR \) funzioni continue su \((a,b)\) e sia \(x_0 \in (a,b)\) tale che \(f(x_0) = g(x_0) = 0\). Supponiamo inoltre che:
\begin{enumerate}
    \item \(f, g\) derivabili in \((a,b) \backslash \{x_0\} \) 
    \item \(g'(x) \neq  0 \quad \forall x \in (a,b) \backslash \{x_0\} \);
    \item ESISTA il limite \(\lim_{x \to x_0} \frac{f'(x)}{g'(x)}= L \in \overline{\RR }\)
\end{enumerate} Allora esiste anche il limite \[
\lim_{x \to x_0} \frac{f(x)}{g(x)}= \lim_{x \to x_0} \frac{f'(x)}{g'(x)}= L \in \overline{\RR }
\] }
\paragraph{Commentiamo l'ipotesi:}
\[
\begin{cases}
    \ f(x_0) = g(x_0) = 0 \\
    \ f, g \ \text{continue in} x_0 \\
\end{cases}
\] 
Quindi \(\lim_{x \to x_0} \frac{f(x)}{g(x)}\) è una forma indeterminata \(\frac{0}{0}\). \\ Ricordiamo che \(f,g\) derivabili in \((a,b) \backslash \{x_0\} \). \\ Questo è in vista del fatto che considero \(\lim_{x \to x_0} \frac{f'(x)}{g'(x)}\).  \[
g'(x) \neq  0 \quad \forall (a,b) \backslash \{x_0\} 
\] 
L'ipotesi misteriosa è \(\exists \lim_{x \to x_0} \frac{f'(x)}{g'(x)}\). Se valgono tutte le ipotesi e esiste il limite allora  \[
\lim_{x \to x_0} \frac{f(x)}{g(x)}= \lim_{x \to x_0} \frac{f'(x)}{g'(x)}
\] la validità dell'uguaglianza è condizionata al fatto che il secondo limite esista.

\thm{Il teorema di de l'Hopital per le f.i. \(\frac{\infty}{\infty}\)}{Siano \(f, g : \ (a, b) \to  \RR \) funzioni continue su \((a,b)\) e sia \(x_0 \in (a,b)\). Supponiamo inoltre che:
\begin{enumerate}
    \item \(f, g\) derivabili in \((a,b) \backslash \{x_0\} \) 
    \item \(\lim_{x \to x_0} f(x) = \lim_{x \to x_0} g(x) = + \infty \)
    \item \(g'(x) \neq  0 \quad \forall x \in (a,b) \backslash \{x_0\} \);
    \item ESISTA il limite \(\lim_{x \to x_0} \frac{f'(x)}{g'(x)}= L \in \overline{\RR }\)
\end{enumerate} Allora esiste anche il limite \[
\lim_{x \to x_0} \frac{f(x)}{g(x)}= \lim_{x \to x_0} \frac{f'(x)}{g'(x)}= L \in \overline{\RR }
\] }

\paragraph{Osservazione:} Anche per le f.i. \(\frac{\infty}{\infty}\), l'uguaglianza \[
\lim_{x \to x_0} \frac{f(x)}{g(x)}= \lim_{x \to x_0} \frac{f'(x)}{g'(x)}
\] è condizionata al fatto che il secondo limite esista.

\ex{Esempio di applicazione}{\[
\lim_{x \to 0} \frac{1-\cos(x)}{x^2}=\lim_{x \to 0} \frac{(1-\cos(x))'}{(x^2)'} = \lim_{x \to 0} \frac{\sin(x)}{2x}= \frac{1}{2}
\]} 
\ex{}{
\[
\lim_{x \to 0} x \log(|x| )
\] 
La riformulo come f.i. quoziente, ho due possibilità 
\begin{itemize}
    \item \(\lim_{x \to 0} x \log(|x| ) = \lim_{x \to 0} \frac{\log(|x| )}{\frac{1}{x}}\) 
    \item \(\lim_{x \to 0} x \log (|x| ) = \lim_{x \to 0} \frac{x}{\frac{1}{\log(|x| )}}\)
\end{itemize} Conviene utilizzare la prima riscrittura per l'applicazione di de l'Hopital. \[
\lim_{x \to 0} x \log (|x| ) = \lim_{x \to 0} \frac{\log(|x| )}{\frac{1}{x}} = \lim_{x \to 0} \frac{f'(x)}{g'(x)}= \lim_{x \to 0} \frac{\frac{|x| }{x} \frac{1}{x}}{-\frac{1}{x^2}} = \lim_{x \to 0} \frac{\frac{1}{x}}{-\frac{1}{x^2}}= \lim_{x \to 0} \frac{1}{x} (-x^2) = 0
\] }

\ex{}{\[
\lim_{x \to 0^{+}} x^{x} = 1
\] La forma esponenziale si riconduce alla forma del prodotto \[
x^{x}= \exp ( \log (x^{x}) ) = \exp (x \log(x)) \to  1
\] \[
\lim_{x \to 0^{+}} x \log (x) = 0 
\] }

A volte è necessario applicare il teorema di de l'Hopital più volte \[
\lim_{x \to 0^{+}} \left( \frac{1}{x}- \frac{1}{\sin(x)} \right) = \lim_{x \to 0^{+}} \frac{\sin(x) - x}{x \sin(x)} = \lim_{x \to 0^{+}} \frac{\cos(x) - 1}{\sin(x) + x \cos(x)} 
\] Non siamo arrivati ad una conclusione quindi dobbiamo ripetere il procedimento \[
\lim_{x \to 0^{+}} \frac{-\sin(x)}{\cos(x) + \cos(x) - x \sin(x)} = \lim_{x  \to 0^{+}} \frac{-\sin(x)}{2\cos(x) - x \sin(x)}=0
\] A posteriori vedo che tutte le uguaglianze condizionate sono vere.

\ex{}{\[
\lim_{x \to +\infty} x^{n}e^{-x}=\lim_{x \to +\infty} \frac{x^{n}}{e^{x}}
\] \[
\lim_{x \to +\infty} \frac{x^{n}}{e^{x}}=\lim_{x \to +\infty}  \frac{n x^{n-1}}{e^{x}}
\] è una forma indeterminata se \(n \ge 2\) \[
= \lim_{x \to +\infty} \frac{n (n-1) x^{n-2}}{e^{x}}
\] è una forma indeterminata se \(n \ge 3\). Applico il teorema di de l'Hopital \(n\) volte \[
\lim_{x \to +\infty} \frac{n!}{e^{x}}= 0
\] }

\ex{}{\[
f(x) = x^2\sin(\frac{1}{x}), \quad g(x) = \sin(x)
\] Provo ad applicare il teorema di de l'Hopital per \[
\lim_{x \to 0} \frac{f(x)}{g(x)}=\lim_{x \to 0} \frac{x^2 \sin (\frac{1}{x})}{\sin(x)} \quad \left[ \frac{0}{0} \right] 
\] \[
= \lim_{x \to 0} \frac{2x \sin(\frac{1}{x}) + x^2 \left( - \frac{1}{x^2} \right) \cos(\frac{1}{x})}{\cos(x)}
\] \[
= \lim_{x \to 0} \frac{2x \sin(\frac{1}{x}) - \cos(\frac{1}{x})}{\cos(x)}
\]Siccome \(\not\exists \lim_{x \to 0} \cos(\frac{1}{x})\) il \(\lim_{x \to 0} \frac{f'(x)}{g'(x)}\) \textbf{non esiste.} Quindi non posso applicare il teorema di de l'Hopital, non sono autorizzato a scrivere \(=\). L'uguaglianza non vale. Infatti \[
\lim_{x \to 0} \frac{x^2 \sin(\frac{1}{x})}{\sin(x)}
\] esiste.}

\thm{Teorema del limite della derivata (enunciato per le derivate destre)}{Sia \(f: [a, b) \to  \RR \) una funzione continua in \(a\) e derivabile in \((a,b)\). Se esiste, finito o no, il limite \(\lim_{x \to a^{+}} f'(x)\), allora esiste anche \(f'_{+}(a)\) e si ha che \[
f'_{+}(a) = \lim_{x \to a^{+}f'(x)} 
\] "La derivata destra di \(f\) in \(a\) coincide con il limite destre di \(f'\) in \(a\)".}
\pf{Dimostrazione}{Considero \[
f'_{+}(a) = \lim_{h \to 0^{+}} \frac{f(a+h) - f(a)}{h} = \lim_{x \to a^{+}} \frac{f(x) - f(a)}{x-a}
\] Quando calcolo questo limite mi trovo in una f.i. \(\frac{0}{0}\), infatti \(f\) è continua in \(a\) per ipotesi \(\implies f(x) \to f(a)\) per \(x \to a^{+} \implies f(x) - f(a) \to 0\) per \(x \to a^{+}\). Le ipotesi del teorema mi permettono di applicare il teorema di de l'Hopital \[
= \lim_{x \to a^{+}} \frac{(f(x) - f(a))'}{(x-a)'}=\lim_{x \to a^{+}} \frac{f'(x)}{1}= \lim_{x \to a^{+}} f'(x)
\] Per ipotesi, ho richiesto che \(\exists \lim_{x \to a^{+}} f'(x)\). L'uguaglianza non è più condizionata e deduco \[
\lim_{x \to a^{+}} \frac{f(x) - f(a)}{x-a} = \lim_{x \to a^{+}} f'(x)
\] }
\paragraph{Osservazione:} Sotto ipotesi analoghe vale anche il risultato per  \[
f'_{-}(b) = \lim_{x \to b ^{-}} f'(x)
\] Ma si consiglia negli esercizi di non usare i limiti unilateri di \(f'\) al posto delle derivate unilatere.

\chapter{Proprietà globali delle funzioni derivabili}
I risultati che ci permetteranno di fare lo studio del grafico qualitativo delle funzioni si basano su altri risultati teorici che legano 
\begin{itemize}
    \item proprietà delle derivate
    \item proprietà globali delle funzioni
\end{itemize}
\end{document}
